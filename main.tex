\documentclass{article}
\usepackage[utf8]{inputenc}
\usepackage{xcolor}
\usepackage{hyperref}
\usepackage{listings}
\usepackage{comment}
\title{CompTIA A+ Bible}
\author{1000+ Questions and Solutions}
\date{December 2020}

\begin{document}

\maketitle
\section{Questions} 
\begin{enumerate}
    \item What are the six step troubleshooting process? 
    \begin{enumerate}
        \item Identify the problem
        \item Establish a theory of probable cause. (Question the obvious) 
        \item Test the theory to determine cause
        \item Establish a plan of action to resolve the problem and implement a solution
        \item Verify full system functionality and, if applicable, implement preventative measures. 
        \item Document findings, actions, and outcomes 
    \end{enumerate}
    \item To which type of technology would you install a x16 card?
    \begin{enumerate}
        \item PCIe (Peripheral Component Interconnect express) is the most common expansion slot for video cards (by sixteen) 
    \end{enumerate}
    \item Which process of the computer checks all your components dur‐
ing boot?
    \begin{enumerate}
        \item POST (Power-On Self Test) is a process that is part of the BIOS (Basic Input Output System) or Unified Extensible Firmware Interface (UEFI). It runs a self-check of the computer system during boot and stores many of the parameters of the components within the CMOS. 
    \end{enumerate}
    \item Which of the following could cause the POST to fail? (Select
the two best answers.) 
    \begin{enumerate}
        \item The CPU and Memory need to be installed properly for the POST to run (and to pass). 
    \end{enumerate}
    
    \item Which of the following might you find as part of a tablet com‐
puter? (Select the two best answers.)   
    \begin{enumerate}
        \item A tablet computer will almost always  contain flash memory as main storage and multi-touch screen. 
    \end{enumerate}
    \item Which kind of socket incorporates “lands” to ensure connectiv‐
ity to a CPU? 
    \begin{enumerate}
        \item LGA (land grid array) is the type of socket that
uses “lands” to connect the socket to the CPU.
    \end{enumerate}
    \item How many pins are inside an SATA 3.0 data connector? 
    \begin{enumerate}
        \item The SATA version 3.0 data connector has seven pins. Note: SATA express uses a tiple connector with 18 pins ( 7 + 7 + 4) 
    \end{enumerate}
    \item What is the minimum number of hard drives necessary to imple‐
ment RAID 5?
    \begin{enumerate}
        \item Because RAID 5 uses striping with parity, a third disk is needed. You can have more than three disks as well. 
    \end{enumerate}
    \item A user’s time and date keep resetting to January 1, 2012. Which
of the following is the most likely cause?
    \begin{enumerate}
        \item If the time and date keep resetting - for example, to a date such as January 1,2012 - chances are that the lithium battery needs to be replaced. These are usually nickle-sized batteries; most PCs use a CR2032 lithium battery. 
    \end{enumerate}
    \item Which type of adapter card is normally plugged into a PCIe x16
adapter card slot?
    \begin{enumerate}
        \item The PCI Express (PCIe) x16 expansion slot is used
primarily for video.
    \end{enumerate}
    \item Which of the following CPU cooling methods is the most com‐
mon?
    \begin{enumerate}
        \item The most common CPU cooling method is the heat
sink and fan combination. The heat sink helps the heat to disperse
away from the CPU, whereas the fan blows the heat down and
through the fins; the power supply exhaust fan and possibly addi‐
tional case fans help the heat escape the case. Heat sink and fan
combinations are known as active cooling methods.
    \end{enumerate}
    \item What type of power connector is used for a x16 video card?
    \begin{enumerate}
        \item PCIe 6-pin
    \end{enumerate}
    \item What does the b in 1000 Mbps stand for?
    \begin{enumerate}
        \item The b in 1000 Mbps stands for bits: 1000 Mbps is
1000 megabits per second or 1 gigabit per second. Remember
that the lowercase b is used to indicate bits when measuring net‐
work data transfer rates, USB data transfer rates, and other simi‐
lar serial data transfers.
    \end{enumerate}
    \item When running cable through drop ceilings, which type of cable do
you need?
    \begin{enumerate}
        \item Plenum-rated cable needs to be installed wherever
a sprinkler system is not able to spray water. This includes ceil‐
ings, walls, and plenums (airways). Plenum-rated cable has a pro‐
tective covering that burns slower and gives off fewer toxic
fumes than regular PVC-based cable.
    \end{enumerate}
    \item Which of the following is the default subnet mask for IP address
192.168.1.1?
    \begin{enumerate}
        \item 192.168.1.1, by default, has the subnet mask
255.255.255.0, which is the standard subnet mask for class C IP
addresses. However, remember that some networks are classless,
which means that a network can use a different subnet mask.
    \end{enumerate}
    \item Which of the following is the minimum category cable needed for
a 1000BASE-T network?
    \begin{enumerate}
        \item The minimum cable needed for 1000BASE-T net‐
works is Category 5e. Of course, Cat 6 would also work, but it is
not the minimum of the listed answers. 1000BASE-T specifies
the speed of the network (1000 Mbps), the type (baseband, single
shared channel), and the cable to be used (T = twisted pair).
    \end{enumerate}
    \item Which of the following IP addresses can be routed across the In‐
ternet?
    \begin{enumerate}
        \item The only listed answer that is a public address
(needed to get onto the Internet) is 129.52.50.13
        \item All the other answers are private IPs, meant
to be behind a firewall. 127.0.0.1 is the IPv4 local loopback IP
address. 192.168.1.1 is a common private IP address used by
SOHO networking devices. 10.52.50.13 is a private address.
Note that the 10 network is common in larger networks.
    \end{enumerate}
    \item Which port number is used by HTTPS by default?
    \begin{enumerate}
        \item The Hypertext Transfer Protocol Secure (HTTPS)
uses port 443 (by default).
        \item Port 21 is used by the File Transfer Protocol
(FTP). Port 25 is used by the Simple Mail Transfer Protocol
(SMTP). Port 80 is used by regular HTTP, which is considered to
be insecure.
    \end{enumerate}
    \item Which of the following cable types have a copper medium? (S‐
elect the three best answers.)
    \begin{enumerate}
        \item Twisted-pair, coaxial, and Category 7 cable are all
examples of network cables with a copper medium. They all send
electricity over copper wire. 
        \item ultimode is a type of fiber-optic cable; it
uses light to send data over a glass or plastic medium. Twisted
pair is the most common type of cabling used in today’s net‐
works.
    \end{enumerate}
    \item Which of the following cable types can protect from electromag‐
netic interference (EMI)? (Select the two best answers.)
    \begin{enumerate}
        \item Shielded twisted pair (STP) and fiber optic can
protect from EMI. 
        \item Unshielded twisted pair (UTP) cannot pro‐
tect from EMI. Unless otherwise mentioned, Category 6 cable is
UTP. STP is shielded twisted pair. Unlike UTP (unshielded
twisted pair), STP provides an aluminum shield that protects
from EMI. UTP and coaxial have no such protection. Fiber optic
uses a different medium altogether, transmitting light rather than
electricity; therefore, EMI cannot affect fiber-optic cables.
    \end{enumerate}
    \item You are configuring Bob’s computer to access the Internet. Which
of the following are required? (Select all that apply.)
    \begin{enumerate}
        \item To get on the Internet, the DNS server address is
required so that the computer can get the resolved IP addresses
from the domain names that are typed in. The gateway address is
necessary to get outside the network.
    \end{enumerate}
    \item A customer wants to access the Internet from many different loca‐
tions in the United States. Which of the following is the best tech‐
nology to enable the customer to do so?
    \begin{enumerate}
        \item Cellular WAN uses a phone or other mobile device
to send data over standard cellular connections.
    \end{enumerate}
    \item You just configured the IP address 192.168.0.105 in Windows.
When you press the Tab key, Windows automatically configures
the default subnet mask of 255.255.255.0. Which of the following
IP addresses is a suitable gateway address?
    \begin{enumerate}
        \item 192.168.0.1 is the only suitable gateway address.
Remember that the gateway address must be on the same net‐
work as the computer. In this case, the network is 192.168.0, as
defined by the 255.255.255.0 subnet mask.
    \end{enumerate}
    \item You have been tasked with blocking remote logins to a server.
Which of the following ports should you block?
    \begin{enumerate}
        \item ort 23 should be blocked. It is associated with the
Telnet service, which is used to remotely log in to a server at the
command line. You can block this service at the company fire‐
wall and individually at the server and other hosts. It uses port 23
by default, but it can be used with other ports as well. Telnet is
considered to be insecure, so it should be blocked and disabled.
    \end{enumerate}
    \item Which	of	the	following	connector	types	is	used	by	fiber-optic	ca‐
bling?

    \begin{enumerate}
        \item The	LC	connector	is	used	by	fiber-optic	cabling.
Other	fiber	connectors	include	SC	and	ST.
        \item RJ45	is	the	connector	used	by	twisted-pair
networks.	RG-6	is	the	cable	used	by	cable	Internet	and	TV;	an	F-
connector	is	attached	to	the	ends	of	an	RG-6	cable.	RJ11	is	the
standard	phone	line	connector.

    \end{enumerate}
    \item Which	protocol	uses	port	53?

    \begin{enumerate}
        \item The	Domain	Name	System	(DNS)	protocol	uses
port	53	by	default. 
        \item TP	uses	port	21.	SMTP	uses	port	25	(or	587 or	465).	HTTP uses port	80.

    \end{enumerate}
    \item Which	of	the	following	terms	best	describes	two	or	more	LANs
connected	over	a	large	geographic	distance?

    \begin{enumerate}
        \item A	wide-area	network	(WAN)	is	a	network	in	which
two	or	more	LANs	are	connected	over	a	large	geographic	dis‐
tance—for	example,	between	two	cities.	The	WAN	requires	con‐
nections	to	be	provided	by	a	telecommunications	or	data	commu‐
nications	company.
    \end{enumerate} 
        \item Which	device	connects	to	the	network	and	has	the	sole	purpose	of
providing	data	to	clients?
    \begin{enumerate}
        \item Network-attached	storage	(NAS)	devices	store	data
for	network	use.	They	connect	directly	to	the	network.
    \end{enumerate}
    \item You	are	making	your	own	networking	patch	cable.	You	need	to
attach	an	RJ45	plug	to	the	end	of	a	twisted-pair	cable.	Which	tool
should	you	use?

    \begin{enumerate}
        \item 	Use	an	RJ45	crimper	tool	to	permanently	attach
RJ45	plugs	to	the	end	of	a	cable

    \end{enumerate}
    \item Which	port	is	used	by	RDP?
    \begin{enumerate}
        \item The	Remote	Desktop	Protocol	(RDP)	uses	port
3389	by	default.	This	protocol	allows	one	computer	to	take control	of	another	remote	system.

    \end{enumerate}
    \item Which	of	the	following	printer	failures	can	be	described	as	a	con‐
dition	in	which	the	internal	feed	mechanism	stopped	working
temporarily?

    \begin{enumerate}
        \item A	failure	that	occurs	due	to	the	internal	feed	mech‐
anism	stopping	is	known	as	a	paper	jam.	For	example,	an	HP
LaserJet	might	show	error	code	13.1	on	the	display,	which	means
a	paper	jam	at	the	paper	feed	area.	You	should	verify	that	the	pa‐
per	trays	are	loaded	and	adjusted	properly.

    \end{enumerate}
    \item Which	type	of	printer	uses	a	toner	cartridge?

    \begin{enumerate}
        \item Laser	printers	use	toner	cartridges.

    \end{enumerate}
    \item Which	of	the	following	should	not	be	connected	to	a	UPS?

    \begin{enumerate}
        \item Laser	printers	use	large	amounts	of	electricity,
which	in	turn	could	quickly	drain	the	battery	of	the	UPS.	They
should	be	plugged	in	to	their	own	individual	power	strips.

    \end{enumerate}
    \item Special	paper	is	needed	to	print	on	which	type	of	printer?

    \begin{enumerate}
        \item 	Regular	paper	can	be	used	on	all	the	listed	printers
except	for	thermal	printers,	which	use	specially	coated	paper	that
is	heated	to	create	the	image.

    \end{enumerate}
    \item Which	of	the	following	channels	should	you	select	for	an	802.11
wireless	network?

    \begin{enumerate}
        \item Of	the	listed	answers,	use	channel	6	for	802.11
wireless	networks.	That	would	imply	a	2.4	GHz	connection	using
either	802.11n,	g,	or	b.	The	2.4	GHz	frequency	range	in	the
United	States	allows	for	channels	1	through	11.

    \end{enumerate}
    \item Which	environmental	issue	affects	a	thermal	printer	the	most?

    \begin{enumerate}
        \item Which	type	of	printer	uses	impact	to	transfer	ink	from	a	ribbon	to
the	paper?

    \end{enumerate}
    \item Which	of	the	following	steps	enables	you	to	take	control	of	a	net‐
work	printer	from	a	remote	computer?

    \begin{enumerate}
        \item 	After	you	install	the	driver	for	the	printer	locally,
you	can	then	take	control	of	it	by	going	to	the	properties	of	the
printer	and	accessing	the	Ports	tab.	Then	click	the	Add	Port	but‐
ton	and	select	the	Standard	TCP/IP	Port	option.	You	have	to
know	the	IP	address	of	the	printer	or	the	computer	that	the	printer
is	connected	to.

    \end{enumerate}
    \item 	A	color	laser	printer	produces	images	that	are	tinted	blue.	Which
of	the	following	steps	should	be	performed	to	address	this	prob‐
lem?

    \begin{enumerate}
        \item After	you	install	a	printer,	it	is	important	to	cali‐
brate	it	for	color	and	orientation,	especially	if	you	are	installing	a
color	laser	printer	or	an	inkjet	printer.	These	calibration	tools	are
usually	built	in	to	the	printer’s	software	and	can	be	accessed	from
Windows,	or	you	can	access	them	from	the	printer’s	display.

    \end{enumerate}
    \item A	desktop	computer	does	not	have	a	lit	link	light	on	the	back	of
the	computer.	Which	of	the	following	is	the	most	likely	reason	for
this?

    \begin{enumerate}
        \item The	most	likely	answer	in	this	scenario	is	that	the
network	cable	is	disconnected.	If	the	desktop	computer	is	using	a
wired	connection,	it	is	most	likely	a	twisted-pair	Ethernet	con‐
nection.	When	this	cable	is	connected	to	the	computer	on	one	end
and	to	a	switch	or	other	central	connecting	device	on	the	other
end,	it	initiates	a	network	connection	over	the	physical	link.	This
link	then	causes	the	network	adapter’s	link	light	to	light	up.	The
link	light	is	directly	next	to	the	RJ45	port	of	the	network	adapter.
The	corresponding	port	on	the	switch	(or	other	similar	device)	is
also	lit.	If	the	cable	is	disconnected,	the	link	light	becomes	unlit,
though	there	are	other	possibilities	for	this	link	light	to	be	dark—
for	example,	if	the	computer	is	off	or	if	the	switch	port	is	dis‐
abled.

    \end{enumerate}
    \item Which	of	the	following	IP	addresses	would	a	technician	see	if	a
computer	running	Windows	is	connected	to	a	multifunction	net‐
work	device	and	is	attempting	to	obtain	an	IP	address	automati‐
cally	but	is	not	receiving	an	IP	address	from	the	DHCP	server?

    \begin{enumerate}
        \item If	the	computer	fails	to	obtain	an	IP	address	from	a
DHCP	server,	Windows	will	take	over	and	apply	an	Automatic
Private	IP	Address	(APIPA).	This	address	will	be	on	the
169.254.0.0	network.

    \end{enumerate}
    \item For	which	type	of	PC	component	are	80	mm	and	120	mm	com‐
mon	sizes?

    \begin{enumerate}
        \item Case	fans	are	measured	in	mm	(millimeters);	80
mm	and	especially	120	mm	are	very	common.	They	are	used	to
exhaust	heat	out	of	the	case.	These	fans	aid	in	keeping	the	CPU
and	other	devices	cool.	The	120	mm	is	quite	common	in	desktop
and	tower	PCs,	and	the	80	mm	is	more	common	in	smaller	sys‐
tems	and	1U	and	2U	rackmount	servers.

    \end{enumerate}
    \item An	exclamation	point	next	to	a	device	in	the	Device	Manager	in‐
dicates	which	of	the	following?

    \begin{enumerate}
        \item If	you	see	an	exclamation	point	in	the	Device	Man‐
ager,	this	indicates	that	the	device	does	not	have	a	proper	driver.

    \end{enumerate}
    \item Beep	codes	are	generated	by	which	of	the	following?

    \begin{enumerate}
        \item As	the	power-on	self-test	(POST)	checks	all	the
components	of	the	computer,	it	may	present	its	findings	on	the
screen	or	in	the	form	of	beep	codes.

    \end{enumerate}
    \item Which	of	the	following	indicates	that	a	printer	is	network-ready?

    \begin{enumerate}
        \item The	RJ45	jack	enables	a	connection	to	a	twisted-
pair	(most	likely	Ethernet)	network.	Printers	with	a	built-in	RJ45
connector	are	network-ready;	so	are	printers	that	are	Wi-Fi	en‐
abled.

    \end{enumerate}
    \item You	just	turned	off	a	printer	to	maintain	it.	Which	of	the	follow‐
ing	should	you	be	careful	of	when	removing	the	fuser?

    \begin{enumerate}
        \item 	The	fuser	heats	paper	to	around	400°	Fahrenheit
(204°	Celsius).	That’s	like	an	oven.	If	you	need	to	replace	the
fuser,	let	the	printer	sit	for	10	or	15	minutes	after	shutting	it	down
and	before	maintenance.

    \end{enumerate}
    \item Which	of	the	following	connectors	is	used	for	musical	equipment?

    \begin{enumerate}
        \item The	Musical	Instrument	Digital	Interface	(MIDI)
connector	is	used	for	musical	equipment	such	as	keyboards,	syn‐
thesizers,	and	sequencers.	MIDI	is	used	to	create	a	clocking	sig‐
nal	that	all	devices	can	synchronize	to.
        
    \end{enumerate}
    \item Which	of	the	following	storage	technologies	is	used	by	hard
disk	drives?

    \begin{enumerate}
        \item Hard	disk	drives	(HDDs)	are	magnetic	disks.	Th‐
ese	are	the	type	with	moving	parts,	as	opposed	to	solid-state
drives	(SSDs)	that	have	no	moving	parts.

    \end{enumerate}
    \item Which	protocol	uses	port	389?

    \begin{enumerate}
        \item LDAP 
    \end{enumerate}
    \item Which	of	the	following	wireless	networking	standards	operates
at	5	GHz	only?	(Select	the	two	best	answers.)

    \begin{enumerate}
        \item :	802.11a	operates	at	5	GHz	only;	so	does	802.11ac.

    \end{enumerate}
    \item Which	of	the	following	types	of	RAM	has	a	peak	transfer	rate	of
21,333	MB/s?

    \begin{enumerate}
        \item DDR4-2666	has	a	peak	transfer	rate	of	21,333
MB/s.	It	runs	at	an	I/O	bus	clock	speed	of	1333	MHz	and	can
send	2666	megatransfers	per	second	(MT/s).	It	is	also	known	as
PC4-21333.

    \end{enumerate}
    \item Which	of	the	following	types	of	printers	uses	a	print	head,	ribbon,
and	tractor	feed?

    \begin{enumerate}
        \item The	impact	printer	uses	a	print	head,	ribbon,	and
tractor	feed.	An	example	of	an	impact	printer	is	the	dot	matrix.

    \end{enumerate}
    \item Which	of	the	following	is	a	possible	symptom	of	a	failing	CPU?

    \begin{enumerate}
        \item If	the	CPU	is	running	beyond	the	recommended
voltage	range	for	extended	periods	of	time,	it	can	be	a	sign	of	a
failing	CPU.	The	problem	could	also	be	caused	by	overclocking.
Check	in	the	UEFI/BIOS	to	see	whether	or	not	the	CPU	is	over‐
clocked.

    \end{enumerate}
    \item Which	of	the	following	cable	types	is	not	affected	by	EMI	but	re‐
quires	specialized	tools	to	install?

    \begin{enumerate}
        \item Fiber-optic	cable	is	the	only	answer	listed	that	is
not	affected	by	electromagnetic	interference	(EMI).	The	reason	is
that	it	does	not	use	copper	wire	or	electricity,	but	instead	uses
glass	or	plastic	fibers	and	light.
    \end{enumerate}
    \item Setting	an	administrator	password	in	the	BIOS	accomplishes
which	of	the	following?

    \begin{enumerate}
        \item Changing boot order
    \end{enumerate}
    \item What	is	an	LCD	display’s	contrast	ratio	defined	as?
    \begin{enumerate}
        \item 	Contrast	ratio	is	the	brightness	of	the	brightest
color	(measured	as	white)	compared	to	the	darkest	color	(mea‐
sured	as	black).	Static	contrast	ratio	measurements	are	static;	they
are	performed	as	tests	using	a	checkerboard	pattern.	But	there	is
also	the	dynamic	contrast	ratio,	a	technology	in	LCD	displays
that	adjusts	dynamically	during	darker	scenes	in	an	attempt	to
give	better	black	levels.	It	usually	has	a	higher	ratio,	but	it	should
be	noted	that	there	is	no	real	uniform	standard	for	measuring	con‐
trast	ratio.

    \end{enumerate}
    \item Which	of	the	following	devices	limits	network	broadcasts,	seg‐
ments	IP	address	ranges,	and	interconnects	different	physical	me‐
dia?

    \begin{enumerate}
        \item A	router	can	limit	network	broadcasts	through	seg‐
menting	and	programmed	routing	of	data.	This	is	part	of	a
router’s	job	when	connecting	two	or	more	networks.	It	is	also
used	with	different	media.	For	example,	you	might	have	a	LAN
that	uses	twisted-pair	cable,	but	the	router	connects	to	the	Inter‐
net	via	a	fiber-optic	connection.	That	one	router	will	have	ports
for	both	types	of	connections.

    \end{enumerate}
    \item Which	of	the	following	traits	and	port	numbers	are	associated
with	POP3?	(Select	the	two	best	answers.)

    \begin{enumerate}
        \item OP3	is	a	protocol	used	by	email	clients	to	receive
email.	It	makes	use	of	either	port	110	(considered	insecure)	or
port	995	(a	default	secure	port).

    \end{enumerate}
    \item Which	of	the	following	uses	port	427?

    \begin{enumerate}
        \item The	Service	Location	Protocol	(SLP)	uses	port
427.	It	enables	access	to	network	services	without	previous	con‐
figuration	of	the	client	computer.

    \end{enumerate}
    \item Which	of	the	following	ports	is	used	by	AFP?

    \begin{enumerate}
        \item The	Apple	Filing	Protocol	(AFP)	uses	port	548.
AFP	offers	file	services	for	Mac	computers	running	macOS	and
can	transfer	files	across	the	network.

    \end{enumerate}
    \item Which of the following is the minimum amount of RAM needed
to install a 64-bit version of Windows 10?
    \begin{enumerate}
        \item 
    \end{enumerate}
    \item 
    \begin{enumerate}
        \item 
    \end{enumerate}
    \item 
    \begin{enumerate}
        \item 
    \end{enumerate}
    \item 
    \begin{enumerate}
        \item 
    \end{enumerate}
    \item 
    \begin{enumerate}
        \item 
    \end{enumerate}
    \item 
    \begin{enumerate}
        \item 
    \end{enumerate}
    \item 
    \begin{enumerate}
        \item 
    \end{enumerate}
    \item 
    \begin{enumerate}
        \item 
    \end{enumerate}
    
\end{enumerate}
\section{Chapter 9}
\section{Tables}
\begin{tabular}{|c|c|c|}
\hline
    Technology & Full Name & Port \\
    \hline
   SLP  & Service Location Protocol & 427\\
   FTP & File Transfer Protocol & 21 \\
   DNS & Domain Name System & 53\\
   AFP & Apple Filing Protocol & 548 \\
   HTTP & HyperText Transfer Protocol & 80 \\
   SSH & Secure Shell & 22 \\
   Telnet & Teletype Network & 23\\
   IMAP & Internet Message Access Protocol & 143\\
   \hline
\end{tabular}



\section{Troy McMillan compTIA A+} 
\subsection{Mobile Devices}
\begin{enumerate}
    \item What is the maximum transmission speed of an ExpressCard in PCIe2 mode?
    \begin{itemize}
        \item D) The maximum transmission speeds are as follows: 280 Mbps effective (USB 2 mode) \\
        1.6 Gbps effective (PCIe 1 mode) \\
        3.2 Gbps effective (PCIe 2 or USB 3 mode) 
    \end{itemize}
    \item Which interface is natively found only in Apple devices?
    \begin{itemize}
        \item C) Thunderbolt ports are most likely to be found on Apple laptops, although they are now
showing up on others as well. USB ports are typically found on all mobile devices, while
Serial and PS/2 connecters are rarely found on mobile devices.
    \end{itemize}
     \item What special screwdriver is typically required to work on a notebook?
    \begin{itemize}
        \item B) Some models of notebook PCs require a special T-8 Torx screwdriver. Most PC toolkits
come with a T-8 bit for a screwdriver with interchangeable bits, but you may find that the
T-8 screws are countersunk in deep holes so that you can’t fit the screwdriver into them. In
such cases, you need to buy a separate T-8 screwdriver, available at most hardware stores
or auto parts stores. Phillips-head screwdrivers have a cross pattern on the tip and may be
required. Hex heads are another type you may encounter, and metric drivers are those that
are sized with the metric system.
    \end{itemize}
     \item What is the easiest thing to damage when removing a laptop keyboard?
    \begin{itemize}
        \item B) When replacing the keyboard, one of the main things you want to keep in mind is to not
damage the data cable connector to the system board.
    \end{itemize}
     \item Which component if damaged can render the hard drive useless?
    \begin{itemize}
        \item C) If required, remove the connector attached to the old drive’s signal pins and attach it
to the new drive. Make sure it’s right side up and do not force it. Damaging the signal pins
may render the drive useless. The caddy, rails, and chassis are not easily damaged.
    \end{itemize}
     \item What size hard drive goes in a laptop?
    \begin{itemize}
        \item C) The 2.5-inch hard drives are smaller (which makes them attractive for a laptop where
space is at a minimum); however, in comparison to 3.5-inch hard drives, they have less
capacity and cache, and they operate at a lower speed.
    \end{itemize}
     \item Which is not an advantage of solid-state drives?
    \begin{itemize}
        \item A) The advantage of solid-state drives is that they are not as susceptible to damage if the
device is dropped, and they are, generally speaking, faster as no moving parts are involved.
They are, however, more expensive, and when they fail they don’t generally give some
advance symptoms like a magnetic drive will do.
    \end{itemize}
     \item Which display uses a row of transistors across the top of the screen and a column of them
down the side?
    \begin{itemize}
        \item A passive matrix screen uses a row of transistors across the top of the screen and a
column of them down the side. It sends pulses to each pixel at the intersection of each row
and column combination, telling it what to display. An active matrix screen uses a separate
transistor for each individual pixel in the display, resulting in higher refresh rates and
brighter display quality. Twisted nematic (TN) is the older of the two major technologies
for flat-panel displays. While it provides the shortest response time, has high brightness,
and draws less power than competing technologies, it suffers from poor quality when
viewed from wide angles. In-Plane Switching (IPS) is a newer technology that solves the
issue of poor quality at angles other than straight on.
    \end{itemize}
    \item Which display is a newer technology that solves the issue of poor quality at angles other
than straight on?
    \begin{itemize}
        \item D) In-Plane Switching (IPS) is a newer technology that solves the issue of poor quality
at angles other than straight on. A passive matrix screen uses a row of transistors across
the top of the screen and a column of them down the side. It sends pulses to each pixel at
the intersection of each row and column combination, telling it what to display. An active
matrix screen uses a separate transistor for each individual pixel in the display, resulting in
higher refresh rates and brighter display quality. Twisted nematic (TN) is the older of the
two major technologies for flat-panel displays. While it provides the shortest response time,
has high brightness, and draws less power than competing technologies, it suffers from
poor quality when viewed from wide angles.
    \end{itemize}
    \item Which of the following lets you quickly connect/disconnect with external peripherals and
may also provide extra ports that the notebook PC doesn’t normally have?
    \begin{itemize}
        \item D) With a hot dock, a laptop once put into suspended mode will recognize plug-and-play
devices. A docking station essentially allows a laptop computer to be converted to a desktop
computer. Laptop and table locks are used to secure mobile devices.
    \end{itemize}
    \item In what mode of plug and play must the laptop be turned off and back on for the change to
be recognized?
    \begin{itemize}
        \item C) In cold docking, the laptop must be turned off and back on for the change to be
recognized. In warm docking, the laptop must be put in and out of suspended mode for the
change to be recognized. In hot docking, the change can be made and is recognized while
running normal operations.
    \end{itemize}
    \item Which of the following is a class of devices that specializes in tracking your movement?
    \begin{itemize}
        \item A)While many smart watches can also act as fitness monitors, there is a class of devices
that specializes in tracking your movement. Fitness monitors read your body temperature,
heart rate, and blood pressure. Extended reality is an exciting new field that includes
both augmented reality and virtual reality. Today’s smartphones are really computers
that can make calls, and tablets have been in existence in some form or fashion since the
early 1990s. Early on they were proprietary devices that didn’t have a lot in common with
desktop computers, but increasingly the two form factors have gravitated toward one
another.
    \end{itemize}
    \item Which of the following uses satellite information to plot the global location of an object
and uses that information to plot the route to a second location?
    \begin{itemize}
        \item A) A global positioning system (GPS) uses satellite information to plot the global location
of an object and uses that information to plot the route to a second location. Geofencing is
the use of GPS to restrict communication to an area. Remote wipe is the cleaning of data
from a lost or stolen device. There is no such thing as local wipe.
    \end{itemize}
    \item Which interface is the most common port found on mobile devices?
    \begin{itemize}
        \item A) he two most common ports found on mobile devices are micro-USB and mini-USB.
Both are small–form-factor implementations of the USB standard, the latest of which is
USB 3.1. Thunderbolt ports are most likely to be found on Apple laptops, but they are
now showing up on others as well. Serial and PS/2 connecters are rarely found on mobile
devices. 
    \end{itemize}
    \item Which is the most common pin code when selecting discovered Bluetooth devices?
    \begin{itemize}
        \item A) Many external devices will ask for a PIN when you select the external device from the
list of discovered devices. In many cases, the PIN is 0000, but you should check the manual
of the external device
    \end{itemize}
    \item Which of the following is the connection between the mobile device and the radio?
    \begin{itemize}
        \item A) The product release information (PRI) is the connection between the mobile device and
the radio. From time to time this may need updating, which, when done, may add features or
increase data speed. The preferred roaming list (PRL) is a list of radio frequencies residing in
the memory of some kinds of digital phones. International Mobile Equipment Identification
(IMEI) is used to identify a physical phone device, while International Mobile Subscriber
Identification (IMSI) is used to identify a Subscriber Identification Module (SIM) card.
    \end{itemize}
    \item Which of the following is a process whereby not only does the server verify the credential of
the client but the client also verifies the credential of the server?
    \begin{itemize}
        \item A) Mutual authentication is a process whereby not only does the server verify the
credential of the client but the client also verifies the credential of the server. It adds
additional security to the process. Single sign-on is a service that allows users to sign in
once and have access to all resources. Multifactor authentication makes use of multiple
factors of authentication to increase security. Biometrics is the use of physical factors of
authentication.
    \end{itemize}
    \item Which of the following is the use of physical factors of authentication?
    \begin{itemize}
        \item D) Mutual authentication is a process whereby not only does the server verify the
credential of the client but the client also verifies the credential of the server. It adds
additional security to the process. Single sign-on is a service that allows users to sign in
once and have access to all resources. Multifactor authentication makes use of multiple
factors of authentication to increase security. Biometrics is the use of physical factors of
authentication.
    \end{itemize}
\end{enumerate}
\subsection{Networking} 
\begin{enumerate}
    \item Which the following uses port 110?
    \begin{itemize}
        \item D) POP3 uses port 110. SSH uses port 22, FTP uses ports 20 and 21, and Telnet uses port 23.
    \end{itemize}
    \item Which of the following uses two ports?
     \begin{itemize}
        \item A) FTP uses ports 20 and 21. POP3 uses port 110, SSH uses port 22, and Telnet uses port 23.
    \end{itemize}
    \item Which the following uses port 22?
     \begin{itemize}
        \item B)SSH uses port 22, POP3 uses port 110, FTP uses ports 20 and 21, and Telnet uses port 23.
    \end{itemize}
    \item Which device operates at layer 2?
     \begin{itemize}
        \item A) Switches operate at layer 2. Routers operate at layer 3. Repeaters and hubs operate at layer 1.
    \end{itemize}
    \item Which device operates at layer 1?
     \begin{itemize}
        \item D. Hubs operate at layer 1. Switches and bridges operate at layer 2. Routers operate at layer 3.
    \end{itemize}
    \item Which device operates at layer 2?
     \begin{itemize}
        \item B. Switches operate at layer 2. Routers operate at layer 3. Hubs and repeaters operate at layer 1.
    \end{itemize}
    \item Which of the following is not a private IP address range?
     \begin{itemize}
        \item B. The class B range is 172.16.0.0–172.31.255.255. The other ranges are correct.
    \end{itemize}
    \item Which of the following delivers an upload speed equal to the download speed?
     \begin{itemize}
        \item A) Symmetric DSL (SDSL) offers an upload equal to the download speed. The other
versions all have slower upload speed than download speed. 
    \end{itemize}
    \item Which of the following is an area where you can place a public server for access by people
you might not trust otherwise?
     \begin{itemize}
        \item B). A demilitarized zone (DMZ) is an area where you can place a public server for access
by people you might not trust otherwise. NAT is a service that maps private IP addresses
to public IP addresses. The intranet is the internal network that should be protected. The
Internet is the untrusted public network.
    \end{itemize}
    \item Which of the following operates in the 5.0 GHz range?
     \begin{itemize}
        \item A)802.11a operates in the 5.0 GHz range. The other standards all operate in the 2.4 GHz
range.
    \end{itemize}
    \item Which of the following operates at a maximum of 2 MB?
     \begin{itemize}
        \item D) 802.11a and 802.11g have a maximum rate of 54 MB, 802.11b has a maximum of 11 MB,
and 802.11 has a maximum of 2 MB.
    \end{itemize}
    \item Which of the following has the largest cell size?
     \begin{itemize}
        \item C) 802.11g has a distance that is the cell size of 125 ft. The others have a distance of 115 ft.
    \end{itemize}
    \item Which type of server resolves IP addresses to hostnames?
     \begin{itemize}
        \item B) DNS servers resolve IP addresses to hostnames. HTTP servers are web servers. DHCP
servers provide automatic IP configurations. SQL is a database server.
    \end{itemize}
    \item Which type of server provides automatic IP configurations?
     \begin{itemize}
        \item C) DHCP servers provide automatic IP configurations. DNS servers resolve IP addresses to
hostnames. HTTP servers are web servers. SQL is a database server.
    \end{itemize}
    
    \item Which type of server is a database server?
     \begin{itemize}
        \item D) A SQL server is a database server. DNS servers resolve IP addresses to hostnames.
HTTP servers are web servers. DHCP servers provide automatic IP configurations.
    \end{itemize}
    \item Which of the following is a Class B address?
     \begin{itemize}
        \item The Class B range is 128–191. The class A range is 1–126. The Class C range is 192–223.
    \end{itemize}
    \item Which of the following is a Class A address?
     \begin{itemize}
        \item The Class A range is 1–126. The class B range is 128–191. The Class C range is 192–223.
    \end{itemize}
    \item Which of the following is a Class C address?
     \begin{itemize}
        \item The Class C range is 192–223. The class A range is 1–126. The class B range is 128–191.
The 224 range is for multicasting.
    \end{itemize}
    \item When personal devices include networking capabilities and can communicate directly with
one another, they create which type of network?
     \begin{itemize}
        \item A personal area network (PAN) is a LAN created by personal devices. A wide area
network (WAN) is a collection of two or more LANs, typically connected by routers and
dedicated leased lines. Occasionally, a WAN will be referenced as a metropolitan area
network (MAN) when it is confined to a certain geographic area, such as a university
campus or city. Wireless mesh networks (WMN) are a form of an ad hoc WLAN that often
consist of mesh clients, mesh routers, and gateways.
    \end{itemize}
    \item Which of the following is a collection of two or more LANs, typically connected by routers
and dedicated leased lines?
     \begin{itemize}
        \item Metropolitan area network (MAN) is the term occasionally used for a WAN that is
confined to a certain geographic area, such as a university campus or city. A personal area
network (PAN) is a LAN created by personal devices. A wide area network (WAN) is a
collection of two or more LANs, typically connected by routers and dedicated leased lines.
Wireless mesh networks (WMN) are a form of an ad hoc WLAN that often consist of mesh
clients, mesh routers, and gateways.
    \end{itemize}
    \item Which of the following is a form of ad hoc WLAN?
     \begin{itemize}
        \item Wireless mesh networks (WMN) are a form of an ad hoc WLAN that often consist
of mesh clients, mesh routers, and gateways. A personal area network (PAN) is a LAN
created by personal devices. A wide area network (WAN) is a collection of two or more
LANs, typically connected by routers and dedicated leased lines. Occasionally, a WAN
will be referenced as a metropolitan area network (MAN) when it is confined to a certain
geographic area, such as a university campus or city.
    \end{itemize}
    \item Which of the following is used to attach media connectors to the ends of cables?
     \begin{itemize}
        \item Wire crimpers look like pliers but are used to attach media connectors to the ends
of cables. A cable stripper is used to remove the outer covering of the cable to get to the
wire pairs within. A multimeter combines a number of tools into one. There can be slight
variations, but a multimeter always includes a voltmeter, an ohmmeter, and an ammeter
(and is sometimes called VOM as an acronym). A toner probe has two parts: the tone
generator (called the toner) and the tone locator (called the probe). The toner sends the
tone, and at the other end of the cable, the probe receives the toner’s signal. This tool makes
it easier to find the beginning and end of a cable.
    \end{itemize}
    \item Which of the following includes a voltmeter, an ohmmeter, and an ammeter?
    \begin{itemize}
        \item A multimeter combines a number of tools into one. There can be slight variations, but a
multimeter always includes a voltmeter, an ohmmeter, and an ammeter (and is sometimes
called VOM as an acronym). Wire crimpers look like pliers but are used to attach media
connectors to the ends of cables. A cable stripper is used to remove the outer covering of the
cable to get to the wire pairs within. A toner probe has two parts: the tone generator (called
the toner) and the tone locator (called the probe). The toner sends the tone, and at the other
end of the cable, the probe receives the toner’s signal. This tool makes it easier to find the
beginning and end of a cable.
    \end{itemize}
    \item Which of the following makes it easier to find the beginning and end of a cable?
    \begin{itemize}
        \item A toner probe has two parts: the tone generator (called the toner) and the tone locator
(called the probe). The toner sends the tone, and at the other end of the cable, the probe
receives the toner’s signal. This tool makes it easier to find the beginning and end of a cable.
Wire crimpers look like pliers but are used to attach media connectors to the ends of cables.
A cable stripper is used to remove the outer covering of the cable to get to the wire pairs
within. A multimeter combines a number of tools into one. There can be slight variations,
but a multimeter always includes a voltmeter, an ohmmeter, and an ammeter (and is
sometimes called VOM as an acronym).
    \end{itemize}
\end{enumerate}
\subsection{Hardware} 
\begin{enumerate}
    \item Which cable type comes in two varieties: unshielded and shielded?
    \begin{itemize}
        \item C) Twisted pair is commonly used in office settings to connect workstations to hubs or
switches. It comes in two varieties: unshielded (UTP) and shielded (STP). Fiber optic, serial,
and coaxial do not come in shielded and unshielded versions.
    \end{itemize}
    \item Which cable type transmits data at speeds up to 100 Mbps and was used with Fast Ethernet
(operating at 100 Mbps) with a transmission range of 100 meters?
    \begin{itemize}
        \item B) Cat 5 transmits data at speeds up to 100 Mbps and was used with Fast Ethernet (operating
at 100 Mbps) with a transmission range of 100 meters. It contains four twisted pairs of copper
wire to give the most protection. Although it had its share of popularity (it’s used primarily for
10/100 Ethernet networking), it is now an outdated standard. Newer implementations use the
5e standard. Cat 4 transmits at 16 Mbps, and Cat 6 transmits at 1 Gbps.
    \end{itemize}
    \item Which cable type has a glass core within a rubber outer coating?
    \begin{itemize}
        \item A) Fiber-optic cabling is the most expensive type of those discussed for this exam.
Although it’s an excellent medium, it’s often not used because of the cost of implementing
it. It has a glass core within a rubber outer coating and uses beams of light rather than
electrical signals to relay data. None of the other options uses glass in its construction.
    \end{itemize}
    \item Which connector is used for telephone cord?
    \begin{itemize}
        \item A) An RJ-11 is a standard connector for a telephone line and is used to connect a computer
modem to a phone line. It looks much like an RJ-45 but is noticeably smaller. The RJ-45 is
used for networking. RS 232 is a serial connector. BNC is a coaxial connector.
    \end{itemize}
    \item Which standard has been commonly used in computer serial ports?
    \begin{itemize}
        \item C) The RS-232 standard had been commonly used in computer serial ports. A serial
cable (and port) uses only one wire to carry data in each direction; all the rest are wires
for signaling and traffic control. An RJ-11 is a standard connector for a telephone line and
is used to connect a computer modem to a phone line. It looks much like an RJ-45 but is
noticeably smaller. The RJ-45 is used for networking.
    \end{itemize}
    \item Which connectors are sometimes used in the place of RCA connectors for video electronics?
    \begin{itemize}
        \item D) Bayonet Neill–Concelman (BNC) connectors are sometimes used in the place of RCA
connectors for video electronics, so you may encounter these connectors, especially when
video equipment connects to a PC. In many cases, you may be required to purchase an
adapter to convert this to another form of connection because it is rare to find one on the
PC. An RJ-11 is a standard connector for a telephone line and is used to connect a computer
modem to a phone line. It looks much like an RJ-45 but is noticeably smaller. The RJ45 is
used for networking. RS-232 is a serial connector.
    \end{itemize}
    \item Which RAM type is used in laptops?
    \begin{itemize}
        \item B)Portable computers (notebooks and subnotebooks) require smaller sticks of RAM
because of their smaller size. One of the two types is small outline DIMM (SODIMM),
which can have 72, 144, or 200 pins. DIMM is a full-size RAM type. Rambus is a type of
RAM but not used in laptops, and BNC is a connector for coaxial cabling.
    \end{itemize}
    \item Which RAM type allows for two memory accesses for each rising and falling clock?
    \begin{itemize}
        \item D) DDR SDRAM is Double Data Rate 2 (DDR2). This allows for two memory accesses
for each rising and falling clock and effectively doubles the speed of DDR. DDR2-667
chips work with speeds at 667 MHz and are also referred to as PC2-5300 modules. DDR3
is the higher-speed successor to DDR and DDR2. Portable computers (notebooks and
subnotebooks) require smaller sticks of RAM because of their smaller size. One of the two
types is small outline DIMM (SODIMM), which can have 72, 144, or 200 pins.
    \end{itemize}
    \item Which RAM type is not compatible with any earlier type of random-access memory?
    \begin{itemize}
        \item B) DDR4 SDRAM is an abbreviation for double data rate fourth-generation synchronous dynamic random-access memory. DDR4 is not compatible with any earlier type of random-access memory (RAM). The DDR4 standard allows for DIMMs of up to 64 GB in capacity, compared to DDR3’s maximum of 16 GB per DIMM. DDR3 and DDR2 are backward compatible, and there is no DDR5
    \end{itemize}
    \item Which of the following is a rewritable optical disc?
    \begin{itemize}
        \item B) Compact Disc-ReWritable (CD-RW) media is a rewritable optical disc. A CD-RW drive requires more sensitive laser optics. It can write data to the disc but also has the ability to erase that data and write more data to the disc. CD, DVD, and CD-ROM are all read-only.
    \end{itemize}
    \item Which of the following is a specification for internally mounted computer expansion cards
and associated connectors that replaces the mSATA?
    \begin{itemize}
        \item A) M.2, formerly known as the Next Generation Form Factor (NGFF), is a specification for internally mounted computer expansion cards and associated connectors. It replaces the mSATA standard. M.2 modules are rectangular, with an edge connector on one side, and a semicircular mounting hole at the center of the opposite edge. Non-Volatile Memory Host Controller Interface Specification (NVME) is an open logical device interface specification for accessing nonvolatile storage media attached via a PCI Express (PCIe) bus. Serial ATA and serial ATA 2.5 are computer bus interfaces that connects host bus adapters to mass storage devices such as hard disk drives, optical drives, and solid-state drives
    \end{itemize}
    \item At what speed will latency on a magnetic drive decrease to about 3 ms?
    \begin{itemize}
        \item C) At 10,000 rpm, the latency will decrease to about 3 ms. Data transfer rates also generally go up with a higher rotational speed but are influenced by the density of the disk (the number of tracks and sectors present in a given area). Latency at 5400 rpm will be  5.56 ms. At 7200 it will be 4.17, and at 15000 it will drop to 2.
    \end{itemize}
    \item Laptops and other portable devices utilize which expansion card?
    \begin{itemize}
        \item A) Laptops and other portable devices utilize an expansion card called the miniPCI. It has the same functionality as the PCI but has a much smaller form factor. PCI and PCIe are used in desktops. SATA is a drive connector.
    \end{itemize}
    \item Which of the following is a standard firmware interface for PCs, designed to replace BIOS?
    \begin{itemize}
        \item A) Unified Extensible Firmware Interface (UEFI) is a standard firmware interface for PCs, designed to replace BIOS. NVRAM is RAM that retains its data during a reboot. CMOS is a battery type found on motherboards, and CHS is a drive geometry concept
    \end{itemize}
    \item Which of the following is memory that does not lose its content when power is lost to the
machine?
    \begin{itemize}
        \item  B) NVRAM is memory that does not lose its content when power is lost to the machine. Unified Extensible Firmware Interface (UEFI) is a standard firmware interface for PCs, designed to replace BIOS. CMOS is a battery type found on motherboards, and CHS is a drive geometry concept.
    \end{itemize}
    \item Which of the following devices allows you to plug multiple PCs (usually servers) into the
device and to switch easily back and forth from system to system using the same mouse,
monitor, and keyboard?
    \begin{itemize}
        \item A) A keyboard, video, and mouse (KVM) device allows you to plug multiple PCs (usually servers) into the device and to switch easily back and forth from system to system using the same mouse, monitor, and keyboard. The KVM is actually a switch that all the systems plug into. There is usually no software to install. Just turn off all the systems, plug them all into the switch and turn them back on; then you can switch from one to another using the same keyboard, monitor, and mouse device connected to the KVM switch. CMOS is a battery type found on motherboards, and CHS is a drive geometry concept. NVRAM is memory that does not lose its content when power is lost to the machine.
    \end{itemize}
    \item Which of the following is a description of light output?
    \begin{itemize}
        \item B) When discussing bulbs for projectors, brightness is a description of light output, which is measured in lumens (not watts). Ensure that you are purchasing the correct bulb for the projector and maximize the life of the bulb by turning the projector off when not in use. A keyboard, video, and mouse (KVM) device allows you to plug multiple PCs (usually servers) into the device and to switch easily back and forth from system to system using the same mouse, monitor, and keyboard. Contrast is the relationship between dark and light. CHS is a drive geometry concept.
    \end{itemize}
    \item Which of the following is a standard managed by the ISO and uses tags that are embedded
in the devices?
    \begin{itemize}
        \item C) NFC components include an initiator and a target; the initiator actively generates an RF field that can power a passive target. This enables NFC targets to take simple form factors such as tags, stickers, key fobs, or cards that do not require batteries. When discussing bulbs for projectors, brightness is a description of light output, which is measured in lumens (not watts). Ensure that you are purchasing the correct bulb for the projector and maximize the life of the bulb by turning the projector off when not in use. A keyboard, video, and mouse (KVM) device allows you to plug multiple PCs (usually servers) into the device and to switch easily back and forth from system to system using the same mouse, monitor, and keyboard. CHS is a drive geometry concept
    \end{itemize}
    \item In 2004, the ATX 12V 2.0 (now 2.03) standard was passed, changing the main connector
from 20 pins to how many?
    \begin{itemize}
        \item C) In 2004, the ATX 12V 2.0 (now 2.03) standard was passed, changing the main connector from 20 pins to 24. The additional pins provide +3.3V, +5V, and +12V (the fourth pin is a ground) for use by PCIe cards. When a 24-pin connector is used, there is no need for the optional four- or six-pin auxiliary power connectors.
    \end{itemize}
    \item When using the AT power connector, the power cable coming from the power supply will
have two separate connectors, labeled what?
    \begin{itemize}
        \item When using the AT power connector, the power cable coming from the power supply will have two separate connectors, labeled P8 and P9. When you are attaching the two parts to the motherboard, the black wires on one should be next to the black wires on the other for proper function
    \end{itemize}
      \item The SATA power connector has how many pins?
    \begin{itemize}
        \item C)The SATA power connector consists of 15 pins, with 3 pins designated for 3.3V, 5V, and 12V and with each pin carrying 1.5 amps. This results in a total draw of 4.95 watts + 7.5 watts + 18 watts, or about 30 watts
    \end{itemize}
      \item Which of the following is a desktop computer system?
    \begin{itemize}
        \item C) A thick client has the applications installed locally and will need to have sufficient resources to support the applications. A thin client only sends commands and displays output with the application on the server. Network attached storage is a storage network that is IP based, while Storage Area Networks use a storage area protocol
    \end{itemize}
     \item Which of the following is a PC that has all the capabilities of a standard PC?
    \begin{itemize}
        \item C) When discussing thin and thick clients, you should understand that a thick client is a PC
that has all the capabilities of a standard PC. It runs all applications locally from its own hard
drive. A thin client is one that has minimal capabilities and runs the applications (and perhaps
even the operating system itself) from a remote server. There is no standard client or thin host.
    \end{itemize}
     \item The amount of RAM that is required in a virtualization workstation depends on which of
the following?
    \begin{itemize}
        \item C) Number of VMs
    \end{itemize}
     \item Which IP setting is optional for network connectivity on a thin client?
    \begin{itemize}
        \item D) proxy server address is optional.
    \end{itemize}
     \item Which of the following needs the most resources?
    \begin{itemize}
        \item B) A thick client is a standard PC. When discussing thin and thick clients, you should understand that a thick client is a PC that has all the capabilities of a standard PC. It runs all applications locally from its own hard drive. A thin client is one that has minimal capabilities and runs the applications (and perhaps even the operating system itself) from a remote server. There is no medium client or stationary client
    \end{itemize}
     \item How is accountability ensured?
    \begin{itemize}
        \item C) Ensure accountability by using no shared accounts. Each user should have a unique username/password combination. Audit trails should always be created.
    \end{itemize}
     \item What software controls how the printer processes the print job?
    \begin{itemize}
        \item A) When you install a printer driver for the printer you are using, it allows the computer to print to that printer correctly (assuming you have the correct interface configured between the computer and printer). Also, keep in mind that drivers are specific to the operating system, so you need to select the one that is both for the correct printer and for the correct operating system.
    \end{itemize}
     \item What printer component turns the printed sheet over so it can be run back through the
printer and allow printing on both sides?
    \begin{itemize}
        \item B) An optional component that can be added to printers (usually laser but also inkjet) is a duplexer. This can be an optional assembly added to the printer, or built into it, but the sole purpose of duplexing is to turn the printed sheet over so it can be run back through the printer and allow printing on both sides
    \end{itemize}
     \item Which of the following refers to how the printed matter is laid out on the page?
    \begin{itemize}
        \item C) .The orientation of a document refers to how the printed matter is laid out on the page. In the landscape orientation, the printing is written across the paper turned on its long side, while in portrait the paper is turned up vertically and printed top to bottom. The driver is the software that talks between the printer and the operating system. Duplexing makes it possible to print on both sides. To collate is to create multiple copies with all sets in correct page order
    \end{itemize}
     \item Which of the following feeds through the printer using a system of sprockets and tractors?
    \begin{itemize}
        \item A)Continuous-feed paper feeds through the printer using a system of sprockets and tractors. Sheet-fed printers accept plain paper in a paper tray. Dot matrix is continuous feed; everything else is sheet fed
    \end{itemize}
     \item Which of the following should not be used more than once?
    \begin{itemize}
        \item B) Never reuse paper in a laser printer that has been through the printer once. Although it may look blank, you’re repeating the charging and fusing process on a piece of paper that most likely has something already on it.
    \end{itemize}
     \item Which of the following is a large circuit board that acts as the motherboard for the printer?
    \begin{itemize}
        \item A) This is a large circuit board that acts as the motherboard for the printer. It contains the processor and RAM to convert data coming in from the computer into a picture of a page to be printed. The imaging drum is the drum where the toner is placed on the correctly charged area. The toner cartridge is the container holding the toner. The maintenance kit contains items that should be changed periodically like rollers
    \end{itemize}
\end{enumerate}
\subsection{Virtualization and Cloud Computing}
\begin{enumerate}
    \item Which of the following involves the vendor providing the entire solution?
    \begin{itemize}
        \item Software as a service (SaaS) involves the vendor providing the entire solution. This
includes the operating system, the infrastructure software, and the application. Infrastructure
as a service ( IaaS) provides only the hardware platform to the customer. Platform as a service
( PaaS) provides a development environment. Security Information and Event Management
(SIEM) is a system that aggregates all log files and analyzes them in real time for attacks.
    \end{itemize}
    \item When a company pays another company to host and manage a cloud environment, it is
called what?
    \begin{itemize}
        \item B) When a company pays another company to host and manage a cloud environment, it is
called a public cloud solution. If the company hosts this environment itself, it is a private
cloud solution. A hybrid cloud solution is one in which both public and private clouds are
part of the solution. A community cloud is one in which multiple entities use the cloud.
    \end{itemize}
    \item Which of the following is the ability to add resources as needed on the fly and release those resources when they are no longer required?
    \begin{itemize}
        \item B) One of the advantages of a cloud environment is the ability to add resources as needed
on the fly and release those resources when they are no longer required. This makes for
more efficient use of resources, placing them where needed at any particular point in
time. These include CPU and memory resources. This is called rapid elasticity because it
occurs automatically according to the rules for resource sharing that have been deployed.
On-demand refers to the ability of the customer to add resources as needed. Virtual sharing
and stretched resources are not terms used when discussing the cloud.
    \end{itemize}
    \item In which VDI model are all desktop instances stored in a single server, requiring significant
processing power on the server?
    \begin{itemize}
        \item A) There are three models for implementing VDI:
Centralized model: All desktop instances are stored in a single server, requiring significant
processing power on the server.
Hosted model: Desktops are maintained by a service provider. This model eliminates capital
cost and is instead subject to operation cost.
Remote virtual desktops model: An image is copied to the local machine, making a constant
network connection unnecessary.
There is no local model.
    \end{itemize}
    \item Which of the following involves the vendor providing the hardware platform or data center
and the software running on the platform?
    \begin{itemize}
        \item A) PaaS (Platform as a Service) 
    \end{itemize}
    \item What is the benefit derived from using hardware-assisted virtualization?
    \begin{itemize}
        \item A) Better performance 
    \end{itemize}
    \item Which of the following is the software that allows the VMs to exist?
    \begin{itemize}
        \item B) Hypervisor
    \end{itemize}
    \item Which hypervisor type runs directly on the host’s hardware?
    \begin{itemize}
        \item B) Type I 
    \end{itemize}
    \item Which of the following is an example of a Type II hypervisor?
    \begin{itemize}
        \item A) Oracle VirtualBox
    \end{itemize}
    \item Which of the following hypervisors runs within a conventional operating system?
    \begin{itemize}
        \item Type I 
    \end{itemize}
\end{enumerate}
\subsection{Hardware and Network Troubleshooting}
\begin{enumerate}
    \item Which of the following is the final step in the CompTIA troubleshooting method?
    \begin{itemize}
        \item B) Document findings, actions, and outcomes.
    \end{itemize}
    \item Which of the following is the first step in the CompTIA troubleshooting method?
    \begin{itemize}
        \item (D) Identify the problem 
    \end{itemize}
    \item What is the most common reason for an unexpected reboot?
    \begin{itemize}
        \item A. One common reason for shutdowns is overheating. Often when that is the case,
however, the system reboots itself rather than just shutting down.
    \end{itemize}
    \item Which of the following is typically not a cause of system lockups?
    \begin{itemize}
        \item D. A bad NIC driver would cause the NIC not to work but would not cause a system
lockup.
    \end{itemize}
    \item What are proprietary screen crashes called in Windows?
    \begin{itemize}
        \item B. Once a regular occurrence when working with Windows, blue screens (also known as
the blue screen of death) have become much less frequent.
    \end{itemize}
    \item Which operating system uses the Pinwheel of Death as a proprietary screen crash?
    \begin{itemize}
        \item A. While Microsoft users have the BSOD to deal with, Apple users have also come to have
the same negative feelings about the Pinwheel of Death. This is a multicolored pinwheel
mouse pointer.
    \end{itemize}
    \item What are the small dots on the screen that are filled with a color?
    \begin{itemize}
        \item A. Pixels are the small dots on the screen that are filled with a color; as a group they
present the image you see on the screen.
    \end{itemize}
    \item What are visual anomalies that appear on the screen called? 
    \begin{itemize}
        \item B. Artifacts are visual anomalies that appear on the screen. They might be pieces of images
left over from a previous image or a “tear in the image” (it looks like the image is divided
into two parts and the parts don’t line up).
    \end{itemize}
    \item What is the light in the device that powers the LCD screen?
    \begin{itemize}
        \item A. The backlight is the light in the device that powers the LCD screen. It can go bad over
time and need to be replaced, and it can also be held captive by the inverter. The inverter
takes the DC power the laptop is providing and boosts it up to AC to run the backlight. If
the inverter goes bad, you can replace it on most models (it’s cheaper than the backlight).
    \end{itemize}
    \item Which of the following is a user interface feature designed by HTC?
    \begin{itemize}
        \item B. Touch flow, or TouchFLO, is a user interface feature designed by HTC. It is used by
dragging your finger up and down or left and right to access common tasks on the screen. This
movement is akin to scrolling the screen up and down or scrolling the screen left and right.
    \end{itemize}
    \item Which of the following indicates that the fuser is not fusing the toner properly on the
paper?
    \begin{itemize}
        \item B. With laser printers, streaks usually indicate that the fuser is not fusing the toner
properly on the paper. It could also be that the incorrect paper is being used. In laser
printers, you can sometimes tell the printer that you are using a heavier paper. For dot-
matrix, you can adjust the platen for thicker paper.
    \end{itemize}
    \item Which of the following indicates that the toner cartridge is just about empty?
    \begin{itemize}
        \item C. In laser printers, faded output usually indicates that the toner cartridge is just about
empty. You can usually remove it, shake it, and replace it and then get a bit more life out of
it before it is completely empty, but it is a signal that you are near the end.
    \end{itemize}
    \item If you can ping resources by IP address but not by name,
functional.
   \begin{itemize}
       \item B) DNS . You may be able to ping the entire network using IP addresses, but most access is done
by name, not IP address. If you can’t ping resources by name, DNS is not functional,
meaning either the DNS server is down or the local machine is not configured with the
correct IP address of the DNS server.
   \end{itemize}
   \item Which of the following should be set to the IP address of the router interface connecting to
the local network?
    \begin{itemize}
        \item C) If the computer cannot connect to the default gateway, it will be confined to
communicating with devices on the local network. This IP address should be that of the
router interface connecting to the local network.
    \end{itemize}
\end{enumerate}
\section{Mike Meyers} 
\subsection{The Visible Computer}
\begin{enumerate}
    \item Which version of Windows introduced the Metro UI?
    \begin{itemize}
        \item B)Microsoft introduced Metro UI with Windows 8.
    \end{itemize}
    \item Which Windows 8 feature did Microsoft not include in Windows 10?
    \begin{itemize}
        \item D) Microsoft did not include the Charms bar in Windows 10. Bye!
    \end{itemize}
    \item What macOS feature is essentially multiple Desktops?
    \begin{itemize}
        \item D) Spaces is the term Apple uses for multiple Desktops in macOS.
    \end{itemize}
    \item What KDE feature is essentially the Start button?
    \begin{itemize}
        \item Kickoff functions like a Start button for KDE desktops.
    \end{itemize}
    \item The user Mike has downloaded files with his Web browser. Where
will they be stored by default?
    \begin{itemize}
        \item C:$\backslash Users\backslash Mike\backslash$ Downloads
    \end{itemize}
    \item 32-bit programs are installed into which folder by default in a 64-bit
edition of Windows?
    \begin{itemize}
        \item By default, 32-bit applications install into the $C:\backslash Program Files
(x86)$ folder.
    \end{itemize}
    \item Which macOS feature is functionally equivalent to Windows File
Explorer?
    \begin{itemize}
        \item A) Finder is the equivalent of File Explorer.
    \end{itemize}
    \item Which of the following paths would open Administrative Tools in
Windows 8.1?
    \begin{itemize}
        \item B) To open Administrative Tools, right-click the Start button and
select Administrative Tools. Easy!
    \end{itemize}
    \item What feature of macOS is the equivalent of the command-line
interface in Windows
    \begin{itemize}
        \item C) Terminal is the equivalent of the Windows command-line
interface.
    \end{itemize}
    \item What Windows app in Windows 10 combines many utilities into a
unified tool?
    \begin{itemize}
        \item C)The Settings app in Windows 10 offers many utilities in a unified
interface.
    \end{itemize}

\end{enumerate}
\subsection{CPUs}
\begin{enumerate}
     \item What do registers provide for the CPU?
    \begin{itemize}
        \item B) The CPU uses registers for temporary storage of internal
commands and data.
    \end{itemize}
    \item C) What function does the external data bus have in the PC?
    \begin{itemize}
        \item The external data bus provides a channel for the flow of data and
commands between the CPU and RAM.
    \end{itemize}
    \item What is the function of the address bus in the PC?
    \begin{itemize}
        \item A) The address bus enables the CPU to communicate with the
memory controller chip.
    \end{itemize}
    \item Which of the following terms are measures of CPU speed?
    \begin{itemize}
        \item A) Megahertz and gigahertz
    \end{itemize}
    \item Which CPU feature enables the microprocessor to support running
multiple operating systems at the same time?
    \begin{itemize}
        \item D) Virtualization support
    \end{itemize}
    \item Into which socket could you place an Intel Core i5?
    \begin{itemize}
        \item B) Socket LGA 1151
    \end{itemize}
    \item Which feature enables a single-core CPU to function like two CPUs?
    \begin{itemize}
        \item A) Hyper-Threading
    \end{itemize}
    \item What steps do you need to take to install a Core i3 CPU into an FM2+
motherboard?
    \begin{itemize}
        \item D) Take all of the steps you want to take because it’s not going to
work
    \end{itemize}
    \item A client calls to complain that his computer starts up, but crasheswhen Windows starts to load. After a brief set of questions, you findout that his nephew upgraded his RAM for him over the weekend andcouldn’t get the computer to work right afterward. What could be theproblem?
    \begin{itemize}
        \item A)   Most likely, the nephew disconnected the CPU fan to get at the RAM slots and simply forgot to plug it back in. 
    \end{itemize}
    \item Darren has installed a new CPU in a client’s computer, but nothinghappens when he pushes the power button on the case. The LED onthe motherboard is lit up, so he knows the system has power. Whatcould the problem be?
    \begin{itemize}
        \item The best answer here is that he forgot the thermal paste, thoughyou can also make an argument for a disconnected fan.
    \end{itemize}
\end{enumerate}
\subsection{RAM}
\begin{enumerate}
    \item Steve adds a second 8-GB 288-pin DIMM to his PC, which should
bring the total RAM in the system up to 16 GB. The PC has an Intel
Core i7 4-GHz processor and four 288-pin DIMM slots on the
motherboard. When he turns on the PC, however, only 8 GB of RAM
shows up in Windows Settings app. Which of the following is most
likely to be the problem?
      \begin{itemize}
          \item A) Steve failed to seat the RAM properly.
      \end{itemize}
    \item Scott wants to add 8 GB of PC3-12800 DDR3 to an aging but still
useful desktop system. The system has a 200-MHz motherboard and
currently has 4 GB of non-ECC DDR3 RAM in the system. What else
does he need to know before installing?
    \begin{itemize}
        \item D) If the system can handle that much RAM.
    \end{itemize}
    \item What is the primary reason that DDR4 RAM is faster than DDR3
RAM?
    \begin{itemize}
        \item The input/output speed of the DDR4 RAM is faster.
    \end{itemize}
    \item What is the term for the delay in the RAM’s response to a request
from the MCC?
    \begin{itemize}
        \item C) Latency (MCC Memory Controller Chip) 
    \end{itemize}
    \item How does an NMI (Non-Maskable Interrupt) manifest on a Windows system?
    \begin{itemize}
        \item A) Blue Screen of Death.
    \end{itemize}
    \item Silas has an AMD-based motherboard with two sticks of DDR3 RAM
installed in two of the three RAM slots, for a total of 8 GB of system
memory. When he runs CPU-Z to test the system, he notices that the
software claims he’s running single-channel memory. What could be
the problem? (Select the best answer.)
    \begin{itemize}
        \item D)He needs to move one of the installed sticks to a different slot to
activate dual-channel memory.
    \end{itemize}
    \item Which of the following Control Panel applets will display the amount
of RAM in your PC?
    \begin{itemize}
        \item A) System 
    \end{itemize}
    \item What is the best way to determine the total capacity and specific type
of RAM your system can handle?
    \begin{itemize}
        \item A) Check the motherboard book.
    \end{itemize}
    \item Gregor installed a third stick of known good RAM into his Core i7
system, bringing the total amount of RAM up to 12 GB. Within a fewdays, though, he started having random lockups and reboots,
especially when doing memory-intensive tasks such as gaming. What
is most likely the problem?
    \begin{itemize}
        \item Gregor installed RAM that didn’t match the speed or quality of
the RAM in the system.
    \end{itemize}
    \item Cindy installs a second stick of DDR4 RAM into her Core i5 system,
bringing the total system memory up to 16 GB. Within a short period
of time, though, she begins experiencing Blue Screens of Death. What
could the problem be?
    \begin{itemize}
        \item She installed faulty RAM.
    \end{itemize}
\end{enumerate}
\subsection{Firmware} 
\begin{enumerate}
    \item What does BIOS provide for the computer? (Choose the best answer.)
    \begin{itemize}
        \item B) BIOS provides the programming that enables the CPU to
communicate with other hardware. 
    \end{itemize}
    \item What is the correct boot sequence for an older BIOS-based PC?
A. CPU,
    \begin{itemize}
        \item D) Power good, CPU, POST, boot loader, operating system 
    \end{itemize}
    \item Jill decided to add a second hard drive to her computer. She thinks
she has it physically installed correctly, but it doesn’t show up in
Windows. Which of the following options will most likely lead Jill
where she needs to go to resolve the issue?
    \begin{itemize}
        \item B) Jill should reboot the computer and watch for instructions to enter
the CMOS setup utility (for example, a message may say to press the
DELETE key). She should do what it says to go into CMOS setup
    \end{itemize}
    \item Henry bought a new card for capturing television on his computer.
When he finished going through the packaging, though, he found no
driver disc, only an application disc for setting up the TV capture
software. After installing the card and software, it all works
flawlessly. What’s the most likely explanation?
    \begin{itemize}
        \item B) The device has an option ROM that loads BIOS, so there’s no
need for a driver disc.
need for a driver disc. 
    \end{itemize}
    \item Which of the following most accurately describes the relationship
between BIOS and hardware?
    \begin{itemize}
        \item A) All hardware needs BIOS. 
    \end{itemize}
    \item After a sudden power outage, Samson’s PC rebooted, but nothing
appeared on the screen. The PC just beeps at him, over and over and
over. What’s most likely the problem?
    \begin{itemize}
        \item A) The power outage toasted his RAM. 
    \end{itemize}
    \item Davos finds that a disgruntled former employee decided to sabotage
her computer when she left by putting a password in CMOS that stops
the computer from booting. What can Davos do to solve this
problem?
    \begin{itemize}
        \item C) Davos should find the CLRTC jumper on the motherboard. Then
he can boot the computer with a shunt on the jumper to clear the
CMOS information. 
    \end{itemize}
    \item Richard over in the sales department went wild in CMOS and made a
bunch of changes that he thought would optimize his PC. Now most
of his PC doesn’t work. The computer powers up, but he can only get
to CMOS, not into Windows. Which of the following tech call
answers would most likely get him up and running again?
    \begin{itemize}
        \item D) Please don’t hand Richard a screwdriver! Having him load
Optimized Default settings will most likely do the trick.
    \end{itemize}
    \item Jill boots an older Pentium system that has been the cause of several
user complaints at the office. The system powers up and starts to run
through POST, but then stops. The screen displays a “CMOS
configuration mismatch” error. Of the following list, what is the most
likely cause of this error?
    \begin{itemize}
        \item A) CMOS battery is likely dying.
    \end{itemize}
    
    \item Where does Windows store device drivers?
    \begin{itemize}
        \item C) Registry
    \end{itemize}
\end{enumerate}
\subsection{Motherboards} 
\begin{enumerate}
    \item Which of the following statements about the expansion bus is true?
    \begin{itemize}
        \item B)The expansion bus crystal sets the speed for the expansion bus.
    \end{itemize}
    \item What does a black down arrow next to a device in Device Manager
indicate?
    \begin{itemize}
        \item D)The device has been disabled.
    \end{itemize}
    \item Which variation of the PCI bus was specifically designed for laptops?
    \begin{itemize}
        \item C) Mini-PCI 
    \end{itemize}
    \item Which of the following form factors dominates the PC industry?
    \begin{itemize}
        \item B) ATX 
    \end{itemize}
    \item Amanda bought a new system that, right in the middle of an important
presentation, gave her a Blue Screen of Death. Now her system won’t
boot at all, not even to CMOS. After extensive troubleshooting, she
determined that the motherboard was at fault and replaced it. Now the
system runs fine. What was the most likely cause of the problem?
    \begin{itemize}
        \item A) Although all of the answers are plausible, the best answer here is
that her system suffered burn-in failure.
    \end{itemize}
    \item Martin bought a new motherboard to replace his older ATX
motherboard. As he left the shop, the tech on duty called after him,
“Check your standoffs!” What could the tech have meant?
    \begin{itemize}
        \item C. Standoffs are the metal connectors that attach the motherboard to
the case.
    \end{itemize}
    \item Solon has a very buggy computer that keeps locking up at odd
moments and rebooting spontaneously. He suspects the motherboard.
How should he test it?
    \begin{itemize}
        \item Check settings, verify good components, replace components,
and document all testing.
    \end{itemize}
    \item When Jane proudly displayed her new motherboard, the senior tech
scratched his beard and asked, “What kind of northbridge does it
have?” What could he possibly be asking about?
    \begin{itemize}
        \item C. The tech is using older terminology to refer to the chips—the
chipset—that help the CPU communicate with devices.
    \end{itemize}
    \item What companies dominate the chipset market? (Select two.)
    \begin{itemize}
        \item A, C) AMD, NVIDIA 
    \end{itemize}
    \item If Windows recognizes a device, where will it appear?
    \begin{itemize}
        \item Device Manager
    \end{itemize}
    
\end{enumerate}
\subsection{Motherboards} 
\begin{enumerate}
    \item 
\end{enumerate}
\section{CompTIA A+ Core 1 Exam Cram David L. Prowse} 
\subsection{Laptop Part I}
\begin{enumerate}
    \item What kinds of hard drives are used by laptops? (Select all that
apply.) 
     \begin{itemize}
         \item A, B and C. Solid state drives (SSDs), M.2 drives, and mag‐
netic disk drives are all found on laptops. Which drive the lap‐
top uses will depend on its age and whether or not it has been
upgraded. DVD-ROM drives are not hard drives, they are opti‐
cal drives, and as of 2015 or so, are not commonly found on
laptops.
     \end{itemize}
     \item What is the module format for a stick of SODIMM DDR4
RAM?
     \begin{itemize}
         \item C. DDR4 SODIMM modules have 260 pins. DDR (known as
DDR1) and DDR2 are 200-pin. DDR3 is 204-pin. 1/8 inch is
the size associated with smaller hard drives used in some lap‐
tops.
     \end{itemize}
     \item You just added a second memory module to a laptop. Howev‐
er, after rebooting the system, the OS reports the same amount
of memory as before. What should you do next?
     \begin{itemize}
         \item D. The next step you should take is to reseat the memory.
SODIMMs can be a bit tricky to install. They must be firmly
installed, but you don’t want to press too hard and damage any
components. If the laptop worked fine before the upgrade, you
shouldn’t have to replace the modules or the motherboard.
Windows Update will not find additional RAM.
     \end{itemize}
     \item Which of the following are ways that a laptop can communi‐
cate with other computers? (Select all that apply.)
     \begin{itemize}
         \item A, B, and D. Some of the methods that laptops use to commu‐
nicate with other computers include: Bluetooth, WLAN, and
cellular WAN wireless connections, plus wired connections
like Ethernet (RJ45) and for older laptops, dial-up (RJ11). The
DC jack is the input on the laptop that accepts power from the
AC adapter.
     \end{itemize}
     \item Which of the following are possible reasons that a laptop’s
keyboard might fail completely? (Select thetwo best answers.)
     \begin{itemize}
         \item C) The user spilled coffee on the laptop. D) The keyboard was disabled in the Device Manager.B and C. A laptop’s keyboard could fail due to a disconnected
or loose keyboard ribbon cable. It could also fail if a user
spilled coffee on the laptop, by being dropped on the ground,
and so on. One stuck key will not cause the entire keyboard tofail, and on most laptops, the keyboard cannot be disabled in
the Device Manager. It can be uninstalled, but not disabled.
     \end{itemize}
     \item A user doesn’t see anything on his laptop’s screen. He tries to
use AC power and thinks that the laptop is not receiving any.
Which of the following are two possible reasons for this? (S‐
elect the two best answers.)
     \begin{itemize}
         \item A and B. An incorrect adapter will usually not power a laptop.
The adapter used must be exact. And of course, if the laptop is
not plugged in properly to the adapter, it won’t get power.
Windows doesn’t play into this scenario. And if the battery
was dead, it could cause the laptop to not power up, but only if
the AC adapter was also disconnected; the scenario states that
the user is trying to use AC power.
     \end{itemize}
     \item One of your customers reports that she walked away from her
laptop for 30 minutes. When she returned, the display was very
dim. She increased the brightness setting and moved the mouse
but to no effect. What should you do first?
     \begin{itemize}
         \item D. It could be that the laptop is now on battery power, which is
usually set to a dimmer display and shorter sleep configura‐
tion. This indicates that the laptop is not getting AC power
from the AC outlet for some reason. The battery power setting
is the first thing you should check; afterward, start trou‐
bleshooting the AC adapter, cable, AC outlet, and so on. It’s
too early to try replacing the display; try not to replace some‐
thing until you have ruled out all other possibilities. A dim
screen is not caused by OS corruption. No need to plug in an
external monitor; you know the video adapter is working, it’s
just dim.
     \end{itemize}
     \item Which are the most common laptop hard drive form factors?
(Select two.)
     \begin{itemize}
         \item B and D. The bulk of the hard drives in laptops are 2.5 inches
wide, though ultra-small laptops and other small portable de‐
vices might use a hard drive as small as 1.8 inches. 5.1 and 7.1
refer to speaker surround sound systems, not hard drive form
factors.
     \end{itemize}
     \item You are helping a customer with a laptop issue. The customer
said that two days ago the laptop was accidentally dropped
while it was charging. You observe that the laptop will not turn
on and that it is connected to the correct power adapter. Which
of the following is the most likely cause?
     \begin{itemize}
         \item . D. The DC jack was probably damaged when the laptop was
dropped. That’s because it was plugged in (charging) and itprobably fell on the plug that connects to the DC jack (which
is easily damaged on many laptops by the way). The customer
probably used the laptop until the battery became discharged
before noticing that the laptop wouldn’t take a charge anymore
—that’s why it won’t turn on at all. So the battery is probably
not the issue. A power adapter can be damaged, but the DC-in
jack is more easily damaged. The hard drive and the BIOS nor‐
mally will not affect whether the laptop will turn on.
     \end{itemize}
\end{enumerate}
\subsection{Laptop Part II} 
\begin{enumerate}
    \item Which kind of video technology do most laptop LCDs use?
    \begin{itemize}
        \item A. TFT active-matrix displays are the most common in laptops that use LCDs. Passive-matrixscreens have been discontinued, but you might see an older laptop that utilizes this technology.OLED technology is a newer and different technology that is not based on TFT displays, but instead uses emissive display technology, meaning that each dot on the screen is illuminated by ase parate diode. OLED displays can however bepassive-matrix or active-matrix controlled. The MAC ID is the hexadecimal address associated with a network adapter, such as a Wi-Fi adapter or network card.
    \end{itemize}
    \item Which of the following uses an organic compound that emits light?
    \begin{itemize}
        \item C. OLED (organic light emitting diode) display suse an organic compound or film that emits light.TFT active-matrix implies LCD, and neither of them use organic compounds the way OLED does.In-plane switching (IPS) is a type of LCD technology that increases the available viewing angle compared to older technologies such as twisted nematic (TN) matrix LCDs. However, IPS is generally considered inferior to OLED screens when it comes to brightness and contrast ratio when viewed from an angle. LED screens use a film and diodes, but not organically in the way that OLED does, and not at such a small size.
    \end{itemize}
    \item Which of the following are two possible reasons why a laptop’s LCD suddenly went blank with no user intervention? (Select the two best answers.)
    \begin{itemize}
        \item  A and C. A damaged inverter or burned-out bulb could cause a laptop’s display to go blank. You can verify whether the LCD is still getting a signal by shining a flashlight at the screen. A damaged LCD usually works to a certain extent and will either be cracked, have areas of Windows missing, or show other signs of damage. An incorrect resolution setting can indeed make the screen suddenly go blank (or look garbled), but that scenario will most likely occur only if the user has changed the resolution setting—the answer specifies with no user intervention.
    \end{itemize}
    \item Which of the following allows us to access a WLAN?
    \begin{itemize}
        \item E. A Wi-Fi card, also known as a Wi-Fi network adapter, allows us to connect to a WLAN (wireless local area network) which is essentially another name for a Wi-Fi network. LED is a type of display. A webcam is used to communicate visually and audibly with others, or to record oneself. A digitizer is the device that converts tapped or written impulses on a screen into digital information that the operating system can use. A stylus is a writing device used with a digitizer or touchscreen.
    \end{itemize}
    \item Which of the following keys should you press to
enable a secondary display on a laptop? (Select the
two best answers.)
    \begin{itemize}
        \item A and D. To enable a secondary, or external,
display on a laptop you would use the Fn key
(called the function key) and a special function
key, for example F3 or F4, whichever one that
corresponds to screen switching. It’s this
combination of keys that allows you to make use
of displays plugged into HDMI or other ports on
the laptop. The caps lock key enables a user to
type in all uppercase letters. The number lock key (if available) turns on the numeric keypad (if the
laptop has one). The insert key is often used by
programs such as word processors in one of two
modes: overtype; where anything that is typed is
written over any existing text; and insert mode,
where typed characters force the existing text
over.
    \end{itemize}
    \item When a user types, a laptop’s screen displays
letters and numbers instead of only letters. What
should you check first?
    \begin{itemize}
        \item Num Lock key
    \end{itemize}
    \item You are required to install an anti-theft solution
for a customer’s laptop. Which of the following
should you perform?
    \begin{itemize}
        \item B. Install a cable lock to increase the security of a
laptop and decrease the chances of theft. Docking
stations and port replicators offer increased
functionality for a laptop but do not increase
security; the laptop can be easily disconnected
from them. Installing Windows is not an anti-theft
solution, nor any type of security precaution.
Configuring a password in the BIOS/UEFI is a
good security practice, but it will not help avoid
theft. However, if the laptop is stolen, a user
password (and administrator password) that is
configured in the BIOS can help prevent a person
    \end{itemize}
    \item You are helping a project manager with a
presentation using a laptop which feeds video to a
projector. During your tests, the projector’s image
begins to flicker. The laptop’s display does not
have any problems. You attempt to change the
resolution on the laptop, but the issue continues.
Which of the following should you do next?
A.
    \begin{itemize}
        \item B. Check the connectivity of the cable. If it is
flickering, chances are that the cable is loose, or
the cable’s quality is lacking. Screen flicker is
more common with VGA cables, but it can happen
with just about any connection. Remember,
always check the basic stuff first: connectivity,
power, and so on. It is unlikely that the projector
settings will make a difference based on this
particular problem. You cannot change the aspect
ratio by itself on most laptops, however, when you
change the resolution (which was already done in
the question) you might be changing the aspect
ratio as well, depending on the resolution
selected. If the power cable was loose or damaged,
it would probably result in more than just screen
flicker; the projector might power off and power
back on, which would prevent the image from
being displayed for at least several seconds while
the projector powers back up. Great job so far!
Two chapters down!
    \end{itemize}
    \item Which of the following is not a mobile device?
    \begin{itemize}
        \item C. The desktop PC is not a mobile device. It is a stationary
computer that is meant to stay at a person’s desk. Tablets,
smartphones, and e-readers are all examples of mobile devices.
    \end{itemize}
    \item Which type of memory do most mobile devices store long-term
data to?
    \begin{itemize}
        \item D. Most mobile devices store their long-term data to solid-state
flash memory. They do not use SATA as the method of con‐
nectivity. LPDDR4 is a common type of RAM used in mobile
devices for short-term storage.
    \end{itemize}
     \item You have been tasked with connecting a wireless earpiece to a
smartphone. Which technology would you most likely use?
    \begin{itemize}
        \item D. When connecting an earpiece (those little cricket-looking
devices) to a smartphone, you would most likely use Bluetooth
—just remember that most of them have a 30-foot range (10
meters). Wi-Fi is less likely to be used; it is more likely to be
used to connect the smartphone to the LAN and ultimately to
the Internet. NFC is used to transmit data between mobile de‐
vices in close proximity to each other. 3.5 mm refers to the au‐dio port on a mobile device. It is quite possible that a user will
utilize a wired headset, but the question focuses on wireless.
    \end{itemize}
    \item You have been tasked with setting up a device for a salesper‐
son’s vehicle. It should be able to display maps and give direc‐
tions to the person while driving. Which of the following de‐
vices would perform these tasks? (Select the two best answers.)
    \begin{itemize}
        \item A and C. A standalone GPS device or a smartphone (equipped
with a GPS app) would do the job here. Both can display maps
and give directions to a person while driving. The other de‐
vices are not designed to function in this manner.
    \end{itemize}
    \item Which type of charging connector would you find on an iPad?
    \begin{enumerate}
        \item B. The Lightning connector is one of Apple’s proprietary
charging and synchronization connectors used by iPads and
iPhones, although Apple also uses USB-C. Micro-USB is used
by older Android-based mobile devices—while USB-C is
more common on newer devices. Thunderbolt is a high-speed
hardware interface used in desktop computers, which we will
discuss more in Chapter 9, “Cables and Connectors.” IP68
deals with ingress protection from dust and water jets.
    \end{enumerate}
    \item You are required to add long-term storage to a smartphone.
Which type would you most likely add?
    \begin{itemize}
        \item B. You would most likely add a microSD card (if the smart‐
phone has a slot available for add-on or upgrading). This is the
most common method for adding long-term storage. DDR4 is
a type of RAM; it is not used for adding long-term memory
storage. Some smartphones will use LPDDR4 as their main
memory, but this is part of the SoC, and not accessible to the
typical user. An SSD is a solid-state drive which generally
means a hard drive that is installed to a PC or laptop, con‐
nected either as SATA or M.2. These are too large for smart‐phones and tablets. A SIM is a subscriber identity module,
usually represented as a small card (mini-SIM) used in smart‐
phones that securely stores authentication information about
the user and device, such as the international mobile subscriber
identity (IMSI) which we will discuss more in the following
chapter.
    \end{itemize}
    \item The organization you work for allows employees to work from
their own mobile devices in a BYOD manner. You have been
tasked with setting up the devices so that they can “beam” in‐
formation back and forth between each other. What is this
known as?
    \begin{itemize}
        \item E. “Beaming” the information back and forth can be accom‐
plished in a couple of ways, primarily by using near field com‐
munication (NFC). This can only be done if the devices are in
close proximity to each other. NFC is commonly used for con‐
tactless payment systems. Another potential option would be
Apple’s AirDrop, but this relies on Bluetooth (for finding de‐
vices) and Wi-Fi (for transmitting data), and of course relies
on using Apple-based devices. A mobile hotspot enables a
smartphone or tablet to act as an Internet gateway for other
mobile devices and computers. IoT stands for the Internet of
Things. In the question, it said employees can use their mobile
devices in a BYOD manner, but CYOD is a bit different. This
means that employees can choose a device to use for work pur‐
poses (most likely whichever type they are more familiar
with). Whether or not the employees can use those for personal
purposes is usually defined by company policy. IR stands for
infrared, which is less commonly found on smartphones as of
2017.
    \end{itemize}
    \item Which of the following can be useful in areas where a smart‐
phone has cellular access, but the PC (or laptop) cannot con‐
nect to the Internet?
    \begin{itemize}
        \item D. Tethering can allow a desktop computer or laptop to share
the mobile device’s Internet connection. Tethering functional‐
ity can be very useful in areas where a smartphone has cellular
access, but the PC/laptop cannot connect to the Internet. Mo‐
bile device accessories such as headsets, speakers, gamepads,extra battery packs and protective covers are useful, however
they are not used to connect to the Internet. IP codes are used
to classify and rate the degree of protection against dust and
water (for example, IP68). A perfect example of a proprietary
vendor-specific connector is the Apple Lightning connector
that can only be used on iOS devices.
    \end{itemize}
\end{enumerate}
\subsection{Smartphones, Tablets, and Other Mobile Devices, Part 2}
\begin{enumerate}
    \item Which of the following connections require a username, pass‐
word, and SMTP server? (Select the two best answers.)
    \begin{itemize}
        \item C and E. POP3 and IMAP e-mail connections require an in‐
coming mail server (either POP3 or IMAP) and an outgoing
mail server (SMTP). Bluetooth and Wi-Fi connections do not
require a username or SMTP server. Bluetooth might require a
PIN, and Wi-Fi will almost always require a passcode. Ex‐
change connections require a username and password, but no
SMTP server. The Exchange server acts as the incoming and
outgoing mail server.
    \end{itemize}
    \item When manually configuring a Wi-Fi connection, which step
occurs after successfully entering the SSID?
    \begin{itemize}
        \item C. After you enter the SSID (if it’s correct) you would enter
the passcode for the network. POP3 has to do with configuring
an e-mail account. If you have already entered the SSID, then
you should be within range of the wireless access point
(WAP). Scanning for networks is the first thing you do when
setting up a Wi-Fi connection.
    \end{itemize}
    \item Which of the following allows other mobile devices to wire‐
lessly share your mobile device’s Internet connection?
    \begin{itemize}
        \item . D. Mobile hotspot technology (sometimes referred to as Wi-Fi
tethering) allows a mobile device to share its Internet connec‐
tion with other Wi-Fi capable devices. Another possibility
would be USB tethering, but that is done in a wired fashion.
NFC stands for near field communication—a technology that
allows two mobile devices to send information to each other
when they are in close proximity. Airplane mode will disableall wireless connectivity including (but not limited to) cellular,
Wi-Fi, and Bluetooth. IMAP is another e-mail protocol similar
to POP3.
    \end{itemize}
     \item Which of the following identifies the user of the device?
    \begin{itemize}
        \item A. International Mobile Subscriber Identity (IMSI) or IMSI ID
is used to identify the user of the device. IMEI stands for Inter‐
national Mobile Station Equipment Identity and identifies the
phone used. In other words, the IMEI ID identifies the device
itself. S/MIME (Secure/Multipurpose Internet Mail Exten‐
sions) is used for authentication and message integrity and is
built-in to some e-mail clients. In other words, it is used to en‐
crypt email. Virtual private networking (VPN) technology is
used to make secure connections—tunneling though the
provider’s radio network.
    \end{itemize}
     \item Which of the following is the most common connection
method when synchronizing data from a mobile device to a
PC?
    \begin{itemize}
        \item C) USB is the most common connection method used when
synchronizing data from a mobile device to a PC. Though Wi-
Fi and Bluetooth are also possible, they are less common.
Lightning is the port found on some of Apple’s mobile de‐
vices, but still ends in USB when connecting to a PC or Mac.
    \end{itemize}
     \item Which of the following is used to synchronize contacts from an
iPad to a PC?
    \begin{itemize}
        \item C. PC users need iTunes to synchronize contacts and other data
from an iPad to a PC. While Gmail can work to synchronize
contacts, it is all based on web storage; nothing is actually
stored on the iPad. Google Play is a place to get applications
and other items for Android. Sync Center is a Control Panel
utility that enables synchronization across Windows 10 de‐
vices.
    \end{itemize}
     \item What is it known as when a user connects to several services
using several apps but with only one username and password?
    \begin{itemize}
        \item B. SSO (single sign-on) is a type of authentication where a user
logs in once but is granted access to multiple services. Android
Auto is a screen sharing/synchronizing app used on Android-
based mobile devices to communicate with a properly
equipped automobile. iTunes is a music/media program that
can be used to sync up a mobile device to a PC or Apple de‐
vice. BT is short for Bluetooth. Exchange ActiveSync is a
client-based protocol that allows a user to sync a mobile device
with an Exchange Server mailbox.
    \end{itemize}
\end{enumerate}
\subsection{Ports, Protocols, and Network Devices} 
\begin{enumerate}
    \item Which protocol uses port 22?
    \begin{itemize}
        \item  C. SSH (Secure Shell) uses port 22, FTP uses port 21, Telnet
uses port 23, and HTTPS uses port 443.
    \end{itemize}
    \item Which of these would be used for streaming media?
    \begin{itemize}
        \item C. User Datagram Protocol (UDP) is used for streaming media.
It is connectionless, whereas TCP is connection-oriented and
not a good choice for streaming media. RDP is the Remote
Desktop Protocol used to make connections to other comput‐
ers. DHCP is the Dynamic Host Configuration Protocol used
to assign IP addresses to clients automatically.
    \end{itemize}
    \item Which ports are used by the IMAP protocol?
    \begin{itemize}
        \item D. The Internet Message Access Protocol (IMAP) uses port
143 by default and port 993 as a secure default. DNS uses port
53. DHCP uses port 68. HTTP uses port 80. HTTPS uses port
443. POP3 uses port 110 and 995 as a secure default. Know
those ports!
    \end{itemize}
    \item A user can receive e-mail but cannot send any. Which protocol
is not configured properly?
    \begin{itemize}
        \item C. The Simple Mail Transfer Protocol (SMTP) is probably not
configured properly. It deals with sending mail. POP3 receives
mail. FTP sends files to remote computers. SNMP is used to
manage networks.
    \end{itemize}
    \item Which of the following is most often used to connect a group
of computers in a LAN? (Select all that apply.)
    \begin{itemize}
        \item B and D. Computers in a LAN are connected by a central con‐
necting device; the most common of which are the switch and
the wireless access point (WAP). Hubs can also be used, but
those are deprecated devices; they are the predecessor of the
switch. A router is designed to connect two networks together.
Now, you might say, “Wait! My router at home has four ports
on the back for computers to talk to each other.” Well, that is
actually the switch portion of a SOHO “router”. The actual
router functionality is in the connection between the two net‐
works—the switched LAN and the Internet. A bridge is used
to connect two LANs or separate a single LAN into two sec‐
tions.
    \end{itemize}
    \item What device protects a network from unwanted intrusion?
    \begin{itemize}
        \item D. A firewall is a hardware appliance or software application
that protects one or more computers from unwanted intrusion.
A switch is a device that connects multiple computers together
on a LAN. A router is used to connect two or more networks.
An access point (or wireless access point) allows Wi-Fi-en‐
abled computers and devices to communicate on the LAN
wirelessly.
    \end{itemize}
    \item Which of the following network devices moves frames of data
between a source and destination based on their MAC address‐
es?
    \begin{itemize}
        \item B) A switch sends frames of data between computers by identi‐
fying the systems by their MAC addresses. A hub broadcasts
data out to all computers. The computer that it is meant for ac‐cepts the data; the rest drop the information. Routers enable
connections with individual high-speed interconnection points
and route signals for all the computers on the LAN out to the
Internet. A modem is a device that allows a computer to access
the Internet by changing the digital signals of the computer to
analog signals used by a typical land-based phone line.
    \end{itemize}
    \item Which of the following network devices allows a remote de‐
vice to obtain Ethernet data as well as electrical power?
    \begin{itemize}
        \item B) A PoE injector sends Ethernet data and power over a single
twisted-pair cable to a remote device. PD stands for “powered
device,” the PoE-compliant remote device that is receiving the
power. A repeater extends the distance of a network connec‐
tion. While a PoE injector can act as a repeater, not all re‐
peaters are PoE injectors. A router makes connections from
one network to another or from the LAN to the Internet.
    \end{itemize}
    \item Which of the following devices can be configured when ac‐
cessed from a browser or SSH or similar configuration tool?
    \begin{itemize}
        \item A)Managed switches can be configured when accessed from a
browser or SSH or similar configuration tool. For example,
you can: change the device’s IP address; configure ports; and
monitor the switch and other devices with SNMP. On the other
hand, unmanaged switches don’t have these capabilities, they
simply connect devices and computers together for transmis‐
sion of data over the Ethernet network. A patch panel is a
physical hardware device that acts as a termination point for all
of the network cables in a building. A network interface card
(NIC) allows for connectivity to a computer network. It is a
physical device that can be added to a computer or networking
device that has an open and compatible slot.
    \end{itemize}

\end{enumerate}
\subsection{SOHO Networks and Wireless Protocols} 
\begin{enumerate}
     \item Which of the following allows for network throttling of indi‐
vidual computers or applications?
    \begin{itemize}
        \item A. Quality of Service (QoS) is a technology used in SOHO
routers that can throttle bandwidth, and give higher priority to
individual computers or applications. Port forwarding is used
to forward outside network ports to internal IP addresses. A
DMZ is a protected area between the LAN and the Internet—
often inhabited by company servers. The Dynamic Host Con‐
figuration Protocol (DHCP) is the protocol in charge of auto‐
matically handing out IP address information to clients.
    \end{itemize}
    \item Which of the following forwards an external network port to an
internal IP address/port on a computer on the LAN?
    \begin{itemize}
        \item B. Port forwarding is used to forward external network ports to
an internal IP and port. This is done so a person can host ser‐
vices such as FTP internally. Network address translation
(NAT) is used by most routers to convert the internal network
of IPs to the single public IP address used by the router. The
demilitarized zone (DMZ) is an area that is protected by the
firewall but separate from the LAN. Servers are often placed
here. DHCP is the protocol that governs the automatic assign‐
ment of IP addresses to clients by a server.
    \end{itemize}
    \item Which of the following is described as the simultaneous send‐
ing and receiving of network data?
    \begin{itemize}
        \item D. Full-duplex is when a network adapter (or other device) can
send and receive information at the same time. Half-duplex is
when only sending or receiving can be done at one time. La‐
tency is the delay it takes for data to reach a computer from a
remote location. PoE is Power over Ethernet, a technology that
allows devices to receive data and power over an Ethernet net‐
work cable.
    \end{itemize}
    \item Which of the following would a company most likely use for
authentication to a server room?
    \begin{itemize}
        \item C. RFID (radio-frequency identification) is commonly used for
access to areas of a building such as a server room. It is often
implemented as a proximity-based ID card or badge. The oth‐
ers are not usually associated with authentication. 802.11ac is
a WLAN (Wi-Fi) standard that runs on 5 GHz and can provide
1 Gbps of data transfer. 802.15-4 is the IEEE standard for Zig‐
bee. Z-Wave, like Zigbee, is a home automation and wirelesssensor control technology. MIMO (multiple-input and multi‐
ple-output) is a multiple propagation technology used to in‐
crease data transfer in 802.11n and 802.11ac wireless net‐
works.
    \end{itemize}
    \item Which standard can attain a data transfer rate of 1 Gbps over a
wireless connection?
    \begin{itemize}
        \item D. 802.11ac can attain speeds in excess of 1 Gbps over wire‐
less. 802.11a and g have a typical maximum of 54 Mbps.
802.11b (rarely used today) has a maximum of 11 Mbps.
802.3ab is the IEEE specification for 1 Gbps transfer over
twisted pair cables—it is wired, not wireless. By the way, this
is also known as 1000BASE-T
    \end{itemize}
    \item Which of the following is often broken down into groups of
channels including 1-5, 6-10, and 11?
    \begin{itemize}
        \item . B. In the United States, the 2.4 GHz frequency range is broken
down into three categories: Channel 1-5, 6-10, and 11. By
placing separate wireless networks on separate distant chan‐
nels (such as 1 and 11), you can avoid overlapping and inter‐
ference. 802.11ac and 802.11a are standards, not frequencies.
5 GHz uses channels such as 36, 40, 149, 153, and so on.
    \end{itemize}
   
\end{enumerate}
\subsection{Networked Hosts and
Network Configuration}
\begin{enumerate}
    \item While looking at the details of a server in your provider’s con‐
trol panel you notice that it says “Apache” in the HTTP sum‐
mary. What kind of server is this?
    \begin{itemize}
        \item B. Apache is a type of web server that runs on Linux. It is also
known as Apache HTTP Server. File servers are used to store
and transfer files but not websites. E-mail servers deal with the
sending and receiving of electronic mail via POP3, IMAP, and
HTTP. Authentication servers verify the identity of users log‐
ging in and computers on the network.
    \end{itemize}
    \item Which type of server acts as a go-between for clients and web‐
sites?
    \begin{itemize}
        \item A. A proxy server is a caching server used to store commonly
accessed websites by clients. It can be incorporated into a web
server but often it runs as a stand-alone server. A print server
manages network printers and their spooling of print jobs, pri‐
orities, and so on. A Syslog server gathers logging data from
network devices and allows for the easy analysis of those logs
from a client workstation. A DHCP server hands out IP ad‐dresses (and other TCP/IP information) to client computers.
    \end{itemize}
    \item Which type of server runs Microsoft Exchange?
    \begin{itemize}
        \item C. Microsoft Exchange is a type of e-mail server software.
While you could run multiple services on a single server—for
example, you could run the web server and e-mail server on
the same machine—it isn’t recommended. Unless you have a
small office, all servers (such as file servers, authentication
servers, e-mail servers, DHCP servers, and so on) should be
separate entities. SCADA is not a server—it stands for Super‐
visory Control and Data Acquisition System, a type of system
used to control larger organizations’ infrastructures such as
heating/cooling, electricity, and so on.
    \end{itemize}
    \item Which of these addresses needs to be configured to enable a
computer access to the Internet or to other networks?
    \begin{itemize}
        \item B. The gateway address must be configured to enable a com‐
puter access to the Internet through the gateway device. By de‐
fault, the subnet mask defines the IP address’s network and
host portions. The DNS server takes care of name resolution.
The MAC address is the address that is burned into the net‐
work adapter; it is configured at the manufacturer.
    \end{itemize}
    \item Which technology assigns addresses on the 169.254.0.0 net‐
work number?
    \begin{itemize}
        \item C. If you see an address with 169.254 as the first two octets,
then it is Automatic Private IP Addressing (APIPA). This is
also the link-local range for IPv4. The Dynamic Host Configu‐
ration Protocol (DHCP) assigns IP addresses automatically to
clients but by default does not use the 169.254 network num‐
ber. Static IP addresses are configured manually by the user in
the IP Properties window. Class B is a range of IP networks
from 128 through 191.
    \end{itemize}
    \item You want to test the local loopback IPv6 address. Which ad‐
dress would you use?
    \begin{itemize}
        \item B. You would use the ::1 address. That is the local loopback
address for IPv6. 127.0.0.1 is the local loopback for IPv4.
FE80::/10 is the range of unicast auto-configured addresses. ::0
is not valid but looks similar to how multiple zeros can be
truncated with a double colon.
    \end{itemize}
    \item You have been tasked with compartmentalizing the network.Which of the following technologies should you use?
    \begin{itemize}
        \item C) VLAN 
    \end{itemize}
\end{enumerate}
\subsection{Network Types and Networking Tools} 
\begin{enumerate}
    \item Which of the following is a group of Windows desktop com‐
puters located in a small area?
    \begin{itemize}
        \item A local area network (LAN) is a group of computers, such
as a SOHO network located in a small area. A wide area net‐
work (WAN) is a group of one or more LANs spread over a
larger geographic area. A personal area network (PAN) is a
smaller computer network used by smartphones and other
small computing devices. A metropolitan area network (MAN)
is a group of LANs in a smaller geographic area of a city.
    \end{itemize}
    \item Which Internet service makes use of PSTN?
    \begin{itemize}
        \item A) Dial-up Internet connections make use of the public
switched telephone network (PSTN) and POTS phone lines.ISDN) was developed to meet the limitations of PSTN. DSL
provides faster data transmissions over phone lines (or separate
data lines). Cable Internet is a broadband service that offers
higher speeds than DSL; it is provided by cable TV compa‐
nies.
    \end{itemize}
    \item You have been tasked with setting up a small office with the
fastest Internet service possible. There is no fiber optic avail‐
ability in the area because of the rocky, hilly terrain. Which In‐
ternet service will typically offer the best data transfer rates?
    \begin{itemize}
        \item D. The Internet service with the best data transfer rates will
typically be cable Internet. Cable Internet service is generally
“faster” than DSL. WMN stands for wireless mesh network, a
type of network that uses multiple access points—but it is not
an Internet service. FTTP stands for fiber to the premises,
which is not available in the scenario. Fixed wireless (or line-
of-sight wireless Internet service) can be offer very high data
transfer rates, but it requires an unobstructed view of the
provider’s tower. This is probably not an option due to the
rocky, hilly area where the customer’s office resides.
    \end{itemize}
    \item Which tool is used to test a network adapter not connected to
the network?
    \begin{itemize}
        \item C. To test a network adapter without a network connection,
you would use a loopback plug. This simulates a network con‐
nection. It can also be used to test a switch port. Punchdown
tools are used to punch individual wires to a patch panel. Cable
testers such as continuity testers test the entire length of a ter‐
minated cable. A tone generator and probe kit can also test a
cable’s length, but only tests one pair of wires at a time.
    \end{itemize}
    \item Your boss is concerned with overlapping wireless networks
from neighboring companies using 802.11ac. Which tool
should you use to analyze the problem, and which frequency
should you display for analysis?
    \begin{itemize}
        \item A. To analyze the problem, use a Wi-Fi analyzer! Because
your boss is concerned about wireless networks using
802.11ac, you would display the results for 5 GHz networks,
not 2.4 GHz. Cable certifiers are used to check long distance
wired connections, for example from a patch panel to an RJ45
jack. The loopback plug is used to simulate a network connec‐
tion which can help with identifying switch ports, and testing a
PC’s network connection.
    \end{itemize}
    \item What would be required to attach RJ45 plugs to the ends of a
single patch cable?
    \begin{itemize}
        \item D. RJ45 plugs are attached to the cable ends with a tool called
a crimper. A tone generator and probe kit is used to trace hard
to find telecommunication and data communication ca‐
bles/wires. A multimeter can be used to test continuity of a
patch cable. A cable stripper is used to strip a portion of the
plastic jacket off the cable, exposing the individual wires.
    \end{itemize}
    
\end{enumerate}
\subsection{Cables and Connectors}
\begin{enumerate}
    \item Which of the following would be suitable for 1000 Mbps net‐
works? (Select all that apply.)
    \begin{itemize}
        \item C and D. Category 5e and Category 6 are suitable for 1000
Mbps networks (and Cat 6 is also suitable for 10 Gbps net‐
works). Category 3 is suitable for 10 Mbps networks only. It is
outdated and you most likely won’t see it. Category 5 is suit‐
able for 100 Mbps networks. In general, Cat 3 and Cat 5 net‐
works should be upgraded.
    \end{itemize}
    \item Which type of cable would you use if you were concerned
about EMI?
    \begin{itemize}
        \item C. STP (shielded twisted pair) is the only cable listed here that
can reduce electromagnetic interference. However, fiber optic
cable is another good solution, though it will be more expen‐
sive, and more difficult to install. Plenum-rated cable is used
where fire code requires it; it doesn’t burn as fast, releasing
fewer PVC chemicals into the air.
    \end{itemize}
    \item You have been tasked with connecting a newer Android-based
smartphone to an external TV so that you can display the
CEO’s smartphone screen during a meeting. Which of the fol‐
lowing adapters would be the best solution typically.?
    \begin{itemize}
        \item E. Typically, you would use USB-C to HDMI. If it is a newer
Android-based smartphone, then chances are that it will have aUSB-C port. If you are attempting to connect it to a TV, then
HDMI is the most likely port to use. Micro-USB is used with
many mobile devices, but newer devices especially Android-
based devices) have switched to, or are moving toward USB-
C. We wouldn’t want USB-C to DVI because TVs normally
don’t have DVI inputs. USB to Ethernet helps to convert from
a computer or mobile device to the Ethernet network. These
devices can ultimately allow a device or computer with a USB
port to access the Internet. This wired connection might be fa‐
vored over wireless for its speed, quality connection, and low
latency.
    \end{itemize}
    \item Which type of cable can connect a computer to another com‐
puter directly?
    \begin{itemize}
        \item B. A crossover cable is used to connect like devices: computer
to computer or switch to switch. Straight-through cables (the
more common patch cable) do not connect like devices (for
example, they connect from a computer to a switch). 568B is
the typical wiring standard you will see in twisted-pair cables;
568A is the less common standard. A crossover cable uses the
568B wiring standard on one end and 568A on the other end.
(By the way, sometimes you will see these written as T568A
and T568B.) SATA is used to connect hard drives internally to
a desktop or laptop computer.
    \end{itemize}
    \item Which connector is used for cable Internet?
    \begin{itemize}
        \item B. Cable Internet connections use RG-6 coaxial cable (usually)
with an F-connector on the end. LC is a type of fiber optic con‐
nector. BNC is an older connector type used by coaxial net‐
works. RJ45 is the connector used on twisted-pair patch ca‐
bles. DE-9 (or DB-9) is a serial connector used with RS-232
connections.
    \end{itemize}
    \item Which cable type would be suitable for longer distances such
as connecting two cities?
    \begin{itemize}
        \item D. Single-mode fiber is used for longer distance runs, perhapsfrom one city to the next (as far as thousands of kilometers).
Coaxial is common for connections between utility poles and
houses/buildings. Twisted pair is common in LANs. Multi‐
mode cables have a larger core diameter than single-mode ca‐
bles. It is the more commonly used fiber optic cable in server
rooms and when making network backbone connections be‐
tween buildings in a campus.
    \end{itemize}

\end{enumerate}
\subsection{RAM and Storage} 
\begin{enumerate}
     \item Which of the following types of cloud services offers e-mail
through a web browser?
    \begin{itemize}
        \item A. Software as a service (SaaS) is the most commonly recog‐
nized cloud service; it allows users to use applications to ac‐
cess data that is stored on the Internet by a third party. Infras‐
tructure as a service (IaaS) is a service that offers computer
networking, storage, load balancing, routing, and VM hosting.
Platform as a service (PaaS) is used for easy-to-configure op‐
erating systems and on-demand computing. A community
cloud is mix of public and private clouds, but one where multi‐
ple organizations can share the public portion.
    \end{itemize}
    \item Your organization requires more control over its data and in‐
frastructure. Money is apparently not an issue. There are only
two admins and about 30 users that will have access to the data
on the cloud. Which of the following types of clouds is the best
option?
    \begin{itemize}
        \item B. The best option listed is a private cloud. This gives the most
control over data and resources in an environment where there
are limited users (and a healthy budget). These resources could
be entirely internal, or a portion of them could also be pro‐
vided by a third-party. Public cloud technology is used for the
general public to access applications over the Internet. Hybrid
is a mixture of the two, but not necessary in this situation be‐
cause of the budget and the limited number of users. Commu‐
nity cloud is similar to hybrid but is meant for multiple organi‐
zations that share data, which is not necessary in this scenario.
    \end{itemize}
    \item You require the ability to add on to your cloud-based network
whenever necessary, rapidly and efficiently. What is this refer‐
ring to?
    \begin{itemize}
        \item C. Rapid elasticity is the ability to build your cloud-based net‐
work, or extend upon an existing one, quickly and efficiently.
Measured services is when a provider monitors a customer’s
services used so that the customer can be properly billed.
Metered services is when the customer can access as many re‐
sources as needed but only be billed for what was accessed.
On-demand service means that the cloud service is available at
all times. The leaders of a successful organization don’t care
what it takes; they simply want high speed, secure access to
services 24/7.
    \end{itemize}
    
\end{enumerate}
\subsection{RAM and Storage} 
\begin{enumerate}
    \item What is the transfer rate of DDR4-2133?
    \begin{itemize}
        \item A. The transfer rate of DDR4-2133 is 17,066
MB/s. It is also known as PC4-17000. 19,200
MB/s is the speed of DDR4-2400 (PC4-19200).
21,333 MB/s is the speed of DDR4-2666 (PC4-
21333). 25,600 MB/s is the speed of DDR4-3200
(PC4-25600).
    \end{itemize}
    \item How many pins are on a DDR3 memory module?
    \begin{itemize}
        \item C. DDR3 is a 240-pin architecture. 288-pin is
DDR4, 184-pin is the first version of DDR
(DDR1), and you can find 200-pin architectures in
laptops; they are known as SODIMMs. To review,
Table 10.2 shows the pin configurations for PCbased
DDR 1 through 4.
    \end{itemize}
    \item Which of the following allows for a 256-bit wide
bus?
    \begin{itemize}
        \item B. The quad-channel memory architecture can
allow for a 256-bit wide bus (64-bit per channel).
However, this will only be the case if all four
channels have memory installed to them. ECC
stands for error correction code which can detect
and correct errors in RAM. Parity is when the
RAM stores an extra bit used for error detection.
DDR2 is a type of DRAM that for the most part
was used in either single-channel or dual-channel
environments.
    \end{itemize}
    \item How much data can a SATA revision 3.0 drive transfer per
second?
    \begin{itemize}
        \item D. SATA Revision 3.0 drives can transfer 6.0 Gb/s, which after
encoding amounts to 600 MB/s. SATA Revision 3.2 is 16 Gb/s
(1969 MB/s) but requires SATA Express or M.2. 50 MB/s is a
typical write speed for Blu-ray discs and some flash media. 90
MB/s is a typical write speed for an SD card.
    \end{itemize}
    \item Which level of RAID stripes data and parity across three or
more disks?
    \begin{itemize}
        \item C. RAID 5 stripes data and parity across three or more disks.
RAID 0 does not stripe parity; it stripes data only and can use
two disks or more. RAID 1 uses two disks only. Striping is an‐
other name for RAID 0. RAID 10 contains two sets of mir‐
rored disks that are then striped.
    \end{itemize}
    \item Which of the following has the largest potential for storage ca‐
pacity?
    \begin{itemize}
        \item D. Blu-ray, at a typical maximum of 50 GB, has the largest
storage capacity. CDs top out just under 1 GB. DVDs have a
maximum of 17 GB.
    \end{itemize}
    \item A customer complains that an important disc is stuck in the
computer’s DVD-ROM drive. What should you recommend to
the customer?
    \begin{itemize}
        \item C. Tell the customer to use a paper clip to eject the DVD-ROM
tray. Disassembling the drive is not necessary; the customer
shouldn’t be told to do this. If the disc is rewritable, formatting
it would erase the contents, even if you could format in thisscenario. Never tell a customer to dispose of a DVD-ROM
drive; they rarely fail.
    \end{itemize}
    \item Which of the following best describes a specification for ac‐
cessing storage while using PCI Express?
    \begin{itemize}
        \item A. Non-Volatile Memory Express (NVMe) is a specification
for accessing storage while using PCI Express. Essentially, the
M.2 slot on a motherboard taps into the PCI Express bus (x4)
and uses a portion of the total bandwidth associated with that
bus. The platters in a hard disk drive (HDD) rotate at a certain
speed, for example 7,200 RPM, which is common; other typi‐
cal speeds include 5,400 RPM and 10,000 RPM. Hot-swap‐
pable capability is when drives can be removed and inserted
while the system is on. SATA-based hard drives come in two
main widths: 3.5” and 2.5”. The 3.5” drive is used in desktop
computers, network-attached storage and other larger devices.
The 2.5” drive is used in laptops and other smaller devices.
    \end{itemize}

\end{enumerate}
\subsection{Motherboards and Cards} 
\begin{enumerate}
    \item Which motherboard form factor measures 12
inches × 9.6 inches (305 mm × 244mm)?
    \begin{itemize}
        \item C. ATX boards measure 12 inches × 9.6 inches
(305 mm × 244mm). microATX boards are square
and measure 9.6 × 9.6 inches (244 mm × 244
mm). SATA is a type of hard drive technology and
the port used to connect hard drives to the
motherboard. mITX (or Mini-ITX), also square,
measures 6.7 × 6.7 inches (17 cm × 17 cm) . 
    \end{itemize}
    \item Which component supplies power to the CMOS
when the computer is off?
    \begin{itemize}
        \item A. The lithium battery supplies power to the
CMOS when the computer is off. This is because
the CMOS is volatile and would otherwise lose the
stored settings when the computer is turned off
    \end{itemize}
    \item To implement a secure boot process, which device
should be listed first in the Boot Device Priority
screen?
    \begin{itemize}
        \item D. To ensure that other users cannot boot the
computer from removable media, set the first
device in the Boot Device Priority screen to hard
drive.
    \end{itemize}
    \item Which of the following connectors would you use
to power a video card?
    \begin{itemize}
        \item B. A video card is normally powered by a 6-pin or
8-pin PCIe connector. Lesser cards are simply
powered by the PCIe bus. The 24-pin power
connector is the main connector that leads from
the power supply to the motherboard. Molex is
used for fans, older IDE drives, and other
secondary devices. 3.5 mm TRS (or 1/8 inch) is an
audio connection.
    \end{itemize}
    \item Which of the following is a chip that stores
encryption keys?
    \begin{itemize}
        \item D. To perform hard drive encryption, some
motherboards come with a trusted platform
module (TPM), a chip that stores encryption keys
—it can be enabled in the BIOS. Intel
Virtualization Technology (VT) is part of the
firmware that supports the use of virtualization
software such as Hyper-V and VMware. Secure
boot can block rootkits and other malware from
launching boot loaders that have been tampered
with. Firmware (such as a motherboard’s BIOS) is
should be updated or “flashed” periodically to take
advantage of the latest functionality and security
updates.
    \end{itemize}
\end{enumerate}
\subsection{CPUs}
\begin{enumerate}
    \item What does Hyper-Threading do?
    \begin{itemize}
        \item C. Hyper-Threading allows for an operating
system to send two simultaneous threads to be
processed by a single CPU core. The OS views the
CPU core as two virtual processors. Multiple cores
would imply multicore technology, which means
there are two physical processing cores within the
CPU package. HyperTransport is a high-speed
connection used by AMD from the CPU to RAM.
    \end{itemize}
    \item What seals the tiny gaps between the CPU cap and
the heat sink
    \begin{itemize}
        \item D. Thermal compound/thermal paste is used to
seal the small gaps between the CPU and heat
sink. It is sometimes referred to as thermal gel or
jelly (among a variety of other names), but not
grape jelly. (Did I ever tell you about the time I
found grape jelly inside a customer’s computer?
Fun times.) Note: Never use petroleum-based
products (such as 3-in-1 oil or WD-40) inside a
computer; the oils can damage the components
over time. TDP stands for thermal design power.
    \end{itemize}
    \item Which of the following can be defined as the
amount of heat generated by the CPU, which the
cooling system is required to dissipate?
    \begin{itemize}
        \item B. TDP (thermal design power) is the amount of
power required to cool a computer and is linked
directly to the amount of heat a CPU creates.
Some CPUs come with a built-in graphics
processing unit (GPU). This means that with a
compatible motherboard, no separate video card
is necessary. PSU stands for power supply unit.
140 watts is a potential TDP rating but does not
define what TDP is.
    \end{itemize}
    \item When deciding on a CPU for use with a specific
motherboard, what does it need to be compatible
with?
    \begin{itemize}
        \item B. The CPU needs to be compatible with the
socket of the motherboard. The case doesn’t
actually make much of a difference when it comes
to the CPU. (Just make sure it’s large enough!)
There is no wattage range, but you should be
concerned with the voltage range of the CPU. PCI
Express (PCIe) slots don’t actually play into this at
all because there is no direct connectivity between
the two
    \end{itemize}
    \item Which kind of socket incorporates “lands” to
ensure connectivity to a CPU?
    \begin{itemize}
        \item C. LGA (Land Grid Array) is the type of socket
that uses “lands” to connect the socket to the CPU.
PGA sockets have pinholes that make for
connectivity to the CPU’s copper pins. AM4 is a
PGA socket that accepts AMD CPUs such as the
Ryzen 7.
    \end{itemize}
    \item Which of the following enables the user to
increase the base clock within the BIOS, thereby
increasing the clock speed of the CPU?
    \begin{itemize}
        \item A. Overclocking enables the user to increase the
clock speed of the CPU within the BIOS. Level 3
(L3) cache comes in the largest capacities of the
three types of cache and has the most latency;
therefore, it is the slowest. If the CPU can’t find
what it needs in L1, it moves to L2 and finally to
L3. An integrated GPU is a video adapter that is
built into the motherboard. The heat sink helps to
dissipate heat from the CPU and is usually aided
by a fan or liquid cooling system.
    \end{itemize}
   
\end{enumerate}
\subsection{Peripherals and Power}
\begin{enumerate}
    \item What does a KVM do?
    \begin{itemize}
        \item D. A KVM connects multiple computers to a single keyboard,
mouse, and monitor. This way, fewer resources in the way of
peripherals (input/output devices) are necessary to use the
computers.
    \end{itemize}
    \item Which of the following are considered both input and output
devices?
    \begin{itemize}
        \item D) Smart TVs, touchscreens, KVMs, and headsets are consid‐
ered both input and output devices. Keyboards, mice, touch‐pads, smart card readers, motion sensors, and biometric de‐
vices are considered input devices. Printers and speakers are
considered output devices.
    \end{itemize}
    \item Which of the following devices can be used to perform combi‐
nation shortcuts?
    \begin{itemize}
        \item A.A keyboard can be used to perform combination shortcuts.
An example of a shortcut key is Ctrl + P which initiates a print
job within an application. While a KVM will have a keyboard
connected to it, it’s at the keyboard that you perform the short‐
cut operation.
    \end{itemize}
    \item Which of the following incorporate the concept of resolution.
(Select all that apply).
    \begin{itemize}
        \item B, C, and E. The mouse, printer, and LED display all deal
with resolution. A mouse’s sensitivity is rated in DPI; for ex‐
ample, 800 DPI is a low resolution for mice. A printer will
commonly print out documents at the resolution 600 DPI
(more on that in Chapter 15, “Printers and Multifunction De‐
vices”. A monitor will commonly have a resolution of 1920 x
1080 (or greater!)
    \end{itemize}
    \item Which of the following terms describes how the light output
from a video projector is measured?
    \begin{itemize}
        \item C. A video projector’s light output is measured in lumens. In-
plane switching (IPS) technology allows for a wider viewing
angle. Some LCDs use a cold cathode fluorescent lamp
(CCFL) as the lighting source instead of LEDs. OLED stands
for organic light-emitting diode—that’s the lighting material
used in the display.
    \end{itemize}
    \item Which power connector should be used to power
an SATA hard drive?
    \begin{itemize}
        \item D. 15-pin connectors power SATA hard drives and
other SATA devices (such as optical drives). Molex
connectors power fans, older IDE devices, and
other secondary devices. 6-pin power connectors
are used for video cards (as are 8-pin connectors).
24-pin refers to the main power connection for
the motherboard.
    \end{itemize}
    \item Which voltages are supplied by a Molex power
connector?
    \begin{itemize}
        \item A. Molex connectors provide 12 volts and 5 volts.
There are four wires: if color-coded, yellow is 12 V,
red is 5 V, and the two blacks are grounds.
    \end{itemize}
    \item A company salesperson just returned to the
United States after three months in Europe. Now
the salesperson tells you that her PC, which
worked fine in Europe, won’t turn on. What is the
best solution?
    \begin{itemize}
        \item D. Most likely, the voltage selector was set to 230
V so that it could function properly in Europe (for
example, in the UK). It needs to be changed to 115
V so that the power supply can work properly in
the United States. Make sure to do this while the
computer is off and unplugged.
    \end{itemize}
    \item Which of the following can have 8 pins? (Select all
that apply).
    \begin{itemize}
        \item A and D. PCIe power can be 8-pin or 6-pin. CPU
power (EATX12V) can be 8-pin or 4-pin. SATA
power is 15-pin (and data is 7-pin). ATX main
power is typically 24-pin. Molex is a 4-wire
connector; it is sometimes also referred to as
“peripheral”.
    \end{itemize}
\end{enumerate}
\subsection{Custom PCs and
Common Devices}
\begin{enumerate}
    \item Which of the following is the best type of custom
computer for use with Pro Tools?
    \begin{itemize}
        \item B. Audio/video editing workstation
    \end{itemize}
    \item Which of the following would include a gigabit
NIC and a RAID array?
    \begin{itemize}
        \item D. NAS
    \end{itemize}
    \item Your organization needs to run Windows in a
virtual environment. The OS is expected to require
a huge amount of resources for a powerful
application it will run. What should you install
Windows to?
    \begin{itemize}
        \item C. Type 1 hypervisor
    \end{itemize}
    \item You have set up a user to work at a thin client. They will be
accessing the OS image and data from a Windows Server 2016
as well as data from the Google Cloud. Which of the following
does this configuration not require? (Select the two best an‐
swers.)
    \begin{itemize}
        \item . C and E. This configuration will not need a hard drive, be it
M.2 or other SSD. Thin clients are meant to use an OS that is
embedded in RAM (or other similar memory) or more often,
grab an image from a server, often as a virtual machine. To do
so, the thin client will need a network connection (wired or
wireless), and every computer needs a CPU.
    \end{itemize}
    \item A customer has a brand new Chromebook and needs help con‐
figuring it. Which of the following should you help the user
with? (Select the three best answers.)
    \begin{itemize}
        \item A, B and D. You should show the user how to configure the
touchpad and touchscreen, and guide the user through the ini‐
tial account setup. Chrome OS is a fairly simple system com‐
pared to Windows and other operating systems. To configure
devices, simply go to the “Home” or app launcher button, then
Settings, then Devices. The registry is a Windows configura‐
tion tool—even if this was a Windows computer, the typical
user has no place in the registry. The App Store is Apple’s ap‐
plication download site. Google uses the Play Store. The Task
Manager is another Windows utility. Consider writing a short
user guide in Word document format if you have multiple
users accessing the same type of system for the first time.
Write it once, and train many!
    \end{itemize}
   \item Where would you go in Windows to enable printer sharing?
    \begin{itemize}
        \item B. The Network and Sharing Center in Windows is where
printer sharing is enabled. Network Connections is the window
that shows the Ethernet and Wi-Fi connections a PC has to the
network. Windows sharing has to be done in Windows, it can’tbe done from the printer’s on-screen display (OSD). Bonjour is
a macOS service, that can also be run on Windows which en‐
ables automatic discovery of devices on the LAN.
    \end{itemize}
    \item Your printer supports printing to both sides of paper. What
should you enable in the Printing preferences?
    \begin{itemize}
        \item C. Duplexing (as it relates to printers) means to print to both
sides. Collating means printing multiple copies of a docu‐
ment’s pages in sequence, instead of printing all of the copies
of one page at a time. Orientation is how the print job is dis‐
played on the paper; it could be portrait (vertically—the de‐
fault), or landscape (horizontal). Quality refers to the clarity of
the print job, usually measured in dots per inch (DPI)—the
higher the DPI the better.
    \end{itemize}
    \item Which of the following address printer data privacy concerns?
(Select the two best answers.)
    \begin{itemize}
        \item A and D. Implement user authentication for the printer or print
server (PIN or password), and clear the cache on the printer.
Bluetooth ad hoc mode network printing can be used by mo‐
bile devices where no wireless access point exists. AirPrint is
an Apple technology for macOS and iOS used to automatically
locate and download drivers for printers.
    \end{itemize}
    \item During which step of the laser printing/imaging process is the
transfer corona wire involved?
    \begin{itemize}
        \item B. The transfer corona wire gets involved in the laser print‐
ing imaging process during the transferring step.
    \end{itemize}
    \item Which stage of the laser printing/imaging process involves ex‐
treme heat?
    \begin{itemize}
        \item A. The fusing step uses heat (up to 400 degrees Fahrenheit/200
degrees Celsius) and pressure to fuse the toner permanently to
the paper.
    \end{itemize}
    \item Which represents the proper order of the laser printing/imaging
process?
    \begin{itemize}
        \item D. The proper order of the laser printing/imaging process is
processing, charging, exposing, developing, transferring, fus‐
ing, cleaning.
    \end{itemize}
    \item Which of the following are associated with inkjet printers?
    \begin{itemize}
        \item .B. Inkjet printer components include ink cartridge, print head,
roller, feeder, duplexing assembly, carriage, and belt. Imaging
drum, fuser assembly, transfer belt, transfer roller, pickup
rollers, separate pads, and duplexing assembly are associated
with laser printers. Feed assembly, thermal heating unit, and
thermal paper are associated with thermal printers. Print head,
ribbon, tractor feed, and impact paper are associated with im‐
pact printers.
    \end{itemize}
    \item When finished installing a new printer and print drivers, what
should you do? (Select all that apply.)
    \begin{itemize}
        \item A and D. After the printer is installed (meaning it has been
connected and the drivers have been installed), you should cal‐
ibrate the printer (if necessary) and print a test page. You
should also consider updating the firmware for the printer. Be‐
fore starting the installation, you should check for compatibil‐
ity with operating systems, applications, and so on.
    \end{itemize}
    
\end{enumerate}

\subsection{Cloud Computing and
Client-side Virtualization}
\begin{enumerate}
    \item Which of the following types of cloud services offers e-mail
through a web browser?
    \begin{itemize}
        \item A. Software as a service (SaaS) is the most commonly recog‐
nized cloud service; it allows users to use applications to ac‐
cess data that is stored on the Internet by a third party. Infras‐
tructure as a service (IaaS) is a service that offers computer
networking, storage, load balancing, routing, and VM hosting.
Platform as a service (PaaS) is used for easy-to-configure op‐
erating systems and on-demand computing. A community
cloud is mix of public and private clouds, but one where multi‐
ple organizations can share the public portion.
    \end{itemize}
    \item Your organization requires more control over its data and in‐
frastructure. Money is apparently not an issue. There are only
two admins and about 30 users that will have access to the data
on the cloud. Which of the following types of clouds is the best
option?
    \begin{itemize}
        \item B. The best option listed is a private cloud. This gives the most
control over data and resources in an environment where there
are limited users (and a healthy budget). These resources could
be entirely internal, or a portion of them could also be pro‐
vided by a third-party. Public cloud technology is used for the
general public to access applications over the Internet. Hybrid
is a mixture of the two, but not necessary in this situation be‐
cause of the budget and the limited number of users. Commu‐
nity cloud is similar to hybrid but is meant for multiple organi‐
zations that share data, which is not necessary in this scenario.
    \end{itemize}
    \item You require the ability to add on to your cloud-based network
whenever necessary, rapidly and efficiently. What is this refer‐
ring to?
    \begin{itemize}
        \item C. Rapid elasticity is the ability to build your cloud-based net‐
work, or extend upon an existing one, quickly and efficiently.
Measured services is when a provider monitors a customer’s
services used so that the customer can be properly billed.
Metered services is when the customer can access as many re‐
sources as needed but only be billed for what was accessed.
On-demand service means that the cloud service is available at
all times. The leaders of a successful organization don’t care
what it takes; they simply want high speed, secure access to
services 24/7.
    \end{itemize}
    \item Of the following listed technologies, which one should you se‐
lect if wanted to run an instance of Ubuntu Linux within your
Windows 10 Pro workstation?
    \begin{itemize}
        \item B. You would need to run virtualization software that includes
a Type 2 hypervisor such as Windows 10 Hyper-V, VMwareWorkstation, or VirtualBox. Type 1 hypervisors are used on
servers; they are also known as bare metal because they allow
virtual machines to access the computer hardware directly. Ex‐
amples include VMware vSphere and Windows Server-based
Hyper-V. An emulator is something that imitates hardware and
firmware, such as an emulated BIOS. They do not use hypervi‐
sors.
    \end{itemize}
    \item Which of the following is the greatest risk of a virtual comput‐
er?
    \begin{itemize}
        \item D. The biggest risk of running a virtual computer is that it will
go offline immediately if the server that it is housed on fails.
All other virtual computers on that particular server will also
go offline immediately.
    \end{itemize}
    \item Which of the following file extensions is used by VMware?
    \begin{itemize}
        \item A. VMware uses the .vmdk file extension for the virtual hard
drive file. VirtualBox uses .vdi by default (though it can use
others). Hyper-V uses .vhdx. VT-x is the Intel virtualization
extension that is incorporated into Intel-based systems and
must be enabled in the UEFI/BIOS for virtualization software
to work.
    \end{itemize}
    \item Which of the following network connection types should be
used to allow for connectivity to the external network, but keep
the VMs on a separate IP network?
    \begin{itemize}
        \item B. Network address translation (NAT)-based network connec‐
tions are the most common default. This allows the VMs to
have their own IP network, but still connect out to the external
network and make use of the Internet. This is the same princi‐
ple behind NAT used in a SOHO network. Bridged means that
the VMs have access to the external network, but they must
use IP addresses from that external network. Private means
that multiple VMs within a host can communicate with each
other, but not beyond the host. The no networking option dis‐
ables any type of networking connectivity for the VM in ques‐
tion.
    \end{itemize}
    \item A customer running Windows 10 Pro wishes to install a Linux
VM in Hyper-V Manager. Which of the following require‐
ments must be met in order for this to happen? (Select all that
apply.)
    \begin{itemize}
        \item B, C and D. Virtualization must be enabled in the UEFI/BIOS.
Then, Hyper-V needs to be turned on in Windows Features.
Finally, the system needs to be restarted. Updating and secur‐
ing the host system is recommended, but is not a requirement.
    \end{itemize}
    
\end{enumerate}
\subsection{}
\begin{enumerate}
    \item What is the second step of the A+ troubleshooting methodology?
    \begin{itemize}
        \item B. The second step is to establish a theory of probable cause.
You need to look for the obvious or most probable cause for
the problem.
    \end{itemize}
    \item When you run out of possible theories for the cause of a prob‐
lem, what should you do?
    \begin{itemize}
        \item A. If you can’t figure out why a problem occurred, it’s time toget someone else involved. Escalate the problem to your super‐
visor.
    \end{itemize}
    \item What should you do before making any changes to the comput‐
er? (Select the best answer.)
    \begin{itemize}
        \item C. Always perform a backup of critical data before making any
changes to the computer.
    \end{itemize}
    \item Which of the following is part of Step 5 in the six-step trou‐
bleshooting process?
    \begin{itemize}
        \item D. Implement preventative measures as part of Step 5 to ensure
that the problem will not happen again.
    \end{itemize}
    \item What should you do next after testing the theory to determine
cause?
    \begin{itemize}
        \item A. After testing the theory to determine cause (Step 3), you
should establish a plan of action to resolve the problem (Step
4). Memorize the six-step troubleshooting process! You will
use it often.
    \end{itemize}
    \item There is a problem with the power supplied to a group of com‐
puters and you do not know how to fix the problem. What
should you do first?
    \begin{itemize}
        \item B. Contact the building supervisor or your manager.
    \end{itemize}
    \item What is the best way to tell if a CR2032 lithium battery has
been discharged?
    \begin{itemize}
        \item C. Although there might be a Windows application that moni‐
tors the battery, the surefire way is to test the voltage of the
lithium battery with a multimeter. A CR2032 lithium battery is
designed to run at 3 volts. Some UEFI/BIOS programs can
also monitor the voltage of the battery.
    \end{itemize}
    \item A PC reboots without any warning. You ruled out any chance
of viruses. When you look at the motherboard, you see that
some of the capacitors appear distended and out of shape.
What should you do?
    \begin{itemize}
        \item A. You should replace the motherboard if it is damaged. It
would be much too time-consuming to even attempt replacing
the capacitors and probably not cost-effective for your compa‐
ny.
    \end{itemize}
    \item A computer you are troubleshooting won’t boot properly.
When you power on the computer, the video display is blank
and you hear a series of beeps. What should you do?
    \begin{itemize}
        \item B. You should check the BIOS version and consult the docu‐
mentation that accompanies the motherboard. You might need
to go online for this information. You can also try performinga POST analysis to discern the problem. The issue could be
video-based, or RAM-based, but the beep code should help to
identify the problem.
    \end{itemize}
    \item You are troubleshooting a CPU and have already cut power,
disconnected the power cable, opened the case, and put on your
antistatic strap. What should you do next?
    \begin{itemize}
        \item B. Check connections first; it is quick, easy, and a common
culprit.
    \end{itemize}
    \item What is a possible symptom of a failing CPU?
    \begin{itemize}
        \item A. If the CPU is running beyond the recommended voltage
range for extended periods of time, it can be a sign of a failing
CPU. If the computer won’t boot at all, another problem might
have occurred, or the CPU might have already failed. Low
case temperatures are a good thing (if they aren’t below freez‐
ing!). Spyware is unrelated, but we talk about it plenty in the
security chapters.
    \end{itemize}
    \item You are repairing a computer that has been used in a ware‐
house for several years. You suspect a problem with a memory
module. What should you do first?
    \begin{itemize}
        \item C. Because the computer is being used in a warehouse (which
is often a fairly dirty environment), you should use compressed
air on the RAM slot and clean the memory module with con‐
tact cleaner. Clean out all of the dust bunnies within the entire
computer. Using MemTest86 or another memory diagnostic
tool is another good answer.
    \end{itemize}
    \item You just investigated a computer that is suffering from inter‐
mittent shutdowns. You note that the RAM modules are over‐
heating. What is the best solution?
    \begin{itemize}
        \item C. The best thing to do in this situation is to install heat sinks
on the RAM modules. On older computers, the memory con‐
troller in a northbridge doesn’t usually overheat because it al‐
ready has a heat sink; on newer computers, it is within the
CPU. A CPU can have only one fan. You can’t install more
(although an additional case fan might help). The chipset also
usually has a heat sink.
    \end{itemize}
    \item You just installed new, compatible RAM into a motherboard,
but when you boot the computer, it does not recognize the
memory. What should you do?
    \begin{itemize}
        \item A. If you are sure that the RAM is compatible and the system doesn’t recognize it during POST, try flashing the UEFI/BIOS.
It could be that the RAM is so new that the motherboard
doesn’t have the required firmware to identify the new RAM.
    \end{itemize}
    \item You are troubleshooting a computer that won’t power on. You
have already checked the AC outlet and the power cord, which
appear to be functioning properly. What should you do next?
    \begin{itemize}
        \item A. You should test the computer with a PSU tester. This can
tell you whether the power supply functions properly. You al‐
ready know that the AC outlet is functional, so there is no rea‐
son to use another outlet. The computer would still turn on if
the RAM wasn’t seated properly. A UPS won’t help the situa‐
tion because it is part of the power flow before the power sup‐
ply.
    \end{itemize}
    \item A computer you are troubleshooting shuts down without warn‐
ing. After a few minutes, it boots back up fine, but after running
for a short time, it shuts down again. Which of the following
components could be the cause? (Select the two best answers.)
    \begin{itemize}
        \item A and D. The two components that could cause the system to
shut down are the power supply and the CPU fan. Check the
CPU fan settings and temperature in the BIOS first before
opening the computer. If those are fine, you most likely need
to replace the power supply. The RAM, video card, and hard
drive should not cause the system to suddenly shut down.
    \end{itemize}
    
\end{enumerate}
\subsection{Troubleshooting Hard Drives and RAID Array} 
\begin{enumerate}
    \item What should you do first to repair a drive that is acting slug‐
gish?
    \begin{itemize}
        \item C. Attempt to defragment the disk. If it is not necessary, Win‐
dows lets you know. Then you can move to other options, such
as scanning the drive for viruses.
    \end{itemize}
    \item Which of the following are possible symptoms of hard drive
failure? (Select the two best answers.)
    \begin{itemize}
        \item A and C. System lockups and failed boot files or other failing
file operations are possible symptoms of hard drive failure.
Antivirus alerts tell you that the operating system has been
compromised, viruses should be quarantined, and a full scan
should be initiated. Sometimes hard drives can fail due to
heavy virus activity, but usually if the malware is caught
quickly enough, the hard drive should survive. Network drives
are separate from the local hard drive; inability to connect to a
network drive suggests a network configuration issue. If theBIOS doesn’t recognize the drive, consider a BIOS update.
    \end{itemize}
    \item You just replaced a SATA hard drive that you suspected had
failed. You also replaced the data cable between the hard drive and the motherboard. When you reboot the computer, you no‐
tice that the SATA drive is not recognized by the BIOS. What
most likely happened to cause this?
    \begin{itemize}
        \item C. Most likely, the SATA port is faulty. It might have been
damaged during the upgrade. To test the theory, you would
plug the SATA data cable into another port on the mother‐
board. We can’t format the drive until it has been recognized
by the BIOS, which, by the way, should recognize SATA
drives if the motherboard has SATA ports! SATA drives don’t
use jumpers unless they need to coexist with older IDE drives.
Most of today’s drives do not come with jumpers.
    \end{itemize}
    \item You are troubleshooting a SATA hard drive that doesn’t func‐
tion on a PC. When you try it on another computer, it works
fine. You suspect a power issue and decide to take voltage
readings from the SATA power connector coming from the
power supply. Which of the following readings should you
find?
    \begin{itemize}
        \item C. If you test a SATA power cable, you should find 3.3 V (or‐
ange wire), 5 V (red wire), and 12 V (yellow wire). If any of
these don’t test properly, try another SATA power connector.
    \end{itemize}
    \item You are troubleshooting a Windows Server that normally boots
from an SATA-based RAID 0 array. The message you receive
is “missing operating system”. As it turns out, another techni‐
cian has been updating the BIOS on several of the servers in
your organization, including this one. What configured setting
needs to be changed? (Select the best answer.)
    \begin{itemize}
        \item B. When the BIOS was updated, the SATA setting in the BIOS
probably reverted back to AHCI. That caused the RAID 0 ar‐
ray to be ignored, and so the OS would not boot, because it is
stored on that array. The setting should be changed from AHCI
to RAID (or similar name). Now, if this was a RAID 1 mirror,
then a copy of the OS would be on each drive, and it might still
boot (though you would probably receive a message as to the
state of the mirror being degraded or broken). But with RAID
0, the OS is striped across two or more drives—all drives need
to be present and accessed via RAID in order for the OS to
boot. That’s one of the reasons why the golden rule for many
years was to “mirror the OS, and stripe the data”. Phew! Any‐
ways, on to the incorrect answers. RAID 1 is incorrect, there
would be no option to set this; in the scenario we are using
RAID 0. S.M.A.R.T. is the monitoring system included in
HDDs and SSDs. NVMe (Non-Volatile Memory Express) isthe specification for non-volatile storage used by M.2 drives,
PCIe card-based drives, and so on. Remember to backup any
and all BIOS configurations!
    \end{itemize}
    \item What should you do first if your SATA magnetic disk begins
to make loud clicking noises? (Select the best answer.)
    \begin{itemize}
        \item A. Don’t hesitate! Copy the data to another drive. Afterward,
update the UEFI/BIOS, replace the drive with a new one, and
consider a new SATA cable while you are at it.
    \end{itemize}
\end{enumerate}
\subsection{Troubleshooting
Video Issues and Mobile Devices}
\begin{enumerate}
    \item A user set the resolution in Windows too high, resulting in a
scrambled, distorted display. What should you do to fix the
problem? (Select the best answer.)
    \begin{itemize}
        \item B. Boot into a low-resolution mode. In Windows, this is called
Enable Low-Resolution Video. Safe Mode is another valid op‐
tion, but keep in mind that Safe Mode loads Windows with a
minimal set of drivers and you can’t access the Internet.
Depending on the display configuration, pressing the monitor
toggle key might actually fix the problem temporarily by dis‐
playing the screen on a secondary monitor, but it doesn’t solve the root cause of the problem.
    \end{itemize}
    \item You are troubleshooting a video issue. Which utility should
you use?
    \begin{itemize}
        \item C. You should use DxDiag to troubleshoot video issues. The
other three answers are not used to troubleshoot video. Regedit
is used to perform advanced configurations in the registry. Ms‐
config is used to change how the system boots and enable/dis‐
able services. The Task Manager is used to see the perfor‐
mance of the computer and view applications and processes
that are running. We’ll discuss the rest of those tools in more
depth in the Windows portion of this book.
    \end{itemize}
    \item You receive a very basic computer that has a broken on-board
DVI connector. What should you attempt first?
    \begin{itemize}
        \item . C. Try installing a video card first to see if the system will still
work. Unless it is a specialized system, the video card should
be less expensive than the motherboard. (Not to mention it will
take a lot less time to install.) As PC techs, we usually do not
replace connectors; it is a possibility, but it should be further
down your troubleshooting list. An adapter cannot help if the
DVI port is broken.
    \end{itemize}
    \item You just replaced a video card in a PC with another card from
a different manufacturer. However, the driver installation does
not complete. What should you do first?
    \begin{itemize}
        \item . C. If the driver installation doesn’t complete, you should roll
back the driver. It could be that you have attempted to install
the incorrect driver. After you roll back the faulty installation,
find the correct latest version of the video driver from the man‐
ufacturer’s website. Installing the driver again can most likely
have the same result. Only reinstall the original video card
temporarily if you cannot find a proper solution right away.
    \end{itemize}
    \item A user’s mobile device is overheating. Which of the following
could be the problem? (Select the two best answers.)
    \begin{itemize}
        \item A and C. The best answers listed are a damaged battery and
excessive gaming. If the brightness setting is low, the device
should use less power and run cooler. If the device is not in a
case, it should not overheat; however, a poorly manufactured
case could cause it to overheat. A defective charging cable will
usually not cause the device to overheat; if it is defective, it
likely is not even charging the device.
    \end{itemize}
    \item You are troubleshooting a user’s smartphone. The user informs
you that he can’t see anything on the screen. Of the following,
what should you do first?
    \begin{itemize}
        \item C. Check the brightness slider first! Chances are that the
brightness is turned all the way down—in a bright environ‐
ment, it might appear that nothing is on the screen. Or, the de‐
vice might simply need to be woken up—use the side button,
home button, or double-tap the screen to wake the device up. If
these are not successful, restart the device. Screen calibration
has nothing to do with brightness, but on some more rare mo‐
bile devices it might be necessary. The volume sliders are not
part of the problem. Don’t open the device until you have ex‐
hausted every other known option!—and only if you are quali‐
fied to work on that device, otherwise send it to an authorized
repair center.
    \end{itemize}
    \item When disassembling a mobile device, what should you not do?
    \begin{itemize}
        \item E. Stay away from magnetically charged tools such as screw‐
drivers and bits. These can potentially damage circuitry andcomponents. All of the other answers were valid procedures;
things you should do when working on mobile devices, and
computers in general.
    \end{itemize}
\end{enumerate}
\subsection{Troubleshooting
Wired and Wireless Network
Problems} 
\begin{enumerate}
    \item A user complains that the computer is not connecting to the
network. Which of the following should be done first?
    \begin{itemize}
        \item C. Check the super-obvious first. Make sure the computer hasa physical cabled connection to the network. Then attempt
things such as ipconfig, ping, and network driver up‐
dates.
    \end{itemize}
    \item One computer loses connectivity. All connectors and settings
appear to be correct. Which tool should be used to fix the prob‐
lem?
    \begin{itemize}
        \item D. Use a patch cable tester to check the patch cable and possi‐
bly use a continuity tester to test longer network cable runs.
Multimeters are great for testing wires inside the computer or
AC outlets, but they are not used for network troubleshooting.
A PSU tester tests power supplies. The loopback plug will ver‐
ify whether the local computer’s network adapter is functional.
    \end{itemize}
    \item One of your customers no longer has access to a frequently ac‐
cessed website. You ping another computer and the router on
the network successfully. Which of the following should be
done next?
    \begin{itemize}
        \item B. This is the concept of pinging outward. Start by pinging the
localhost, then a computer, and then the router on the network.
Then ping a domain name or website. If you can ping the web‐
site but the browser cannot get through, the browser might
have been compromised. If you cannot ping the website, you
should check the IP configuration; the DNS server address
might be incorrectly configured. Updating the OS and AV
software should be done right away if you guess that the
browser has been compromised.
    \end{itemize}
    \item A user moves a laptop from one office to another. The patch
cable and the network adapter do not appear to be working
properly at the new office. The cable is plugged in correctly
and tests okay when checked with a patch tester. Which of the
following should be done first?
    \begin{itemize}
        \item A. Some routers and switches can disable physical ports (a
smart security measure). Check that first. Later, you can check
whether the network adapter is compatible with the OS and up‐
date it if necessary. Do not replace the cable with a crossover;
those are used to connect one computer to another.
    \end{itemize}
\end{enumerate}
\section{CompTIA A+ Premium} 
\begin{enumerate}
    \item Which of the following are components you might find inside a PC? 
    \begin{itemize}
        \item \textbf{Answers: A, B, and E}
Explanation: Common components inside a PC include the CPU, motherboard, and
RAM, along with the power supply, adapter cards, and hard drives
    \end{itemize}
    \item Which device stores data over the long term? 
    \begin{itemize}
        \item \textbf{Answer: C}
Explanation: The hard drive stores data over the long term. The hard drive stores the
OS and data in a nonvolatile fashion, meaning the data won’t be erased when the com-
puter is turned off.
    \end{itemize}
    \item To which type of technology would you install a x16 card?
    \begin{itemize}
        \item \textbf{Answer: B } Explanation: PCI Express (PCIe) expansion slots accept x1, x4, and x16 cards
(pronounced “by one”, “by four”, and “by sixteen” respectively).
    \end{itemize}
    \item Which process of the computer checks all your components
during boot?
    \begin{itemize}
        \item \textbf{Answer: B}
Explanation: The POST (power-on self-test) is part of the Basic Input/Output System
(BIOS) or Unified Extensible Firmware Interface (UEFI). It runs a self-check of the com-
puter system during boot and stores many of the parameters of the components within
the CMOS.
    \end{itemize}
    \item Tim installs a new CPU in a computer. After a few hours, the
processor starts to overheat. Which of the following might be the
cause?
    \begin{itemize}
        \item \textbf{Answer: C}
Explanation: Without the thermal compound applied, the processor might overheat
after a few hours.
    \end{itemize}
    \item Which of the following could cause the POST to fail? (Select the
two best answers.)
    \begin{itemize}
        \item \textbf{Answers: A and D} 
Explanation: The CPU and memory need to be installed properly for the POST to run
(and to pass).
    \end{itemize}
    \item Which of the following might you find as part of a tablet
computer? (Select the two best answers.)
    \begin{itemize}
        \item \textbf{Answers: A and C}
Explanation: A tablet computer will almost always contain flash memory as main stor-
age and a multi-touch touchscreen.
    \end{itemize}
    \item Which kind of socket incorporates “lands” to ensure connectivity
to a CPU?
    \begin{itemize}
        \item \textbf{Answer: C}
Explanation: LGA (land grid array) is the type of socket that uses “lands” to connect
the socket to the CPU.
    \end{itemize}
     \item How should you hold RAM when installing it?
    \begin{itemize}
        \item  \textbf{Answer: A} Explanation: Hold RAM by the edges to avoid contact with the pins, chips, and cir-
cuitry.
    \end{itemize}
     \item You have been tasked with setting up a specialized computer for
video editing. Which of the following should you include with the
computer? (Select the two best answers.)
    \begin{itemize}
        \item \textbf{Answers: C and E}
Explanation: You should incorporate a solid-state drive (SSD) and dual monitors for
the video editing workstation. Video files, such as .MP4 and .AVI, are considerable;
they are big files that require a decent-size hard drive. An SSD, possibly an M.2 or
similar drive, helps to work with these files efficiently, especially in the video rendering
stage. And having dual monitors (or more than two) is important when editing. You will
generally want to have an editing window and a separate playback window.
    \end{itemize}
     \item How many pins are inside an SATA 3.0 data connector?
    \begin{itemize}
        \item \textbf{ Answer: B}
Explanation: The SATA version 3.0 data connector has seven pins. Note: SATA Express
uses a triple connector with 18 pins (7 + 7 + 4).
    \end{itemize}
     \item What is the delay in the RAM’s response to a request from the
memory controller called?
    \begin{itemize}
        \item \textbf{Answer: A}
Explanation: Memory latency or CAS (column address strobe) latency happens
when a memory controller tries to access data from a memory module. It is a slight
delay (usually measured in nanoseconds) while the memory module responds to the
memory controller. It is also known as CL. The memory controller (also known as the
northbridge) has a specific speed at which it operates. If the CPU asks the chip for too
much information at once, this might increase latency time while the memory control-
ler works.
    \end{itemize}
     \item What is the minimum number of hard drives necessary to imple-
ment RAID 5?
    \begin{itemize}
        \item \textbf{Answer: C}
Explanation: Because RAID 5 uses striping with parity, a third disk is needed. You can
have more than three disks as well.
    \end{itemize}
     \item A user’s time and date keep resetting to January 1, 2012. Which of
the following is the most likely cause?
    \begin{itemize}
        \item \textbf{Answer: D}
Explanation: If the time and date keep resetting—for example, to a date such as
January 1, 2012—chances are that the lithium battery needs to be replaced. These are
usually nickel-sized batteries; most PCs use a CR2032 lithium battery.
    \end{itemize}
     \item Which type of adapter card is normally plugged into a PCIe x16
adapter card slot?
    \begin{itemize}
        \item Answer: B
Explanation: The PCI Express (PCIe) x16 expansion slot is used primarily for video
    \end{itemize}
     \item Which of the following is a common type of CPU for a smart-
phone?
    \begin{itemize}
        \item Answer: B
Explanation: System on a chip (SoC) is a type of CPU used in smartphones and tablet
computers. The 64-bit versions are common in mobile devices; they incorporate a vari-
ety of functionality within the CPU.
    \end{itemize}
     \item Which of the following components could cause the POST to beep
several times and fail during boot?
    \begin{itemize}
        \item \textbf{Answer: D}
Explanation: RAM is one of the big four (RAM, CPU, motherboard, and video) that can
cause the POST to fail. Different RAM errors can cause the POST to make a different
series of beeps. Consult your motherboard documentation for more information about
the different beep codes.
    \end{itemize}
     \item Which of the following are ports you might find on smartphones and tablets? 
    \begin{itemize}
        \item \textbf{Answers: B and C}
Explanation: Mobile devices such as smartphones and tablets commonly incorporate
ports such as USB-C (and micro USB) and Apple’s Lightning connector.
    \end{itemize}
    \item Which of the following has the fastest data throughput? 
    \begin{itemize}
        \item \textbf{Answer: C}
Explanation: RAM is much faster than the rest of the options listed. For instance, if
you have PC4-25600 DDR4 RAM (aka DDR4-3200), your peak transfer rate is 25,600
MB/s.
    \end{itemize}
    \item Which of the following CPU cooling methods is the most common? 
    \begin{itemize}
        \item \textbf{ Answer: B}
Explanation: The most common CPU cooling method is the heat sink and fan combi-
nation. The heat sink helps the heat to disperse away from the CPU, whereas the fan
blows the heat down and through the fins; the power supply exhaust fan and possibly
additional case fans help the heat escape the case. Heat sink and fan combinations are
known as active cooling methods.
    \end{itemize}
    \item You are tasked with fixing a problem with a video editing workstation. There is an unexpected clicking noise 
    every time the video editing program is started. The case fans have been replaced, but the noise remains. Diagnostics
    have also been run on a the video card, and it appears to be operating normally. What action should you take first? 
    \begin{itemize}
        \item \textbf{ Answer: D}
Explanation: The first action you should take is to perform a full data backup. Clicking
noises can indicate that the computer’s magnetic-based hard drive is damaged and
might fail. Immediately back up the drive’s contents before taking any other action.
    \end{itemize}
    \item Which of the following is not a video port? 
    \begin{itemize}
        \item \textbf{ Answer: D}
Explanation: S/PDIF is not a video port. It stands for Sony/Philips digital interface
format—a digital audio interconnect that can be used with fiber-optic TOSLINK con-
nectors or coaxial RCA connectors.
    \end{itemize}
    \item Which of the following is necessary for a CAD/CAM workstation? 
    \begin{itemize}
        \item Answers: A and D
Explanation: A computer-aided design/computer-aided manufacturing workstation
requires an SSD, high-end video, and as much RAM as possible. The CPU can also
be important to run the latest design applications such as AutoCAD. Always check the
minimum and recommended requirements for applications.
    \end{itemize}
    \item Which of the following technologies allow two mobile devices to transfer data simply by touching them together? 
    \begin{itemize}
        \item \textbf{Answer: B}
Explanation: Near field communication (NFC) allows two mobile devices such as
smartphones to transfer data simply by touching them together (or bringing them in
very close proximity of each other).
    \end{itemize}
    \item What type of power connector is used for a x16 video card? 
    \begin{itemize}
        \item \textbf{Answer: C}
Explanation: A x16 card is a PCI Express card. It can have one or two PCIe 6-pin
power connectors (or 8-pin).
    \end{itemize}
    \item Which of the following are output devices? 
    \begin{itemize}
        \item \textbf{Answers: A, D, and E}
Explanation: Speakers, printers, and displays are output devices. A speaker outputs
sound. A printer outputs paper with text and graphics. A display (or monitor) displays
video.
    \end{itemize}
    \item What does the b in 1000 Mbps stand for? 
    \begin{itemize}
        \item \textbf{Answer: B}
Explanation: The b in 1000 Mbps stands for bits: 1000 Mbps is 1000 megabits per
second or 1 gigabit per second. Remember that the lowercase b is used to indicate bits
when measuring network data transfer rates, USB data transfer rates, and other similar
serial data transfers.
    \end{itemize}
    \item When running cable through drop ceilings, which type of cable do
you need?
    \begin{itemize}
        \item \textbf{Answer: D}
Explanation: Plenum-rated cable needs to be installed wherever a sprinkler system is
not able to spray water. This includes ceilings, walls, and plenums (airways). Plenum-
rated cable has a protective covering that burns slower and gives off fewer toxic fumes
than regular PVC-based cable.
    \end{itemize}
    \item Which device connects multiple computers in a LAN?
    \begin{itemize}
        \item \textbf{Answer: C}
Explanation: A switch connects computers together in a local-area network (LAN). In
SOHO networks, it is usually a part of a multifunction network device. In larger net-
works, the switch is an individual device that has 24, 48, or 96 ports.
    \end{itemize}
    \item Which of the following is the default subnet mask for IP address
192.168.1.1?
    \begin{itemize}
        \item Answer: B
Explanation: 192.168.1.1, by default, has the subnet mask 255.255.255.0, which is the
standard subnet mask for class C IP addresses. However, remember that some net-
works are classless, which means that a network can use a different subnet mask.
    \end{itemize}
    \item Which of the following is the minimum category cable needed for a
1000BASE-T network?
    \begin{itemize}
        \item Answer: C
Explanation: The minimum cable needed for 1000BASE-T networks is Category 5e.
Of course, Cat 6 would also work, but it is not the minimum of the listed answers.
1000BASE-T specifies the speed of the network (1000 Mbps), the type (baseband,
single shared channel), and the cable to be used (T = twisted pair).
    \end{itemize}
    \item Which of the following IP addresses can be routed across
the Internet?
    \begin{itemize}
        \item \textbf{Answer: C}
Explanation: The only listed answer that is a public address (needed to get onto the
Internet) is 129.52.50.13.
    \end{itemize}
    \item Which port number is used by HTTPS by default?
    \begin{itemize}
        \item \textbf{Answer: D}
Explanation: The Hypertext Transfer Protocol Secure (HTTPS) uses port 443 (by
default).
    \end{itemize}
    \item Which of the following cable types have a copper medium?
(Select the three best answers.)
    \begin{itemize}
        \item\textbf{ Answers: A, B, and D}
Explanation: Twisted-pair, coaxial, and Category 7 cable are all examples of network
cables with a copper medium. They all send electricity over copper wire.
    \end{itemize}
    \item Which of the following cable types can protect from electromag-
netic interference (EMI)? (Select the two best answers.)
    \begin{itemize}
        \item  \textbf{Answers: B and C} Explanation: Shielded twisted pair (STP) and fiber optic can protect from EMI.
    \end{itemize}
    \item You are configuring Bob’s computer to access the Internet. Which
of the following are required? (Select all that apply.)
    \begin{itemize}
        \item \textbf{Answers: A and B}
Explanation: To get on the Internet, the DNS server address is required so that the
computer can get the resolved IP addresses from the domain names that are typed in.
The gateway address is necessary to get outside the network.
    \end{itemize}
    \item Which of the following translates a computer name into an IP
address?
    \begin{itemize}
        \item \textbf{Answer: C}
Explanation: The Domain Name System (DNS) protocol translates a computer name
into an IP address. Whenever you type a web server name such as dprocomputer.com,
a DNS server translates that name to its corresponding IP address.
    \end{itemize}
    \item A customer wants to access the Internet from many different loca-
tions in the United States. Which of the following is the best tech-
nology to enable the customer to do so?
    \begin{itemize}
        \item \textbf{Answer: B}
Explanation: Cellular WAN uses a phone or other mobile device to send data over stan-
dard cellular connections.
    \end{itemize}
    \item You just configured the IP address 192.168.0.105 in Windows.
When you press the Tab key, Windows automatically configures
the default subnet mask of 255.255.255.0. Which of the following
IP addresses is a suitable gateway address?
    \begin{itemize}
        \item Answer: D
Explanation: 192.168.0.1 is the only suitable gateway address. Remember that the
gateway address must be on the same network as the computer. In this case, the net-
work is 192.168.0, as defined by the 255.255.255.0 subnet mask.
    \end{itemize}
    \item A wireless network is referred to as which of the following?
    \begin{itemize}
        \item Explanation: The Service Set Identifier (SSID) is the name of the wireless network.
This is the name you look for when locating a wireless network.
    \end{itemize}
    \item You have been tasked with blocking remote logins to a server.
Which of the following ports should you block?
    \begin{itemize}
        \item \textbf{Answer: B}
Explanation: Port 23 should be blocked. It is associated with the Telnet service, which
is used to remotely log in to a server at the command line. You can block this service
at the company firewall and individually at the server and other hosts. It uses port 23
by default, but it can be used with other ports as well. Telnet is considered to be inse-
cure, so it should be blocked and disabled.
    \end{itemize}
    \item Which of the following connector types is used by fiber-
optic cabling?
    \begin{itemize}
        \item Explanation: The LC connector is used by fiber-optic cabling. Other fiber connectors
include SC and ST.
    \end{itemize}
    \item Which protocol uses port 53?
    \begin{itemize}
        \item C. DNS
    \end{itemize}
    \item Which of the following Internet services are wireless? (Select the
two best answers.)
    \begin{itemize}
        \item B. Satellite, D. Cellular
    \end{itemize}
    \item Which of the following terms best describes two or more LANs
connected over a large geographic distance?
    \begin{itemize}
        \item \textbf{Answer: B}
Explanation: A wide-area network (WAN) is a network in which two or more LANs are
connected over a large geographic distance—for example, between two cities. The
WAN requires connections to be provided by a telecommunications or data communi-
cations company.
    \end{itemize}
    \item Which device connects to the network and has the sole purpose of
providing data to clients?
    \begin{itemize}
        \item \textbf{Answer: A}
Explanation: Network-attached storage (NAS) devices store data for network use. They
connect directly to the network.
    \end{itemize}
    \item You are making your own networking patch cable. You need to
attach an RJ45 plug to the end of a twisted-pair cable. Which tool
should you use?
    \begin{itemize}
        \item C. Crimper
    \end{itemize}
    \item Which port is used by RDP?
    \begin{itemize}
        \item \textbf{Answer: D}
Explanation: The Remote Desktop Protocol (RDP) uses port 3389 by default. This pro-
tocol allows one computer to take control of another remote system.
    \end{itemize}
    \item Which key on a laptop aids in switching to an external monitor?
    \begin{itemize}
        \item \textbf{Answer: A}
Explanation: The Fn (Function) key is used for a variety of things, including toggling
between the built-in LCD screen and an external monitor/TV. The Fn key could be a
different color (for example, blue) and offers a sort of “second” usage for keys on the
laptop.
    \end{itemize}
    \item Which of the following printer failures can be described as a
condition in which the internal feed mechanism stopped working
temporarily?
    \begin{itemize}
        \item \textbf{Answer: C}
Explanation: A failure that occurs due to the internal feed mechanism stopping is
known as a paper jam. For example, an HP LaserJet might show error code 13.1 on
the display, which means a paper jam at the paper feed area. You should verify that the
paper trays are loaded and adjusted properly.
    \end{itemize}
    \item A customer can barely hear sound from the speakers on her lap-
top. What should you do first?
    \begin{itemize}
        \item \textbf{Answer: C}
Explanation: Sometimes, the volume key(s) on laptops can be a little difficult to locate
and may be muted or set to the lowest position (which might still make a slight audible
noise). The volume is usually controlled by pressing the Fn key and the Volume Up or
Volume Down key (or mute) simultaneously.
    \end{itemize}
    \item After you replace a motherboard in a PC, the system overheats and
fails to boot. Which of the following is the most likely cause?
    \begin{itemize}
        \item C. Thermal paste was not applied between the heat sink
and the CPU.
    \end{itemize}
    \item You use your laptop often. Which of the following is a simple, free
way to keep your laptop running cool?
    \begin{itemize}
        \item Explanation: Laptops have airflow underneath them; if the unit is not on a flat surface,
that airflow will be reduced or stopped altogether, leading to component damage.
    \end{itemize}
    \item Which of the following are important factors when purchasing a
replacement laptop AC adapter? (Select the two best answers.)
    \begin{itemize}
        \item \textbf{Answers: A and B}
Explanation: Make sure to purchase an AC adapter that is a true replacement. You can
find one on the laptop manufacturer’s website. When you enter your model number,
the website will tell you everything you need to know about current, voltage, and con-
nector type (these details are also listed on the brick portion of the power adapter).
    \end{itemize}
    \item Eric uses an external monitor with his laptop. He tells you that
his laptop will boot, but the system won’t display anything on
the external screen. Which of the following solutions enables the
display?
    \begin{itemize}
        \item \textbf{Answer: B}
Explanation: The Screen key (also known as the display toggle) is one of the keys
available when you use the Function (Fn) key. It enables you to switch between the lap-
top display and an external display (or if you want to use both).
    \end{itemize}
    \item Which type of printer uses a toner cartridge?
    \begin{itemize}
        \item Answer: B
Explanation: Laser printers use toner cartridges.
    \end{itemize}
    \item Which of the following should not be connected to a UPS?
    \begin{itemize}
        \item \textbf{Answer: C}
Explanation: Laser printers use large amounts of electricity, which in turn could
quickly drain the battery of the UPS. They should be plugged in to their own individual
power strips.
    \end{itemize}
    \item Terri finishes installing a printer for a customer. What should she
do next?
    \begin{itemize}
        \item \textbf{Answer: B}
Explanation: Print a test page after installation. If the test page prints properly, printing
a page in Word should be unnecessary.
    \end{itemize}
    \item Which of the following best describes printing in duplex?
    \begin{itemize}
        \item \textbf{Answer: A}
Explanation: When you are printing “duplex,” you are printing on both sides of the
paper (if the printer has that capability). Some laser printers can do this, but printing
this way creates a longer total paper path, which leads to more frequent paper jams.
    \end{itemize}
    \item Special paper is needed to print on which type of printer?
    \begin{itemize}
        \item Answer: B
Explanation: Regular paper can be used on all the listed printers except for thermal
printers, which use specially coated paper that is heated to create the image.
    \end{itemize}
    \item Which of the following channels should you select for an 802.11
wireless network?
    \begin{itemize}
        \item \textbf{Answer: A}
Explanation: Of the listed answers, use channel 6 for 802.11 wireless networks.
That would imply a 2.4 GHz connection using either 802.11n, g, or b. The 2.4 GHz
frequency range in the United States allows for channels 1 through 11.
    \end{itemize}
    \item Which environmental issue affects a thermal printer the most
    \begin{itemize}
        \item \textbf{Answer: D}
Explanation: Heat is the number-one enemy to a thermal printer. Keeping a thermal
printer or the thermal paper in a location where the temperature is too high could
cause failure of the printer and damage to the paper.
    \end{itemize}
    \item Which of the following occurs last in the laser printing process?
    \begin{itemize}
        \item A\textbf{nswer: E}
Explanation: In the laser printing process, also known as the imaging process, the
cleaning stage happens last.
    \end{itemize}
    \item Which type of printer uses impact to transfer ink from a ribbon to
the paper?
    \begin{itemize}
        \item \textbf{Answer: C}
Explanation: Impact is a type of impact printer. It uses a printhead to physically impact
the ribbon and transfer ink to the paper.
    \end{itemize}
    \item Which of the following steps enables you to take control of a net-
work printer from a remote computer?
    \begin{itemize}
        \item \textbf{Answer: C}
Explanation: After you install the driver for the printer locally, you can then take control
of it by going to the properties of the printer and accessing the Ports tab. Then click
the Add Port button and select the Standard TCP/IP Port option. You have to know the
IP address of the printer or the computer that the printer is connected to.
    \end{itemize}
    \item A color laser printer produces images that are tinted blue. Which of
the following steps should be performed to address this problem?
    \begin{itemize}
        \item \textbf{Answer: B}
Explanation: After you install a printer, it is important to calibrate it for color and ori-
entation, especially if you are installing a color laser printer or an inkjet printer. These
calibration tools are usually built in to the printer’s software and can be accessed from
Windows, or you can access them from the printer’s display.
    \end{itemize}
    \item A laptop cannot access the wireless network. Which of the fol-
lowing statements best describes the most likely causes for this?
(Select the two best answers.)
    \begin{itemize}
        \item \textbf{Answers: A and C}
Explanation: Most laptops have a special function key (for example, F12) that allows
you to enable or disable the Wi-Fi connection just by pressing it. Also, a wireless net-
work can be “forgotten” in the operating system. If this happens, the laptop has to be
reconnected to the wireless network.
    \end{itemize}
    \item A desktop computer does not have a lit link light on the back of
the computer. Which of the following is the most likely reason for
this?
    \begin{itemize}
        \item \textbf{ Answer: D}
Explanation: The most likely answer in this scenario is that the network cable is dis-
connected. If the desktop computer is using a wired connection, it is most likely a
twisted-pair Ethernet connection. When this cable is connected to the computer on one
end and to a switch or other central connecting device on the other end, it initiates a
network connection over the physical link. This link then causes the network adapter’s
link light to light up. The link light is directly next to the RJ45 port of the network
adapter. The corresponding port on the switch (or other similar device) is also lit. If the
cable is disconnected, the link light becomes unlit, though there are other possibilities
for this link light to be dark—for example, if the computer is off or if the switch port is
disabled.
    \end{itemize}
    \item Which of the following IP addresses would a technician see if a
computer running Windows is connected to a multifunction net-
work device and is attempting to obtain an IP address automati-
cally but is not receiving an IP address from the DHCP server?
    \begin{itemize}
        \item \textbf{Answer: C} Explanation: If the computer fails to obtain an IP address from a DHCP server,
Windows will take over and apply an Automatic Private IP Address (APIPA). This
address will be on the 169.254.0.0 network.
    \end{itemize}
    \item For which type of PC component are 80 mm and 120 mm common
sizes?
    \begin{itemize}
        \item \textbf{Answer: A}
Explanation: Case fans are measured in mm (millimeters); 80 mm and especially 120
mm are very common. They are used to exhaust heat out of the case. These fans aid
in keeping the CPU and other devices cool. The 120 mm is quite common in desktop
and tower PCs, and the 80 mm is more common in smaller systems and 1U and 2U
rackmount servers.
    \end{itemize}
    \item An exclamation point next to a device in the Device Manager indi-
cates which of the following?
    \begin{itemize}
        \item \textbf{Answer: A}
Explanation: If you see an exclamation point in the Device Manager, this indicates that
the device does not have a proper driver.
    \end{itemize}
    \item Beep codes are generated by which of the following?
    \begin{itemize}
        \item \textbf{Answer: C}
Explanation: As the power-on self-test (POST) checks all the components of the com-
puter, it may present its findings on the screen or in the form of beep codes.
    \end{itemize}
    \item Which of the following indicates that a printer is network-ready?
    \begin{itemize}
        \item \textbf{Answer: C}
Explanation: The RJ45 jack enables a connection to a twisted-pair (most likely
Ethernet) network. Printers with a built-in RJ45 connector are network-ready; so are
printers that are Wi-Fi enabled.
    \end{itemize}
    \item You just turned off a printer to maintain it. Which of the following
should you be careful of when removing the fuser?
    \begin{itemize}
        \item \textbf{Answer: A}
Explanation: The fuser heats paper to around 400° Fahrenheit (204° Celsius). That’s
like an oven. If you need to replace the fuser, let the printer sit for 10 or 15 minutes
after shutting it down and before maintenance.
    \end{itemize}
    \item Which of the following connectors is used for musical equipment?
    \begin{itemize}
        \item \textbf{Answer: A}
Explanation: The Musical Instrument Digital Interface (MIDI) connector is used for
musical equipment such as keyboards, synthesizers, and sequencers. MIDI is used to
create a clocking signal that all devices can synchronize to.
    \end{itemize}
    \item Which of the following tools is not used as often as a Phillips
screwdriver but is sometimes used to remove screws from the
outside of a computer case or from within a laptop?
    \begin{itemize}
        \item \textbf{Answer: B}
Explanation: The Torx screwdriver (also known as a Torx wrench) is a special tool
used to remove screws from the outside of a case; often, proprietary computer manu-
facturers use these screws. This tool can also be used to remove screws (albeit smaller
ones) from a laptop. The standard is the size T10 Torx screwdriver, but you might also
use a T8 or even a T6 on laptops.
    \end{itemize}
    \item Moving your CPU’s speed beyond its normal operating range is
called .
    \begin{itemize}
        \item \textbf{Answer: A}
Explanation: Overclocking is the act of increasing your CPU’s operating speed beyond
its normal rated speed.
    \end{itemize}
    \item Which of the following is the most important piece of information
needed to connect to a specific wireless network?
    \begin{itemize}
        \item \textbf{Answer: C}
Explanation: The Service Set Identifier (SSID) is the most important piece of informa-
tion required to connect to a wireless network; it is the name of the wireless network.
    \end{itemize}
     \item You are considering a cloud-based service for file storage and syn-
chronization. Which of the following resources is the most critical
to your design?
    \begin{itemize}
        \item \textbf{Answer: D}
Explanation: I/O bandwidth is the most critical of the listed resources in the question.
When you are considering file storage and file synchronization, you need to know
the maximum input/output operations per second (IOPS) that the cloud provider can
deliver. (Get actual reports of previous customers as proof!) IOPS gives you a concrete
measurement of data that you can use for analysis.
    \end{itemize}
     \item Which of the following statements describes why the display on a
laptop gets dimmer when the power supply from the AC outlet is
disconnected?
    \begin{itemize}
        \item \textbf{Answer: B}
Explanation: The power management settings on the laptop can cause the display
to automatically dim when the AC adapter is unplugged. In fact, this is the default
on many laptops to conserve battery power. These settings can be configured within
Power Options in Windows.
    \end{itemize}
     
\end{enumerate}
\subsection{Exam 2} 
\begin{enumerate}
    \item Which of the following servers is responsible for resolving a name
such as dprocomputer.com to its corresponding IP address?
    \begin{itemize}
        \item Answer: C
Explanation: A Domain Name System (DNS) server is responsible for resolving (or
converting) hostnames and domain names to their corresponding IP addresses. To
see this in action, open the command line and try connecting to a domain (such
as dprocomputer.com) and run a ping, tracert (or traceroute in Linux and macOS),
nslookup, or dig (or all of those) against the domain name. These commands can
supply you with the IP address of the host. Remember, always practice in a hands-on
manner on real computers to reinforce your knowledge!
    \end{itemize}
     \item Which of the following statements describe advantages of using
the Dynamic Host Configuration Protocol (DHCP)? (Select the two
best answers.)
    \begin{itemize}
        \item Answers: A and E
Explanation: Advantages of using DHCP include the following: IP addresses can be
managed from a central location, and computers can automatically get new addressing
when moved to a different network segment (perhaps one that uses a different DHCP
server).
    \end{itemize}
     \item Which of the following storage technologies is used by hard
disk drives?
    \begin{itemize}
        \item Answer: A
Explanation: Hard disk drives (HDDs) are magnetic disks. These are the type with
moving parts, as opposed to solid-state drives (SSDs) that have no moving parts.
    \end{itemize}
     \item On which type of computer is RAM the most important
    \begin{itemize}
        \item Answer: B
Explanation: RAM is more essential to the virtualization workstation than any of the
other types of custom PCs listed. Virtual operating systems (virtual machines or VMs)
require a lot of RAM to run, much more than any other application. Plus, a virtualization
workstation often has more than one virtual machine running, increasing its need
for RAM even further.
    \end{itemize}
     \item A client brings in a printer that is giving a paper-feed error. Which
of the following is the most likely cause?
    \begin{itemize}
        \item Answer: D
Explanation: Paper-feed errors are often caused by the pickup rollers, which are in
charge of feeding the paper into the printer.
    \end{itemize}
     \item Which protocol uses port 389?
    \begin{itemize}
        \item Answer: C
Explanation: Port 389 is used by the Lightweight Directory Access Protocol (LDAP).
Although 389 is the default port, 636 is also used for secure LDAP.
    \end{itemize}
    \item What is the maximum distance at which a Class 2 Bluetooth device
can receive signals from a Bluetooth access point?
    \begin{itemize}
        \item Answer: B
Explanation: Class 2 Bluetooth devices have a maximum range of approximately
10 meters. Class 2 devices (such as Bluetooth headsets) are the most common.
    \end{itemize}
    \item Which of the following wireless networking standards operates at
5 GHz only? (Select the two best answers.)
    \begin{itemize}
        \item Answers: A and E
Explanation: 802.11a operates at 5 GHz only; so does 802.11ac.
    \end{itemize}
    \item Which of the following types of RAM has a peak transfer rate of
21,333 MB/s?
    \begin{itemize}
        \item Answer: D
Explanation: DDR4-2666 has a peak transfer rate of 21,333 MB/s. It runs at an I/O bus
clock speed of 1333 MHz and can send 2666 megatransfers per second (MT/s). It is
also known as PC4-21333.
The math: To figure out the data transfer rate of DDR4 from the name “DDR4-2666,”
simply multiply the 2666 by 8 (bytes) and solve for megabytes: 21,333 MB/s. To figure
out the data transfer rate of DDR4 by the consumer name “PC4-21333,” just look at
the number within the name and add “MB/s” to the end. To figure out the data transfer
rate when given only the I/O bus clock speed (for example, 1333 MHz), multiply
    \end{itemize}
    \item Which of the following types of printers uses a print head, ribbon,
and tractor feed?
    \begin{itemize}
        \item Answer: B
Explanation: The impact printer uses a print head, ribbon, and tractor feed. An example
of an impact printer is the dot matrix.
    \end{itemize}
    \item Several users on your network are reporting that a network printer,
which is controlled by a print server, is not printing. What is the
first action you should take to fix the problem?
    \begin{itemize}
        \item Answer: C
Explanation: The first thing you should try (of the listed answers) is to clear the print
queue. How this is done will vary depending on the type of print server being used. For
example, if the printer is being controlled by a Windows Server, that server will have
the Print Management role installed. From the Print Management console window, you
can access the printer in question and locate the queue. In most client Windows operating
systems (such as Windows 10), you can find the queue simply by double-clicking
the printer. Regardless of the OS, it is where you would normally go to manage print
jobs. Then, after you have found the queue, delete any pending or stalled jobs. One of
those print jobs might have been causing a delay, possibly if the print job was too big
or was corrupted. Of course, you will have to notify users that any jobs as of x time
frame have been deleted and they will need to be printed again. On some printers you
can clear the print queue from the onscreen display (OSD) directly on the printer. Or, if
the print server is built into a SOHO router or similar device, you would have to access
it from a browser.
    \end{itemize}
    \item Which of the following is a possible symptom of a failing CPU?
    \begin{itemize}
        \item Answer: A
Explanation: If the CPU is running beyond the recommended voltage range for extended
periods of time, it can be a sign of a failing CPU. The problem could also be caused
by overclocking. Check in the UEFI/BIOS to see whether or not the CPU is overclocked.
    \end{itemize}
    \item Which of the following cable types is not affected by EMI but
requires specialized tools to install?
    \begin{itemize}
        \item Answer: C
Explanation: Fiber-optic cable is the only answer listed that is not affected by electromagnetic
interference (EMI). The reason is that it does not use copper wire or electricity,
but instead uses glass or plastic fibers and light.
    \end{itemize}
    \item Setting an administrator password in the BIOS accomplishes
which of the following?
    \begin{itemize}
        \item Answer: A
Explanation: Setting an admin password in the UEFI/BIOS prevents a user from rearranging
the boot order. The idea behind this is to stop a person from attempting to
boot off an optical disc or USB flash drive. As an administrator, you should change the
BIOS boot order to hard drive first. Then apply an administrative password. That’ll stop
‘em right in their tracks!
    \end{itemize}
    \item Which of the following functions is performed by the external
power supply of a laptop?
    \begin{itemize}
        \item Explanation: The external power supply of the laptop converts AC to DC for the system
to use and for charging the battery. It is known as the power adapter, and it needs to
run at a very specific voltage. In fact, different make and model power adapters usually
do not work with different laptops, even if the voltages are only slightly different.
    \end{itemize}
    \item How many pins would you see in a high-quality printhead on a
dot-matrix printer?
    \begin{itemize}
        \item Answer: A
Explanation: High-quality dot-matrix printheads can come in 9, 18, or 24 pins, with 24
being the highest quality.
    \end{itemize}
    \item What is an LCD display’s contrast ratio defined as?
    \begin{itemize}
        \item Answer: C
Explanation: Contrast ratio is the brightness of the brightest color (measured as white)
compared to the darkest color (measured as black). Static contrast ratio measurements
are static; they are performed as tests using a checkerboard pattern. But there is also
the dynamic contrast ratio, a technology in LCD displays that adjusts dynamically during
darker scenes in an attempt to give better black levels. It usually has a higher ratio,
but it should be noted that there is no real uniform standard for measuring contrast
ratio.
    \end{itemize}
    \item Which of the following tools can protect you in the case of a
surge?
    \begin{itemize}
        \item Answer: B
Explanation: Most antistatic straps come with a 1-megaohm resistor, which can protect
against surges. However, the best way to avoid a surge is to (1) make sure the
computer is unplugged before working on it, (2) not touch any components that hold
a charge, and (3) stay away from live electricity sources. In other words, this means:
don’t open power supplies or CRT monitors; don’t touch capacitors on any circuit
boards such as motherboards; and, of course, stay away from any other electrical
devices, wires, and circuits when working on computers
    \end{itemize}
    \item Which of the following connectors can have audio and video pass
through it?
    \begin{itemize}
        \item Answer: D
Explanation: High-Definition Multimedia Interface (HDMI), as the word multimedia
implies, can transmit video and audio signals.
Incorrect answers: VGA, RGB, and DVI are video
    \end{itemize}
    \item Which of the following devices limits network broadcasts, seg-
ments IP address ranges, and interconnects different physi-
cal media?
    \begin{itemize}
        \item Answer: D
Explanation: A router can limit network broadcasts through segmenting and programmed
routing of data. This is part of a router’s job when connecting two or more
networks. It is also used with different media. For example, you might have a LAN that
uses twisted-pair cable, but the router connects to the Internet via a fiber-optic connection.
That one router will have ports for both types of connections.
    \end{itemize}
    \item Which of the following technologies can be used to make wireless
payments?
    \begin{itemize}
        \item Answer: C
Explanation: Near-field communication (NFC) is a wireless technology that allows
mobile devices to transfer data simply by touching or being in close proximity to each
other, as well as make payments wirelessly to some point-of-sale (POS) systems.
    \end{itemize}
    \item You just upgraded the CPU. Which of the following issues can
make your computer shut down automatically after a few minutes?
(Select the best answer.)
    \begin{itemize}
        \item Answer: D
Explanation: The CPU could overheat if thermal compound has not been applied correctly
(which is common) or if it is not seated properly (which is rare).
    \end{itemize}
    \item Which of the following addresses is a valid IPv4 address for a net-
work host?
    \begin{itemize}
        \item Answer: C
Explanation: Of the answers listed, 172.17.58.254 is the only valid IPv4 address for a
network host. A host on the network is any computer or network device that uses an IP
address to communicate with other computers or devices. 172.17.58.254 is a class B
private IP address, so it fits the description of a valid IPv4 address for a network host.
    \end{itemize}
    \item You want to upgrade memory in your computer. Which of the fol-
lowing is user-replaceable memory in a PC?
    \begin{itemize}
        \item Answer: C
Explanation: Dynamic random-access memory (DRAM) is a module (or stick) of memory
that you can install into a motherboard. SDRAM, DDR, DDR2, DDR3, and DDR4
are all examples of DRAM.
    \end{itemize}
    \item Which of the following IP addresses is private?
    \begin{itemize}
        \item Answer: C
Explanation: 172.31.1.1 is the only address listed that is private. It is within the class B
range of private addresses: 172.16.0.0–172.31.255.255.
    \end{itemize}
    \item Which of the following is the local loopback IPv6 address?
    \begin{itemize}
        \item Answer: B
Explanation: The IPv6 loopback address used for testing is ::1. This determines if IPv6
is working correctly on the network card but does not generate network traffic. It exists
on every computer that runs IPv6.
    \end{itemize}
    \item Which of the following statements is correct concerning
IPv6 addresses?
    \begin{itemize}
        \item Answer: C
Explanation: The only statement that is correct concerning IPv6 is that it uses 128-bit
addressing. This is compared to IPv4, which uses 32-bit addresses.
    \end{itemize}
    \item Which of the following operating CPU temperatures is typical?
    \begin{itemize}
        \item Answer: B
Explanation: A typical operating temperature for CPUs is 60° Celsius. Keep in mind
that the operating range may be above or below that. Many computers hover around
30° to 35° Celsius.
    \end{itemize}
    \item Which of the following printer technologies should be used to print
payroll checks on paper forms that have a carbon backing?
    \begin{itemize}
        \item Answer: A
Explanation: The impact printer technology is what you want. This strikes the ribbon
and consequently the paper with a printhead. The physical hammering action causes
the carbon backing to apply text to the next layer of paper. Multipart forms such as
these are commonly used for receipts.
    \end{itemize}
    \item Which of the following is not a configuration that can be made in
the UEFI/BIOS?
    \begin{itemize}
        \item Answer: D
Explanation: You cannot install drivers to the UEFI/BIOS. Drivers are software that allows
the operating system to communicate with hardware; they can be configured in the
Device Manager in Windows.
    \end{itemize}
    \item Which of the following traits and port numbers are associated with
POP3? (Select the two best answers.)
    \begin{itemize}
        \item Answers: A and E
Explanation: POP3 is a protocol used by email clients to receive email. It makes use of
either port 110 (considered insecure) or port 995 (a default secure port).
    \end{itemize}
    \item Which type of cable should be used to connect a laptop directly to
a PC?
    \begin{itemize}
        \item Answer: D
Explanation: To connect one computer to another directly by way of network adapter
cards, use a crossover cable. (The category such as 5e or 6 doesn’t matter.) That cable
is designed to connect like devices. It is wired as T568B on one end and T568A on the
other. Those standards are ratified by the Telecommunications Industry Association/
Electronic Industries Alliance (TIA/EIA).
    \end{itemize}
    \item To perform a network installation of Windows, which of the following
must be supported by the computer’s network interface card?
    \begin{itemize}
        \item Answer: A
Explanation: Network installations require that the network card be configured for
Preboot eXecution Environment (PXE). This allows the network card to boot off the
network, locate a network installation server, and request that the installation begin.
This configuration might be done in the UEFI/BIOS of the computer (if the network
adapter is integrated to the motherboard), within a special program in Windows, or
in one that boots from disc or other removable media (if the network adapter is an
adapter card).
    \end{itemize}
    \item Which of the following devices is the least likely to be replaced on
a laptop?
    \begin{itemize}
        \item Answer: A
Explanation: The CPU is the least likely to be replaced. You would probably need to
replace other equipment, too, in this case. Just like PCs, though, the CPU should rarely
fail.
Incorrect
    \end{itemize}
    \item If your “bandwidth” is 1000 Mbps, how many bits are you sending/
receiving? (Select the two best answers.)
    \begin{itemize}
        \item Explanation: 1000 Mbps is 1000 megabits per second, otherwise notated as
1,000,000,000 bits per second, or 1 gigabit per second.
    \end{itemize}
    \item Which of the following can send data the farthest?
    \begin{itemize}
        \item Answer: B
Explanation: Single-mode fiber-optic cable can send data farther than any of the other
answers—up to hundreds of kilometers.
    \end{itemize}
    \item You need to expand the peripherals of a computer, but the system
doesn’t have enough ports. Which type of card should be
installed?
    \begin{itemize}
        \item Answer: C
Explanation: You should install a USB add-on card. This will give you more ports than
the computer already has for use with peripherals. Another option—and a more common
option at that—would be to purchase a USB hub.
    \end{itemize}
    \item Which of the following is the typical speed of an SATA hard disk
drive?
    \begin{itemize}
        \item Answer: D
Explanation: A typical speed of a magnetic hard disk drive is 7200 RPM—rotational
speed, that is. Other common rotational speeds include 5400 RPM, 10,000 RPM, and
15,000 RPM. Note: Solid-state hard drives do not have a magnetic disk and therefore
are not given an RPM rating or a latency rating.
    \end{itemize}
    \item Which of the following best describes the differences between a
switch and a router?
    \begin{itemize}
        \item A. A switch interconnects devices on the same network
so that they can communicate; a router interconnects
one or more networks.
    \end{itemize}
    \item A group of users in ABC Corp. needs to back up several gigabytes of
data daily. Which of the following is the best media for this scenario?
    \begin{itemize}
        \item Answer: D
Explanation: In a large corporation (or enterprise environment), tape backup such as
Linear Tape-Open (LTO) or Digital Linear Tape (DLT) is the best media for backing up.
    \end{itemize}
    \item  The organization you work for has three locations within a city that
need to be networked together. The network requirement for all three
locations is a minimum data throughput of 1 Gbps. Which of the
following network types are most likely to be used for internal office
and office-to-office communications? (Select the two best answers.)
    \begin{itemize}
        \item Answers: A and D
Explanation: The organization will most l kely use local-area networks (LAN) for each office
and a metropolitan-area network (MAN) to connect the three networks. The LANs meet
the requirements for each office’s internal communications. Most LANs operate at 1 Gbps
or faster. The MAN meets the requirement for the connection between the offices within
the city. A MAN is the right choice because it can harness the power of fiber-optic cables
and other technologies that already exist in the city limits and provide for 1 Gbps or more
throughput.
    \end{itemize}
    \item Which of the following is a type of virtualization software?
    \begin{itemize}
        \item Answer: D
Explanation: VMware is an example of virtualization software. Actually, VMware is the
company, and it makes a multitude of software including virtualization software such
as VMware Workstation and VMware ESXi.
    \end{itemize}
    \item You need to replace and upgrade the memory card in a
smartphone. Which type of memory is most likely used by the
smartphone?
    \begin{itemize}
        \item Answer: D
Explanation: Smartphones typically use Secure Digital (SD) cards—more to the point,
microSD cards.
    \end{itemize}
    \item Your organization relies heavily on its server farm for resources
and is less reliant on the client computers. Which type of client
computer is most likely used by the organization?
    \begin{itemize}
        \item Answer: C
Explanation: In this scenario, the organization probably has thin client computers
for its users. These computers have operating systems that are embedded in flash
memory, and the rest of the information they require comes from a server. Thin clients
normally have no hard drive; this is why they are referred to as diskless workstations.
    \end{itemize}
    \item Which of the following statements describe the respective functions
of the two corona wires in a laser printer? (Select the two
best answers.)
    \begin{itemize}
        \item Answers: A and B
Explanation: Know the main steps of the laser printer imaging process: processing,
charging, exposing, developing, transferring, fusing, and cleaning. In the charging step,
the drum is conditioned/charged by the primary corona wire (negatively charging it)
and is prepared for writing. In the transferring step, the paper is positively charged by
the transfer corona wire, preparing it to accept the toner from the drum.
    \end{itemize}
    \item Which of the following tools should be used to test a 24-pin ATX
12v power connector?
    \begin{itemize}
        \item Answer: B
Explanation: When testing the main 24-pin ATX power connector that leads from the
power supply to the motherboard, use a power supply unit tester (PSU tester) or a
multimeter. The multimeter can test each individual wire’s voltage, but the PSU tester
can test them all in one shot.
    \end{itemize}
    \item Which of the following uses port 427?
    \begin{itemize}
        \item Answer: C
Explanation: The Service Location Protocol (SLP) uses port 427. It enables access to
network services without previous configuration of the client computer.
    \end{itemize}
    \item Which of the following ports is used by AFP?
    \begin{itemize}
        \item Answer: F
Explanation: The Apple Filing Protocol (AFP) uses port 548. AFP offers file services for
Mac computers running macOS and can transfer files across the network.
    \end{itemize}
    \item Which of the following does a laptop have yet a tablet does not?
    \begin{itemize}
        \item Answer: A
Explanation: Tablets do not have a touchpad; instead, you use your finger(s) or a stylus
to tap on the display (known as a touchscreen).
    \end{itemize}
    \item At the beginning of the workday, a user informs you that her computer
is not working. When you examine the computer, you notice
that nothing is on the display. Which of the following should be
done first?
    \begin{itemize}
        \item Answer: B
Explanation: When troubleshooting a computer system, always look for the most likely
and simplest solutions first. The fact that the user might not have turned on her monitor
when she first came in is a likely scenario.
    \end{itemize}
    \item A customer reports that when his computer is turned on, the
screen is blank except for some text and a flashing cursor. He also
tells you that there are numbers counting upward when the computer
beeps and then freezes. Which of the following is the most
likely cause of this problem?
    \begin{itemize}
        \item Answer: A
Explanation: Chances are that the computer has faulty memory or a memory module
that needs to be reseated properly. The flashing cursor on the screen tells you that the
system is not posting properly. The numbers counting up are the system checking the
RAM. If the system beeps and freezes during this count-up, the RAM has an issue. It
could also be incompatible with the motherboard.
    \end{itemize}
    \item Joey’s computer was working fine for weeks, yet suddenly it cannot
connect to the Internet. Joey runs the command ipconfig and
sees that the IP address his computer is using is 169.254.50.68.
Which of the following statements describes the most likely issue?
    \begin{itemize}
        \item Answer: A
Explanation: If you get any address that starts with 169.254, the computer has self assigned
that address. It is known as an APIPA (Automatic Private IP Addressing)
address, a type of link-local address. Normally, DHCP servers do not use this network
number. A simple ipconfig/release and ipconfig/renew might fix the problem, if a
DHCP server is actually available
    \end{itemize}
    \item Which of the following could cause a ghosted image on the paper
outputted by a laser printer?
    \begin{itemize}
        \item Answer: D
Explanation: Ghosted images or blurry marks could be a sign that the drum has some
kind of imperfection or is dirty, especially if the image reappears at equal intervals.
Replace the drum (or toner cartridge). Another possibility is that the fuser assembly
has been damaged and needs to be replaced.
    \end{itemize}
    \item Mary installed a new sound card and speakers; however, she cannot
get any sound from the speakers. Which of the following statements
describe the most likely cause? (Select all that apply.)
    \begin{itemize}
        \item  
    \end{itemize}
    \item A laptop with an integrated 802.11 WLAN card is unable to connect
to any wireless networks. Just yesterday the laptop was able
to connect to wireless networks. Which of the following statements
describes the most likely cause?
    \begin{itemize}
        \item C. The wireless card firmware requires an update.
    \end{itemize}
    \item The IP address of dprocomputer.com is 216.97.236.245. You can
ping that IP address, but you cannot ping dprocomputer.com.
Which of the following statements describes the most likely cause?
    \begin{itemize}
        \item Answer: C
Explanation: The purpose of a DNS server is to resolve (convert) hostnames and
domain names to the IP address. Computers normally communicate via IP address,
but it is easier for humans to type in names. If dprocomputer.com is down, you cannot
ping the corresponding IP address at all.
    \end{itemize}
    \item A newly built computer runs through the POST, but it doesn’t recognize
the specific CPU that was just installed. Instead, it recognizes
it as a generic CPU. Which of the following is the first thing
you should check?
    \begin{itemize}
        \item Answer: B
Explanation: You must have the correct firmware to recognize the latest CPUs. A BIOS
flash can fix many problems related to unidentified hardware.
    \end{itemize}
    \item Which of the following commands displays a network interface
card’s MAC address?
    \begin{itemize}
        \item Answer: B
Explanation: Ipconfig/all shows a lot of information, including the MAC address, as
well as the DNS server, DHCP information, and more.
    \end{itemize}
    \item A customer reports that print jobs sent to a local printer are printing
as blank pieces of paper. Which of the following can help you
determine the cause?
    \begin{itemize}
        \item Answer: D
Explanation: First, try printing an internal test page, meaning from the printer’s
onscreen display. If that doesn’t work, you need to start troubleshooting the printer.
Perhaps the toner cartridge is empty, or maybe a corona wire is malfunctioning.
    \end{itemize}
    \item Signal strength for a laptop’s wireless connection is low (yellow in
color and only one bar). The laptop is on the first floor of a house.
The wireless access point (WAP) is in the basement. Which of
the following can improve signal strength? (Select the two best
answers.)
    \begin{itemize}
        \item Answers: A and B
Explanation: The easiest and (probably) cheapest way is to move the WAP. Basements
are usually the worst place for an access point because of concrete foundations and
walls, electrical interference, and so on. Signal boosters might also work, but often the
cost of a signal booster is the same as buying a newer, more powerful WAP.
    \end{itemize}
    \item You are troubleshooting a computer that can’t communicate with
other computers on the network. An ipconfig shows an APIPA IP
address (169.254.21.184). What is the most likely problem?
    \begin{itemize}
        \item Answer: C
Explanation: The most likely issue is a DHCP failure. Link-local IP addresses such as
APIPA on 169.254.0.0 usually kick in when the client computer cannot locate a DHCP
server. There are several possible reasons why this might happen: lack of network connectivity,
incorrect client configuration, DHCP server failure, and so on. The best way to
troubleshoot this situation would be to start at the client and check its network cable (or
wireless connection) and then its TCP/IP configuration. You could also try configuring a
static IP address on the client to see if it allows the client to communicate over the network.
If those things don’t solve the problem or give you any clues, you can look at the
network switch, DHCP server, and so on.
    \end{itemize}
    \item Which of the following is the first thing you should check when a
computer cannot get on the Internet?
    \begin{itemize}
        \item Answer: C
Explanation: The simplest solution is often the most common. Check cables and see
whether the power is on for your devices and computers.
    \end{itemize}
    \item You just built a PC, and when it first boots, you hear some beep
codes. If you don’t have the codes memorized, which of the following
are the best devices to examine first? (Select the two best
answers.)
    \begin{itemize}
        \item Answers: A and C
Explanation: It is common to have an unseated RAM stick or video card. These are the
most common culprits of beep codes during the POST.
    \end{itemize}
    \item Which of the following are examples of virtual printing? (Select the
two best answers.)
    \begin{itemize}
        \item Answers: B and D
Explanation: Two examples of virtual printing include XPS printing and print to file
(one example of which saves the file as a .prn). Other examples include printing to PDF
and printing to image.
    \end{itemize}
    \item A coworker needs to print to a printer from a laptop running
Windows. The printer has a USB and an Ethernet connector. Which
of the following is the easiest way to connect the printer to the
laptop?
    \begin{itemize}
        \item Answer: C
Explanation: Use the USB connector. By far, this is the easiest method. Windows will
sense the USB connection and attempt to install the print driver automatically (though
you should still install the latest proper driver from the printer manufacturer’s website).
    \end{itemize}
    \item You have had several support requests for a PC located in a school
cafeteria kitchen that is experiencing a problem. You have already
reseated the PCIe and PCI cards and replaced the hard drive in the
PC. Computers located in the business office or the classrooms
have not had the same problem as the computer in the cafeteria.
Which of the following is the most likely issue?
    \begin{itemize}
        \item Answer: A
Explanation: Excessive heat is the most likely cause of the problem. This could be an
unfortunate result of ovens and other equipment. Computers in environments such as
these are often prone to dirt collecting inside the CPU fans and other devices inside the
case.
    \end{itemize}
    \item A user complains that his network interface card (NIC) is not
functioning and has no link lights. The weather has been changing
drastically over the past few days, and humidity and temperature
have been rising and falling every day. Which of the following
could be the direct cause of this problem? (Select the best
answer.)
    \begin{itemize}
        \item Answer: A
Explanation: Thermal expansion and contraction happen when humidity changes
quickly. This can lead to what some technicians refer to as “chip creep” or “card
creep.”
    \end{itemize}
    \item Which network type enables high-speed data communication and
is the most difficult to eavesdrop on?
    \begin{itemize}
        \item Answer: C
Explanation: Fiber-optic networks use fiber-optic cables that have a core of plastic or
glass fibers. They are much more difficult to eavesdrop on than any copper cable.
    \end{itemize}
    \item Which of the following properties of a heat sink has the greatest
effect on heat dissipation?
    \begin{itemize}
        \item Answer: C
Explanation: The surface area of the heat sink has the greatest effect on heat dispersion.
The more solid the bond between the heat sink and CPU cap, the better the
transition of heat out of the CPU. To aid in this transition, you must use thermal
compound.
    \end{itemize}
    \item You need to set up a server system that will run in a VM. It will
have the bulk of the network computers’ resources and will supply
much of the resources necessary to the client computers that will
connect to it. You are also required to set up the client computers.
Which two types of systems (server and client) will you be implementing?
    \begin{itemize}
        \item Answer: B
Explanation: You will be implementing a virtualization workstation and thin clients.
The virtualization workstation will run virtual software that will allow you to install the
server software to a virtual machine (VM). This server provides most of the resources
for the clients on the network—the thin clients. Thin clients normally have very limited
resources of their own and rely on the server (be it a regular or virtual server) for the
additional resources they need.
    \end{itemize}
    \item Your boss can receive email but can’t seem to send email with the
installed email client software. Which protocol is not configured
properly?
    \begin{itemize}
        \item Answer: A
Explanation: The Simple Mail Transfer Protocol (SMTP) is not configured properly.
That is the protocol used to send mail.
    \end{itemize}
    \item A customer’s laptop LCD needs replacement. Which of the following
tools should be used to remove the display bezel?
    \begin{itemize}
        \item Answer: D
Explanation: Use a plastic shim to open the display or remove the bezel that surrounds
it.
    \end{itemize}
    \item One of your customers is running Windows on a PC that has a
3.0-GHz CPU, 16-GB RAM, and an integrated video card. The customer
tells you that performance is slow when editing video files.
Which of the following solutions increases performance on the
computer?
    \begin{itemize}
        \item Answer: B
Explanation: The only solution listed is to upgrade the video card. This is the only way
that computer performance can be increased while editing video files.
    \end{itemize}
    \item Which of the following is configured in the UEFI/BIOS? (Select the
four best answers.)
    \begin{itemize}
        \item Answers: A, C, D, and F
Explanation: The time/date, boot priority (boot sequence), passwords, and
Wake-on-LAN (WOL) can all be configured in the UEFI/BIOS.
    \end{itemize}
    \item You want to test whether IPv4 and IPv6 are working properly on a
computer. Which of the following commands should be issued?
    \begin{itemize}
        \item Answer: C
Explanation: To test IPv4, use the command ping 127.0.0.1. To test IPv6, use the
command ping ::1.
    \end{itemize}
    \item Which of the following measurements is the typical latency of an
SATA hard drive?
    \begin{itemize}
        \item Answer: D
Explanation: The typical latency of an SATA magnetic-based hard disk drive is 4.2 ms
(milliseconds). When you are dealing with magnetic drives, latency is the delay in time
before a particular sector on the platter can be read. It is directly linked to rotational
speed. A hard drive with a rotational speed of 7200 RPM has an average latency of 4.2
    \end{itemize}
    \item The marketing printer has been used for four years. Which of the
following statements represents a best practice for ensuring the
printer remains in good working order?
    \begin{itemize}
        \item Answer: B
Explanation: A maintenance kit includes a new fuser assembly, rollers, and more.
Installing a maintenance kit is like changing a car’s oil (although it isn’t done as often).
    \end{itemize}
    \item A customer wants users to be able to store additional files in a
cloud-based manner when necessary and remove them when they
are not needed anymore. What is this known as?
    \begin{itemize}
        \item Answer: D
Explanation: On-demand means that a customer can get access to cloud-based services
24 hours a day, 7 days a week and that the customer can get additional space if
needed on a temporary basis. This links with scalability to a certain extent
    \end{itemize}
    \item You are troubleshooting a computer that is no longer able to
browse the Internet. The user says that the computer worked
before he went on vacation. However, now the user is able to
navigate the local intranet but cannot connect to any outside sites.
You ping a well-known website on the Internet by name, but you
receive no replies. Then you ping the website’s IP address, and you
do receive a reply. Which of the following commands will fix the
problem?
    \begin{itemize}
        \item Answer: A
Explanation: You should run ipconfig /flushdns in an attempt to fix the problem. The
problem could be that the DNS cache on the client computer is outdated and cannot
associate Internet domain names with their corresponding IP addresses. So you run
ipconfig /flushdns and potentially ipconfig /registerdns (or computer restart) to fix
the problem. That clears the DNS cache of name resolutions and forces the client to
request new DNS information from a DNS server either on the Internet or within the
company
    \end{itemize}
    \item Answer: B
Explanation: The thermal design power (TDP) of a CPU is measured in watts. For
example, a typical Core i7 CPU might be rated at 140 watts or less. The less the watt-
age rating, the less the computer’s cooling system needs to dissipate heat generated
by the CPU.
    \begin{itemize}
        \item B. Watts
    \end{itemize}
\end{enumerate}
\begin{enumerate}
    \item A - Application 
    \item Penguin - Presentation 
    \item Said - Session 
     \item That - Transfer
    \item Nobody - Network 
     \item Drinks - Data Link
    \item Pepsi  - Physical 
\end{enumerate}
\section{----------}
\section{Core 2 Troy McMillan} 
\subsection{Operating Systems} 
\begin{enumerate}
    \item Which of the following is an interface that offers a glass design that includes translucent
windows?
    \begin{itemize}
        \item B. The Aero interface offers a glass design that includes translucent windows. It was new
with Windows Vista. The Sidebar is an area of the desktop where gadgets can be placed.
Metro is the name of an interface type, and the Start screen is the new opening interface to
Windows.
    \end{itemize}
    \item Which of the following are mini programs, introduced with Windows Vista, that can be
placed on the desktop (Windows 7) or on the Sidebar (Windows Vista)?
    \begin{itemize}
        \item A. Windows 7 renamed these Windows Desktop Gadgets. Metro apps are a type of
application new in Windows, and widgets are an element of a graphical user interface (GUI)
that displays information such as temperature. Shims are pieces of software that allow an
application to be supported by a system that normally does not support it.
    \end{itemize}
    \item Into which tool has the Security Center been rolled in Windows 7?
    \begin{itemize}
        \item A. Rolled into the Action Center in Windows 7, this interface shows the status of, and
allows you to configure, the firewall, Windows Update, virus protection, spyware and
unwanted software protection, Internet security settings, UAC, and network access
protection. Control Panel is a holding spot for many tools used in Windows. The Windows
Firewall is a built in firewall for Windows systems. Defender is the Windows anti-malware
solution.
    \end{itemize}
    \item What is the name of the user interface in Windows 8 and 8.1?
    \begin{itemize}
        \item B. In both Windows and Windows 8.1, the user interface is very different from earlier
versions of Windows. The Start menu was removed, and the desktop replaced with a
new look called Metro. The Aero interface offers a glass design that includes translucent
windows. It was new with Windows Vista. Sidebar is an area of the desktop where gadgets
can be placed. Start or the Start screen is a new opening interface to Windows.
    \end{itemize}
    \item What is the minimum RAM required for 64-bit Windows 7?
    \begin{itemize}
        \item D. Windows 7 requires 1 GB for 32-bit and 2 GB for 64-bit Windows 7 Professional.
    \end{itemize}
    \item Which version of Windows 7 can be upgraded to Windows 8?
    \begin{itemize}
        \item D. Only Windows Home Starter can be upgraded to Windows 8.
    \end{itemize}
    \item Which Windows command is used to view a listing of the files and folders that exist within
a directory, subdirectory, or folder?
    \begin{itemize}
        \item B. The DIR command is simply used to view a listing of the files and folders that exist
within a directory, subdirectory, or folder. The net use command allows for connecting
to, removing, and configuring connections to shared resources. The cd command is used
to change directories, and the ipconfig command is used to view and change network
settings.
    \end{itemize}
    \item Which Windows command is used to move to another folder or directory?
    \begin{itemize}
        \item C. The change directory (cd) command is used to move to another folder or directory.
It is used in both Unix and Windows. The dir command is simply used to view a listing
of the files and folders that exist within a directory, subdirectory, or folder. The net use
command allows for connecting to, removing, and configuring connections to shared
resources. The ipconfig command is used to view and change network settings.
    \end{itemize}
    \item Which Windows tool shows a list of all installed hardware and lets you add items, remove
items, update drivers, and more?
    \begin{itemize}
        \item A. Device Manager shows a list of all installed hardware and lets you add items, remove
items, update drivers, and more. Event Viewer allows for viewing various event logs. The
Users And Groups tool is used to create and manage user and group accounts. The Sync
Center is used to manage the synchronization of data between the local device and other
location
    \end{itemize}
    \item Which Windows tool tracks all events on a particular Windows computer?
    \begin{itemize}
        \item  B. Every program and process theoretically could have its own logging utility, but
Microsoft has come up with a rather slick utility, Event Viewer, which, through log files,
tracks all events on a particular Windows computer. Device Manager shows a list of all
installed hardware and lets you add items, remove items, update drivers, and more. The
Users And Groups tool is used to create and manage user and group accounts. The Sync
Center is used to manage the synchronization of data between the local device and other
locations.
    \end{itemize}
    \item Which of the following is either the number of bits used to indicate the color of a single
pixel, in a bitmapped image or video frame buffer, or the number of bits used for each color
component of a single pixel?
    \begin{itemize}
        \item C. Color depth is either the number of bits used to indicate the color of a single pixel,
in a bitmapped image or video frame buffer, or the number of bits used for each color
component of a single pixel. In Windows 7, this can be set on the Monitor tab of the
properties of the adapter. Resolution has to do with the number of pixels (individual points
of color) contained on a display monitor. Refresh rate describes how often the device
“repaints” the display image. Pixel count is the same thing as resolution.
    \end{itemize}
    \item Which of the following is the number of times in a second that a display updates its buffer
and is expressed in hertz?
    \begin{itemize}
        \item B. The refresh rate is the number of times in a second that a display updates its buffer and
is expressed in Hertz. In Windows 7, the refresh rate is set using a drop-down box just
above the setting for color depth. Color depth is either the number of bits used to indicate
the color of a single pixel, in a bitmapped image or video frame buffer, or the number of
bits used for each color component of a single pixel. In Windows 7, this can be set on the
Monitor tab of the properties of the adapter. Resolution has to do with the number of pixels
(individual points of color) contained on a display monitor. Pixel count is the same thing as
resolution
    \end{itemize}
    \item Which of the following should you exceed for good performance?
    \begin{itemize}
        \item A. Minimum RAMA. The minimum of RAM required should be viewed as just that, a minimum. Make
sure you have more than required for satisfactory performance. You should not exceed
the maximum resolution recommended for a device. Extra disk space will nor improve
performance unless the disk is very full. As pixel count is the same as resolution you should
not exceed the recommended count either.
    \end{itemize}
    \item Which type of installation is most likely to take place in a SOHO?
    \begin{itemize}
        \item C. Outside of the enterprise most installations occur by using the CD that came with the
software or by placing these same files on a USB stick and accessing them from the USB
drive. A network installation is when you load installation files over the network. A Remote
Installation Service installation is a legacy method no longer supported. An unattended
installation is one that uses an answer file to answer the prompts that appear during
installation.
    \end{itemize}
    \item Which of the following offers a simplified way to set up a home network?
    \begin{itemize}
        \item C. HomeGroup offers a simplified way to set up a home network. It allows you to share files
(including libraries) and prevent changes from being made to those files by those sharing
them (unless you give them permission to do so). A peer-to-peer network is the type of
network created with a workgroup. A domain is a network with a directory service such as
Active Directory. A workgroup is a small collection of computers grouped without a server.
    \end{itemize}
    \item Which of the following can be used to connect to a shared printer
    \begin{itemize}
        \item  A. net use can also be used to connect to a shared printer: net use lpt1: \\printername.
The net user command is used to add, remove, and make changes to the user accounts
on a computer. The robocopy command (Robust File Copy for Windows) is included with
Windows 7, Windows 8, Windows 8.1, and Windows 10 and has the big advantage of
being able to accept a plethora of specifications and keep NTFS permissions intact in its
operations. If you are comfortable with the copy command, learning xcopy shouldn’t pose
too many problems. It’s basically an extension of copy with one notable exception—it’s
designed to copy directories as well as files.
    \end{itemize}
    \item Which of the following is a command-line interface in Linux?
    \begin{itemize}
        \item A. In Linux, a shell is a command-line interface, of which there are several types. A
domain is a network type that includes a directory service. cmd is a command that when
executed in the Run box opens the command prompt. A disk operating system (abbreviated
DOS) is a computer operating system that can use a disk storage device, such as a floppy
disk, hard disk drive, or optical disc.
    \end{itemize}
    \item Which of the following provides a quick way to see everything that’s currently open on
your Mac?
    \begin{itemize}
        \item B. In Apple, Mission Control provides a quick way to see everything that’s currently open
on your Mac. A shell is simply a command-line interface. A sandbox is an area where a
process can be executed without impacting any other processes. Beeker is not a term used
when discussing Apple.
    \end{itemize}
\end{enumerate}
\subsection{Security}
\begin{enumerate}
    \item Which of the following is a series of two doors with a small room between them?
    \begin{itemize}
        \item A. A mantrap is a series of two doors with a small room between them. The user is
authenticated at the first door and then allowed into the room. At that point, additional
verification will occur (such as a guard visually identifying the person), and then the
person is allowed through the second door. A trapdoor is a doorway that is usually hidden.
A saferoom is a room that is impenetrable from outside, and badgetrap is not a term used
when disusing doorway systems.
    \end{itemize}
    \item Which of the following physical characteristics is used to identify the user?
    \begin{itemize}
        \item B. Biometric devices use physical characteristics to identify the user. Such devices are
becoming more common in the business environment. Biometric systems include hand
scanners, retinal scanners, and, possibly soon, DNA scanners. Hardware tokens are devices
that contain security credentials. Smart cards are cards that contain a chip and credentials.
Badge readers are devices that read the information on a card and allow or disallow entry.
    \end{itemize}
    \item In which filtering is the physical address used?
    \begin{itemize}
        \item A. As physical addresses are MAC addresses, MAC address filtering is the correct answer.
Email filtering is the filtering of email addresses from which one is allowed to receive. IP
address filtering is the type of filtering done on a router or firewall, based on IP addresses.
URL filtering restricts the URLs that can be reached with the browser.
    \end{itemize}
    \item What firewall only passes or blocks traffic to specific addresses based on the type of
application?
    \begin{itemize}
        \item A. A firewall operating as a packet filter passes or blocks traffic to specific addresses based
on the type of application. The packet filter doesn’t analyze the data of a packet; it decides
whether to pass it based on the packet’s addressing information. A proxy firewall is one
that makes the Internet connection on behalf of the user and can control where the user
goes. Stateful firewalls monitor the state of every TCP connection, thus preventing network
mapping. A new-generation firewall is one that operates on all levels of the OSI model.
    \end{itemize}
    \item Which of the following was created as a first stab at security for wireless devices?
    \begin{itemize}
        \item D. Wired Equivalent Privacy (WEP) is a standard that was created as a first stab at security
for wireless devices. Using WEP-encrypted data to provide data security has always been
under scrutiny for not being as secure as initially intended. Wi-Fi Protected Access (WPA)
and WPA2 are later methods that cane after WEP. Temporal Key Integrity Protocol is the
encryption method used in WPA.
    \end{itemize}
    \item Which of the following was used to increase security in WPA?
    \begin{itemize}
        \item A. WPA was able to increase security by using a Temporal Key Integrity Protocol (TKIP) to
scramble encryption keys using a hashing algorithm. Temporal Key Integrity Protocol is the
encryption method used in WPA. Advanced Encryption Standard (AES) is the encryption
used in WPA2. IPSec is an industry-standard encryption method, and Secure Sockets Layer
(SSL) is an encryption method used in many VPNs.
    \end{itemize}
    \item Which type of virus covers itself with protective code that stops debuggers or disassemblers
from examining critical elements of the virus?
    \begin{itemize}
        \item  C. An armored virus is designed to make itself difficult to detect or analyze. Armored
viruses cover themselves with protective code that stops debuggers or disassemblers from
examining critical elements of the virus. A companion virus is one that attaches to a file
or adopts the name of a file. A macro virus is one that hides in macros, and a multipartite
virus is one that has multiple propagation methods.
    \end{itemize}
    \item What element of a virus uniquely identifies it?
    \begin{itemize}
        \item B. A signature is an algorithm or other element of a virus that uniquely identifies it.
Because some viruses have the ability to alter their signature, it is crucial that you keep
signature files current, whether you choose to manually download them or configure the
antivirus engine to do so automatically. An ID is any types of identifying badge or marker.
A badge is something worn to provide identification. Marking is not a word typically used
when discussing algorithms or attacks.
    \end{itemize}
    \item Which of the following is the term used for someone being so close to you when you enter a
building that they are able to come in right behind you without needing to use a key, a card,
or any other security device?
    \begin{itemize}
        \item C. Tailgating is the term used for someone being so close to you when you enter a building
that they are able to come in right behind you without needing to use a key, a card, or any
other security device. Many social-engineering intruders needing physical access to a site
will use this method of gaining entry. Shadowing is when one user monitors another for
training. Spoofing is the adoption of another’s email address, IP address, or MAC address.
Keyriding is not a word typically used when discussing social engineering.
    \end{itemize}
    \item Which of the following is the process of masquerading as another user or device?
    \begin{itemize}
        \item B. Spoofing is the process of masquerading as another user or device. It is usually done for
the purpose of accessing a resource to which the hacker should not have access or to get
through a security device such as a firewall that may be filtering traffic based on source
IP address. Shadowing is when one user monitors another for training. Duplication is the
creation of a matching object. Masking is not a term used when discussing impersonation.
    \end{itemize}
    \item Which Windows group allows members to install most software but keeps them from
changing key operating system files?
    \begin{itemize}
        \item A. The Power Users group is not as powerful as the Administrators group. Membership
in this group gives read/write permission to the system, allowing members to install most
software but keeping them from changing key operating system files. This is a good group
for those who need to test software (such as programmers) and junior administrators. The
Guest group is used to allow restricted access to the device. The Administrators group
allows full access to the device. The rights held by the Users group are a compromise
between Admin and Guest.
    \end{itemize}
    \item Which NTFS permission is the least required to run a program?
    \begin{itemize}
        \item D. This combines the permissions of Read with those of List Folder Contents and adds the
ability to run executables. List Folder Contents allows viewing what items are in a folder.
Full Control allows everything, and Read only allows reading documents.D. Write
    \end{itemize}
    \item Which of the following passwords is the strongest?
    \begin{itemize}
        \item D. Although length is now considered the most important password security factor,
complexity is also a factor, and these examples are all the same length. The password
P@ssw0rd contains four character types, the most of any of the options, which increases
the strength of the password. Password and pAssword contains only two types of
characters. Pa\$\$word contains three types.
    \end{itemize}
    \item What principle should drive the granting of permissions?
    \begin{itemize}
        \item B. When assigning user permissions, follow the principle of least privilege by giving users only
the bare minimum they need to do their job. Separation of duties prescribes that any operation
prone to fraud should be broken up into two operations with different users performing each.
Job rotation has the same goal but accomplishes it by requiring users to move around from job
to job. Open rights is not a term used when discussing permission and rights.
    \end{itemize}
    \item Which type of screen lock uses gestures?
    \begin{itemize}
        \item C. Swipe locks use a gesture or series of gestures, sometimes involving the movement
of an icon to open the screen. In some cases, they require only knowledge of the mobile
platform in use; they offer no security to the process because no authentication of the user
is occurring. Fingerprint locks open when the correct fingerprint is presented. Facial locks
require a matching face scan to open. Passcode locks require the configured passcode to
unlock.
    \end{itemize}
    \item Which method is good for a lost mobile device?
    \begin{itemize}
        \item A. Remote wipe gives you the ability to delete all content when a device is stolen or lost.
Geofencing allows you to restrict use of the device to a geographic area. Screen locks
prevent access to the home screen on the device. Segmentation of data is the separation of
personal data from enterprise data on a device.
    \end{itemize}
    \item Which of the following involves applying a strong magnetic field to initialize the media?
    \begin{itemize}
        \item A. Degaussing involves applying a strong magnetic field to initialize the media (this is also
referred to as disk wiping). This process helps ensure that information doesn’t fall into the
wrong hands. Incineration is the burning of the storage device. Hammers can be used to
destroy the device. Deleting is the least effective way of removing information.
    \end{itemize}
    \item Which method of destroying the data on a hard drive is most effective?
    \begin{itemize}
        \item B. Physically destroying the drive involves rendering the component no longer usable.
Incineration is the burning of the storage device. Degaussing involves applying a strong
magnetic field to initialize the media (this is also referred to as disk wiping). This process
helps ensure that information doesn’t fall into the wrong hands. Clearing is a method that
still leaves the data recoverable with data forensics. Deleting is the least effective way of
removing information.
    \end{itemize}
    \item Which of the following was a concept that was designed to make it easier for less knowl-
edgeable users to add a new client to the WLAN without manually entering the security
information on the client?
    \begin{itemize}
        \item B. Wi-Fi protected setup (WPS) was a concept that was designed to make it easier for
less knowledgeable users to add a new client to the WLAN without manually entering
the security information on the client. One method involves pushing a button on the AP
at the same time a client is attempting to join the network so that the settings are sent
to the client. Other methods involve placing the client close to the AP, and near-field
communication is used for the process. Service Set identifier (SSID) is the name of the
WLAN. Wired Equivalent Privacy ( WEP) and Wi-Fi protected Access (WPA) are wireless
security protocols.
    \end{itemize}
    \item Which of the following should always be changed from the default?
    \begin{itemize}
        \item A. Every wireless AP or wireless router on the market comes with a default SSID. Cisco
models use the name tsunami, for example. You should change these defaults and create a
new SSID to represent your WLAN. Wi-Fi protected setup (WPS) was a concept that was
designed to make it easier for less knowledgeable users to add a new client to the WLAN
without manually entering the security information on the client. One method involves
pushing a button on the AP at the same time a client is attempting to join the network so
that the settings are sent to the client. Other methods involve placing the client close to the
AP, and near-field communication is used for the process. Wired Equivalent Privacy (WEP)
and Wi-Fi protected Access (WPA) are wireless security protocols.
    \end{itemize}
\end{enumerate}
\subsection{Software Troubleshooting} 
\begin{enumerate}
    \item Which type of file cannot be copied from another machine if missing or corrupted?
    \begin{itemize}
        \item B. The boot.ini file is specific to the machine. A .dll file is a file type and not a specific
file. The ntldr is the file that loads the operating system. Bootmgr is a file in later systems
that manages the boot process.
    \end{itemize}
    \item Which of the following is not a possible cause of a “no operating system found” message?
    \begin{itemize}
        \item D. If there was a disk with system files in the DVD drive, the system would boot to it.
A nonsystem disk, a corrupted or missing boot sector, and an incorrect boot order in the
BIOS could all be possible causes.
    \end{itemize}
    \item What is external code that changes your Internet Explorer settings?
    \begin{itemize}
        \item B. A browser redirection is one of the most serious security problems. Browser hijacking
software is external code that changes your Internet Explorer settings. It may include
changing your home page or adding or removing items from your favorites. A man-
in-the-middle attack is when the malicious individual positions himself between two
communicating system, receiving all data. A SYN flood is a form of a DoS attack. Fraggle is
an attack using UDP packets.
    \end{itemize}
    \item Which of the following is not a symptom of malware?
    \begin{itemize}
        \item A. Malware decreases performance. It can cause Internet connectivity issues, browser
redirection, and pop-ups.
    \end{itemize}
    \item Which of the following is the first step in malware removal?
    \begin{itemize}
        \item D. The steps are as follows:
1. Identify and research malware symptoms.
2. Quarantine the infected systems.
3. Disable System Restore (in Windows).
4. Remediate the infected systems.
5. Schedule scans and run updates.
6. Enable System Restore and create a restore point (in Windows).
7. Educate the end user.
    \end{itemize}
    \item What Windows service should be disabled before cleaning an infection?
    \begin{itemize}
        \item B. Although it is recommended that you disable System Restore before cleaning an
infection, it is a good idea to create a restore point after an infection is cleaned. This gives
you a clean restore point going forward in case the system becomes infected again at some
point. Network address translation, the Windows Firewall, and your antivirus should not
be disabled.
    \end{itemize}
    \item Which of the following does not negatively impact mobile battery life?
    \begin{itemize}
        \item A. A low brightness setting does not negatively impact battery life. A high setting, however,
does. Location services, Bluetooth, and overheating do not negatively affect battery life.
    \end{itemize}
    \item Which of the following does not cause overheating of a mobile device?
    \begin{itemize}
        \item B. While leaving the phone on will run down the battery, it will not alone cause it to
overheat. Excessive gaming, using an old battery, and continuous online browsing will
cause overheating.
    \end{itemize}
    \item Which of the following is not an indication of a security issue with a mobile device?
    \begin{itemize}
        \item D. On the contrary, evidence of malware or other issues is usually accompanied by very
high resource utilization. Unusual loss of power, slow speeds, and a weak signal are all
signs of security issues.
    \end{itemize}
    \item Which of the following is an indication of a security issue with a mobile device?
    \begin{itemize}
        \item C. When cameras have been enabled when they weren’t previously, it is an indication of
compromise. Low resource utilization, a disabled microphone, and authorized use of the
device are not symptoms of a security issue.
    \end{itemize}
\end{enumerate}
\subsection{Operational Procedures} 
\begin{enumerate}
    \item Which of the following is the least important piece of information to record about each
device for proper asset inventory?
    \begin{itemize}
        \item D. While the OS may be important, for warranty issues these other pieces are more
important. The make, model, and serial number are all important.
    \end{itemize}
    \item Which of the following is false with respect to change management?
    \begin{itemize}
        \item C. All costs and effects of the methods of implementation should be reviewed prior to
formal approval. The other statements are true.
    \end{itemize}
    \item What is the process called that ensures all configuration changes are beneficial?
    \begin{itemize}
        \item A. During the change management process, the relative costs and benefits to the overall
organization will be weighed by a change management board or team. Acceptable use is
a policy that defines what users can and cannot do. Separation of duties is a concept that
says that any operation prone to fraud should be broken into two jobs and assigned to two
people. Risk analysis is a process that identifies risk and mitigations.
    \end{itemize}
    \item Which of the following ensures an escape from changes that break something?
    \begin{itemize}
        \item A. During implementation, incremental testing should occur, relying on a predetermined
fallback strategy if necessary. A phased deployment is one in which parts of the network are
done at a time. While the communication process is important, it is not what allows for an
escape. Request control is a process where change requests are managed and approved.
    \end{itemize}
    \item If you use incremental backups every day except Monday, when you do a full backup,
how many backup tapes will be required if there is a drive failure on Wednesday after the
backup has been made?
    \begin{itemize}
        \item B. Since an incremental backup backs up everything that has changed since the last backup
of any type, each day’s tape is unique, so you will need the Monday full backup and the
incremental tapes from Tuesday and Wednesday.
    \end{itemize}
    \item If you use differential backups every day except Monday when you do a full backup, how
many backup tapes will be required if there is a drive failure on Wednesday after the
backup has been made?
    \begin{itemize}
        \item C. Since a differential backup backs up everything that has changed since the last full
backup, each day’s incremental tape contains what was on the previous day’s tape. So, you
only need the last differential and the last full backups.
    \end{itemize}
    \item Which of the following is not a safe lifting technique to keep in mind?
    \begin{itemize}
        \item A. Lift with your legs, not your back. When you have to pick something up, bend at the
knees, not at the waist. The other options are all safety recommendation.
    \end{itemize}
    \item What class of fire extinguisher is used for paper fires?
    \begin{itemize}
        \item A. A is for wood and paper fires, B is for flammable liquids, C is for electrical fires, and D
is for metal fires.
    \end{itemize}
    \item Any type of chemical, equipment, or supply that has the potential to harm the environment
or people has to have what document associated with it?
    \begin{itemize}
        \item B. Any type of chemical, equipment, or supply that has the potential to harm the
environment or people has to have a material safety data sheet (MSDS) associated with
it. These are traditionally created by the manufacturer, and you can obtain them from
the manufacturer or from the Environmental Protection Agency. A statement of work
(SOW) is a document that indicates the work to be performed. A service level agreement
is a document that indicates what is being paid and what the service consists of. A
memorandum of understanding (MOU) is a document that indicates the intent of two
parties to do something together.
    \end{itemize}
    \item What humidity level should be maintained for computing equipment?
    \begin{itemize}
        \item A. Another preventive measure you can take is to maintain the relative humidity at around
50 percent. Be careful not to increase the humidity too far—to the point where moisture
starts to condense on the equipment!
    \end{itemize}
    \item Which of the following is the not part of the first response to an incident?
    \begin{itemize}
        \item A. You never shut down the system until all volatile evidence has been collected. The other
options are correct guidelines.
    \end{itemize}
    \item Which of the following applies to EU-based organizations that collect or process the
personal data of EU residents?
    \begin{itemize}
        \item C. Beginning on May 25, 2018, the members of the EU began applying the General
Data Protection Regulation (GDPR). The GDPR applies to EU-based organizations that
collect or process the personal data of EU residents and to organizations outside the EU
that monitor behavior or offer goods and services to EU residents. Personally identifiable
information (PII) is data like an XSN number that is unique to the individual. Personal
health information (PHI) is confidential medical records. Payment Card Industry/ Data
Security Standards (PCI-DSS) is a standard for protecting credit card data.
    \end{itemize}
    \item Which of the following is false regarding dealing with customers?
    \begin{itemize}
        \item B. If you’re providing phone support, do the following:
Always answer the telephone in a professional manner, announcing the name of the com-
pany and yourself.
Make a concentrated effort to ascertain the customer’s technical level, and communicate at
that level, not above or below it.
The other options are all valid recommendations.
    \end{itemize}
    \item Which of the following should the IT professional do when dealing with customers?
    \begin{itemize}
        \item A. You should use appropriate professional titles, when applicable, and never take personal
calls, use the customers equipment for personal messages, or talk to co-workers while
interacting with customers.
    \end{itemize}
    \item Which of the following is written in Python?
    \begin{itemize}
        \item D. A .pyc file is one written in the Python language. Python runs on Windows, Mac OS
X, and Linux/Unix. A .vbs file is a Visual Basic file. An SH file is a script programmed for
Bash, a type of Unix shell.
    \end{itemize}
    \item Which of the following is a number that can be written without a fractional component?
    \begin{itemize}
        \item A. An integer (from the Latin integer meaning “whole”) is a number that can be written
without a fractional component (1, 2, 3 but not 1.5, 1.6). A string is a series of characters.
Variables are used to store information to be referenced and manipulated in a computer
program. A loop is a section of code that goes back to an earlier part of the script.
    \end{itemize}
    \item Which of the following is a command-line tool?
    \begin{itemize}
        \item D. If you don’t need access to the graphical interface and you just want to operate at the
command line, you have two options, Telnet and SSH. While Telnet works just fine, it
transmits all of the data in clear text, which obviously would be security issue. Remote
Desktop and screen sharing are graphical concepts, while file sharing is not a command-line
utility.
    \end{itemize}
    \item Which of the following is the least secure remote access technology?
    \begin{itemize}
        \item C. While Telnet works just fine, it transmits all of the data in clear text, which obviously
would be a security issue. Remote Desktop and screen sharing are graphical concepts that’s
can be secured, while Secure Shell (SSH) is an encrypted technology.
    \end{itemize}
\end{enumerate}
\section{220-1002 Practice Exam A}
\begin{enumerate}
    \item Which of the following are Microsoft operating systems? 
    \begin{itemize}
        \item Answers: A, C, and F
Explanation: Windows 8.1, Windows 10 (version 1803), and Windows 7 are all
Microsoft operating systems that you should know for the exam.
Incorrect Answers: iOS is the operating system Apple uses on its mobile devices.
Android is the competitor of iOS and is an open-source operating system used on many
other manufacturers’ mobile devices. Android is developed from Linux. The original Linux
was made for PCs with the goal of being a freely accessible, open-source platform; today
it is more commonly found in server form. Okay, that was an easy one…moving on!
    \end{itemize}
    \item Which of the following is the default file system used by Windows? 
    \begin{itemize}
        \item Answer: C
Explanation: The New Technology File System (NTFS) is the default file system that
Windows uses.
Incorrect Answers: FAT32 is an older, less desirable file system that offers less functionality
and less security and accesses smaller partition sizes. Compact Disc File System
(CDFS) is the file system used by an optical disc. exFAT is another file system supported
by Windows that works best with flash-based drives (such as USB thumb drives).
    \end{itemize}
    \item Where is the Notification Area located in Windows?
    \begin{itemize}
        \item Answer: C
Explanation: The Notification Area is the area toward the bottom right of your screen
within the taskbar. It contains the time and any applications (shown as icons) currently
running in memory. The System Properties dialog box contains configuration tabs for
the computer name and network, hardware, system restore, and more. You can access
any of the tabs in that dialog box quickly by going to Run and typing systempropertiescomputername.
exe, systempropertiesadvanced.exe, and so on.
Incorrect Answers: The System32 folder resides within the Windows folder; it contains
the critical Windows system files such as ntoskrnl.exe as well as applications such
as cmd.exe. The Start menu gives access to most programs and configurations in
Windows.
    \end{itemize}
    \item Which of the following is the minimum amount of RAM needed to
install a 64-bit version of Windows 10?
     \begin{itemize}
         \item Answer: C
Explanation: Windows 10 64-bit requires a minimum of 2 GB of RAM.
Incorrect Answers: The 32-bit version requires 1 GB. The same goes for Windows 8.1.
A minimum requirement for older versions of Windows was 512 MB. No doubt, 4 GB
will be the minimum for some Windows versions at some point.
     \end{itemize}
     \item In Windows, an MMC is blank by default. Which of the following
should be added to the MMC to populate it with programs?
     \begin{itemize}
         \item Answer: D
Explanation: The MMC (Microsoft Management Console) is a blank shell until you
add snap-ins (such as Computer Management or the Performance Monitor) for
functionality
     \end{itemize}
     \item Which of the following files is the boot loader in Windows?
     \begin{itemize}
         \item A. Winload.exe
     \end{itemize}
     \item Which specific tool enables you to create a partition in Windows?
     \begin{itemize}
         \item A. Disk Management
     \end{itemize}
     \item Which type of partition should an operating system be installed to?
     \begin{itemize}
         \item Answer: A
Explanation: Primary partitions are the first partitions created on a drive. An OS should
always be installed to a primary partition, but before you install the OS, you should set
the primary partition to active. If you are installing to a new hard drive, Windows will
automatically set the partition to active for you. A Master Boot Record (MBR)-based
hard drive can have four primary partitions maximum, each with its own drive letter.
Incorrect Answers: If you need to subdivide the hard drive further, you can also use an
extended partition, which is then broken up into logical drives. A GUID Partition Table
(GPT)-based hard drive is not limited to this number; it can have up to 128 primary
partitions. Dynamic refers to a dynamic drive; if you want to resize partitions, you have
to convert the drive to dynamic in Disk Management. By the way, any drive in Windows
that has a drive letter is known as a volume.
     \end{itemize}
     \item Which of the following tools enables you to find out how much
memory a particular application is using?
     \begin{itemize}
         \item Answer: B
Explanation: The Task Manager enables you, via a click of the Processes tab, to view
all current running processes and see how much memory each is using. You can open
the Task Manager by right-clicking the Taskbar and selecting it, by going to Run and
typing taskmgr, by pressing Ctrl+Shift+Esc, or by pressing Ctrl+Alt+Del and selecting
Task Manager.
Incorrect Answers: Msconfig is a utility in Windows that allows you to enable and
disable services and boot Windows in different modes. Chkdsk is a Command Prompt
utility that searches for errors and fixes them (with the /F or /R switches). The
System Information tool gives a summary of hardware resources, components, and the
software environment; you can open it by going to Run and typing msinfo32.
     \end{itemize}
     \item Which of the following features is used to both start and stop ser-
vices? (Select the two best answers.)
     \begin{itemize}
         \item  Answer: A and B
Explanation: You can start, stop, and restart services within Computer Management >
Services and Applications > Services. From there, right-click the service in question
and configure it as you wish. You can also open Services by going to the Run prompt
and typing services.msc. The Task Manager can also be used to start and stop
services, as well as to analyze the performance of the CPU, RAM, and the networking
connections. You can also start and stop services with the net start / net stop
and sc start / sc stop commands.
Incorrect Answers: Performance Monitor analyzes the computer in much more depth
than the Task Manager. The MMC is the Microsoft Management Console, which is the
index that can store other console windows such as Computer Management. Among
other things, Msconfig is used to enable/disable services, but not to start them.
     \end{itemize}
     \item Which of the following user account permissions are needed to
install device drivers on Windows?
     \begin{itemize}
         \item Answer: C
Explanation: The administrator is the only account level that can install device drivers.
Incorrect Answers: Standard user, and especially guest, accounts cannot install drivers
or programs. The Power Users group is an older group from the Windows XP days
that was carried over to newer versions of Windows for application compatibility, but it
has no real power in those operating systems.
     \end{itemize}
     \item Which of the following commands creates a new directory in the
Windows Command Prompt?
     \begin{itemize}
         \item  Answer: B
Explanation: MD is short for make directory and is the command to use when creating
directories in the Command Prompt.
Incorrect Answers: CD is change directory. RD is remove directory, and SD, which
deals with memory cards, is not a valid command in the Command Prompt.
     \end{itemize}
     \item Which of the following commands is entered at the Command
Prompt to learn more about the dir command? (Select the two
best answers.)
     \begin{itemize}
         \item Answers: B and C
Explanation: To learn more about any command in Windows, open the Command
Prompt, type the command and then /?, or type help dir.
Incorrect Answers: dir help would attempt to find the file named HELP within the
current directory. dir man would most likely result in a “file not found” error. MAN
pages are help pages used in Linux and macOS.
     \end{itemize}
     \item Which interface is used to launch the ipconfig command?
     \begin{itemize}
         \item Answer: A
Explanation: Use the Command Prompt to launch the command ipconfig. ipconfig is
a networking command that displays the configuration of your network adapter. You can
open the Command Prompt in a variety of ways. You can open the default Command
Prompt by going to Run and typing cmd.exe. However, many commands require you
to open the Command Prompt as an administrator (in elevated mode). To run it as an
administrator, locate it in Windows, right-click it, and select Run as Administrator. Or,
you could type cmd in the search field and then press Ctrl+Shift+Enter. You can also
locate it in Windows 10 or 8.1 by right-clicking the Start button.
Incorrect Answers: The other tools are used in the GUI and cannot run commands
such as ipconfig. Use the Command Prompt or the PowerShell to run commands in
Windows.
     \end{itemize}
     \item A customer’s computer is using FAT32. Which file system can you
upgrade it to when using the convert command?
     \begin{itemize}
         \item 
     \end{itemize}
     \item Which of the following can be used to keep hard drives free of
errors and ensure that Windows runs efficiently? (Select the two
best answers.)
     \begin{itemize}
         \item B. Disk Defragmenter C. Check Disk
     \end{itemize}
     \item What is Windows Recovery Environment known as? (Select the
two best answers.)
     \begin{itemize}
         \item A. WinRE D. System Recovery Options
     \end{itemize}
     \item Which log file contains information about Windows setup errors?
     \begin{itemize}
         \item B. setuperr.log
     \end{itemize}
     \item Which of the following represents the RAM limitation of Windows
8.1 Pro 64-bit?
     \begin{itemize}
         \item C 512GB
     \end{itemize}
     \item A customer’s Device Manager shows an arrow pointing down over
one of the devices. What does this tell you?
     \begin{itemize}
         \item Answer:   C                Explanation:        The arrow pointing down tells you that the device is disabled in Windows. In many cases, you can easily enable it by right-clicking and selecting En
     \end{itemize}
     \item Which of the following is not an advantage of NTFS over FAT32?
     \begin{itemize}
         \item Answer:   D                Explanation:        NTFS and FAT32 support the same number of file formats. This is actually the only listed similarity between the t
     \end{itemize}
     \item A coworker just installed a second hard drive in his Windows com-
puter. However, he does not see the drive in Explorer. What did he
forget to do? (Select the three best answers.)
     \begin{itemize}
         \item  Answers: A, B, and D                Explanation:        For secondary drives, you must go to Disk Management and initialize, partition, and format them. Explorer in the question could mean File Explorer (Windows 10) or Windows Explorer (Windows 7
     \end{itemize}
     \item How would you create a restore point in Windows?
     \begin{itemize}
         \item Answer:   C                Explanation:        System Restore is the tool used to create restore points. In all versions of Windows, you can find it with the Search utility or by going to the Control Panel > All Control Panel Icons > System, and then clicking the System Protection link. (Or, go to Run and type systempropertiesprotection.exe.) In Windows 7, you can find it in Start > All Programs > Accessories > System Tools as well.
     \end{itemize}
     \item Which of the following tasks cannot be performed from the Printer
Properties screen?
     \begin{itemize}
         \item   Answer:   C                Explanation:        To pause printing in general and pause individual documents, double-click on the printer in question and make the modifications from the ensuing window.
     \end{itemize}
     \item You are setting up auditing on a Windows computer. If it’s set up
properly, which of the following logs should contain entries?
     \begin{itemize}
         \item   Answer:   C                Explanation:        After Auditing is turned on and specific resources are configured for auditing, you need to check the Event Viewer’s Security log for the entries. These could be successful logons or misfired attempts at deleting files; there are literally hundreds of option
     \end{itemize}
     \item Which type of virus propagates itself by tunneling through the
Internet and networks?
     \begin{itemize}
         \item   Answer:   D                Explanation:        Worms travel through the Internet and through local-area networks (LANs). They are similar to viruses but differ in that they self-replicate
     \end{itemize}
     \item Which component of Windows enables users to perform common
tasks as non administrators and, when necessary, as administrators
without having to switch users, log off, or use Run As?
     \begin{itemize}
         \item   Answer:   B                Explanation:        With User Account Control (UAC) enabled, users perform common tasks as nonadministrators and, when necessary, as administrators without having to switch users, log off, or use Run As. If the user is logged in as an administrator, a pop-up win-dow will appear verifying that the user has administrative privileges before action is taken; the user need only click Yes. If the user is not logged on as an administrator, clicking Yes will cause Windows to prompt the user for an administrative username and pas
     \end{itemize}
     \item Which of the following tasks can be performed to secure your
WAP/router? (Select all that apply.)
     \begin{itemize}
         \item  Answers: A, B, and D                Explanation:        A multifunction network device that acts as both a wireless access point (WAP) and a router may come with a standard, default SSID name (that everyone knows). It is a good idea to change it (if the router doesn’t ask you to do so auto-matically). After PCs and laptops have been associated with the wireless network, turn off SSID broadcasting so that no one else can find your WAP (with normal means). Disabling DHCP and instead using static IP addresses removes one of the types of packets that are broadcast from the WAP, making it more difficult to hack, but of course less functional and useful. Other ways to secure the wireless access point include changing the password; incorporating strong encryption such as Wi-Fi Protected Access version 2 (WPA2) with Advanced Encryption Standard (AES); dis-abling WPS; and initiating MAC filtering, which only allows the computers with the MAC addresses you specify access to the wireless network.
     \end{itemize}
     \item When you connect to a website to make a purchase by credit card,
you want to make sure the website is secure. Which of the follow-
ing statements best describes how to determine whether a site is
secure? (Select the two best answers.)
     \begin{itemize}
         \item   Answers: A and D                Explanation:        Although it could possibly be spoofed, the padlock in the locked posi-tion gives you a certain level of assurance and tells you that the website is using a secure certificate to protect your session. This padlock could be in different locations depending on the web browser used. Hypertext Transfer Protocol Secure (HTTPS) also defines that the session is using either the Secure Sockets Layer (SSL) protocol or the Transport Layer Security (TLS) protocol.                Incorrect        Answers:        HTTP by itself is enough for regular web sessions when you read documents and so on, but HTTPS is required when you log in to a site, purchase items, or do online banking. HTTPS opens a secure channel on port 443 as opposed to the default, insecure HTTP port 80. To be sure that you have a secure session, you can analyze the certificate and verify it against the certificate auth
     \end{itemize}
     \item Which type of software helps protect against viruses that are
attached to email?
     \begin{itemize}
         \item  Answer:   B                Explanation:        Antivirus software (from vendors such as McAfee or Symantec) updates automatically to protect you against the latest viruses, whether they are attached to emails or are lying in wait on removable media. You might also choose to use Windows Defender on newer versions of Window
     \end{itemize}
     \item Which of the following is an example of social engineering?
     \begin{itemize}
         \item   Answer:   A                Explanation:        Social engineering is the practice of obtaining confidential information by manipulating people. Asking for a username and password over the phone is a type of phishing attack (known as vishin
     \end{itemize}
     \item Where are software-based firewalls most commonly implemented?
     \begin{itemize}
         \item Answer:   C                Explanation:        Software-based firewalls, such as the Windows Defender Firewall, nor-mally run on client computers.                Incorrect        Answers:        It is possible that software-based firewalls will run on servers, especially if the server is acting as a network firewall, but the servers might rely on a hardware-based network firewall or an IDS/IPS solution. Hardware-based firewalls are also found in multifunction network devices. Some people might refer to these devices as routers, but the router functionality is really just one of the roles of the multifunction network device—separate from the firewall role. Plus, higher-end routers for larger networks are usually not combined with firewall functionality. Switches don’t employ software firewalls, mainly because they don’t use software (for the mos
     \end{itemize}
     \item Making data appear as if it is coming from somewhere other than
its original source is known as which of the following terms?
     \begin{itemize}
         \item D. Spoofing
     \end{itemize}
     \item A fingerprint reader is known as which type of security
technology?
     \begin{itemize}
         \item A. Biometrics
     \end{itemize}
     \item Which of the following is the most secure password?
     \begin{itemize}
         \item D. Marqu1s\_De\_S0d\_ver\_2
     \end{itemize}
     \item Which shortcut key combination immediately locks Windows?
     \begin{itemize}
         \item B. Windows+L
     \end{itemize}
     \item Which of the following is the most secure file system in Windows?
     \begin{itemize}
         \item Answer:   C Explanation:        NTFS is Windows’ New Technology File System. It secures files and fold-ers (and, in fact, the whole partition) much better than the older FAT32 system does.EFS, BitLocker, and NTFS permissions are just a few of the advantages of an NTFSpartition
     \end{itemize}
     \item Which of the following is the most secure for your wireless
network?
     \begin{itemize}
         \item Answer:   B        Explanation:        WPA2 is superior to WPA and WEP and takes much longer to crack (if itis crackable at all). It works best with AES
     \end{itemize}
     \item Which of the following terms refers to when people are
manipulated into giving access to network resources?
     \begin{itemize}
         \item Answer:   B        Explanation:        Social engineering is when fraudulent individuals try to get information from users through manipulation.
     \end{itemize}
     \item A customer’s Windows computer needs a new larger, faster hard
drive. Another technician in your company installs the new drive
and then formats the old drive before delivering it to you for
disposal. How secure is the customer’s data?
     \begin{itemize}
         \item Answer:   B        Explanation:        The data is very insecure. Many tools can recover data from a drive after it is formatted. Some companies will “low-level” format the drive, or sanitize the drive(as opposed to a standard format in Windows, for example) and keep it in storage indefinitely. The organization might go further and use data wiping software; in fact,this might be a policy for the organization. Always check your organization’s policies to be sure you are disposing of or recycling hard drives properly.
     \end{itemize}
     \item Which of the following offers hardware-based authentication?
     \begin{itemize}
         \item Answer:   B                Explanation:        Smart cards are actual physical cards that you use as authentica-tion tools. They are sometimes referred to as tokens and have built-in processors. Examples of smart cards include the Personal Identity Verification (PIV) card used by U.S. government employees and the Common Access Card (CAC) used by Department of Defense personnel.
     \end{itemize}
     \item Which protocol encrypts transactions through a website?
     \begin{itemize}
         \item Answer:   B                Explanation:        Secure Sockets Layer (SSL) and the newer Transport Layer Security (TLS) encrypt the transactions through the website. These SSL certificates are often accompanied by the protocol HTTPS.
     \end{itemize}
     \item Which of the following is a common local security policy?
     \begin{itemize}
         \item Answer:   B                Explanation:        Common local security policies include password length, duration, and complexity. Just the use of a password doesn’t constitute a password policy. An exam-ple of a password policy would be when an organization mandates that passwords be 15 characters in length with at least 1 capital letter, 1 number, and 1 special character. In Windows you would access Local Security Policy > Security Settings > Account Policies > Password Policy to make changes to these things
     \end{itemize}
     \item A coworker downloads a game that ends up stealing information
from the computer system. What is this known as?
     \begin{itemize}
         \item Answer:   C                Explanation:        A Trojan is a disguised program that is used to gain access to a computer and either steal information or take control of the computer.
     \end{itemize}
     \item Which of the following is an open-source operating system?
     \begin{itemize}
         \item Answer: A
Explanation: Android is an open-source OS. It is freely downloadable and can be modi-
fied by manufacturers of mobile devices to suit their specific hardware.
     \end{itemize}
     \item Where can you obtain applications for mobile devices? (Select the
three best answers.)
     \begin{itemize}
         \item Answers: B, C, and D
Explanation: Android users download applications (apps) from Google Play. Apple
users download apps from the App Store or from within iTunes.
     \end{itemize}
     \item You need to locate a mobile device that was stolen. Which technol-
ogy can aid in this?
     \begin{itemize}
         \item Answer: A
Explanation: The Global Positioning System (GPS) technology (or location services)
can be instrumental in locating lost or stolen mobile devices. Many devices have this
installed; others rely on geotracking or Wi-Fi hotspot locating techniques. (You are
being watched!)
     \end{itemize}
     \item Which kinds of data are typically synchronized on a smartphone?
(Select the two best answers.)
     \begin{itemize}
         \item Answers: A and C
Explanation: Some of the things you might synchronize on a smartphone include con-
tacts, email, programs, pictures, music, and videos.
     \end{itemize}
     \item Which of the following is the second step of the A+ troubleshoot-
ing theor
     \begin{itemize}
         \item Answer: B
Explanation: The second step is to establish a theory of probable cause. You are look-
ing for the obvious or most probable cause for the problem.
     \end{itemize}
     \item You successfully modified the Registry on a customer’s PC. Now
the customer’s system gets onto the Internet normally. Which of
the following steps should be performed next?
     \begin{itemize}
         \item Answer: C
Explanation: Documentation is the final step in the troubleshooting theory. This helps
you better understand and articulate exactly what the problem (and solution) was. If
you see this problem in the future, you can consult your documentation for the solu-
tion. Plus, others on your team can do the same. In addition, it is common company
policy to document all findings as part of a trouble ticket.
     \end{itemize}
     \item Buzz gets an error that says “Error log full.” Where should you go
to clear his error log?
     \begin{itemize}
         \item Answer: D
Explanation: The Event Viewer contains the error logs; they are finite in size. You could
either clear the log or increase the size of the log.
     \end{itemize}
     \item Which of the following tools checks protected system files?
     \begin{itemize}
         \item Answer: D
Explanation: System File Checker ( SFC ) checks protected system files and replaces
incorrect versions.
     \end{itemize}
     \item After installing a new hard drive on a Windows computer, Len tries
to format the drive. Windows does not show the format option in
Disk Management. What did Len forget to do first?
     \begin{itemize}
         \item Answer: B
Explanation: You must partition the drive before formatting.
     \end{itemize}
     %54
     \item Which Windows System Recovery Option attempts to automati-
cally fix problems?
     \begin{itemize}
         \item Answer: B
Explanation: The best answer is Startup repair. Startup repair attempts to fix issues
automatically. This is available in the Windows RE System Recovery Options.
     \end{itemize}
     \item Which utility enables you to troubleshoot an error with a file such
as ntoskrnl.exe?
     \begin{itemize}
         \item Answer: B
Explanation: The Event Viewer logs all errors that occur on a system. Particularly, the
System log would contain the information useful in troubleshooting this error.
     \end{itemize}
     \item A blue screen is most often caused by
     \begin{itemize}
         \item Answer: A
Explanation: The most common reason for a BSOD (blue screen of death, otherwise
known as a stop error) is driver failure.
     \end{itemize}
     \item A technician is installing a program on a Windows computer
and the installation fails. Which of the following statements
describes the next best step?
     \begin{itemize}
         \item Answer: A
Explanation: Run the installer as an administrator. Programs cannot be installed by
standard users or guests. You must have administrative rights to do so.
     \end{itemize}
     \item Which of the following statements best describes how to apply
spray cleaner to a monitor?
     \begin{itemize}
         \item Answer: D
Explanation: Spray on a lint-free cloth first, and then wipe the display gently. A lot of
companies sell products that are half isopropyl alcohol and half water. You could also
make this cleaner yourself. Again, remember to put the solution on a lint-free cloth
first.
     \end{itemize}
      \item You and a coworker are running network cables above the drop
ceiling. The coworker accidentally touches a live AC power line and
is thrown off the ladder and onto the ground. He is dazed and can’t
stand. He is no longer near the AC power line. Which of the follow-
ing statements best describes the first step you should take?
     \begin{itemize}
         \item Answer: D
Explanation: Because the immediate danger is gone, call 911 right away.
     \end{itemize}
      \item A computer you are working on has a lot of dust inside it. Which
of the following statements best describes how to clean this?
     \begin{itemize}
         \item Answer: D
Explanation: Compressed air is safe. However, you might want to do this outside and
vacuum up the leftover residue. Or if you are working inside, use compressed air to
blow dust out of the computer while using an antistatic vacuum to suck up the dust at
the same time.
     \end{itemize}
      \item You are working on a very old printer and it begins to smoke.
Which of the following statements best describes the first step you
should take?
     \begin{itemize}
         \item Answer: C
Explanation: Turning off the printer might not be enough. It might be seriously mal-
functioning, so pull the plug.
Incorrect Answers: Dialing 911 is not necessary unless a fire has started. Wait at least
15 minutes before opening the printer to see what caused the smoke. Printer power sup-
plies can fail just like a PC’s power supply can. In fact, a laser printer power supply does
more work because it needs to convert for high voltages in the 600-V range. If you have a
maintenance contract with a printer company, and the printer is under warranty or contained
in the service contract, you could call the maintenance company to fix the problem. Be
ready to give a detailed account of exactly what happened. You could tell the printer that it
is bad to smoke, but that would be belligerent and would probably show that you have been
working too hard. All kidding aside, be ready to disconnect power at a moment’s notice.
     \end{itemize}
      \item Which of the following statements best describes the recommended
method for handling an empty toner cartridge?
     \begin{itemize}
         \item Answer: D
Explanation: Recycle toner cartridges according to your company’s policies and proce-
dures, or according to municipality rules and regulations.
     \end{itemize}
      \item One of your technicians is on a service call and is dealing with a
furious customer who has been shouting loudly. The technician
tries but cannot calm down the customer. Which of the following
statements best describes the next step the technician should
take?
     \begin{itemize}
         \item Answer: C
Explanation: The technician should leave the customer site and document the incident.
In rare cases, there is no way to calm down the customer, and you might have to leave
the site if there is no other alternative.
incident.
     \end{itemize}
      \item While you are working at a customer site, a friend calls you on
your cell phone. Which of the following statements best describes
the recommended course of action?
     \begin{itemize}
         \item Answer: A
Explanation: While you’re on the job site, limit phone calls to only emergencies or calls
from your employer about other customers.
Incorrect Answers: Taking a personal phone call, texting, or using social media sites
while working at a client site is considered unprofessional. Be professional when you’re
on the job.
     \end{itemize}
      \item Which of the following tools is used when setting the computer to
boot with the Selective Startup feature?
     \begin{itemize}
         \item  Answer: D
Explanation: Msconfig enables you to modify the startup selection. You can boot the
computer in different modes with Msconfig . You can also enable and disable services.220-1002 Practice Exam A
169
Incorrect Answers: The Task Manager gives you a snapshot of your system’s per-
formance and allows you to shut down applications (tasks) or processes, even if the
application is hanging or frozen. Windows RE is the Windows Recovery Environment,
a special repair environment that is used to fix issues in the operating system. From
here, you can fix system file issues and repair the boot sector, along with GPT and
MBR-related issues. Safe Mode is one of the options in the Startup Settings/Advanced
Boot Options menu. It starts the computer with a basic set of drivers so that you can
troubleshoot why devices have failed. It is also instrumental when dealing with viruses.
     \end{itemize}
      \item Which of the following file extensions is used when saving
PowerShell scripts?
     \begin{itemize}
         \item Answer: C
Explanation: The .ps1 extension is used for PowerShell files and scripts.
     \end{itemize}
     \item You have been given the task of installing a new hard drive on
a server for a customer. The customer will be supervising your
work. Which of the following questions should you ask the customer
first?
     \begin{itemize}
         \item Answer: B
Explanation: Always check whether there are backups, and physically inspect and
verify the backups before changing out any drives. Making sure that a backup is avail-
able is the first order of business.
     \end{itemize}
      \item You just upgraded the president’s computer’s video driver. Now,
the Windows 10 system will not boot. Which of the following steps
should be taken first?
     \begin{itemize}
         \item Answer: B
Explanation: By rolling back the driver (which is done in the Device Manager) while in
Safe Mode, you can go back in time to the old working video driver.
Incorrect Answers: The Windows Recovery Environment might help (for example, if
you used Startup Settings > Safe Mode, or System Restore), but the WinRE Command
Prompt is not the best answer because it is a different tool. Reinstalling the OS would
wipe the partition of the president’s data (and probably wipe you of your job). Directory
Services Restore mode (although listed in the Advanced Startup Options) is only for
Windows Server domain controllers.
     \end{itemize}
      \item Which tool is used to analyze and diagnose a video card, including
its DirectX version?
     \begin{itemize}
         \item Answer: B
Explanation: The DxDiag utility is used to analyze a video card and its DirectX version
and to check if drivers are digitally signed. You can access it by going to Run and typ-
ing dxdiag .
     \end{itemize}
      \item Which of the following statements best describes a common risk
when installing Windows drivers that are unsigned?
     \begin{itemize}
         \item Answer: A
Explanation: By installing a driver that is not signed by Microsoft, you are risking
instability of the operating system.
     \end{itemize}
      \item Which of the following settings must be established when you
want to make a secure wireless connection? (Select all that apply.)
     \begin{itemize}
         \item  Answer: C and D
Explanation: To make a secure connection, you first need to know the Service Set
Identifier (SSID) of the AP and then the encryption being used (for example, WPA or
WPA2). The SSID takes care of the “connection” portion, and the encryption takes care
of the “secure” portion. After all computers are connected, consider disabling the SSID
for increased security.
Incorrect Answers: Knowing the wireless standard being used can help you verify
whether your computer is compatible (802.11ac, n, or g), but the brand of access point
isn’t really helpful.
     \end{itemize}
      \item Which Windows utility is used to prepare a drive image for duplication
across the network?
     \begin{itemize}
         \item Answer: B
Explanation: Sysprep is one of the utilities built into Windows for image deployment
over the network.
Incorrect Answers: Ghost and Image Clone are third-party offerings. Robocopy copies
entire directories (in the same physical order, too). Sysprep preps the system to be
moved as an image file.
     \end{itemize}
      \item In Windows, when will a computer dump the physical memory
     \begin{itemize}
         \item Answer: D
Explanation: If the computer fails and cannot recover, you usually see some type of
critical or stop error. At this point, you must restart the computer to get back into the
operating system (unless it is configured to do so automatically, which is the default
setting in Windows). The reason for the physical dump of memory is for later debug-
ging. The physical dump writes the contents of memory (when the computer failed) to
a file on the hard drive.
Incorrect Answers: Missing drivers do not cause this error, but a failed driver might.
If the wrong processor is installed, you can probably not get the system to boot at all.
Shutting down the computer improperly just means that the computer recognizes this
upon the next reboot and might attempt to automatically fix errors if any occurred.
     \end{itemize}
      \item When a person takes control of a session between a server and a
client, it is known as which type of attack?
     \begin{itemize}
         \item Answer: C
Explanation: Session hijacking occurs when an unwanted mediator takes control of
the session between a client and a server (for example, an FTP or HTTP session). An
example of this would be a man-in-the-middle (MITM) attack.
Incorrect Answers: DDoS is a distributed denial-of-service attack, an attack perpetu-
ated by hundreds or thousands of computers in an effort to take down a single server;
the computers, individually known as zombies, are often unknowingly part of a botnet.
A brute-force attack is an attempt to crack an encryption code or password. Malicious220-1002 Practice Exam A
171
software is any compromising code or software that can damage a computer’s files;
examples include viruses, spyware, worms, rootkits, ransomware, and Trojans.
     \end{itemize}
      \item The message “The Windows Boot Configuration Data File Is
Missing Required Information” appears on the screen. Which command
can repair this issue?
     \begin{itemize}
         \item Answer: C
Explanation: Bootrec /rebuildbcd attempts to rebuild the boot configuration store.
Incorrect Answers: Bootrec /fixboot is one of the methods you can try to repair
bootmgr.exe in Windows. Bootrec /fixmbr rewrites the master boot record in a
Windows system that has an MBR-based hard drive (doesn’t affect the more common
GPT-based drive). boot bcd is where the boot configuration store is located.
     \end{itemize}
     \item Which of the following should be performed during a hard drive
replacement to best maintain data privacy?
     \begin{itemize}
         \item Answer: A
Explanation: The drive should be completely erased with bit-level erasure software.
If it is to be disposed of or is to leave the building, it should also be shredded or
degaussed (or both.
     \end{itemize}
     \item Which tool is used to back up data on the C: drive in Windows 10?
     \begin{itemize}
         \item Answer: D
Explanation: The Windows 10 and 8 File History utility (accessible in the Control
Panel) enables a user to back up files or the entire PC
     \end{itemize}
     \item Which of the following is the minimum processor requirement for
Windows 10?
     \begin{itemize}
         \item Answer: B
Explanation: Windows 10 (and Windows 8 and Windows 7) requires a minimum pro-
cessor frequency of 1 GHz.
Incorrect Answers: Windows 10 64-bit requires 32 GB of hard drive space. As of the
writing of this book, 2 GHz is not a valid answer for Windows. The minimum RAM
requirement for 64-bit versions of Windows is 2 GB.
     \end{itemize}
     \item You create an answer file to aid in installing Windows. Which type
of installation are you performing? (Select the best answer.)
     \begin{itemize}
         \item Answer: D
Explanation: An unattended installation of Windows requires an answer file. This file is
normally named unattend.xml. Unattended installations can be done locally or as part
of a network installation using Windows Deployment Services (WDS) in Server 2008 or
higher.
     \end{itemize}
     \item Which of the following utilities can be used to view the startup
programs?
     \begin{itemize}
         \item C. Regedit
     \end{itemize}
     
\end{enumerate}
\subsection{Practice Exam B}
\begin{enumerate}
    \item Which of the following statements best describes how to restart the print spooler service? 
    \begin{itemize}
        \item Answer: A and D Explanation: In the command line, this service is simply known as Spooler. Type net
stop spooler and net start spooler to restart the service. In Computer
Management, the Print Spooler service is found in Services and Applications >
Services. Or you could open the Run prompt and type services.msc. From there,
you can start, stop, pause, resume, or restart services and also set their Startup type
to Automatic, Manual, or Disabled.
    \end{itemize}
    \item Where is Registry hive data stored? 
    \begin{itemize}
        \item Answer: D
        \begin{lstlisting}
            \%systemroot%\System32\Config
        \end{lstlisting}
    \end{itemize}
    \item Clinton needs a more secure partition on his hard drive. Currently the only partition on the drive (C:) is formatted as FAT32. He cannot lose the data on the drive but must have a higher level of security, so he is asking you to change the drive to NTFS. Which of the following is the proper syntax for this procedure
    \begin{itemize}
        \item Answer: C
Explanation: The convert command turns a FAT32 drive into an NTFS drive without
data loss, allowing for a higher level of data security. The proper syntax is convert
volume /FS:NTFS .
    \end{itemize}
    \item Tom has a 200GB hard drive partition (known as C:) on a Windows computer. He has 15GB free space on the partition. Which of the following statements best describes how he can defrag the
    partition? 
    \begin{itemize}
        \item Answer: B
Explanation: Use defrag.exe -f . You need to have 15 percent free space on your
partition to defrag it in the Disk Defragmenter GUI-based utility. In the scenario, Tom
would need 30 GB free on the 200 GB drive. However, you can force a defrag on a par-
tition even if you don’t have enough free space by using the -f switch in the command
line. ( -f may not be necessary in some versions of Windows.)
    \end{itemize}
    \item You are utilizing WSUS and are testing new updates on PCs what is this example of? 
    \begin{itemize}
        \item Answer: C
Explanation: Patch management is the patching of many systems from a central
location. It includes the planning, testing, implementing, and auditing stages. There
are various software packages you can use to perform patch management. Windows
Server Update Services (WSUS) is an example of Microsoft patch management soft-
ware. Other Microsoft examples include the System Center Configuration Manager
(SCCM) and its predecessor Systems Management Center (SMS), but there are plenty
of third-party offerings as well.
    \end{itemize}
    \item Which versions of Windows 8 allow for joining domains? 
    \begin{itemize}
        \item Answers: B and D
Explanation: Windows 8 Pro and Enterprise allow for the joining of domains.
    \end{itemize}
    \item One of your customers reports that there is a large amount of
spam in her email inbox. Which of the following statements
describes the best course of action to recommend to her?
\begin{itemize}
        \item Answer: B
Explanation: You should recommend that the user add the senders to the junk email
sender list. This blocks those senders’ email addresses (or the entire domain can be
blocked). However, this option could take a lot of time; another option is to increase
the level of security on the spam filter within the email program. Any further spam can
then be sent to the junk email sender list.
    \end{itemize}
    \item In Windows, where can devices like the display and hard drives be
configured to turn off after a certain amount of time?
    \begin{itemize}
        \item Answer: A
Explanation: To turn off devices after a specified period of time in Windows, access
Control Panel > Power Options. Then click Change Plan Settings for the appropriate
power plan.
    \end{itemize}
    \item Which of the following procedures best describes how to find out
which type of connection the printer is using?
    \begin{itemize}
        \item Answer: D
Explanation: On the Ports tab, you can find how the printer is connected to the com-
puter. This can be a USB, COM, LPT, or TCP/IP port. You might get to this tab by
selecting Properties or Printer Properties, depending on the printer.
    \end{itemize}
    \item Your customer is having problems printing from an application.
You attempt to send a test page to the printer. Which of the following
statements best describes why a test page should be used to
troubleshoot the issue?
    \begin{itemize}
        \item Answer: C
Explanation: The test page verifies connectivity and gives you insight as to possible
application problems at the computer that is attempting to print.
application problems.
    \end{itemize}
    \item A user’s hard drive seems very slow in its reaction time when
opening applications. Which of the following statements best
describes the most likely cause of this?
    \begin{itemize}
        \item Answer: C
Explanation: The drive is fragmented. This is why it is very slow in its reaction time.
It’s also possible that the OS is infected with a virus. You should analyze and defrag-
ment the drive and run an AV sweep of the system.
    \end{itemize}
    \item Which of the following actions will not secure a functioning computer
workstation?
    \begin{itemize}
        \item Answer: D
Explanation: Sanitizing the hard drive does not secure a computer workstation. It does,
however, prevent anyone from accessing data on the drive, but it also ensures the
computer workstation won’t be functional anymore. A data sanitization method is the
specific way in which a data destruction program or file shredder overwrites the data
on a hard drive or other storage device.
    \end{itemize}
    \item Which utility enables you to implement auditing on a single
Windows computer?
    \begin{itemize}
        \item Answer: A
Explanation: Because there is only one computer, you can implement auditing only
locally. This is done with the Local Security Policy. (This policy is not available in all
editions of Windows.)
    \end{itemize}
    \item Which of the following statements best describes the main function
of a device driver?
    \begin{itemize}
        \item Answer: D
Explanation: Device drivers are the connection between the operating system and the
device itself. It is a program that makes the interaction between the two run efficiently.
It simplifies programming by using high-level application code. The best device driv-
ers come from the manufacturer of the device. They are the ones who developed the
device, so it stands to reason that their code would be the most thoroughly tested and
debugged.
    \end{itemize}
    \item Where are restore points stored after they are created?
    \begin{itemize}
        \item Answer: D
Explanation: After a restore point is made, it is stored in the System Volume
Information folder. To view this folder, you must log on as an administrator, show hid-
den files and folders, and then assign permissions to the account that wants to view
that folder. It is located in the root of the volume that the restore point was created for.
    \end{itemize}
    \item Which of the following is considered to be government-regulated
data?
    \begin{itemize}
        \item Answer: C
Explanation: PII stands for personally identifiable information. It is regulated by many
laws such as the Privacy Act of 1974 and several others, including GDPR and PCI-DSS
    \end{itemize}
    \item Which of the following are types of social engineering? (Select the
two best answers.)
    \begin{itemize}
        \item Answers: B and C
Explanation: Shoulder surfing and tailgating are both types of social engineering. A
shoulder surfer is someone who attempts to view information on a person’s desk or
display without the person’s knowledge.
    \end{itemize}
    \item Which of the following is the service that controls the printing of
documents in a Windows computer?
    \begin{itemize}
        \item Answer: D
Explanation: The Print Spooler controls the queue and the printing of documents.
    \end{itemize}
    \item Which of the following is the best way to ensure that a hard drive
is secure for disposal?
    \begin{itemize}
        \item Answer: A
Explanation: Magnetically erase the drive; for example, degauss the drive. Degaussing
a drive is an excellent way to remove all traces of data, but only if the drive is elec-
tromagnetic! Of course, physical destruction is better (shredding, pulverizing); and
degaussing might be used on top of physical destruction.
    \end{itemize}
    \item A month ago, you set up a wireless access point/router for a small
business that is a customer of yours. Now, the customer calls and
complains that Internet access is getting slower and slower. As you
look at the WAP/router, you notice that it was reset at some point
and is now set for open access. You then guess that neighboring
companies are using the service connection. Which of the following
statements best describes how you can restrict access to your
customer’s wireless connection? (Select the two best answers.)
    \begin{itemize}
        \item Answers: A and C
Explanation: If the WAP/router was reset, any security settings that you originally set
up are most likely gone. If you backed up the settings previously, you could restore
them. Either way, some type of encryption protocol (preferably WPA2) is necessary.
The passphrase or network key generated by the WAP/router needs to be installed on
each client before it can be recognized on the network. This passphrase/key should be
kept secret, of course. After all the clients have been associated with the WAP/router,
disable SSID broadcasting so that no one else can “see” the router (without more
advanced software).
    \end{itemize}
    \item A first-level help desk support technician receives a call from a
customer and works with the customer for several minutes to
resolve the call, but the technician is unsuccessful. Which of the
following steps should the technician perform next?
    \begin{itemize}
        \item Answer: B
Explanation: The tech should escalate the call to another technician. This is exactly
why help desks are configured in groups: Level 1, Level 2, the masters (Level 3), and
possibly beyond. Don’t try to be a superhuman. In technology, there is almost always
someone who knows more than you about a specific subject. First, route the call to the
next-level tech, and then let the customer know that you are doing so.
    \end{itemize}
    \item A customer complains that there is nothing showing on the display
of his laptop. Which of the following should you attempt first on
the computer?
    \begin{itemize}
        \item Answer: D
Explanation: The computer might need a special keystroke, a press of the power but-
ton, or just a little more time to come out of Hibernation mode. Remember, check the
simple, quick solutions first because they are usually the culprits.
    \end{itemize}
    \item During an installation of Windows, you are given an opportunity
to load alternative third-party drivers. Which device are you most
likely loading drivers for?
    \begin{itemize}
        \item Answer: B
Explanation: The SCSI hard drive is the most likely answer. SCSI hard drives (such as
SAS SCSI) and RAID controllers need special drivers during the Windows installation
process if they are not recognized automatically. Click the option for loading third-party
drivers when the installation begins.
    \end{itemize}
    \item A Windows 10 computer in a Windows workgroup can have how
many concurrent connections?
    \begin{itemize}
        \item Answer: C
Explanation: A Windows 10 computer in a Windows workgroup can have 20 maximum
concurrent connections to it over the network.
    \end{itemize}
    \item Megan’s laptop runs perfectly when she is at work, but when she
takes it on the road, it cannot get on the Internet. Internally, the
company uses static IP addresses for all computers. What should
you do to fix the problem?
    \begin{itemize}
        \item Answer: D
Explanation: The issue is that Megan needs to obtain an IP address through DHCP
when on the road. But setting the network adapter to obtain an IP address auto-
matically is not enough. To connect to the internal company network, the Alternate
Configuration tab must be configured as a “User Configured” static IP address. This
solution enables Megan to connect to networks while on the road by obtaining IP
addresses automatically and allows her to connect to the internal company network
with the static IP address.
    \end{itemize}
    \item Which power-saving mode enables for the best power savings,
while still allowing the session to be reactivated later?
    \begin{itemize}
        \item  Answer: C
Explanation: Hibernate mode saves all the contents of RAM (as hiberfil.sys in the root
of C:) and then shuts down the system so that it is using virtually no power. To reacti-
vate the system, you must press the power button. At that point, the entire session is
loaded from RAM, and you can continue on with the session.
    \end{itemize}
    \item John’s computer has two hard drives, each 1 TB. The first is the
system drive and is formatted as NTFS. The second is the data
drive and is formatted as FAT32. Which of the following statements
are true? (Select the two best answers.)
    \begin{itemize}
        \item Answers: A and D
Explanation: NTFS can use NTFS file-level security, whereas FAT32 cannot. NTFS clus-
ter sizes are smaller than FAT32 clusters. NTFS partitions are therefore more efficient
(when installed correctly) than FAT32 partitions.
    \end{itemize}
    \item When using the command line, a switch
    \begin{itemize}
        \item Answer: D
Explanation: A switch (aka option) alters the action of the command but not by forcing
it to perform unrelated actions.
    \end{itemize}
    \item You need to view any application errors that have occurred today.
Which tool should be used?
    \begin{itemize}
        \item Answer: A
Explanation: The Event Viewer contains the log files of all the errors that occur on the
machine. In this case, you would go to the Application log. Another common log is the
System log, which shows errors concerning the OS and drivers.
    \end{itemize}
    \item Which of the following commands can help you modify the startup
environment?
    \begin{itemize}
        \item Answer: A
Explanation: The msconfig utility enables you to modify the startup environment via
the General, Boot, and Startup tabs (in Windows 7), and the General and Boot tabs (in
Windows 8 and Windows 10).
    \end{itemize}
    \item Which of the following log files references third-party software
error messages?
    \begin{itemize}
        \item Answer: C
Explanation: The Application log in the Event Viewer displays errors concerning
Windows applications as well as third-party applications.
    \end{itemize}
    \item Which of the following provides the lowest level of wireless security
protection?
    \begin{itemize}
        \item Answer: A
Explanation: Disabling the SSID broadcast is a security precaution, but it only keeps
out the average user. Any attacker with two bits of knowledge can scan for other things
the wireless access point broadcasts.
    \end{itemize}
    \item A customer uses an unencrypted wireless network. One of the
users has shared a folder for access by any computer. The customer
complains that files sometimes appear and disappear from
the shared folder. Which of the following statements best describes
how to fix the problem? (Select the two best answers.)
    \begin{itemize}
        \item Answers: A and C
Explanation: Use WPA or WPA2 on the router (and clients) to deny wardrivers and
other stragglers access to the customer’s network and, ultimately, any shared folders
on the network. Increase the level of NTFS security by changing the permissions in the
Security tab of the shared folder.
    \end{itemize}
    \item A customer is having difficulties with his hard drive, and the system
won’t boot. You discover that the operating system has to be
reloaded. Which of the following statements best describes how to
explain this to the customer?
    \begin{itemize}
        \item Answer: D
Explanation: Always explain specifically and exactly what you must do and what the
ramifications are. Verify that the customer agrees to the proposed work (in writing).
    \end{itemize}
    \item Users in your accounting department are prompted to provide
usernames and passwords to access the payroll system. Which
type of authentication method is being requested in this scenario?
    \begin{itemize}
        \item . Answer: B
Explanation: The type of authentication method being used here is single-factor. The
only factor of authentication is something the users know—usernames and passwords.
    \end{itemize}
    \item Which of the following commands makes a duplicate of a file?
    \begin{itemize}
        \item Answer: B
Explanation: Copy is used to make a duplicate of the file in another location
    \end{itemize}
    \item Which tool in Windows enables a user to easily see how much
memory a particular process uses?
    \begin{itemize}
        \item Answer: C
Explanation: The Task Manager enables a user to see the amount of memory and the
percentage of processing power a particular process uses in real time. This can be
done on the Processes tab.
    \end{itemize}
    \item Windows was installed on a computer with two hard drives: a C:
drive and a D: drive. Windows is installed to C:, and it works normally.
The user of this computer complains that his applications are
drive intensive and that they slow down the computer. Which of the
following statements best describes how to resolve the problem?
    \begin{itemize}
        \item Answer: A
Explanation: By moving the paging file (or swap file, aka virtual memory) to the D:
drive, you are freeing up C: to deal with those drive-intensive programs.
    \end{itemize}
    \item Which of the following tools should be used to protect a computer
from electrostatic discharge (ESD) while you are working inside it?
    \begin{itemize}
        \item Answer: C
Explanation: Use an antistatic wrist strap when working inside a computer to protect
against electrostatic discharge (ESD). Other ways to prevent ESD include using an anti-
static mat, touching the chassis of the case (self-grounding), and using antistatic bags
    \end{itemize}
    \item You are running some cable from an office to a computer located
in a warehouse. As you are working in the warehouse, a 55-gallon
drum falls from a pallet and spills what smells like ammonia.
Which of the following statements best describes the first step you
should take in your efforts to resolve this problem?
    \begin{itemize}
        \item Answer: C
Explanation: If something is immediately hazardous to you, you must leave the area
right away.
    \end{itemize}
    \item While you are upgrading a customer’s server hard drives, you notice
looped network cables lying all over the server room floor. Which of
the following statements best describes how to resolve this issue?
    \begin{itemize}
        \item Answer: B
Explanation: You need to explain to the customer that there is a safer way. Cable man-
agement is very important when it comes to the safety of employees. Trip hazards such
as incorrectly routed network cables can have devastating effects on a person.
    \end{itemize}
    \item Which of the following statements best describes the recommended
solution for a lithium-ion battery that won’t hold a charge any longer?
    \begin{itemize}
        \item Answer: C
Explanation: Every municipality has its own way of recycling batteries. They might be
collected by the town or county yearly, or perhaps there are other recycling programs
that are sponsored by recycling companies. Always call the municipality to find out
exactly what to do.
    \end{itemize}
    \item Which of the following statements is not assertive communication?
    \begin{itemize}
        \item Answer: C
Explanation: Asking a customer if employees always cause issues is just plain rude;
this type of communication should be avoided
    \end{itemize}
    \item A customer has a malfunctioning PC, and as you are about to
begin repairing it, the customer proceeds to tell you about the
problems with the server. Which of the following statements best
describes how to respond to the customer?
    \begin{itemize}
        \item Answer: C
Explanation: Ask if the server problem is related to the PC problem. Try to understand
the customer before making any judgments about the problems. Make sure it isn’t a
bigger problem than you realize before making repairs that could be futile. If you find
out that it is a separate problem, ask the customer which issue should be resolved
first.
    \end{itemize}
    \item Which of the following could be described as the chronological
paper trail of evidence?
    \begin{itemize}
        \item Answer: B
Explanation: Chain of custody is the chronological paper trail of evidence that may or
may not be used in court.
    \end{itemize}
    \item Which of the following statements best describes what not to do
when moving servers and server racks?
    \begin{itemize}
        \item Answer: B
Explanation: Don’t attempt to move heavy objects by yourself. Ask someone to
help you.
    \end{itemize}
    \item Active communication includes which of the following?
    \begin{itemize}
        \item Answer: C
Explanation: One example of active communication is clarifying a customer’s state-
ments. For instance, if you are unsure exactly what the customer wants, always clarify
the information or repeat it back to the customer so that everyone is on the same page.
    \end{itemize}
    \item You are troubleshooting a tablet PC that has a frozen application.
You have attempted to end the underlying task of the application
but have not succeeded. Which of the following statements best
describes the next recommended course of action?
    \begin{itemize}
        \item Answer: C
Explanation: The next attempt you should make (from the listed answers) is a soft
reset of the device. Resetting often requires pressing a special combination of but-
tons. That keypress (hopefully) restarts the device with the RAM cleared. Then you can
troubleshoot the problem application further if necessary.
    \end{itemize}
    \item Which of the following statements best describes the first course
of action to removing malware?
    \begin{itemize}
        \item Answer: A
Explanation: The first step in the malware removal best practices procedure is to iden-
tify malware symptoms.
    \end{itemize}
    \item You are working on a Windows computer that is performing
slowly. Which of the following commands should you use to
resolve the problem? (Select the two best answers.)
    \begin{itemize}
        \item Answers: B and D
Explanation: The best listed answers are dism and chkdsk . For a computer that
is running slow, try using the chkdsk (check disk) and SFC (system file checker)
commands. Then, if those run into problems, try using the dism (Deployment Image
Servicing and Management) command. Chkdsk and SFC can repair problems with the
drive and with system files. Dism can repair problems with the system image (where
SFC will draw information from).
    \end{itemize}
    \item A customer reports that an optical drive in a PC is no longer
responding. Which of the following statements best describes the
first question you should ask the customer?
    \begin{itemize}
        \item Answer: A
Explanation: You should first ask if anything has changed since the optical drive
worked properly.
    \end{itemize}
    \item A coworker is traveling to Europe and is bringing her desktop
computer. She asks you what concerns there might be. Which of
the following statements best describes how to respond to the
customer? (Select the two best answers.)
    \begin{itemize}
        \item Answer: B and D
Explanation: Your coworker might need an adapter; otherwise, the plug may not fit in
some countries’ outlets. Some power supplies have selectors for the United States and
Europe (115 and 230 volts). If the wrong voltage is selected, the power supply will not
work and the computer will not boot; it can also be a safety concern if the voltage is
set incorrectly. Newer power supplies might auto-sense the voltage. If the power sup-
ply doesn’t have one of those red switches, check the documentation to see if it can
switch the voltage automatically.
    \end{itemize}
    \item After you remove malware/spyware from a customer’s PC for the
third time, which of the following steps should be taken next?
    \begin{itemize}
        \item Answer: C
Explanation: Teach the user how to avoid this problem by recommending safe comput-
ing practices. The customer will then be more likely to come back to you with other
computer problems. ’Nuff said.
    \end{itemize}
    \item You are asked to fix a problem with a customer’s Active Directory
Domain Services domain controller that is outside the scope of
your knowledge. Which of the following statements best describes
the recommended course of action?
    \begin{itemize}
        \item Answer: D
Explanation: Make sure that the customer has a path toward a solution before dismiss-
ing the issue.
    \end{itemize}
    \item When you are working on a computer, which of the following
should be disconnected to prevent electrical shock? (Select the
two best answers.)
    \begin{itemize}
        \item  Answers: C and D
Explanation: The power cord carries 120 volts at 15 amps or 20 amps, with all of the
obvious danger that such voltage and amperage entails. While normally low voltage, a
landline telephone cord carries 80 volts when the phone rings. That and network cables
can also be the victims of power surges from central office or networking equipment. It
is important to disconnect these before servicing a computer.
    \end{itemize}
    \item You are troubleshooting a Windows Server computer that you have
little knowledge about. The message on the screen says that there
is a “DHCP partner down” error. No other technicians are available
to help you, and your manager wants the server fixed ASAP or you
are fired. Which of the following statements best describes the
recommended course of action? (Select the two best answers.)
    \begin{itemize}
        \item Answers: A and D
Explanation: You should attempt to identify the problem and call Microsoft tech sup-
port (or contact them in another manner). The message tells you that the DHCP part-
ner is down. This means that there are two DHCP servers, one acting as a failover. As
part of your identification of the problem, you should access the TechNet, for example:
    \end{itemize}
    \item Which of the following protects confidential information from
being disclosed publicly?
    \begin{itemize}
        \item Answer: A
Explanation: The classification of data helps prevent confidential information from
being publicly disclosed. Some organizations have a classification scheme for their
data, such as normal, secret, and top secret. Policies are implemented to make top
secret data the most secure on the network. By classifying data, you are determining
who has access to it. This is generally done on a need-to-know basis.
    \end{itemize}
    \item Programs that run when Windows starts are stored in which of the
following registry hives?
    \begin{itemize}
        \item Explanation: HKEY\_LOCAL\_MACHINE is the Registry hive that stores information
about the programs Windows runs when it starts
    \end{itemize}
    \item Typically, which of the following Windows tools enables you to
configure a SOHO router?
    \begin{itemize}
        \item Answer: D
Explanation: The first thing you need to supply is the driver for any special drives,
such as new SCSI drives, SAS drives, or RAID controllers. That, of course, is optional.
If you have a typical SATA drive, Windows should recognize it automatically.
    \end{itemize}
    \item Which of the following steps is performed first when running a
clean install of Windows on a brand new SAS hard drive?
    \begin{itemize}
        \item Answer: D
Explanation: The first thing you need to supply is the driver for any special drives,
such as new SCSI drives, SAS drives, or RAID controllers. That, of course, is optional.
If you have a typical SATA drive, Windows should recognize it automatically.
    \end{itemize}
    \item A coworker maps a network drive for a user, but after rebooting,
the drive is not seen within Explorer. Which of the following steps
should be taken first to ensure that the drive remains mapped?
    \begin{itemize}
        \item Answer: A
Explanation: Although Windows has the Reconnect at Sign In check box selected by
default, it could have been disabled.
    \end{itemize}
    \item Based on the physical hardware address of the client’s network
device, which of the following is commonly used to restrict access
to a network?
    \begin{itemize}
        \item Answer: C
Explanation: MAC filtering is used to restrict computers from connecting to a network;
it is based on the physical Media Access Control (MAC) address of the computer’s net-
work adapter. It works with wired or wireless connections.
    \end{itemize}
    \item A print job fails to leave the print queue. Which of the following
services may need to be restarted?
    \begin{itemize}
        \item Answer: B
Explanation: The Print Spooler needs to be restarted on the computer that started the
print job or the computer that controls the printer. This can be done in the Services
console window or in the Command Prompt with the net stop spooler and net
start spooler commands, or anywhere else that services can be started and
stopped, such as the Task Manager.
    \end{itemize}
    \item After installing a network application on a computer running
Windows 10, the application does not communicate with the
server. Which of the following actions should be taken first?
    \begin{itemize}
        \item Answer: C
Explanation: Adding the port number and name of service to the Windows Defender
Firewall Exceptions list is the correct answer. But I’m going to pontificate more, as I
usually do.
    \end{itemize}
    \item A customer reports a problem with a PC located in the same room
as cement testing equipment. The room appears to have adequate
cooling. The PC will boot up but locks up after 5–10 minutes of
use. After a lockup, it will not reboot immediately. Which the following
statements best describes the most likely problem?
    \begin{itemize}
        \item Answer: B
Explanation: The PC air intakes are probably clogged with cement dust. This stops
fresh, cool air from entering the PC and causes the CPU to overheat. That’s why the
system doesn’t reboot immediately; the CPU needs some time to cool down. You
should install a filter in front of the PC air intake and instruct the customer to clean the
filter often. While you are working on the computer, you should clean out the inside
of the system and vacuum out the exhaust of the power supply (without opening the
power supply, of course).
    \end{itemize}
    \item One of your Windows users is trying to install a local printer and
is unsuccessful based on the permissions for the user account.
Which of the following types best describes this user account?
    \begin{itemize}
        \item Answer: C
Explanation: The Guest account is the most likely answer here. This account has the
fewest privileges of all Windows accounts. It cannot install printers or printer drivers.
By the way, Standard users can also have issues with printers depending on the ver-
sion of Windows and the policies involved. But the Guest has absolutely no administra-
tive powers whatsoever.
    \end{itemize}
    \item When accessing an NTFS shared resource, which of the following
are required? (Select the two best answers.)
    \begin{itemize}
        \item Answers: B and D
Explanation: The share-level permissions must first be set to enable access to the
user. Then the NTFS file-level “user” permissions must also be set; the most restrictive
of the two will take precedence (usually this is configured as NTFS being more restrictive).
    \end{itemize}
    \item You are contracted to recover data from a laptop. In which two
locations might you find irreplaceable, valuable data? (Select the
two best answers.)
    \begin{itemize}
        \item Answers: C and D
Explanation: Pictures and email are possibly valuable, and definitely irreplaceable, if
there is no backup.
    \end{itemize}
    \item Which utility enables auditing at the local level?
    \begin{itemize}
        \item Answer: B
Explanation: Of all the answers, the only one that deals with the local level is Local
Security Policy.
    \end{itemize}
    \item A customer has forgotten his password. He can no longer access
his company email address. Which of the following statements
best describes the recommended course of action?
    \begin{itemize}
        \item Answer: B
Explanation: In many cases, passwords cannot be reset by the user or by the systems
admin. If that is the case, you need to verify the identity of the person first. You might
need to do so just as a matter of organizational policy
    \end{itemize}
    \item Which of the following can help locate a lost or stolen mobile
device?
    \begin{itemize}
        \item Answer: C
Explanation: GPS can help to locate a stolen or lost mobile device. Plenty of third party
programs allow the user to track the device, as long as it is on and has GPS
installed and functioning. If the device is off, the program will display the last known
good location.
    \end{itemize}
    \item Which of the following can be disabled to help prevent access to a
wireless network?
    \begin{itemize}
        \item Answer: B
Explanation: To aid in preventing access to a wireless network, disable the SSID. But
only do this when all computers have been connected. If more computers need to be
connected later, they will have to connect manually, or the SSID will have to be reenabled.
    \end{itemize}
    \item Which of the following commands sets the time on a workstation?
    \begin{itemize}
        \item Answer: A
Explanation: If you are just setting the time on the computer, use the time command.
Time can also be set in Windows within the Notification Area. This is a bit of a trick
question because you are dealing only with local time, not anything network-related.
So the rest of the answers are incorrect
    \end{itemize}
    \item In Windows, which utility enables you to select and copy characters
from any font?
    \begin{itemize}
        \item Answer: D
Explanation: The Character Map enables you to copy characters from any font type.
To open it, go to Run and type charmap. In Windows 10, go to Start > Windows
Accessories > Character Map. In Windows 7, go to Start > All Programs > Accessories >
System Tools > Character Map. Otherwise, in any version of Windows, you can locate
it simply by searching for it by name.
    \end{itemize}
    \item Which of the following can be described as removing the limitations
of Apple iOS?
    \begin{itemize}
        \item Answer: B
Explanation: Jailbreaking is the process of removing the limitations of an Apple
device’s iOS. It enables a user to gain root access to the system and download previously
unavailable applications, most likely unauthorized by Apple.
    \end{itemize}
    \item In Windows, which of the following built-in applets should be used
by a technician to enable and manage offline files, view conflicts
and partnerships, and ensure locally stored files match those
stored on an external device or server?
    \begin{itemize}
        \item Answer: D
Explanation: The Sync Center is located within the Control Panel or can be found using
the search tool. It allows you to set up synchronization partnerships with external
devices and enables you to manage offline files. Sometimes, the individual icons within
the Control Panel are referred to as applets.
    \end{itemize}
    \item Which language support for representing characters is built into
Windows?
    \begin{itemize}
        \item Answer: A
Explanation: Unicode is the code used to represent characters among multiple computers’
language platforms. It is commonly used in Microsoft Word and other Office
programs. For example, to show the logical equivalence symbol, you would type
U+2261, then highlight that text, and then press the Alt+X shortcut on the keyboard,
which changes the text into the symbol .Unicode works regardless of the language
a person is working in.
    \end{itemize}
    \item Which of the following is the best source of information about
malicious software detected on a computer?
    \begin{itemize}
        \item Answer: B
Explanation: New malicious software (malware) is always being created. Because of
this, the best place to find information about spyware, a virus, rootkit, ransomware,
or other malware is at a place that can be updated often and easily: the anti-malware
company’s website.
    \end{itemize}
    \item You are working for a company as a roaming PC tech and have
been assigned work by a network administrator. The admin notifies
you that the company is experiencing a DDoS attack. Half a dozen
internal Windows PCs are the source of the traffic. The admin
gives you the Windows computer names and tells you that they
must be scanned and cleaned immediately. Which of the following
effects to the PCs should you as a PC technician focus on fixing?
(Select the two best answers.)
    \begin{itemize}
        \item Answers: A and D
Explanation: The Windows PCs have probably been infected by a worm and have been
compromised and turned into zombies (bots). Trojans could also be involved in this scenario.
The Windows PCs are probably part of a botnet that includes other computers as
well. The botnet is orchestrated by a master computer that initiates the DDoS (distributed
denial-of-service) attack. The infections that you as the technician will have to remove
include the worm and the zombie program (or script).You might also be informed that
the systems need to be isolated, wiped, and re-imaged before they can be used again.
    \end{itemize}
    \item You are troubleshooting a networking problem with Windows, and
you can’t seem to fix it using the typical Windows GUI-based troubleshooting
tools or with the Command Prompt. You have identified
the problem and established a theory of probable cause. (In
fact, you are on your fourth theory.) Which tool should be used to
troubleshoot the problem, and in what stage of the troubleshooting
process should you do so?
    \begin{itemize}
        \item Answer: D
Explanation: Use the Registry Editor (regedit.exe) to try troubleshooting the problem
if typical GUI-based and Command Prompt methods have provided no resolution.
The Registry Editor allows you to do any configuration necessary in Windows, and
using it may be necessary for more complex troubleshooting problems. At this point
you are testing the theory to determine cause because you have already identified the
problem and established a theory of probable cause. Remember your CompTIA A+
troubleshooting theory from the 220-1001 objectives. I’ve listed them below.
    \end{itemize}
    
\end{enumerate}
\subsection{Practice Exam C}
\begin{enumerate}
    \item You work as a technician for an organization that has a custom web-based application that is used for the monitoring of networking devices. While using a web browser to access the application, you press F12, and 
    within the js folder, you see the following code: 
    \begin{lstlisting}
    $(function(){
     (function())
     {
        var selects = $('select[data-toggle="collapse"]')$
     }
    }$
    \end{lstlisting}
    \begin{itemize}
        \item D. avaScript
    \end{itemize}
    \item Viruses have been detected and removed on a customer's computer several times during the course of several weeks. Which of the following methods will best help prevent future occurrences? 
    \begin{itemize}
        \item Answer: D
Explanation: Because this situation happens often, you should school the user on safer
web browsing habits such as being very careful when clicking on links brought up by
search engines, not clicking on pop-up windows, and being conservative about the
websites that are accessed. Also, the browser can be updated, add-ons can be installed
to the web browser for increased protection, phishing filters can be enabled, and
so on.
    \end{itemize}
    \item Which of the following sends an invitation by email asking for
help?
    \begin{itemize}
        \item Answer: D
Explanation: Connections can be made by sending Remote Assistance invitations by
email (Outlook or other email client) or Easy Connect. These invitations could be to ask
for help or to offer help. This approach is often implemented in help desk scenarios
in which a user invites a technician to take control of his computer so that it can be
repaired. It’s effectively a virtual service call. The technician doesn’t need to come
physically to the user’s desk but instead connects remotely.
    \end{itemize}
    \item When you are performing a clean installation, which of the follow-
ing is the default location for the system files of Windows?
    \begin{itemize}
        \item C. C:WindowsSystem32
    \end{itemize} 
    \item You are required to set up a remote backup solution for music and
photos stored on an Android tablet. The files cannot be stored at
any company location. Which technology should be used?
    \begin{itemize}
        \item Answer: B
Explanation: You would use the Google Cloud solution so that files can be backed up
to a location outside the company. This backup—or full synchronization method—is
great for Android-based smartphones or tablets as well as Google Chromebooks.
Several other third-party solutions are available as well.
    \end{itemize}
    \item You have been contracted to repair a computer at an organization
that has strict rules about information leaving the premises. While
troubleshooting the computer, you determine that the computer
should be taken offsite to complete the repair. Which of the follow-
ing should you do next?
    \begin{itemize}
        \item Answer: C
Explanation: You should check the company’s policies and procedures first (or inquire
with a compliance officer). If there is confidential or proprietary information that should
not leave the premises (under normal c
    \end{itemize}
    \item You need to copy and paste information from a web page, but you
want to remove all formatting so that it can be pasted cleanly into
Word. Which program should be used as an intermediary?
    \begin{itemize}
        \item Answer: C
Explanation: Use Notepad. This text-based editor applies virtually no formatting. Text
and other information can be copied from a web page, pasted to a Notepad document,
and then copied again and pasted into Word; all formatting is removed. Notepad (and
third-party tools such as Notepad++) can also be used for scripting and web page
development.
    \end{itemize}
    \item A computer is responding slowly, and the Windows Task Manager
shows that spoolsv.exe is using 95 percent of system resources.
Which of the following is most likely the cause of this problem?
    \begin{itemize}
        \item Answer: D
Explanation: The printing subsystem is most likely failing for one of a variety of reasons.
The first solution is to terminate spoolsv.exe (which is the Print Spooler service)
in the Task Manager or in the Command Prompt with the taskkill command.
Then restart the computer. If that approach doesn’t work, the system may have to be
repaired, restored, or modified in the Registry (which could be an in-depth process).
It is also possible that a virus has compromised the system. There are viruses that are
also called spoolsv.exe; a quick sweep of the system folders with AV software should
uncover this…hopefully.
    \end{itemize}
    \item Which of the following descriptions classifies the protocol IMAP?
    \begin{itemize}
        \item Answer: B
Explanation: IMAP is the Internet Message Access Protocol, which allows an email
client to access email on a remote mail server. Generally, the email client software
leaves the messages on the server until the user specifically deletes them. So, the user
can selectively download messages. This allows multiple users to manage the same
mailbox.
    \end{itemize}
    \item From which of the following locations could you disable a
hardware component on a laptop in Windows?
    \begin{itemize}
        \item Answer: A
Explanation: Use the Device Manager to disable a component in Windows, regardless
of whether it is a laptop or a PC. When you disable a device, a down arrow appears
over the icon of the device, next to the name.
    \end{itemize}
    \item Which command-line tool in Windows finds all of the unsigned
drivers in the computer?
    \begin{itemize}
        \item Answer: A
Explanation: The sigverif.exe tool can be used to check for unsigned drivers within
your Windows operating system. Unsigned drivers are those that have not been verified
by Microsoft. If you receive error messages and are troubleshooting, run this command
from the Run prompt. When the check is finished, unsigned drivers are displayed. This
list is also stored in a file called sigverif.txt within the
    \end{itemize}
    \item Users are reporting to you that a Windows feature asks them for
confirmation before running certain applications or when making
system changes. What is the name of this Windows feature, and
where should you direct users to turn the functionality off?
    \begin{itemize}
        \item Answer: D
Explanation: User Account Control (UAC) is the portion of Windows that asks for confirmation
of administrative rights before allowing a user to make system changes or
run certain applications. It can be disabled within the User Accounts applet within the
Control Panel by clicking the Change User Account Control Settings link. But beware;
only users who have administrative rights should even be permitted to turn off this
setting. UAC can be further configured in the Group Policy Editor and in the Registry
Editor. For more information about how UAC works, see the following link:
    \end{itemize}
    \item James is a LAN administrator in charge of printers. Which of the
following should he check first when a Windows user is trying to
print a document and gets the error message “Print sub-system
not available”?
    \begin{itemize}
        \item Answer: C
Explanation: If a “print sub-system not available” message or similar message
appears, it most likely means the spooler has stalled. You can turn it back on within the
Services section of Computer Management or by issuing the command net start
spooler at the Command Prompt.
    \end{itemize}
    \item Your manager’s Windows computer locks up after the graphical
user interface starts to load. However, the computer will boot in
Safe Mode. When you access the Event Viewer, you see an entry
stating that a driver failed. Which of the following steps will help
you further diagnose the problem?
    \begin{itemize}
        \item Answer: B
Explanation: Boot Logging can be enabled from the Windows Recovery Environment
(WinRE) in Startup Settings or in the Windows Advanced Boot Options menu. After
this option is enabled, the system automatically creates a file called ntbtlog.txt.
Afterward, you can access the system by booting into Safe Mode, once again from the
recovery environment.
    \end{itemize}
    \item Which of the following commands is used to fix errors on the
system disk?
    \begin{itemize}
        \item Answer: D
Explanation: Chkdsk /F allows you to fix errors on a disk. It does not fix all errors,
but it checks for disk integrity, bad sectors, and similar issues.
    \end{itemize}
    \item You are troubleshooting a computer that has a web browser issue.
The end user says that multiple browser pages open by themselves
when surfing the Internet. Also, you observe that the computer is run-
ning slowly. Which of the following actions should you perform first?
    \begin{itemize}
        \item Answer: A
Explanation: The first thing you should do is install anti-malware software. It would be
surprising if the computer doesn’t have any, but it happens.
    \end{itemize}
    \item A new program is crashing and causing the computer to lock up.
What is the best location to check for further information about the
cause of the crash?
    \begin{itemize}
        \item Answer: C
Explanation: The Application log is the location for all events concerning Windows
applications and third-party programs.
    \end{itemize}
    \item You are tasked with disabling services from starting on a Windows
PC. Which command should be run to bring up a window to make
these changes?
    \begin{itemize}
        \item Answer: C
Explanation: The Application log is the location for all events concerning Windows
applications and third-party programs.
    \end{itemize}
    \item In Windows, which of the following folders might be stored in a
hidden partition by default?
    \begin{itemize}
        \item Answer: A
Explanation: The Boot folder can be located in a hidden partition (100 MB in size), by
default, which is separate from the C: drive.
    \end{itemize}
    \item One of your customers has a wireless network that is secured with
WEP. The customer wants to improve data encryption so that the
transmission of data has less of a chance of being compromised.
Which of the following statements best describes the recommend-
ed course of action?
    \begin{itemize}
        \item Answer: A
Explanation: The best solution is to upgrade the wireless network from WEP to at least
WPA2. WEP is a deprecated wireless encryption protocol and should be updated to a
newer and more powerful protocol if at all possible. If this is not possible, it would be
wise to use a strong WEP key and modify it often.
    \end{itemize}
    \item Which of the following commands is used to display hidden files?
    \begin{itemize}
        \item Answer: B
Explanation: Dir /a can be used to display hidden files. Specifically, dir /ah can
be used to show hidden files only.
    \end{itemize}
    \item After you install a new video card, the PC loads Windows and
continuously reboots. Which of the following statements best
describes the first course of action?
    \begin{itemize}
        \item Answer: A
Explanation: Try accessing Safe Mode first and see if the problem continues. It probably
won’t, and you will need to roll back the driver and locate, download, and install
the correct one. Remember to get your drivers from the manufacturer’s website, and
don’t forget to download the correct driver for your particular operating system.
    \end{itemize}
    \item Which of the following statements best describes how to prepare
a mobile device in case it is stolen or lost? (Select the three best
answers.)
    \begin{itemize}
        \item Answers: B, D, and F
Explanation: First, you should configure some kind of remote backup. This way, if the
device is compromised, you have the confidential data backed up outside of the device
at another location. The other half of this solution (not mentioned in the answers) is
remote wipe. When you are positive that the device is stolen or lost, and you know the
data was backed up at some point, trigger a remote wipe to remove all data from the
device. Second, enable GPS on the device so that it can be tracked if it is lost or stolen.
Third, configure a screenlock of some sort, be it a pattern that is drawn on the display,
a PIN, or a password. A strong password is usually the best form of screenlock and
the hardest to crack.
    \end{itemize}
    \item Two coworkers share the same file inside a folder. User A works on
the file, makes changes, and saves the file. User B then works on
the file, makes changes, and saves the file as well. The next time
User A attempts to open the file, she receives an access denied
error. Which of the following statements best describes the most
likely cause of this error message?
    \begin{itemize}
        \item Answer: D
Explanation: Most likely User B moved the file to another location outside of the current
partition, made the changes (which is possible since User B is the one who moved
it), and then moved it back to the original location. Whenever a file is moved to another
partition or volume, the file takes on the permissions of the parent folder. However, if
the file had been moved within the volume, the permissions would have been retained.
Tricky. Remember this: If the file is moved within the same volume, it retains permissions,
so the permissions don’t change. But if a file is moved to another volume, it
takes on the permissions of the folder it is moved into. As for copying, the file’s copy
always takes on the permissions of the parent regardless of where that copy is placed.
    \end{itemize}
    \item In Windows, which of the following commands should be used
to verify that a previous system shutdown was completed
successfully?
    \begin{itemize}
        \item Answer: B
Explanation: Chkntfs can check to see whether a previous system shutdown completed
successfully. This command must be run in elevated mode to function properly.
Generally, you would check this on the system drive (for example, C:). If the drive is
okay and the system did complete the shutdown successfully, you’ll get a message
such as “C: is not dirty.” Otherwise, you’ll get a message telling of the error.
    \end{itemize}
    \item Which of the following are the best answers for securing a data
center? (Select the two best answers.)
    \begin{itemize}
        \item Answers: B and E
Explanation: The badge reader and biometric lock are the best of the listed answers
(although all kinds of other security methods are possible). This scenario is an example
of multifactor authentication (MFA). An RFID-based badge reader relies on something
a person has, and the biometric lock system relies on something the user is. MFA
systems are more secure because they layer the security.
    \end{itemize}
    \item Which of the following is the best Windows utility to use if an
administrator wants to perform administrative tasks that integrate
scripts over a network?
    \begin{itemize}
        \item Answer: A
Explanation: The Windows PowerShell is the best of the listed Windows utilities that
enables administrators to perform administrative tasks that integrate scripts and executables
and can be run over a network. For even more power and flexibility, use the
PowerShell Integrated Scripting Environment (PowerShell ISE).
    \end{itemize}
    \item Which of the following can be used to kill a running process?
    \begin{itemize}
        \item Answer: A
Explanation: The Task Manager can end (or “kill”) a running process. It is also used to
end applications that lock up, and it analyzes the performance of the system.
    \end{itemize}
    \item Which of the following file systems is suited specifically for USB
flash drives?
    \begin{itemize}
        \item Answer: B
Explanation: exFAT (also known as FAT64) is suited specifically for USB flash drives
and many other mobile storage solutions. It is the successor to FAT32 and can format
media that is larger than 32 GB with a single partition.
    \end{itemize}
    \item A program has been detected collecting information such as the
computer name and IP address and sending that information to a
specific IP address on the Internet. Which kind of threat is this an
example of?
    \begin{itemize}
        \item Answer: A
Explanation: Spyware is a type of malicious software that is usually downloaded unwittingly
by a user or is installed by third-party software. It collects information about the
user and the user’s computer without the user’s consent.
    \end{itemize}
    \item You are required to stop the Windows Defender Firewall service.
Which of the following best describes how to accomplish this?
(Select the three best answers.)
    \begin{itemize}
        \item Answers: B, D, and G
Explanation: You can stop a service in a variety of ways. The easiest and most common
is to go to the Services console window. You can do this by typing services.
msc at the Run prompt. You can also stop services in the Task Manager by accessing
the Services tab and right-clicking the service in question. But in the Task Manager you
have to know the executable name of the service. The name of the Windows Firewall
service is mpssvc. So, the third way (of the listed answers) is to use the net stop
mpssvc command in the Command Prompt.
    \end{itemize}
    \item You spill a chemical on your hands. It does not appear to be life
threatening. Which of the following statements best describes the
recommended course of action?
    \begin{itemize}
        \item Answer: C
Explanation: If the chemical spill is not life threatening, consult the material safety data
sheet (MSDS) to determine the proper first aid (if any).
    \end{itemize}
    \item Which command allows a user to change a file’s permissions in Linux?
    \begin{itemize}
        \item Answer: D
Explanation: The chmod command allows a user to modify file and folder permissions
at the Linux command line.
    \end{itemize}
    \item While you are working on a computer at a customer’s home, the
customer informs you that he needs to leave for about 10 minutes
and that his eight-year-old son can help you with anything if you
need it. Which of the following statements best describes the recommended
course of action??
    \begin{itemize}
        \item Answer: C
Explanation: Whenever you’re working in someone’s home, make sure that an adult is
available.
you work.
    \end{itemize}
    \item You want a cloud provider that will offer you service which is
quickly scalable. Which of the following should be requested when
you contact potential cloud providers?
    \begin{itemize}
        \item Answer: B
Explanation: Rapid elasticity means that the service can be scalable at need and can
grow in real time with your company’s growth.
    \end{itemize}
    \item You have been asked to recommend an anti-malware program for
a home user. However, the user does not want to pay for a license.
Which of the following should you suggest?
    \begin{itemize}
        \item Answer: C
Explanation: An open license means that the software can be downloaded and used for
free.
    \end{itemize}
    \item A customer experiences a server crash. When you arrive, the
manager is upset about this problem. Which of the following statements
best describes the recommended course of action?
    \begin{itemize}
        \item Answer: A
Explanation: Stay calm and do the job as efficiently as possible. There isn’t much you
can do when a customer is upset except fix the problem.
    \end{itemize}
    \item Which type of web server is designed to resolve hostnames to
IP addresses?
    \begin{itemize}
        \item Answer: D
Explanation: A Domain Name System (DNS) server is designed to translate hostnames
(such as dprocomputer.com) to their corresponding IP addresses (for example,
65.18.242.1).
    \end{itemize}
    \item As you are servicing a manager’s PC at your company, you run
across a list of names of employees who are supposedly about to
be let go from the company. Some of these people are coworkers.
Which of the following statements best describes the recommended
course of action?B. Act as if you never saw the list.
    \begin{itemize}
        \item Answer: B
Explanation: There isn’t much you can do in a situation like this, especially if you
already saw what was printed on the document. The best thing is to ignore it and act
as if it never happened. It’s not your place to take action based on a document that
is lying around. Without intense scrutiny, it is hard to know exactly what a document
is. The purported list might be real, but it might not be. It isn’t your call to make.
However, before working at a customer site, you should ask that all confidential materials
be removed before you begin work. If something is left out in plain sight, you could
let a manager know that there could be confidential data lying around.
    \end{itemize}
    \item Which macOS utility is most like Windows’ “end task” feature?
    \begin{itemize}
        \item Answer: D
Explanation: The force quit option in Apple’s macOS is most like the “end task” feature
in the Windows Task Manager. It helps when an application is not functioning as
intended and is either frozen or intermittently slows down the system.
    \end{itemize}
    \item Which of the following statements best describes how to reduce
the chance of ESD? (Select the three best answers.)
    \begin{itemize}
        \item Answers: A, B, and D
Explanation: To reduce the chance of electrostatic discharge (ESD), use an antistatic
wrist strap and mat. If connected properly, they become suitable methods of selfgrounding.
Also, consider raising the humidity. The more humidity there is, the less
friction, and ultimately, less ESD.
    \end{itemize}
    \item While you explain a technical concept to a customer, which of the
following statements best describes the recommended course of
action?
    \begin{itemize}
        \item Answer: B
Explanation: Make the customer truly feel comfortable by sitting down next to her and
taking the time to explain the technical concept from a simple and concise point of
view. The less jargon, the better.
    \end{itemize}
    \item You are viewing the contents of an ipconfig /all on a
Windows computer. You see the name dpro42.com toward the
beginning of the results. Which type of network is this Windows
computer most likely a part of?
    \begin{itemize}
        \item Answer: C
Explanation: If you see the name dpro42.com toward the beginning of the results of
an ipconfig /all command, the computer is most likely a part of the dpro42.com
domain. This would be listed in the Primary DNS Suffix entry, which is usually directly
after the Host Name entry. The .com is the giveaway. Some kind of DNS extension
(such as .com or .net) is necessary when you have a domain
    \end{itemize}
    \item Which of the following should be used to clean a monitor’s screen
when you are not sure how to do so?
    \begin{itemize}
        \item Answer: C
Explanation: If you are not sure about what to clean a screen with, use water. Water
will most likely not damage the screen.
    \end{itemize}
    \item You are required to register an ActiveX control in the Command
Prompt. Which utility should be used?
    \begin{itemize}
        \item Answer: A
Explanation: The regsvr32 command is used to register and unregister ActiveX controls
and Dynamic-Link Libraries (DLLs). For example, to register the Sample ActiveX
control, you would type regsvr32 sample.ocx.
    \end{itemize}
    \item As part of the risk management of your company, you have been
tasked with backing up three physical servers on a daily basis.
These backups will be stored to a NAS device on the LAN. Which
of the following can you do to make sure the backup will work if
needed?
    \begin{itemize}
        \item Answer: A
Explanation: The best option here is to create alerts to let any and all administrators
know if a backup failure occurs. These alerts would either be created at the network attached
storage (NAS) device or at the individual servers to be backed up. If an admin
receives an alert, that person will know to either rerun the backup or (more likely) fix
the backup task and then run it. One of the issues here is that you might not know if a
backup fails—without the alerts, that is.
    \end{itemize}
    \item You have an Intel Core i7 system with a UEFI-enabled motherboard.
Which of the following types of hard drive partitioning
schemes should be selected when installing Windows?
    \begin{itemize}
        \item  Answer: D
Explanation: If your system’s motherboard is equipped with a UEFI BIOS, you should
definitely take advantage of the GUID Partitioning Table (GPT). It is superior to Master
Boot Record (MBR) technology. It allows for up to 128 partitions, is not limited to the
2-TB maximum partition size of MBR, and stores multiple copies of itself on the
system.
    \end{itemize}
    \item Which of the following statements best describes the recommended
course of action to take prior to attempting to remediate
infected Windows systems of malware?
    \begin{itemize}
        \item Answer: B
Explanation: You should disable System Restore on Windows systems just before
attempting to remediate the system of malware. This is step 3 of the CompTIA A+ best
practices/procedure for malware removal. The entire procedure is as follows:
    \end{itemize}
    \item A customer’s mobile device cannot connect to Wi-Fi. According to
the customer, it was working fine yesterday. Troubleshoot! Which
of the following statements best describes the recommended
course of action? (Select the three best answers.)
    \begin{itemize}
        \item Answers: A, D, and E
Explanation: If a mobile device cannot connect to the network, you should attempt to
power cycle the device, forget and reconnect to the Wi-Fi network, and check if the
correct SSID was entered in the first place. Perhaps the \#1 method would be to power
cycle Wi-Fi (not listed in the answers).
    \end{itemize}
    \item Which of the following utilities enables a Windows user to edit a
file offline and then automatically update the changes when the
user returns to the office?
    \begin{itemize}
        \item Answer: A
Explanation: The Sync Center is a Windows feature that enables you to keep information
synchronized between your computer and network servers. You can still access
the files and modify them even if you don’t have physical access to the server; in this
case they are modified “offline” and are synchronized automatically when you return
to the network. Some mobile devices are also compatible with Sync Center. The Sync
Center can be configured within the Control Panel.
    \end{itemize}
    \item A help desk phone support technician is finding it difficult to
understand the customer due to a heavy accent. Which of the
following statements best describes the first course of action the
technician should take to help the customer resolve the problem?
    \begin{itemize}
        \item Answer: A
Explanation: The technician should repeat the problem back to the customer to make
sure that everyone is talking about the same thing and that both parties understand
each other. Always clarify.
    \end{itemize}
    \item Which of the following relies on PPTP to create a secure tunnel?
    \begin{itemize}
        \item Answer: C
Explanation: Virtual private networks (VPNs) rely on a tunneling protocol such as
Point-to-Point Tunneling Protocol (PPTP) or Layer 2 Tunneling Protocol (L2TP) to
create a secure connection between a network and a remote computer or group of
computers. The preferred method for Windows clients is to use Internet Key Exchange
version 2 (IKEv2). You might also make use of a RADIUS server for authentication or
use an always-on VPN solution such as OpenVPN.
    \end{itemize}
    \item Which of the following will occur if temp is executed from
Run?
    \begin{itemize}
        \item Answer: C
Explanation: Entering
    \end{itemize}
    \item Which group is best to assign to a home user to prevent software
installation?
    \begin{itemize}
        \item Answer: D
Explanation: The standard user cannot install software or make changes to the system
without knowing an administrative login.
    \end{itemize}
    \item A Windows PC is not booting correctly. You need to locate bad
sectors and recover information. Which command is best?
    \begin{itemize}
        \item Answer: A
Explanation: Chkdsk /R locates bad sectors and recovers the information from
them.
    \end{itemize}
    \item One of your coworkers has a smartphone that contains PII.
Because the data is required for use and is valuable, the coworker
cannot have the phone automatically wiped if it is lost or stolen.
Which of the following is the best way to secure the device?
    \begin{itemize}
        \item Answer: D
Explanation: Of the listed answers, a fingerprint is the best way to secure the smartphone.
If the smartphone is lost or stolen, another person would have a difficult time
unlocking the device (though not impossible). For a device that cannot be remote
wiped (for various reasons), the best alternatives are the use of biometric authentication,
in combination with a strong password (for MFA); plus encryption.
    \end{itemize}
    \item Where can a user’s Desktop folder be found in Windows by
default?
    \begin{itemize}
        \item Answer: A
Explanation: Every user profile gets a Desktop folder by default. This folder is located
within the user profile folder, which is shown in the answer as a variable
    \end{itemize}
    \item A user who is part of a workgroup reports that she cannot print to
a new printer. Everyone else in the workgroup can print to the new
printer, and the user can still automatically send print jobs to the
old printer. Which of the following statements describes how to
remedy the problem? (Select the two best answers.)
    \begin{itemize}
        \item Answers: A and D
Explanation: If a user cannot print to a brand-new printer, yet everyone else can print
to it, you should check whether the printer is installed on that user’s computer and if it
is set as the default printer.
    \end{itemize}
    \item Your organization has an Active Directory domain. One of the
users, Bill, should not have read access to a folder named
Accounting. The Accounting folder is shared on a network server,
on a partition formatted as NTFS. Which of the following statements
best describes how to stop Bill from having read access to
the folder without impacting any other users on the network?
    \begin{itemize}
        \item Answer: D
Explanation: The best option in this scenario would be to deny read access to the
Accounting folder for Bill through shared access security.
    \end{itemize}
    \item Examine the following figures 
    \begin{itemize}
        \item Answer: B
Explanation: The Disk Management component of Computer Management is displayed
in the figure. You can tell because it shows each disk and the volumes within each
disk.
    \end{itemize}
    \item Which of the following the best Windows utility to back up important sytem settings without requiring external storage? 
    
    \begin{itemize}
        \item Answer: C
Explanation: System Protection is a feature that creates and saves data about the computer’s
system files and settings. It does this by creating restore points. You access it
by going to the System Properties dialog box and clicking the System Protection tab.
External storage is not necessary for these restore points; they are automatically stored
in the system volume. 
    \end{itemize}
    \item Your boss wants to encrypt a hard drive that will store critical data. Your boss needs to be able to drag and drop folders onto the volume and have them encrypted in real time. Which encryption techniques should you suggests? 
    \begin{itemize}
        \item Answer: A
Explanation: BitLocker is a type of whole-disk encryption, or WDE. It encrypts all of
the contents that are created on it or copied to it in real time. It requires a trusted
platform module (TPM) on the motherboard or an encrypted USB flash drive. Only
select editions of Windows support BitLocker when used in this manner. Other lesser
versions of Windows are compatible with BitLocker To Go for reading encrypted documents
from USB flash drives.
    \end{itemize}
    \item Your boss asks you to troubleshoot a computer with a virus.
Which of the following statements best describes the first step you
should take to remedy the problem?
    \begin{itemize}
        \item Answer: B
Explanation: The first thing you should do is identify the malware. (By the way, if the
computer is on the network, disconnect it first.) Then you can research that malware
and any possible cures by searching the Internet and accessing your AV provider’s
website.
    \end{itemize}
    \item User A is part of the Users Group
    \begin{itemize}
        \item Answer: A
Explanation: User A will end up having the Read Only level of access to the share.
Generally, a user gets the more restrictive level of access. The only thing that is different
between the share’s permissions and the parent directory’s permissions is the level
of control for the Users group. Normally, a share will obtain its permissions from the
parent folder—that is, unless that option is unchecked in the properties of the folder.
Then the folder can be reconfigured for whatever permissions an admin wants to set
for it. That must be what happened in this scenario.
    \end{itemize}
    \item Your boss wants to implement BitLocker on yet a second laptop
for traveling purposes. Which of the following should be performed
before implementing BitLocker?
    \begin{itemize}
        \item Answer: A
Explanation: Before implementing the BitLocker solution in Windows, you should
enable the trusted platform module (TPM) in the BIOS. This is the chip on the motherboard
that includes the encryption code.
    \end{itemize}
    \item You need to edit a protected .dll file on a Windows 8.1 Pro PC, but
you cannot find the file you are looking for in the System32 folder.
Which of the following Control Panel utilities should you configure?
    \begin{itemize}
        \item Answer: D
Explanation: Use the Folder Options utility in the Control Panel of Windows 8.1.
From there, you go to the View tab and then deselect the check box labeled “Hide
protected operating system files (Recommended).” You might also deselect the “Hide
extensions for known file types” check box to see which ones are .dll files. Note
that this utility was removed from the Windows 10 Control Panel. Instead, you can
access it with File Explorer Options or from the File Explorer program by choosing
View > Options > Change Folder and Search Options. Or you can go to Run and type
control folders.
    \end{itemize}
    \item One of your customers has a defective disk. Which command
can be used to extract readable information?
    \begin{itemize}
        \item Answer: A
Explanation: The Recover command can recover readable information from a bad or
defective disk. The disk should be attached (slaved) to a working computer to get back
the data
    \end{itemize}
    \item You have been asked to load a copy of the company’s purchased
software on a personal computer. Which of the following statements
best describes the first step you should take to remedy the
problem?
    \begin{itemize}
        \item Answer: A
Explanation: You should first verify that the installation is allowed under a company’s
licensing agreement. It probably isn’t, but you should check first. Most organizations
do not allow purchased software to be installed on an employee’s home computer.
If doing so is against organization policy, you should notify your supervisor. There
are many types of licenses that you should be aware of, including end-user licensing
agreements (EULA), digital rights management (DRM), commercial and enterprise
licenses (such as client access licenses or CALs), open source versus closed source
(that is, Android versus iOS), personal licenses, and so on. Again, be sure to follow
and incorporate corporate end-user policies and security best practices when it comes
to these types of licenses.
    \end{itemize}
    \item Your friend is playing the latest first-person game on a PC, but the
screen is pausing during game play. Your friend has a high-end
graphics card and the maximum memory for the motherboard.
Which of the following statements best describes how to remedy
the problem?
    \begin{itemize}
        \item Answer: A
Explanation: If you see video issues such as pausing during game play, upgrade the
video drivers. Make sure that you download the latest video driver from the manufacturer’s
website. Gamers cannot rely on Microsoft drivers.
    \end{itemize}
    \item You have been asked to move data from one user’s laptop to
another user’s laptop, each of which has EFS functioning. Which of
the following statements best describes the first step you should
take to remedy the problem?
    \begin{itemize}
        \item Answer: B
Explanation: The first thing you should do is export the user’s certificate from the first
laptop to the second laptop. You can do this by clicking Start and typing certmgr.
msc in the Search box; then locate and export the correct Personal Certificate. The
Certificates console window can also be added to an MMC. The Encrypting File System
(EFS) is the standard single-file encryption method for Windows (if the version supports
it).
    \end{itemize}
    \item Which of the following statements is true?
    \begin{itemize}
        \item Answer: B
Explanation: Authentication can be carried out by utilizing something a user is, such as
a fingerprint; something a user knows, such as a password or PIN; something a user
has, such as a smart card or token; and something a user does, such as writing a signature
or speaking words.
    \end{itemize}
    \item You are required to implement an organizational policy that states
user passwords can’t be used twice in a row. Which of the following
policies should be configured?
    \begin{itemize}
        \item Answer: B
Explanation: You should configure the Enforce password history policy and set it to
a number higher than zero. This way, when a user is prompted to change her password
every 42 days (which is the default minimum password age), that user will not
be able to use the same password. Password policies can be accessed in Windows
within Local Security Policy window > Security Settings > Account Policies > Password
Policy.
    \end{itemize}
    \item You are working on a computer in which you just installed a new
hard drive. The system already runs Windows. The new hard drive
does not appear in Explorer. Which of the following statements
best describes the next step you should take to ensure the drive
will be recognized by the operating system?
    \begin{itemize}
        \item Answer: B
Explanation: When you add a second drive to a system that already has Windows
installed, you will probably have to initialize the drive and format it in the Disk
Management utility.
    \end{itemize}
    \item An attacker is constantly trying to hack into one of your customer’s
SOHO networks. Which of the following statements best describes
the easiest, most practical way to protect the network from
intrusion?
    \begin{itemize}
        \item Answer: D
Explanation: The most practical way to prevent intrusion to the network is to install a
firewall. In fact, if this is a SOHO network, chances are the network is controlled by a
multifunction network device that already acts as a switch and a router and probably
has built-in firewall technology; it just has to be enabled. Usually, these are enabled by
default, but perhaps someone inadvertently disabled this feature, and that’s one of the
reasons an attacker keeps trying to get into the network.
    \end{itemize}
    \item One of the administrators recently moved a large chunk of data
from one server to another. Now, several users are reporting they
cannot access certain data shares and get the following error:
Access Denied. The admin confirms that the users are in the
proper security groups, but the users are still unable to access the
shares. Which of the following are the most likely causes of the
problem? (Select the two best answers.)
    \begin{itemize}
        \item Answers: A and C
Explanation: The most likely reasons the users cannot connect are because of denied
permissions and mapped drives. If the data was moved to another computer, the
folders will inherit new permissions from the parent (by default). That will most likely
eliminate the current user access. Also, the path to the share will change (again by
default). Either the server name/IP address, the sharename, or both will be different
when the data is moved to another server. So, to fix the problem, the user and group
permissions will have to be modified for the new share, and new mapped drives will
need to be configured.
    \end{itemize}
    \item Which command in the Linux terminal enables you to find out
information about a wireless network adapter?
    \begin{itemize}
        \item Answer: D
Explanation: Use iwconfig (or ifconfig) to analyze a wireless network adapter
in the Linux terminal. (Note that iwconfig does not work in macOS, but ifconfig
does.) Also, as of the writing of this book, the ip a command can be used as well.
    \end{itemize}
    \item You have a Windows computer for which you wish to write a batch
file. You want the batch file to turn off the computer after a certain
amount of time. Which main command should be run in the
batch file?
    \begin{itemize}
        \item D. Shutdown
    \end{itemize}
    \item Which switch of the Robocopy command copies subdirectories
but skips empty ones?
    \begin{itemize}
        \item /s copies subdirectories but skips any empty ones
    \end{itemize}
    \item Which of the following are components of dealing with prohibited
content? (Select the three best answers.)
    \begin{itemize}
        \item A. C. Preserving data D. Creating a chain of custody
    \end{itemize}
    \item You are designing the environmental controls for a server room
that contains several servers and other network devices. Which of
the following statements best describes the role of an HVAC
system in this environment? (Select the two best answers.)
    \begin{itemize}
        \item C. It provides an appropriate ambient temperature. D. 
    \end{itemize}
\end{enumerate}
\end{document}



