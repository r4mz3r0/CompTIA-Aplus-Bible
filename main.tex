\documentclass{article}
\usepackage[utf8]{inputenc}
\usepackage{xcolor}
\usepackage{hyperref}
\usepackage{comment}
\title{CompTIA A+ Bible}
\author{1000+ Questions and Solutions}
\date{December 2020}

\begin{document}

\maketitle
\section{Questions} 
\begin{enumerate}
    \item What are the six step troubleshooting process? 
    \begin{enumerate}
        \item Identify the problem
        \item Establish a theory of probable cause. (Question the obvious) 
        \item Test the theory to determine cause
        \item Establish a plan of action to resolve the problem and implement a solution
        \item Verify full system functionality and, if applicable, implement preventative measures. 
        \item Document findings, actions, and outcomes 
    \end{enumerate}
    \item To which type of technology would you install a x16 card?
    \begin{enumerate}
        \item PCIe (Peripheral Component Interconnect express) is the most common expansion slot for video cards (by sixteen) 
    \end{enumerate}
    \item Which process of the computer checks all your components dur‐
ing boot?
    \begin{enumerate}
        \item POST (Power-On Self Test) is a process that is part of the BIOS (Basic Input Output System) or Unified Extensible Firmware Interface (UEFI). It runs a self-check of the computer system during boot and stores many of the parameters of the components within the CMOS. 
    \end{enumerate}
    \item Which of the following could cause the POST to fail? (Select
the two best answers.) 
    \begin{enumerate}
        \item The CPU and Memory need to be installed properly for the POST to run (and to pass). 
    \end{enumerate}
    
    \item Which of the following might you find as part of a tablet com‐
puter? (Select the two best answers.)   
    \begin{enumerate}
        \item A tablet computer will almost always  contain flash memory as main storage and multi-touch screen. 
    \end{enumerate}
    \item Which kind of socket incorporates “lands” to ensure connectiv‐
ity to a CPU? 
    \begin{enumerate}
        \item LGA (land grid array) is the type of socket that
uses “lands” to connect the socket to the CPU.
    \end{enumerate}
    \item How many pins are inside an SATA 3.0 data connector? 
    \begin{enumerate}
        \item The SATA version 3.0 data connector has seven pins. Note: SATA express uses a tiple connector with 18 pins ( 7 + 7 + 4) 
    \end{enumerate}
    \item What is the minimum number of hard drives necessary to imple‐
ment RAID 5?
    \begin{enumerate}
        \item Because RAID 5 uses striping with parity, a third disk is needed. You can have more than three disks as well. 
    \end{enumerate}
    \item A user’s time and date keep resetting to January 1, 2012. Which
of the following is the most likely cause?
    \begin{enumerate}
        \item If the time and date keep resetting - for example, to a date such as January 1,2012 - chances are that the lithium battery needs to be replaced. These are usually nickle-sized batteries; most PCs use a CR2032 lithium battery. 
    \end{enumerate}
    \item Which type of adapter card is normally plugged into a PCIe x16
adapter card slot?
    \begin{enumerate}
        \item The PCI Express (PCIe) x16 expansion slot is used
primarily for video.
    \end{enumerate}
    \item Which of the following CPU cooling methods is the most com‐
mon?
    \begin{enumerate}
        \item The most common CPU cooling method is the heat
sink and fan combination. The heat sink helps the heat to disperse
away from the CPU, whereas the fan blows the heat down and
through the fins; the power supply exhaust fan and possibly addi‐
tional case fans help the heat escape the case. Heat sink and fan
combinations are known as active cooling methods.
    \end{enumerate}
    \item What type of power connector is used for a x16 video card?
    \begin{enumerate}
        \item PCIe 6-pin
    \end{enumerate}
    \item What does the b in 1000 Mbps stand for?
    \begin{enumerate}
        \item The b in 1000 Mbps stands for bits: 1000 Mbps is
1000 megabits per second or 1 gigabit per second. Remember
that the lowercase b is used to indicate bits when measuring net‐
work data transfer rates, USB data transfer rates, and other simi‐
lar serial data transfers.
    \end{enumerate}
    \item When running cable through drop ceilings, which type of cable do
you need?
    \begin{enumerate}
        \item Plenum-rated cable needs to be installed wherever
a sprinkler system is not able to spray water. This includes ceil‐
ings, walls, and plenums (airways). Plenum-rated cable has a pro‐
tective covering that burns slower and gives off fewer toxic
fumes than regular PVC-based cable.
    \end{enumerate}
    \item Which of the following is the default subnet mask for IP address
192.168.1.1?
    \begin{enumerate}
        \item 192.168.1.1, by default, has the subnet mask
255.255.255.0, which is the standard subnet mask for class C IP
addresses. However, remember that some networks are classless,
which means that a network can use a different subnet mask.
    \end{enumerate}
    \item Which of the following is the minimum category cable needed for
a 1000BASE-T network?
    \begin{enumerate}
        \item The minimum cable needed for 1000BASE-T net‐
works is Category 5e. Of course, Cat 6 would also work, but it is
not the minimum of the listed answers. 1000BASE-T specifies
the speed of the network (1000 Mbps), the type (baseband, single
shared channel), and the cable to be used (T = twisted pair).
    \end{enumerate}
    \item Which of the following IP addresses can be routed across the In‐
ternet?
    \begin{enumerate}
        \item The only listed answer that is a public address
(needed to get onto the Internet) is 129.52.50.13
        \item All the other answers are private IPs, meant
to be behind a firewall. 127.0.0.1 is the IPv4 local loopback IP
address. 192.168.1.1 is a common private IP address used by
SOHO networking devices. 10.52.50.13 is a private address.
Note that the 10 network is common in larger networks.
    \end{enumerate}
    \item Which port number is used by HTTPS by default?
    \begin{enumerate}
        \item The Hypertext Transfer Protocol Secure (HTTPS)
uses port 443 (by default).
        \item Port 21 is used by the File Transfer Protocol
(FTP). Port 25 is used by the Simple Mail Transfer Protocol
(SMTP). Port 80 is used by regular HTTP, which is considered to
be insecure.
    \end{enumerate}
    \item Which of the following cable types have a copper medium? (S‐
elect the three best answers.)
    \begin{enumerate}
        \item Twisted-pair, coaxial, and Category 7 cable are all
examples of network cables with a copper medium. They all send
electricity over copper wire. 
        \item ultimode is a type of fiber-optic cable; it
uses light to send data over a glass or plastic medium. Twisted
pair is the most common type of cabling used in today’s net‐
works.
    \end{enumerate}
    \item Which of the following cable types can protect from electromag‐
netic interference (EMI)? (Select the two best answers.)
    \begin{enumerate}
        \item Shielded twisted pair (STP) and fiber optic can
protect from EMI. 
        \item Unshielded twisted pair (UTP) cannot pro‐
tect from EMI. Unless otherwise mentioned, Category 6 cable is
UTP. STP is shielded twisted pair. Unlike UTP (unshielded
twisted pair), STP provides an aluminum shield that protects
from EMI. UTP and coaxial have no such protection. Fiber optic
uses a different medium altogether, transmitting light rather than
electricity; therefore, EMI cannot affect fiber-optic cables.
    \end{enumerate}
    \item You are configuring Bob’s computer to access the Internet. Which
of the following are required? (Select all that apply.)
    \begin{enumerate}
        \item To get on the Internet, the DNS server address is
required so that the computer can get the resolved IP addresses
from the domain names that are typed in. The gateway address is
necessary to get outside the network.
    \end{enumerate}
    \item A customer wants to access the Internet from many different loca‐
tions in the United States. Which of the following is the best tech‐
nology to enable the customer to do so?
    \begin{enumerate}
        \item Cellular WAN uses a phone or other mobile device
to send data over standard cellular connections.
    \end{enumerate}
    \item You just configured the IP address 192.168.0.105 in Windows.
When you press the Tab key, Windows automatically configures
the default subnet mask of 255.255.255.0. Which of the following
IP addresses is a suitable gateway address?
    \begin{enumerate}
        \item 192.168.0.1 is the only suitable gateway address.
Remember that the gateway address must be on the same net‐
work as the computer. In this case, the network is 192.168.0, as
defined by the 255.255.255.0 subnet mask.
    \end{enumerate}
    \item You have been tasked with blocking remote logins to a server.
Which of the following ports should you block?
    \begin{enumerate}
        \item ort 23 should be blocked. It is associated with the
Telnet service, which is used to remotely log in to a server at the
command line. You can block this service at the company fire‐
wall and individually at the server and other hosts. It uses port 23
by default, but it can be used with other ports as well. Telnet is
considered to be insecure, so it should be blocked and disabled.
    \end{enumerate}
    \item Which	of	the	following	connector	types	is	used	by	fiber-optic	ca‐
bling?

    \begin{enumerate}
        \item The	LC	connector	is	used	by	fiber-optic	cabling.
Other	fiber	connectors	include	SC	and	ST.
        \item RJ45	is	the	connector	used	by	twisted-pair
networks.	RG-6	is	the	cable	used	by	cable	Internet	and	TV;	an	F-
connector	is	attached	to	the	ends	of	an	RG-6	cable.	RJ11	is	the
standard	phone	line	connector.

    \end{enumerate}
    \item Which	protocol	uses	port	53?

    \begin{enumerate}
        \item The	Domain	Name	System	(DNS)	protocol	uses
port	53	by	default. 
        \item TP	uses	port	21.	SMTP	uses	port	25	(or	587 or	465).	HTTP uses port	80.

    \end{enumerate}
    \item Which	of	the	following	terms	best	describes	two	or	more	LANs
connected	over	a	large	geographic	distance?

    \begin{enumerate}
        \item A	wide-area	network	(WAN)	is	a	network	in	which
two	or	more	LANs	are	connected	over	a	large	geographic	dis‐
tance—for	example,	between	two	cities.	The	WAN	requires	con‐
nections	to	be	provided	by	a	telecommunications	or	data	commu‐
nications	company.
    \end{enumerate} 
        \item Which	device	connects	to	the	network	and	has	the	sole	purpose	of
providing	data	to	clients?
    \begin{enumerate}
        \item Network-attached	storage	(NAS)	devices	store	data
for	network	use.	They	connect	directly	to	the	network.
    \end{enumerate}
    \item You	are	making	your	own	networking	patch	cable.	You	need	to
attach	an	RJ45	plug	to	the	end	of	a	twisted-pair	cable.	Which	tool
should	you	use?

    \begin{enumerate}
        \item 	Use	an	RJ45	crimper	tool	to	permanently	attach
RJ45	plugs	to	the	end	of	a	cable

    \end{enumerate}
    \item Which	port	is	used	by	RDP?
    \begin{enumerate}
        \item The	Remote	Desktop	Protocol	(RDP)	uses	port
3389	by	default.	This	protocol	allows	one	computer	to	take control	of	another	remote	system.

    \end{enumerate}
    \item Which	of	the	following	printer	failures	can	be	described	as	a	con‐
dition	in	which	the	internal	feed	mechanism	stopped	working
temporarily?

    \begin{enumerate}
        \item A	failure	that	occurs	due	to	the	internal	feed	mech‐
anism	stopping	is	known	as	a	paper	jam.	For	example,	an	HP
LaserJet	might	show	error	code	13.1	on	the	display,	which	means
a	paper	jam	at	the	paper	feed	area.	You	should	verify	that	the	pa‐
per	trays	are	loaded	and	adjusted	properly.

    \end{enumerate}
    \item Which	type	of	printer	uses	a	toner	cartridge?

    \begin{enumerate}
        \item Laser	printers	use	toner	cartridges.

    \end{enumerate}
    \item Which	of	the	following	should	not	be	connected	to	a	UPS?

    \begin{enumerate}
        \item Laser	printers	use	large	amounts	of	electricity,
which	in	turn	could	quickly	drain	the	battery	of	the	UPS.	They
should	be	plugged	in	to	their	own	individual	power	strips.

    \end{enumerate}
    \item Special	paper	is	needed	to	print	on	which	type	of	printer?

    \begin{enumerate}
        \item 	Regular	paper	can	be	used	on	all	the	listed	printers
except	for	thermal	printers,	which	use	specially	coated	paper	that
is	heated	to	create	the	image.

    \end{enumerate}
    \item Which	of	the	following	channels	should	you	select	for	an	802.11
wireless	network?

    \begin{enumerate}
        \item Of	the	listed	answers,	use	channel	6	for	802.11
wireless	networks.	That	would	imply	a	2.4	GHz	connection	using
either	802.11n,	g,	or	b.	The	2.4	GHz	frequency	range	in	the
United	States	allows	for	channels	1	through	11.

    \end{enumerate}
    \item Which	environmental	issue	affects	a	thermal	printer	the	most?

    \begin{enumerate}
        \item Which	type	of	printer	uses	impact	to	transfer	ink	from	a	ribbon	to
the	paper?

    \end{enumerate}
    \item Which	of	the	following	steps	enables	you	to	take	control	of	a	net‐
work	printer	from	a	remote	computer?

    \begin{enumerate}
        \item 	After	you	install	the	driver	for	the	printer	locally,
you	can	then	take	control	of	it	by	going	to	the	properties	of	the
printer	and	accessing	the	Ports	tab.	Then	click	the	Add	Port	but‐
ton	and	select	the	Standard	TCP/IP	Port	option.	You	have	to
know	the	IP	address	of	the	printer	or	the	computer	that	the	printer
is	connected	to.

    \end{enumerate}
    \item 	A	color	laser	printer	produces	images	that	are	tinted	blue.	Which
of	the	following	steps	should	be	performed	to	address	this	prob‐
lem?

    \begin{enumerate}
        \item After	you	install	a	printer,	it	is	important	to	cali‐
brate	it	for	color	and	orientation,	especially	if	you	are	installing	a
color	laser	printer	or	an	inkjet	printer.	These	calibration	tools	are
usually	built	in	to	the	printer’s	software	and	can	be	accessed	from
Windows,	or	you	can	access	them	from	the	printer’s	display.

    \end{enumerate}
    \item A	desktop	computer	does	not	have	a	lit	link	light	on	the	back	of
the	computer.	Which	of	the	following	is	the	most	likely	reason	for
this?

    \begin{enumerate}
        \item The	most	likely	answer	in	this	scenario	is	that	the
network	cable	is	disconnected.	If	the	desktop	computer	is	using	a
wired	connection,	it	is	most	likely	a	twisted-pair	Ethernet	con‐
nection.	When	this	cable	is	connected	to	the	computer	on	one	end
and	to	a	switch	or	other	central	connecting	device	on	the	other
end,	it	initiates	a	network	connection	over	the	physical	link.	This
link	then	causes	the	network	adapter’s	link	light	to	light	up.	The
link	light	is	directly	next	to	the	RJ45	port	of	the	network	adapter.
The	corresponding	port	on	the	switch	(or	other	similar	device)	is
also	lit.	If	the	cable	is	disconnected,	the	link	light	becomes	unlit,
though	there	are	other	possibilities	for	this	link	light	to	be	dark—
for	example,	if	the	computer	is	off	or	if	the	switch	port	is	dis‐
abled.

    \end{enumerate}
    \item Which	of	the	following	IP	addresses	would	a	technician	see	if	a
computer	running	Windows	is	connected	to	a	multifunction	net‐
work	device	and	is	attempting	to	obtain	an	IP	address	automati‐
cally	but	is	not	receiving	an	IP	address	from	the	DHCP	server?

    \begin{enumerate}
        \item If	the	computer	fails	to	obtain	an	IP	address	from	a
DHCP	server,	Windows	will	take	over	and	apply	an	Automatic
Private	IP	Address	(APIPA).	This	address	will	be	on	the
169.254.0.0	network.

    \end{enumerate}
    \item For	which	type	of	PC	component	are	80	mm	and	120	mm	com‐
mon	sizes?

    \begin{enumerate}
        \item Case	fans	are	measured	in	mm	(millimeters);	80
mm	and	especially	120	mm	are	very	common.	They	are	used	to
exhaust	heat	out	of	the	case.	These	fans	aid	in	keeping	the	CPU
and	other	devices	cool.	The	120	mm	is	quite	common	in	desktop
and	tower	PCs,	and	the	80	mm	is	more	common	in	smaller	sys‐
tems	and	1U	and	2U	rackmount	servers.

    \end{enumerate}
    \item An	exclamation	point	next	to	a	device	in	the	Device	Manager	in‐
dicates	which	of	the	following?

    \begin{enumerate}
        \item If	you	see	an	exclamation	point	in	the	Device	Man‐
ager,	this	indicates	that	the	device	does	not	have	a	proper	driver.

    \end{enumerate}
    \item Beep	codes	are	generated	by	which	of	the	following?

    \begin{enumerate}
        \item As	the	power-on	self-test	(POST)	checks	all	the
components	of	the	computer,	it	may	present	its	findings	on	the
screen	or	in	the	form	of	beep	codes.

    \end{enumerate}
    \item Which	of	the	following	indicates	that	a	printer	is	network-ready?

    \begin{enumerate}
        \item The	RJ45	jack	enables	a	connection	to	a	twisted-
pair	(most	likely	Ethernet)	network.	Printers	with	a	built-in	RJ45
connector	are	network-ready;	so	are	printers	that	are	Wi-Fi	en‐
abled.

    \end{enumerate}
    \item You	just	turned	off	a	printer	to	maintain	it.	Which	of	the	follow‐
ing	should	you	be	careful	of	when	removing	the	fuser?

    \begin{enumerate}
        \item 	The	fuser	heats	paper	to	around	400°	Fahrenheit
(204°	Celsius).	That’s	like	an	oven.	If	you	need	to	replace	the
fuser,	let	the	printer	sit	for	10	or	15	minutes	after	shutting	it	down
and	before	maintenance.

    \end{enumerate}
    \item Which	of	the	following	connectors	is	used	for	musical	equipment?

    \begin{enumerate}
        \item The	Musical	Instrument	Digital	Interface	(MIDI)
connector	is	used	for	musical	equipment	such	as	keyboards,	syn‐
thesizers,	and	sequencers.	MIDI	is	used	to	create	a	clocking	sig‐
nal	that	all	devices	can	synchronize	to.
        
    \end{enumerate}
    \item Which	of	the	following	storage	technologies	is	used	by	hard
disk	drives?

    \begin{enumerate}
        \item Hard	disk	drives	(HDDs)	are	magnetic	disks.	Th‐
ese	are	the	type	with	moving	parts,	as	opposed	to	solid-state
drives	(SSDs)	that	have	no	moving	parts.

    \end{enumerate}
    \item Which	protocol	uses	port	389?

    \begin{enumerate}
        \item LDAP 
    \end{enumerate}
    \item Which	of	the	following	wireless	networking	standards	operates
at	5	GHz	only?	(Select	the	two	best	answers.)

    \begin{enumerate}
        \item :	802.11a	operates	at	5	GHz	only;	so	does	802.11ac.

    \end{enumerate}
    \item Which	of	the	following	types	of	RAM	has	a	peak	transfer	rate	of
21,333	MB/s?

    \begin{enumerate}
        \item DDR4-2666	has	a	peak	transfer	rate	of	21,333
MB/s.	It	runs	at	an	I/O	bus	clock	speed	of	1333	MHz	and	can
send	2666	megatransfers	per	second	(MT/s).	It	is	also	known	as
PC4-21333.

    \end{enumerate}
    \item Which	of	the	following	types	of	printers	uses	a	print	head,	ribbon,
and	tractor	feed?

    \begin{enumerate}
        \item The	impact	printer	uses	a	print	head,	ribbon,	and
tractor	feed.	An	example	of	an	impact	printer	is	the	dot	matrix.

    \end{enumerate}
    \item Which	of	the	following	is	a	possible	symptom	of	a	failing	CPU?

    \begin{enumerate}
        \item If	the	CPU	is	running	beyond	the	recommended
voltage	range	for	extended	periods	of	time,	it	can	be	a	sign	of	a
failing	CPU.	The	problem	could	also	be	caused	by	overclocking.
Check	in	the	UEFI/BIOS	to	see	whether	or	not	the	CPU	is	over‐
clocked.

    \end{enumerate}
    \item Which	of	the	following	cable	types	is	not	affected	by	EMI	but	re‐
quires	specialized	tools	to	install?

    \begin{enumerate}
        \item Fiber-optic	cable	is	the	only	answer	listed	that	is
not	affected	by	electromagnetic	interference	(EMI).	The	reason	is
that	it	does	not	use	copper	wire	or	electricity,	but	instead	uses
glass	or	plastic	fibers	and	light.
    \end{enumerate}
    \item Setting	an	administrator	password	in	the	BIOS	accomplishes
which	of	the	following?

    \begin{enumerate}
        \item Changing boot order
    \end{enumerate}
    \item What	is	an	LCD	display’s	contrast	ratio	defined	as?
    \begin{enumerate}
        \item 	Contrast	ratio	is	the	brightness	of	the	brightest
color	(measured	as	white)	compared	to	the	darkest	color	(mea‐
sured	as	black).	Static	contrast	ratio	measurements	are	static;	they
are	performed	as	tests	using	a	checkerboard	pattern.	But	there	is
also	the	dynamic	contrast	ratio,	a	technology	in	LCD	displays
that	adjusts	dynamically	during	darker	scenes	in	an	attempt	to
give	better	black	levels.	It	usually	has	a	higher	ratio,	but	it	should
be	noted	that	there	is	no	real	uniform	standard	for	measuring	con‐
trast	ratio.

    \end{enumerate}
    \item Which	of	the	following	devices	limits	network	broadcasts,	seg‐
ments	IP	address	ranges,	and	interconnects	different	physical	me‐
dia?

    \begin{enumerate}
        \item A	router	can	limit	network	broadcasts	through	seg‐
menting	and	programmed	routing	of	data.	This	is	part	of	a
router’s	job	when	connecting	two	or	more	networks.	It	is	also
used	with	different	media.	For	example,	you	might	have	a	LAN
that	uses	twisted-pair	cable,	but	the	router	connects	to	the	Inter‐
net	via	a	fiber-optic	connection.	That	one	router	will	have	ports
for	both	types	of	connections.

    \end{enumerate}
    \item Which	of	the	following	traits	and	port	numbers	are	associated
with	POP3?	(Select	the	two	best	answers.)

    \begin{enumerate}
        \item OP3	is	a	protocol	used	by	email	clients	to	receive
email.	It	makes	use	of	either	port	110	(considered	insecure)	or
port	995	(a	default	secure	port).

    \end{enumerate}
    \item Which	of	the	following	uses	port	427?

    \begin{enumerate}
        \item The	Service	Location	Protocol	(SLP)	uses	port
427.	It	enables	access	to	network	services	without	previous	con‐
figuration	of	the	client	computer.

    \end{enumerate}
    \item Which	of	the	following	ports	is	used	by	AFP?

    \begin{enumerate}
        \item The	Apple	Filing	Protocol	(AFP)	uses	port	548.
AFP	offers	file	services	for	Mac	computers	running	macOS	and
can	transfer	files	across	the	network.

    \end{enumerate}
    \item Which of the following is the minimum amount of RAM needed
to install a 64-bit version of Windows 10?
    \begin{enumerate}
        \item 
    \end{enumerate}
    \item 
    \begin{enumerate}
        \item 
    \end{enumerate}
    \item 
    \begin{enumerate}
        \item 
    \end{enumerate}
    \item 
    \begin{enumerate}
        \item 
    \end{enumerate}
    \item 
    \begin{enumerate}
        \item 
    \end{enumerate}
    \item 
    \begin{enumerate}
        \item 
    \end{enumerate}
    \item 
    \begin{enumerate}
        \item 
    \end{enumerate}
    \item 
    \begin{enumerate}
        \item 
    \end{enumerate}
    
\end{enumerate}
\section{Chapter 9}
\section{Tables}
\begin{tabular}{|c|c|c|}
\hline
    Technology & Full Name & Port \\
    \hline
   SLP  & Service Location Protocol & 427\\
   FTP & File Transfer Protocol & 21 \\
   DNS & Domain Name System & 53\\
   AFP & Apple Filing Protocol & 548 \\
   HTTP & HyperText Transfer Protocol & 80 \\
   SSH & Secure Shell & 22 \\
   Telnet & Teletype Network & 23\\
   IMAP & Internet Message Access Protocol & 143\\
   \hline
\end{tabular}



\section{Troy McMillan compTIA A+} 
\subsection{Mobile Devices}
\begin{enumerate}
    \item What is the maximum transmission speed of an ExpressCard in PCIe2 mode?
    \begin{itemize}
        \item D) The maximum transmission speeds are as follows: 280 Mbps effective (USB 2 mode) \\
        1.6 Gbps effective (PCIe 1 mode) \\
        3.2 Gbps effective (PCIe 2 or USB 3 mode) 
    \end{itemize}
    \item Which interface is natively found only in Apple devices?
    \begin{itemize}
        \item C) Thunderbolt ports are most likely to be found on Apple laptops, although they are now
showing up on others as well. USB ports are typically found on all mobile devices, while
Serial and PS/2 connecters are rarely found on mobile devices.
    \end{itemize}
     \item What special screwdriver is typically required to work on a notebook?
    \begin{itemize}
        \item B) Some models of notebook PCs require a special T-8 Torx screwdriver. Most PC toolkits
come with a T-8 bit for a screwdriver with interchangeable bits, but you may find that the
T-8 screws are countersunk in deep holes so that you can’t fit the screwdriver into them. In
such cases, you need to buy a separate T-8 screwdriver, available at most hardware stores
or auto parts stores. Phillips-head screwdrivers have a cross pattern on the tip and may be
required. Hex heads are another type you may encounter, and metric drivers are those that
are sized with the metric system.
    \end{itemize}
     \item What is the easiest thing to damage when removing a laptop keyboard?
    \begin{itemize}
        \item B) When replacing the keyboard, one of the main things you want to keep in mind is to not
damage the data cable connector to the system board.
    \end{itemize}
     \item Which component if damaged can render the hard drive useless?
    \begin{itemize}
        \item C) If required, remove the connector attached to the old drive’s signal pins and attach it
to the new drive. Make sure it’s right side up and do not force it. Damaging the signal pins
may render the drive useless. The caddy, rails, and chassis are not easily damaged.
    \end{itemize}
     \item What size hard drive goes in a laptop?
    \begin{itemize}
        \item C) The 2.5-inch hard drives are smaller (which makes them attractive for a laptop where
space is at a minimum); however, in comparison to 3.5-inch hard drives, they have less
capacity and cache, and they operate at a lower speed.
    \end{itemize}
     \item Which is not an advantage of solid-state drives?
    \begin{itemize}
        \item A) The advantage of solid-state drives is that they are not as susceptible to damage if the
device is dropped, and they are, generally speaking, faster as no moving parts are involved.
They are, however, more expensive, and when they fail they don’t generally give some
advance symptoms like a magnetic drive will do.
    \end{itemize}
     \item Which display uses a row of transistors across the top of the screen and a column of them
down the side?
    \begin{itemize}
        \item A passive matrix screen uses a row of transistors across the top of the screen and a
column of them down the side. It sends pulses to each pixel at the intersection of each row
and column combination, telling it what to display. An active matrix screen uses a separate
transistor for each individual pixel in the display, resulting in higher refresh rates and
brighter display quality. Twisted nematic (TN) is the older of the two major technologies
for flat-panel displays. While it provides the shortest response time, has high brightness,
and draws less power than competing technologies, it suffers from poor quality when
viewed from wide angles. In-Plane Switching (IPS) is a newer technology that solves the
issue of poor quality at angles other than straight on.
    \end{itemize}
    \item Which display is a newer technology that solves the issue of poor quality at angles other
than straight on?
    \begin{itemize}
        \item D) In-Plane Switching (IPS) is a newer technology that solves the issue of poor quality
at angles other than straight on. A passive matrix screen uses a row of transistors across
the top of the screen and a column of them down the side. It sends pulses to each pixel at
the intersection of each row and column combination, telling it what to display. An active
matrix screen uses a separate transistor for each individual pixel in the display, resulting in
higher refresh rates and brighter display quality. Twisted nematic (TN) is the older of the
two major technologies for flat-panel displays. While it provides the shortest response time,
has high brightness, and draws less power than competing technologies, it suffers from
poor quality when viewed from wide angles.
    \end{itemize}
    \item Which of the following lets you quickly connect/disconnect with external peripherals and
may also provide extra ports that the notebook PC doesn’t normally have?
    \begin{itemize}
        \item D) With a hot dock, a laptop once put into suspended mode will recognize plug-and-play
devices. A docking station essentially allows a laptop computer to be converted to a desktop
computer. Laptop and table locks are used to secure mobile devices.
    \end{itemize}
    \item In what mode of plug and play must the laptop be turned off and back on for the change to
be recognized?
    \begin{itemize}
        \item C) In cold docking, the laptop must be turned off and back on for the change to be
recognized. In warm docking, the laptop must be put in and out of suspended mode for the
change to be recognized. In hot docking, the change can be made and is recognized while
running normal operations.
    \end{itemize}
    \item Which of the following is a class of devices that specializes in tracking your movement?
    \begin{itemize}
        \item A)While many smart watches can also act as fitness monitors, there is a class of devices
that specializes in tracking your movement. Fitness monitors read your body temperature,
heart rate, and blood pressure. Extended reality is an exciting new field that includes
both augmented reality and virtual reality. Today’s smartphones are really computers
that can make calls, and tablets have been in existence in some form or fashion since the
early 1990s. Early on they were proprietary devices that didn’t have a lot in common with
desktop computers, but increasingly the two form factors have gravitated toward one
another.
    \end{itemize}
    \item Which of the following uses satellite information to plot the global location of an object
and uses that information to plot the route to a second location?
    \begin{itemize}
        \item A) A global positioning system (GPS) uses satellite information to plot the global location
of an object and uses that information to plot the route to a second location. Geofencing is
the use of GPS to restrict communication to an area. Remote wipe is the cleaning of data
from a lost or stolen device. There is no such thing as local wipe.
    \end{itemize}
    \item Which interface is the most common port found on mobile devices?
    \begin{itemize}
        \item A) he two most common ports found on mobile devices are micro-USB and mini-USB.
Both are small–form-factor implementations of the USB standard, the latest of which is
USB 3.1. Thunderbolt ports are most likely to be found on Apple laptops, but they are
now showing up on others as well. Serial and PS/2 connecters are rarely found on mobile
devices. 
    \end{itemize}
    \item Which is the most common pin code when selecting discovered Bluetooth devices?
    \begin{itemize}
        \item A) Many external devices will ask for a PIN when you select the external device from the
list of discovered devices. In many cases, the PIN is 0000, but you should check the manual
of the external device
    \end{itemize}
    \item Which of the following is the connection between the mobile device and the radio?
    \begin{itemize}
        \item A) The product release information (PRI) is the connection between the mobile device and
the radio. From time to time this may need updating, which, when done, may add features or
increase data speed. The preferred roaming list (PRL) is a list of radio frequencies residing in
the memory of some kinds of digital phones. International Mobile Equipment Identification
(IMEI) is used to identify a physical phone device, while International Mobile Subscriber
Identification (IMSI) is used to identify a Subscriber Identification Module (SIM) card.
    \end{itemize}
    \item Which of the following is a process whereby not only does the server verify the credential of
the client but the client also verifies the credential of the server?
    \begin{itemize}
        \item A) Mutual authentication is a process whereby not only does the server verify the
credential of the client but the client also verifies the credential of the server. It adds
additional security to the process. Single sign-on is a service that allows users to sign in
once and have access to all resources. Multifactor authentication makes use of multiple
factors of authentication to increase security. Biometrics is the use of physical factors of
authentication.
    \end{itemize}
    \item Which of the following is the use of physical factors of authentication?
    \begin{itemize}
        \item D) Mutual authentication is a process whereby not only does the server verify the
credential of the client but the client also verifies the credential of the server. It adds
additional security to the process. Single sign-on is a service that allows users to sign in
once and have access to all resources. Multifactor authentication makes use of multiple
factors of authentication to increase security. Biometrics is the use of physical factors of
authentication.
    \end{itemize}
\end{enumerate}
\subsection{Networking} 
\begin{enumerate}
    \item Which the following uses port 110?
    \begin{itemize}
        \item D) POP3 uses port 110. SSH uses port 22, FTP uses ports 20 and 21, and Telnet uses port 23.
    \end{itemize}
    \item Which of the following uses two ports?
     \begin{itemize}
        \item A) FTP uses ports 20 and 21. POP3 uses port 110, SSH uses port 22, and Telnet uses port 23.
    \end{itemize}
    \item Which the following uses port 22?
     \begin{itemize}
        \item B)SSH uses port 22, POP3 uses port 110, FTP uses ports 20 and 21, and Telnet uses port 23.
    \end{itemize}
    \item Which device operates at layer 2?
     \begin{itemize}
        \item A) Switches operate at layer 2. Routers operate at layer 3. Repeaters and hubs operate at layer 1.
    \end{itemize}
    \item Which device operates at layer 1?
     \begin{itemize}
        \item D. Hubs operate at layer 1. Switches and bridges operate at layer 2. Routers operate at layer 3.
    \end{itemize}
    \item Which device operates at layer 2?
     \begin{itemize}
        \item B. Switches operate at layer 2. Routers operate at layer 3. Hubs and repeaters operate at layer 1.
    \end{itemize}
    \item Which of the following is not a private IP address range?
     \begin{itemize}
        \item B. The class B range is 172.16.0.0–172.31.255.255. The other ranges are correct.
    \end{itemize}
    \item Which of the following delivers an upload speed equal to the download speed?
     \begin{itemize}
        \item A) Symmetric DSL (SDSL) offers an upload equal to the download speed. The other
versions all have slower upload speed than download speed. 
    \end{itemize}
    \item Which of the following is an area where you can place a public server for access by people
you might not trust otherwise?
     \begin{itemize}
        \item B). A demilitarized zone (DMZ) is an area where you can place a public server for access
by people you might not trust otherwise. NAT is a service that maps private IP addresses
to public IP addresses. The intranet is the internal network that should be protected. The
Internet is the untrusted public network.
    \end{itemize}
    \item Which of the following operates in the 5.0 GHz range?
     \begin{itemize}
        \item A)802.11a operates in the 5.0 GHz range. The other standards all operate in the 2.4 GHz
range.
    \end{itemize}
    \item Which of the following operates at a maximum of 2 MB?
     \begin{itemize}
        \item D) 802.11a and 802.11g have a maximum rate of 54 MB, 802.11b has a maximum of 11 MB,
and 802.11 has a maximum of 2 MB.
    \end{itemize}
    \item Which of the following has the largest cell size?
     \begin{itemize}
        \item C) 802.11g has a distance that is the cell size of 125 ft. The others have a distance of 115 ft.
    \end{itemize}
    \item Which type of server resolves IP addresses to hostnames?
     \begin{itemize}
        \item B) DNS servers resolve IP addresses to hostnames. HTTP servers are web servers. DHCP
servers provide automatic IP configurations. SQL is a database server.
    \end{itemize}
    \item Which type of server provides automatic IP configurations?
     \begin{itemize}
        \item C) DHCP servers provide automatic IP configurations. DNS servers resolve IP addresses to
hostnames. HTTP servers are web servers. SQL is a database server.
    \end{itemize}
    
    \item Which type of server is a database server?
     \begin{itemize}
        \item D) A SQL server is a database server. DNS servers resolve IP addresses to hostnames.
HTTP servers are web servers. DHCP servers provide automatic IP configurations.
    \end{itemize}
    \item Which of the following is a Class B address?
     \begin{itemize}
        \item The Class B range is 128–191. The class A range is 1–126. The Class C range is 192–223.
    \end{itemize}
    \item Which of the following is a Class A address?
     \begin{itemize}
        \item The Class A range is 1–126. The class B range is 128–191. The Class C range is 192–223.
    \end{itemize}
    \item Which of the following is a Class C address?
     \begin{itemize}
        \item The Class C range is 192–223. The class A range is 1–126. The class B range is 128–191.
The 224 range is for multicasting.
    \end{itemize}
    \item When personal devices include networking capabilities and can communicate directly with
one another, they create which type of network?
     \begin{itemize}
        \item A personal area network (PAN) is a LAN created by personal devices. A wide area
network (WAN) is a collection of two or more LANs, typically connected by routers and
dedicated leased lines. Occasionally, a WAN will be referenced as a metropolitan area
network (MAN) when it is confined to a certain geographic area, such as a university
campus or city. Wireless mesh networks (WMN) are a form of an ad hoc WLAN that often
consist of mesh clients, mesh routers, and gateways.
    \end{itemize}
    \item Which of the following is a collection of two or more LANs, typically connected by routers
and dedicated leased lines?
     \begin{itemize}
        \item Metropolitan area network (MAN) is the term occasionally used for a WAN that is
confined to a certain geographic area, such as a university campus or city. A personal area
network (PAN) is a LAN created by personal devices. A wide area network (WAN) is a
collection of two or more LANs, typically connected by routers and dedicated leased lines.
Wireless mesh networks (WMN) are a form of an ad hoc WLAN that often consist of mesh
clients, mesh routers, and gateways.
    \end{itemize}
    \item Which of the following is a form of ad hoc WLAN?
     \begin{itemize}
        \item Wireless mesh networks (WMN) are a form of an ad hoc WLAN that often consist
of mesh clients, mesh routers, and gateways. A personal area network (PAN) is a LAN
created by personal devices. A wide area network (WAN) is a collection of two or more
LANs, typically connected by routers and dedicated leased lines. Occasionally, a WAN
will be referenced as a metropolitan area network (MAN) when it is confined to a certain
geographic area, such as a university campus or city.
    \end{itemize}
    \item Which of the following is used to attach media connectors to the ends of cables?
     \begin{itemize}
        \item Wire crimpers look like pliers but are used to attach media connectors to the ends
of cables. A cable stripper is used to remove the outer covering of the cable to get to the
wire pairs within. A multimeter combines a number of tools into one. There can be slight
variations, but a multimeter always includes a voltmeter, an ohmmeter, and an ammeter
(and is sometimes called VOM as an acronym). A toner probe has two parts: the tone
generator (called the toner) and the tone locator (called the probe). The toner sends the
tone, and at the other end of the cable, the probe receives the toner’s signal. This tool makes
it easier to find the beginning and end of a cable.
    \end{itemize}
    \item Which of the following includes a voltmeter, an ohmmeter, and an ammeter?
    \begin{itemize}
        \item A multimeter combines a number of tools into one. There can be slight variations, but a
multimeter always includes a voltmeter, an ohmmeter, and an ammeter (and is sometimes
called VOM as an acronym). Wire crimpers look like pliers but are used to attach media
connectors to the ends of cables. A cable stripper is used to remove the outer covering of the
cable to get to the wire pairs within. A toner probe has two parts: the tone generator (called
the toner) and the tone locator (called the probe). The toner sends the tone, and at the other
end of the cable, the probe receives the toner’s signal. This tool makes it easier to find the
beginning and end of a cable.
    \end{itemize}
    \item Which of the following makes it easier to find the beginning and end of a cable?
    \begin{itemize}
        \item A toner probe has two parts: the tone generator (called the toner) and the tone locator
(called the probe). The toner sends the tone, and at the other end of the cable, the probe
receives the toner’s signal. This tool makes it easier to find the beginning and end of a cable.
Wire crimpers look like pliers but are used to attach media connectors to the ends of cables.
A cable stripper is used to remove the outer covering of the cable to get to the wire pairs
within. A multimeter combines a number of tools into one. There can be slight variations,
but a multimeter always includes a voltmeter, an ohmmeter, and an ammeter (and is
sometimes called VOM as an acronym).
    \end{itemize}
\end{enumerate}
\subsection{Hardware} 
\begin{enumerate}
    \item Which cable type comes in two varieties: unshielded and shielded?
    \begin{itemize}
        \item C) Twisted pair is commonly used in office settings to connect workstations to hubs or
switches. It comes in two varieties: unshielded (UTP) and shielded (STP). Fiber optic, serial,
and coaxial do not come in shielded and unshielded versions.
    \end{itemize}
    \item Which cable type transmits data at speeds up to 100 Mbps and was used with Fast Ethernet
(operating at 100 Mbps) with a transmission range of 100 meters?
    \begin{itemize}
        \item B) Cat 5 transmits data at speeds up to 100 Mbps and was used with Fast Ethernet (operating
at 100 Mbps) with a transmission range of 100 meters. It contains four twisted pairs of copper
wire to give the most protection. Although it had its share of popularity (it’s used primarily for
10/100 Ethernet networking), it is now an outdated standard. Newer implementations use the
5e standard. Cat 4 transmits at 16 Mbps, and Cat 6 transmits at 1 Gbps.
    \end{itemize}
    \item Which cable type has a glass core within a rubber outer coating?
    \begin{itemize}
        \item A) Fiber-optic cabling is the most expensive type of those discussed for this exam.
Although it’s an excellent medium, it’s often not used because of the cost of implementing
it. It has a glass core within a rubber outer coating and uses beams of light rather than
electrical signals to relay data. None of the other options uses glass in its construction.
    \end{itemize}
    \item Which connector is used for telephone cord?
    \begin{itemize}
        \item A) An RJ-11 is a standard connector for a telephone line and is used to connect a computer
modem to a phone line. It looks much like an RJ-45 but is noticeably smaller. The RJ-45 is
used for networking. RS 232 is a serial connector. BNC is a coaxial connector.
    \end{itemize}
    \item Which standard has been commonly used in computer serial ports?
    \begin{itemize}
        \item C) The RS-232 standard had been commonly used in computer serial ports. A serial
cable (and port) uses only one wire to carry data in each direction; all the rest are wires
for signaling and traffic control. An RJ-11 is a standard connector for a telephone line and
is used to connect a computer modem to a phone line. It looks much like an RJ-45 but is
noticeably smaller. The RJ-45 is used for networking.
    \end{itemize}
    \item Which connectors are sometimes used in the place of RCA connectors for video electronics?
    \begin{itemize}
        \item D) Bayonet Neill–Concelman (BNC) connectors are sometimes used in the place of RCA
connectors for video electronics, so you may encounter these connectors, especially when
video equipment connects to a PC. In many cases, you may be required to purchase an
adapter to convert this to another form of connection because it is rare to find one on the
PC. An RJ-11 is a standard connector for a telephone line and is used to connect a computer
modem to a phone line. It looks much like an RJ-45 but is noticeably smaller. The RJ45 is
used for networking. RS-232 is a serial connector.
    \end{itemize}
    \item Which RAM type is used in laptops?
    \begin{itemize}
        \item B)Portable computers (notebooks and subnotebooks) require smaller sticks of RAM
because of their smaller size. One of the two types is small outline DIMM (SODIMM),
which can have 72, 144, or 200 pins. DIMM is a full-size RAM type. Rambus is a type of
RAM but not used in laptops, and BNC is a connector for coaxial cabling.
    \end{itemize}
    \item Which RAM type allows for two memory accesses for each rising and falling clock?
    \begin{itemize}
        \item D) DDR SDRAM is Double Data Rate 2 (DDR2). This allows for two memory accesses
for each rising and falling clock and effectively doubles the speed of DDR. DDR2-667
chips work with speeds at 667 MHz and are also referred to as PC2-5300 modules. DDR3
is the higher-speed successor to DDR and DDR2. Portable computers (notebooks and
subnotebooks) require smaller sticks of RAM because of their smaller size. One of the two
types is small outline DIMM (SODIMM), which can have 72, 144, or 200 pins.
    \end{itemize}
    \item Which RAM type is not compatible with any earlier type of random-access memory?
    \begin{itemize}
        \item B) DDR4 SDRAM is an abbreviation for double data rate fourth-generation synchronous dynamic random-access memory. DDR4 is not compatible with any earlier type of random-access memory (RAM). The DDR4 standard allows for DIMMs of up to 64 GB in capacity, compared to DDR3’s maximum of 16 GB per DIMM. DDR3 and DDR2 are backward compatible, and there is no DDR5
    \end{itemize}
    \item Which of the following is a rewritable optical disc?
    \begin{itemize}
        \item B) Compact Disc-ReWritable (CD-RW) media is a rewritable optical disc. A CD-RW drive requires more sensitive laser optics. It can write data to the disc but also has the ability to erase that data and write more data to the disc. CD, DVD, and CD-ROM are all read-only.
    \end{itemize}
    \item Which of the following is a specification for internally mounted computer expansion cards
and associated connectors that replaces the mSATA?
    \begin{itemize}
        \item A) M.2, formerly known as the Next Generation Form Factor (NGFF), is a specification for internally mounted computer expansion cards and associated connectors. It replaces the mSATA standard. M.2 modules are rectangular, with an edge connector on one side, and a semicircular mounting hole at the center of the opposite edge. Non-Volatile Memory Host Controller Interface Specification (NVME) is an open logical device interface specification for accessing nonvolatile storage media attached via a PCI Express (PCIe) bus. Serial ATA and serial ATA 2.5 are computer bus interfaces that connects host bus adapters to mass storage devices such as hard disk drives, optical drives, and solid-state drives
    \end{itemize}
    \item At what speed will latency on a magnetic drive decrease to about 3 ms?
    \begin{itemize}
        \item C) At 10,000 rpm, the latency will decrease to about 3 ms. Data transfer rates also generally go up with a higher rotational speed but are influenced by the density of the disk (the number of tracks and sectors present in a given area). Latency at 5400 rpm will be  5.56 ms. At 7200 it will be 4.17, and at 15000 it will drop to 2.
    \end{itemize}
    \item Laptops and other portable devices utilize which expansion card?
    \begin{itemize}
        \item A) Laptops and other portable devices utilize an expansion card called the miniPCI. It has the same functionality as the PCI but has a much smaller form factor. PCI and PCIe are used in desktops. SATA is a drive connector.
    \end{itemize}
    \item Which of the following is a standard firmware interface for PCs, designed to replace BIOS?
    \begin{itemize}
        \item A) Unified Extensible Firmware Interface (UEFI) is a standard firmware interface for PCs, designed to replace BIOS. NVRAM is RAM that retains its data during a reboot. CMOS is a battery type found on motherboards, and CHS is a drive geometry concept
    \end{itemize}
    \item Which of the following is memory that does not lose its content when power is lost to the
machine?
    \begin{itemize}
        \item  B) NVRAM is memory that does not lose its content when power is lost to the machine. Unified Extensible Firmware Interface (UEFI) is a standard firmware interface for PCs, designed to replace BIOS. CMOS is a battery type found on motherboards, and CHS is a drive geometry concept.
    \end{itemize}
    \item Which of the following devices allows you to plug multiple PCs (usually servers) into the
device and to switch easily back and forth from system to system using the same mouse,
monitor, and keyboard?
    \begin{itemize}
        \item A) A keyboard, video, and mouse (KVM) device allows you to plug multiple PCs (usually servers) into the device and to switch easily back and forth from system to system using the same mouse, monitor, and keyboard. The KVM is actually a switch that all the systems plug into. There is usually no software to install. Just turn off all the systems, plug them all into the switch and turn them back on; then you can switch from one to another using the same keyboard, monitor, and mouse device connected to the KVM switch. CMOS is a battery type found on motherboards, and CHS is a drive geometry concept. NVRAM is memory that does not lose its content when power is lost to the machine.
    \end{itemize}
    \item Which of the following is a description of light output?
    \begin{itemize}
        \item B) When discussing bulbs for projectors, brightness is a description of light output, which is measured in lumens (not watts). Ensure that you are purchasing the correct bulb for the projector and maximize the life of the bulb by turning the projector off when not in use. A keyboard, video, and mouse (KVM) device allows you to plug multiple PCs (usually servers) into the device and to switch easily back and forth from system to system using the same mouse, monitor, and keyboard. Contrast is the relationship between dark and light. CHS is a drive geometry concept.
    \end{itemize}
    \item Which of the following is a standard managed by the ISO and uses tags that are embedded
in the devices?
    \begin{itemize}
        \item C) NFC components include an initiator and a target; the initiator actively generates an RF field that can power a passive target. This enables NFC targets to take simple form factors such as tags, stickers, key fobs, or cards that do not require batteries. When discussing bulbs for projectors, brightness is a description of light output, which is measured in lumens (not watts). Ensure that you are purchasing the correct bulb for the projector and maximize the life of the bulb by turning the projector off when not in use. A keyboard, video, and mouse (KVM) device allows you to plug multiple PCs (usually servers) into the device and to switch easily back and forth from system to system using the same mouse, monitor, and keyboard. CHS is a drive geometry concept
    \end{itemize}
    \item In 2004, the ATX 12V 2.0 (now 2.03) standard was passed, changing the main connector
from 20 pins to how many?
    \begin{itemize}
        \item C) In 2004, the ATX 12V 2.0 (now 2.03) standard was passed, changing the main connector from 20 pins to 24. The additional pins provide +3.3V, +5V, and +12V (the fourth pin is a ground) for use by PCIe cards. When a 24-pin connector is used, there is no need for the optional four- or six-pin auxiliary power connectors.
    \end{itemize}
    \item When using the AT power connector, the power cable coming from the power supply will
have two separate connectors, labeled what?
    \begin{itemize}
        \item When using the AT power connector, the power cable coming from the power supply will have two separate connectors, labeled P8 and P9. When you are attaching the two parts to the motherboard, the black wires on one should be next to the black wires on the other for proper function
    \end{itemize}
      \item The SATA power connector has how many pins?
    \begin{itemize}
        \item C)The SATA power connector consists of 15 pins, with 3 pins designated for 3.3V, 5V, and 12V and with each pin carrying 1.5 amps. This results in a total draw of 4.95 watts + 7.5 watts + 18 watts, or about 30 watts
    \end{itemize}
      \item Which of the following is a desktop computer system?
    \begin{itemize}
        \item C) A thick client has the applications installed locally and will need to have sufficient resources to support the applications. A thin client only sends commands and displays output with the application on the server. Network attached storage is a storage network that is IP based, while Storage Area Networks use a storage area protocol
    \end{itemize}
     \item Which of the following is a PC that has all the capabilities of a standard PC?
    \begin{itemize}
        \item C) When discussing thin and thick clients, you should understand that a thick client is a PC
that has all the capabilities of a standard PC. It runs all applications locally from its own hard
drive. A thin client is one that has minimal capabilities and runs the applications (and perhaps
even the operating system itself) from a remote server. There is no standard client or thin host.
    \end{itemize}
     \item The amount of RAM that is required in a virtualization workstation depends on which of
the following?
    \begin{itemize}
        \item C) Number of VMs
    \end{itemize}
     \item Which IP setting is optional for network connectivity on a thin client?
    \begin{itemize}
        \item D) proxy server address is optional.
    \end{itemize}
     \item Which of the following needs the most resources?
    \begin{itemize}
        \item B) A thick client is a standard PC. When discussing thin and thick clients, you should understand that a thick client is a PC that has all the capabilities of a standard PC. It runs all applications locally from its own hard drive. A thin client is one that has minimal capabilities and runs the applications (and perhaps even the operating system itself) from a remote server. There is no medium client or stationary client
    \end{itemize}
     \item How is accountability ensured?
    \begin{itemize}
        \item C) Ensure accountability by using no shared accounts. Each user should have a unique username/password combination. Audit trails should always be created.
    \end{itemize}
     \item What software controls how the printer processes the print job?
    \begin{itemize}
        \item A) When you install a printer driver for the printer you are using, it allows the computer to print to that printer correctly (assuming you have the correct interface configured between the computer and printer). Also, keep in mind that drivers are specific to the operating system, so you need to select the one that is both for the correct printer and for the correct operating system.
    \end{itemize}
     \item What printer component turns the printed sheet over so it can be run back through the
printer and allow printing on both sides?
    \begin{itemize}
        \item B) An optional component that can be added to printers (usually laser but also inkjet) is a duplexer. This can be an optional assembly added to the printer, or built into it, but the sole purpose of duplexing is to turn the printed sheet over so it can be run back through the printer and allow printing on both sides
    \end{itemize}
     \item Which of the following refers to how the printed matter is laid out on the page?
    \begin{itemize}
        \item C) .The orientation of a document refers to how the printed matter is laid out on the page. In the landscape orientation, the printing is written across the paper turned on its long side, while in portrait the paper is turned up vertically and printed top to bottom. The driver is the software that talks between the printer and the operating system. Duplexing makes it possible to print on both sides. To collate is to create multiple copies with all sets in correct page order
    \end{itemize}
     \item Which of the following feeds through the printer using a system of sprockets and tractors?
    \begin{itemize}
        \item A)Continuous-feed paper feeds through the printer using a system of sprockets and tractors. Sheet-fed printers accept plain paper in a paper tray. Dot matrix is continuous feed; everything else is sheet fed
    \end{itemize}
     \item Which of the following should not be used more than once?
    \begin{itemize}
        \item B) Never reuse paper in a laser printer that has been through the printer once. Although it may look blank, you’re repeating the charging and fusing process on a piece of paper that most likely has something already on it.
    \end{itemize}
     \item Which of the following is a large circuit board that acts as the motherboard for the printer?
    \begin{itemize}
        \item A) This is a large circuit board that acts as the motherboard for the printer. It contains the processor and RAM to convert data coming in from the computer into a picture of a page to be printed. The imaging drum is the drum where the toner is placed on the correctly charged area. The toner cartridge is the container holding the toner. The maintenance kit contains items that should be changed periodically like rollers
    \end{itemize}
\end{enumerate}
\subsection{Virtualization and Cloud Computing}
\begin{enumerate}
    \item Which of the following involves the vendor providing the entire solution?
    \begin{itemize}
        \item Software as a service (SaaS) involves the vendor providing the entire solution. This
includes the operating system, the infrastructure software, and the application. Infrastructure
as a service ( IaaS) provides only the hardware platform to the customer. Platform as a service
( PaaS) provides a development environment. Security Information and Event Management
(SIEM) is a system that aggregates all log files and analyzes them in real time for attacks.
    \end{itemize}
    \item When a company pays another company to host and manage a cloud environment, it is
called what?
    \begin{itemize}
        \item B) When a company pays another company to host and manage a cloud environment, it is
called a public cloud solution. If the company hosts this environment itself, it is a private
cloud solution. A hybrid cloud solution is one in which both public and private clouds are
part of the solution. A community cloud is one in which multiple entities use the cloud.
    \end{itemize}
    \item Which of the following is the ability to add resources as needed on the fly and release those resources when they are no longer required?
    \begin{itemize}
        \item B) One of the advantages of a cloud environment is the ability to add resources as needed
on the fly and release those resources when they are no longer required. This makes for
more efficient use of resources, placing them where needed at any particular point in
time. These include CPU and memory resources. This is called rapid elasticity because it
occurs automatically according to the rules for resource sharing that have been deployed.
On-demand refers to the ability of the customer to add resources as needed. Virtual sharing
and stretched resources are not terms used when discussing the cloud.
    \end{itemize}
    \item In which VDI model are all desktop instances stored in a single server, requiring significant
processing power on the server?
    \begin{itemize}
        \item A) There are three models for implementing VDI:
Centralized model: All desktop instances are stored in a single server, requiring significant
processing power on the server.
Hosted model: Desktops are maintained by a service provider. This model eliminates capital
cost and is instead subject to operation cost.
Remote virtual desktops model: An image is copied to the local machine, making a constant
network connection unnecessary.
There is no local model.
    \end{itemize}
    \item Which of the following involves the vendor providing the hardware platform or data center
and the software running on the platform?
    \begin{itemize}
        \item A) PaaS (Platform as a Service) 
    \end{itemize}
    \item What is the benefit derived from using hardware-assisted virtualization?
    \begin{itemize}
        \item A) Better performance 
    \end{itemize}
    \item Which of the following is the software that allows the VMs to exist?
    \begin{itemize}
        \item B) Hypervisor
    \end{itemize}
    \item Which hypervisor type runs directly on the host’s hardware?
    \begin{itemize}
        \item B) Type I 
    \end{itemize}
    \item Which of the following is an example of a Type II hypervisor?
    \begin{itemize}
        \item A) Oracle VirtualBox
    \end{itemize}
    \item Which of the following hypervisors runs within a conventional operating system?
    \begin{itemize}
        \item Type I 
    \end{itemize}
\end{enumerate}
\subsection{Hardware and Network Troubleshooting}
\begin{enumerate}
    \item Which of the following is the final step in the CompTIA troubleshooting method?
    \begin{itemize}
        \item B) Document findings, actions, and outcomes.
    \end{itemize}
    \item Which of the following is the first step in the CompTIA troubleshooting method?
    \begin{itemize}
        \item (D) Identify the problem 
    \end{itemize}
    \item What is the most common reason for an unexpected reboot?
    \begin{itemize}
        \item A. One common reason for shutdowns is overheating. Often when that is the case,
however, the system reboots itself rather than just shutting down.
    \end{itemize}
    \item Which of the following is typically not a cause of system lockups?
    \begin{itemize}
        \item D. A bad NIC driver would cause the NIC not to work but would not cause a system
lockup.
    \end{itemize}
    \item What are proprietary screen crashes called in Windows?
    \begin{itemize}
        \item B. Once a regular occurrence when working with Windows, blue screens (also known as
the blue screen of death) have become much less frequent.
    \end{itemize}
    \item Which operating system uses the Pinwheel of Death as a proprietary screen crash?
    \begin{itemize}
        \item A. While Microsoft users have the BSOD to deal with, Apple users have also come to have
the same negative feelings about the Pinwheel of Death. This is a multicolored pinwheel
mouse pointer.
    \end{itemize}
    \item What are the small dots on the screen that are filled with a color?
    \begin{itemize}
        \item A. Pixels are the small dots on the screen that are filled with a color; as a group they
present the image you see on the screen.
    \end{itemize}
    \item What are visual anomalies that appear on the screen called? 
    \begin{itemize}
        \item B. Artifacts are visual anomalies that appear on the screen. They might be pieces of images
left over from a previous image or a “tear in the image” (it looks like the image is divided
into two parts and the parts don’t line up).
    \end{itemize}
    \item What is the light in the device that powers the LCD screen?
    \begin{itemize}
        \item A. The backlight is the light in the device that powers the LCD screen. It can go bad over
time and need to be replaced, and it can also be held captive by the inverter. The inverter
takes the DC power the laptop is providing and boosts it up to AC to run the backlight. If
the inverter goes bad, you can replace it on most models (it’s cheaper than the backlight).
    \end{itemize}
    \item Which of the following is a user interface feature designed by HTC?
    \begin{itemize}
        \item B. Touch flow, or TouchFLO, is a user interface feature designed by HTC. It is used by
dragging your finger up and down or left and right to access common tasks on the screen. This
movement is akin to scrolling the screen up and down or scrolling the screen left and right.
    \end{itemize}
    \item Which of the following indicates that the fuser is not fusing the toner properly on the
paper?
    \begin{itemize}
        \item B. With laser printers, streaks usually indicate that the fuser is not fusing the toner
properly on the paper. It could also be that the incorrect paper is being used. In laser
printers, you can sometimes tell the printer that you are using a heavier paper. For dot-
matrix, you can adjust the platen for thicker paper.
    \end{itemize}
    \item Which of the following indicates that the toner cartridge is just about empty?
    \begin{itemize}
        \item C. In laser printers, faded output usually indicates that the toner cartridge is just about
empty. You can usually remove it, shake it, and replace it and then get a bit more life out of
it before it is completely empty, but it is a signal that you are near the end.
    \end{itemize}
    \item If you can ping resources by IP address but not by name,
functional.
   \begin{itemize}
       \item B) DNS . You may be able to ping the entire network using IP addresses, but most access is done
by name, not IP address. If you can’t ping resources by name, DNS is not functional,
meaning either the DNS server is down or the local machine is not configured with the
correct IP address of the DNS server.
   \end{itemize}
   \item Which of the following should be set to the IP address of the router interface connecting to
the local network?
    \begin{itemize}
        \item C) If the computer cannot connect to the default gateway, it will be confined to
communicating with devices on the local network. This IP address should be that of the
router interface connecting to the local network.
    \end{itemize}
\end{enumerate}
\section{Mike Meyers} 
\subsection{The Visible Computer}
\begin{enumerate}
    \item Which version of Windows introduced the Metro UI?
    \begin{itemize}
        \item B)Microsoft introduced Metro UI with Windows 8.
    \end{itemize}
    \item Which Windows 8 feature did Microsoft not include in Windows 10?
    \begin{itemize}
        \item D) Microsoft did not include the Charms bar in Windows 10. Bye!
    \end{itemize}
    \item What macOS feature is essentially multiple Desktops?
    \begin{itemize}
        \item D) Spaces is the term Apple uses for multiple Desktops in macOS.
    \end{itemize}
    \item What KDE feature is essentially the Start button?
    \begin{itemize}
        \item Kickoff functions like a Start button for KDE desktops.
    \end{itemize}
    \item The user Mike has downloaded files with his Web browser. Where
will they be stored by default?
    \begin{itemize}
        \item C:$\backslash Users\backslash Mike\backslash$ Downloads
    \end{itemize}
    \item 32-bit programs are installed into which folder by default in a 64-bit
edition of Windows?
    \begin{itemize}
        \item By default, 32-bit applications install into the $C:\backslash Program Files
(x86)$ folder.
    \end{itemize}
    \item Which macOS feature is functionally equivalent to Windows File
Explorer?
    \begin{itemize}
        \item A) Finder is the equivalent of File Explorer.
    \end{itemize}
    \item Which of the following paths would open Administrative Tools in
Windows 8.1?
    \begin{itemize}
        \item B) To open Administrative Tools, right-click the Start button and
select Administrative Tools. Easy!
    \end{itemize}
    \item What feature of macOS is the equivalent of the command-line
interface in Windows
    \begin{itemize}
        \item C) Terminal is the equivalent of the Windows command-line
interface.
    \end{itemize}
    \item What Windows app in Windows 10 combines many utilities into a
unified tool?
    \begin{itemize}
        \item C)The Settings app in Windows 10 offers many utilities in a unified
interface.
    \end{itemize}

\end{enumerate}
\subsection{CPUs}
\begin{enumerate}
     \item What do registers provide for the CPU?
    \begin{itemize}
        \item B) The CPU uses registers for temporary storage of internal
commands and data.
    \end{itemize}
    \item C) What function does the external data bus have in the PC?
    \begin{itemize}
        \item The external data bus provides a channel for the flow of data and
commands between the CPU and RAM.
    \end{itemize}
    \item What is the function of the address bus in the PC?
    \begin{itemize}
        \item A) The address bus enables the CPU to communicate with the
memory controller chip.
    \end{itemize}
    \item Which of the following terms are measures of CPU speed?
    \begin{itemize}
        \item A) Megahertz and gigahertz
    \end{itemize}
    \item Which CPU feature enables the microprocessor to support running
multiple operating systems at the same time?
    \begin{itemize}
        \item D) Virtualization support
    \end{itemize}
    \item Into which socket could you place an Intel Core i5?
    \begin{itemize}
        \item B) Socket LGA 1151
    \end{itemize}
    \item Which feature enables a single-core CPU to function like two CPUs?
    \begin{itemize}
        \item A) Hyper-Threading
    \end{itemize}
    \item What steps do you need to take to install a Core i3 CPU into an FM2+
motherboard?
    \begin{itemize}
        \item D) Take all of the steps you want to take because it’s not going to
work
    \end{itemize}
    \item A client calls to complain that his computer starts up, but crasheswhen Windows starts to load. After a brief set of questions, you findout that his nephew upgraded his RAM for him over the weekend andcouldn’t get the computer to work right afterward. What could be theproblem?
    \begin{itemize}
        \item A)   Most likely, the nephew disconnected the CPU fan to get at the RAM slots and simply forgot to plug it back in. 
    \end{itemize}
    \item Darren has installed a new CPU in a client’s computer, but nothinghappens when he pushes the power button on the case. The LED onthe motherboard is lit up, so he knows the system has power. Whatcould the problem be?
    \begin{itemize}
        \item The best answer here is that he forgot the thermal paste, thoughyou can also make an argument for a disconnected fan.
    \end{itemize}
\end{enumerate}
\subsection{RAM}
\begin{enumerate}
    \item Steve adds a second 8-GB 288-pin DIMM to his PC, which should
bring the total RAM in the system up to 16 GB. The PC has an Intel
Core i7 4-GHz processor and four 288-pin DIMM slots on the
motherboard. When he turns on the PC, however, only 8 GB of RAM
shows up in Windows Settings app. Which of the following is most
likely to be the problem?
      \begin{itemize}
          \item A) Steve failed to seat the RAM properly.
      \end{itemize}
    \item Scott wants to add 8 GB of PC3-12800 DDR3 to an aging but still
useful desktop system. The system has a 200-MHz motherboard and
currently has 4 GB of non-ECC DDR3 RAM in the system. What else
does he need to know before installing?
    \begin{itemize}
        \item D) If the system can handle that much RAM.
    \end{itemize}
    \item What is the primary reason that DDR4 RAM is faster than DDR3
RAM?
    \begin{itemize}
        \item The input/output speed of the DDR4 RAM is faster.
    \end{itemize}
    \item What is the term for the delay in the RAM’s response to a request
from the MCC?
    \begin{itemize}
        \item C) Latency (MCC Memory Controller Chip) 
    \end{itemize}
    \item How does an NMI (Non-Maskable Interrupt) manifest on a Windows system?
    \begin{itemize}
        \item A) Blue Screen of Death.
    \end{itemize}
    \item Silas has an AMD-based motherboard with two sticks of DDR3 RAM
installed in two of the three RAM slots, for a total of 8 GB of system
memory. When he runs CPU-Z to test the system, he notices that the
software claims he’s running single-channel memory. What could be
the problem? (Select the best answer.)
    \begin{itemize}
        \item D)He needs to move one of the installed sticks to a different slot to
activate dual-channel memory.
    \end{itemize}
    \item Which of the following Control Panel applets will display the amount
of RAM in your PC?
    \begin{itemize}
        \item A) System 
    \end{itemize}
    \item What is the best way to determine the total capacity and specific type
of RAM your system can handle?
    \begin{itemize}
        \item A) Check the motherboard book.
    \end{itemize}
    \item Gregor installed a third stick of known good RAM into his Core i7
system, bringing the total amount of RAM up to 12 GB. Within a fewdays, though, he started having random lockups and reboots,
especially when doing memory-intensive tasks such as gaming. What
is most likely the problem?
    \begin{itemize}
        \item Gregor installed RAM that didn’t match the speed or quality of
the RAM in the system.
    \end{itemize}
    \item Cindy installs a second stick of DDR4 RAM into her Core i5 system,
bringing the total system memory up to 16 GB. Within a short period
of time, though, she begins experiencing Blue Screens of Death. What
could the problem be?
    \begin{itemize}
        \item She installed faulty RAM.
    \end{itemize}
\end{enumerate}
\subsection{Firmware} 
\begin{enumerate}
    \item What does BIOS provide for the computer? (Choose the best answer.)
    \begin{itemize}
        \item B) BIOS provides the programming that enables the CPU to
communicate with other hardware. 
    \end{itemize}
    \item What is the correct boot sequence for an older BIOS-based PC?
A. CPU,
    \begin{itemize}
        \item D) Power good, CPU, POST, boot loader, operating system 
    \end{itemize}
    \item Jill decided to add a second hard drive to her computer. She thinks
she has it physically installed correctly, but it doesn’t show up in
Windows. Which of the following options will most likely lead Jill
where she needs to go to resolve the issue?
    \begin{itemize}
        \item B) Jill should reboot the computer and watch for instructions to enter
the CMOS setup utility (for example, a message may say to press the
DELETE key). She should do what it says to go into CMOS setup
    \end{itemize}
    \item Henry bought a new card for capturing television on his computer.
When he finished going through the packaging, though, he found no
driver disc, only an application disc for setting up the TV capture
software. After installing the card and software, it all works
flawlessly. What’s the most likely explanation?
    \begin{itemize}
        \item B) The device has an option ROM that loads BIOS, so there’s no
need for a driver disc.
need for a driver disc. 
    \end{itemize}
    \item Which of the following most accurately describes the relationship
between BIOS and hardware?
    \begin{itemize}
        \item A) All hardware needs BIOS. 
    \end{itemize}
    \item After a sudden power outage, Samson’s PC rebooted, but nothing
appeared on the screen. The PC just beeps at him, over and over and
over. What’s most likely the problem?
    \begin{itemize}
        \item A) The power outage toasted his RAM. 
    \end{itemize}
    \item Davos finds that a disgruntled former employee decided to sabotage
her computer when she left by putting a password in CMOS that stops
the computer from booting. What can Davos do to solve this
problem?
    \begin{itemize}
        \item C) Davos should find the CLRTC jumper on the motherboard. Then
he can boot the computer with a shunt on the jumper to clear the
CMOS information. 
    \end{itemize}
    \item Richard over in the sales department went wild in CMOS and made a
bunch of changes that he thought would optimize his PC. Now most
of his PC doesn’t work. The computer powers up, but he can only get
to CMOS, not into Windows. Which of the following tech call
answers would most likely get him up and running again?
    \begin{itemize}
        \item D) Please don’t hand Richard a screwdriver! Having him load
Optimized Default settings will most likely do the trick.
    \end{itemize}
    \item Jill boots an older Pentium system that has been the cause of several
user complaints at the office. The system powers up and starts to run
through POST, but then stops. The screen displays a “CMOS
configuration mismatch” error. Of the following list, what is the most
likely cause of this error?
    \begin{itemize}
        \item A) CMOS battery is likely dying.
    \end{itemize}
    
    \item Where does Windows store device drivers?
    \begin{itemize}
        \item C) Registry
    \end{itemize}
\end{enumerate}
\subsection{Motherboards} 
\begin{enumerate}
    \item Which of the following statements about the expansion bus is true?
    \begin{itemize}
        \item B)The expansion bus crystal sets the speed for the expansion bus.
    \end{itemize}
    \item What does a black down arrow next to a device in Device Manager
indicate?
    \begin{itemize}
        \item D)The device has been disabled.
    \end{itemize}
    \item Which variation of the PCI bus was specifically designed for laptops?
    \begin{itemize}
        \item C) Mini-PCI 
    \end{itemize}
    \item Which of the following form factors dominates the PC industry?
    \begin{itemize}
        \item B) ATX 
    \end{itemize}
    \item Amanda bought a new system that, right in the middle of an important
presentation, gave her a Blue Screen of Death. Now her system won’t
boot at all, not even to CMOS. After extensive troubleshooting, she
determined that the motherboard was at fault and replaced it. Now the
system runs fine. What was the most likely cause of the problem?
    \begin{itemize}
        \item A) Although all of the answers are plausible, the best answer here is
that her system suffered burn-in failure.
    \end{itemize}
    \item Martin bought a new motherboard to replace his older ATX
motherboard. As he left the shop, the tech on duty called after him,
“Check your standoffs!” What could the tech have meant?
    \begin{itemize}
        \item C. Standoffs are the metal connectors that attach the motherboard to
the case.
    \end{itemize}
    \item Solon has a very buggy computer that keeps locking up at odd
moments and rebooting spontaneously. He suspects the motherboard.
How should he test it?
    \begin{itemize}
        \item Check settings, verify good components, replace components,
and document all testing.
    \end{itemize}
    \item When Jane proudly displayed her new motherboard, the senior tech
scratched his beard and asked, “What kind of northbridge does it
have?” What could he possibly be asking about?
    \begin{itemize}
        \item C. The tech is using older terminology to refer to the chips—the
chipset—that help the CPU communicate with devices.
    \end{itemize}
    \item What companies dominate the chipset market? (Select two.)
    \begin{itemize}
        \item A, C) AMD, NVIDIA 
    \end{itemize}
    \item If Windows recognizes a device, where will it appear?
    \begin{itemize}
        \item Device Manager
    \end{itemize}
    
\end{enumerate}
\subsection{Motherboards} 
\begin{enumerate}
    \item 
\end{enumerate}
\section{CompTIA A+ Core 1 Exam Cram David L. Prowse} 
\subsection{Laptop Part I}
\begin{enumerate}
    \item What kinds of hard drives are used by laptops? (Select all that
apply.) 
     \begin{itemize}
         \item A, B and C. Solid state drives (SSDs), M.2 drives, and mag‐
netic disk drives are all found on laptops. Which drive the lap‐
top uses will depend on its age and whether or not it has been
upgraded. DVD-ROM drives are not hard drives, they are opti‐
cal drives, and as of 2015 or so, are not commonly found on
laptops.
     \end{itemize}
     \item What is the module format for a stick of SODIMM DDR4
RAM?
     \begin{itemize}
         \item C. DDR4 SODIMM modules have 260 pins. DDR (known as
DDR1) and DDR2 are 200-pin. DDR3 is 204-pin. 1/8 inch is
the size associated with smaller hard drives used in some lap‐
tops.
     \end{itemize}
     \item You just added a second memory module to a laptop. Howev‐
er, after rebooting the system, the OS reports the same amount
of memory as before. What should you do next?
     \begin{itemize}
         \item D. The next step you should take is to reseat the memory.
SODIMMs can be a bit tricky to install. They must be firmly
installed, but you don’t want to press too hard and damage any
components. If the laptop worked fine before the upgrade, you
shouldn’t have to replace the modules or the motherboard.
Windows Update will not find additional RAM.
     \end{itemize}
     \item Which of the following are ways that a laptop can communi‐
cate with other computers? (Select all that apply.)
     \begin{itemize}
         \item A, B, and D. Some of the methods that laptops use to commu‐
nicate with other computers include: Bluetooth, WLAN, and
cellular WAN wireless connections, plus wired connections
like Ethernet (RJ45) and for older laptops, dial-up (RJ11). The
DC jack is the input on the laptop that accepts power from the
AC adapter.
     \end{itemize}
     \item Which of the following are possible reasons that a laptop’s
keyboard might fail completely? (Select thetwo best answers.)
     \begin{itemize}
         \item C) The user spilled coffee on the laptop. D) The keyboard was disabled in the Device Manager.B and C. A laptop’s keyboard could fail due to a disconnected
or loose keyboard ribbon cable. It could also fail if a user
spilled coffee on the laptop, by being dropped on the ground,
and so on. One stuck key will not cause the entire keyboard tofail, and on most laptops, the keyboard cannot be disabled in
the Device Manager. It can be uninstalled, but not disabled.
     \end{itemize}
     \item A user doesn’t see anything on his laptop’s screen. He tries to
use AC power and thinks that the laptop is not receiving any.
Which of the following are two possible reasons for this? (S‐
elect the two best answers.)
     \begin{itemize}
         \item A and B. An incorrect adapter will usually not power a laptop.
The adapter used must be exact. And of course, if the laptop is
not plugged in properly to the adapter, it won’t get power.
Windows doesn’t play into this scenario. And if the battery
was dead, it could cause the laptop to not power up, but only if
the AC adapter was also disconnected; the scenario states that
the user is trying to use AC power.
     \end{itemize}
     \item One of your customers reports that she walked away from her
laptop for 30 minutes. When she returned, the display was very
dim. She increased the brightness setting and moved the mouse
but to no effect. What should you do first?
     \begin{itemize}
         \item D. It could be that the laptop is now on battery power, which is
usually set to a dimmer display and shorter sleep configura‐
tion. This indicates that the laptop is not getting AC power
from the AC outlet for some reason. The battery power setting
is the first thing you should check; afterward, start trou‐
bleshooting the AC adapter, cable, AC outlet, and so on. It’s
too early to try replacing the display; try not to replace some‐
thing until you have ruled out all other possibilities. A dim
screen is not caused by OS corruption. No need to plug in an
external monitor; you know the video adapter is working, it’s
just dim.
     \end{itemize}
     \item Which are the most common laptop hard drive form factors?
(Select two.)
     \begin{itemize}
         \item B and D. The bulk of the hard drives in laptops are 2.5 inches
wide, though ultra-small laptops and other small portable de‐
vices might use a hard drive as small as 1.8 inches. 5.1 and 7.1
refer to speaker surround sound systems, not hard drive form
factors.
     \end{itemize}
     \item You are helping a customer with a laptop issue. The customer
said that two days ago the laptop was accidentally dropped
while it was charging. You observe that the laptop will not turn
on and that it is connected to the correct power adapter. Which
of the following is the most likely cause?
     \begin{itemize}
         \item . D. The DC jack was probably damaged when the laptop was
dropped. That’s because it was plugged in (charging) and itprobably fell on the plug that connects to the DC jack (which
is easily damaged on many laptops by the way). The customer
probably used the laptop until the battery became discharged
before noticing that the laptop wouldn’t take a charge anymore
—that’s why it won’t turn on at all. So the battery is probably
not the issue. A power adapter can be damaged, but the DC-in
jack is more easily damaged. The hard drive and the BIOS nor‐
mally will not affect whether the laptop will turn on.
     \end{itemize}
\end{enumerate}
\subsection{Laptop Part II} 
\begin{enumerate}
    \item Which kind of video technology do most laptop LCDs use?
    \begin{itemize}
        \item A. TFT active-matrix displays are the most common in laptops that use LCDs. Passive-matrixscreens have been discontinued, but you might see an older laptop that utilizes this technology.OLED technology is a newer and different technology that is not based on TFT displays, but instead uses emissive display technology, meaning that each dot on the screen is illuminated by ase parate diode. OLED displays can however bepassive-matrix or active-matrix controlled. The MAC ID is the hexadecimal address associated with a network adapter, such as a Wi-Fi adapter or network card.
    \end{itemize}
    \item Which of the following uses an organic compound that emits light?
    \begin{itemize}
        \item C. OLED (organic light emitting diode) display suse an organic compound or film that emits light.TFT active-matrix implies LCD, and neither of them use organic compounds the way OLED does.In-plane switching (IPS) is a type of LCD technology that increases the available viewing angle compared to older technologies such as twisted nematic (TN) matrix LCDs. However, IPS is generally considered inferior to OLED screens when it comes to brightness and contrast ratio when viewed from an angle. LED screens use a film and diodes, but not organically in the way that OLED does, and not at such a small size.
    \end{itemize}
    \item Which of the following are two possible reasons why a laptop’s LCD suddenly went blank with no user intervention? (Select the two best answers.)
    \begin{itemize}
        \item  A and C. A damaged inverter or burned-out bulb could cause a laptop’s display to go blank. You can verify whether the LCD is still getting a signal by shining a flashlight at the screen. A damaged LCD usually works to a certain extent and will either be cracked, have areas of Windows missing, or show other signs of damage. An incorrect resolution setting can indeed make the screen suddenly go blank (or look garbled), but that scenario will most likely occur only if the user has changed the resolution setting—the answer specifies with no user intervention.
    \end{itemize}
    \item Which of the following allows us to access a WLAN?
    \begin{itemize}
        \item E. A Wi-Fi card, also known as a Wi-Fi network adapter, allows us to connect to a WLAN (wireless local area network) which is essentially another name for a Wi-Fi network. LED is a type of display. A webcam is used to communicate visually and audibly with others, or to record oneself. A digitizer is the device that converts tapped or written impulses on a screen into digital information that the operating system can use. A stylus is a writing device used with a digitizer or touchscreen.
    \end{itemize}
    \item Which of the following keys should you press to
enable a secondary display on a laptop? (Select the
two best answers.)
    \begin{itemize}
        \item A and D. To enable a secondary, or external,
display on a laptop you would use the Fn key
(called the function key) and a special function
key, for example F3 or F4, whichever one that
corresponds to screen switching. It’s this
combination of keys that allows you to make use
of displays plugged into HDMI or other ports on
the laptop. The caps lock key enables a user to
type in all uppercase letters. The number lock key (if available) turns on the numeric keypad (if the
laptop has one). The insert key is often used by
programs such as word processors in one of two
modes: overtype; where anything that is typed is
written over any existing text; and insert mode,
where typed characters force the existing text
over.
    \end{itemize}
    \item When a user types, a laptop’s screen displays
letters and numbers instead of only letters. What
should you check first?
    \begin{itemize}
        \item Num Lock key
    \end{itemize}
    \item You are required to install an anti-theft solution
for a customer’s laptop. Which of the following
should you perform?
    \begin{itemize}
        \item B. Install a cable lock to increase the security of a
laptop and decrease the chances of theft. Docking
stations and port replicators offer increased
functionality for a laptop but do not increase
security; the laptop can be easily disconnected
from them. Installing Windows is not an anti-theft
solution, nor any type of security precaution.
Configuring a password in the BIOS/UEFI is a
good security practice, but it will not help avoid
theft. However, if the laptop is stolen, a user
password (and administrator password) that is
configured in the BIOS can help prevent a person
    \end{itemize}
    \item You are helping a project manager with a
presentation using a laptop which feeds video to a
projector. During your tests, the projector’s image
begins to flicker. The laptop’s display does not
have any problems. You attempt to change the
resolution on the laptop, but the issue continues.
Which of the following should you do next?
A.
    \begin{itemize}
        \item B. Check the connectivity of the cable. If it is
flickering, chances are that the cable is loose, or
the cable’s quality is lacking. Screen flicker is
more common with VGA cables, but it can happen
with just about any connection. Remember,
always check the basic stuff first: connectivity,
power, and so on. It is unlikely that the projector
settings will make a difference based on this
particular problem. You cannot change the aspect
ratio by itself on most laptops, however, when you
change the resolution (which was already done in
the question) you might be changing the aspect
ratio as well, depending on the resolution
selected. If the power cable was loose or damaged,
it would probably result in more than just screen
flicker; the projector might power off and power
back on, which would prevent the image from
being displayed for at least several seconds while
the projector powers back up. Great job so far!
Two chapters down!
    \end{itemize}
    \item Which of the following is not a mobile device?
    \begin{itemize}
        \item C. The desktop PC is not a mobile device. It is a stationary
computer that is meant to stay at a person’s desk. Tablets,
smartphones, and e-readers are all examples of mobile devices.
    \end{itemize}
    \item Which type of memory do most mobile devices store long-term
data to?
    \begin{itemize}
        \item D. Most mobile devices store their long-term data to solid-state
flash memory. They do not use SATA as the method of con‐
nectivity. LPDDR4 is a common type of RAM used in mobile
devices for short-term storage.
    \end{itemize}
     \item You have been tasked with connecting a wireless earpiece to a
smartphone. Which technology would you most likely use?
    \begin{itemize}
        \item D. When connecting an earpiece (those little cricket-looking
devices) to a smartphone, you would most likely use Bluetooth
—just remember that most of them have a 30-foot range (10
meters). Wi-Fi is less likely to be used; it is more likely to be
used to connect the smartphone to the LAN and ultimately to
the Internet. NFC is used to transmit data between mobile de‐
vices in close proximity to each other. 3.5 mm refers to the au‐dio port on a mobile device. It is quite possible that a user will
utilize a wired headset, but the question focuses on wireless.
    \end{itemize}
    \item You have been tasked with setting up a device for a salesper‐
son’s vehicle. It should be able to display maps and give direc‐
tions to the person while driving. Which of the following de‐
vices would perform these tasks? (Select the two best answers.)
    \begin{itemize}
        \item A and C. A standalone GPS device or a smartphone (equipped
with a GPS app) would do the job here. Both can display maps
and give directions to a person while driving. The other de‐
vices are not designed to function in this manner.
    \end{itemize}
    \item Which type of charging connector would you find on an iPad?
    \begin{enumerate}
        \item B. The Lightning connector is one of Apple’s proprietary
charging and synchronization connectors used by iPads and
iPhones, although Apple also uses USB-C. Micro-USB is used
by older Android-based mobile devices—while USB-C is
more common on newer devices. Thunderbolt is a high-speed
hardware interface used in desktop computers, which we will
discuss more in Chapter 9, “Cables and Connectors.” IP68
deals with ingress protection from dust and water jets.
    \end{enumerate}
    \item You are required to add long-term storage to a smartphone.
Which type would you most likely add?
    \begin{itemize}
        \item B. You would most likely add a microSD card (if the smart‐
phone has a slot available for add-on or upgrading). This is the
most common method for adding long-term storage. DDR4 is
a type of RAM; it is not used for adding long-term memory
storage. Some smartphones will use LPDDR4 as their main
memory, but this is part of the SoC, and not accessible to the
typical user. An SSD is a solid-state drive which generally
means a hard drive that is installed to a PC or laptop, con‐
nected either as SATA or M.2. These are too large for smart‐phones and tablets. A SIM is a subscriber identity module,
usually represented as a small card (mini-SIM) used in smart‐
phones that securely stores authentication information about
the user and device, such as the international mobile subscriber
identity (IMSI) which we will discuss more in the following
chapter.
    \end{itemize}
    \item The organization you work for allows employees to work from
their own mobile devices in a BYOD manner. You have been
tasked with setting up the devices so that they can “beam” in‐
formation back and forth between each other. What is this
known as?
    \begin{itemize}
        \item E. “Beaming” the information back and forth can be accom‐
plished in a couple of ways, primarily by using near field com‐
munication (NFC). This can only be done if the devices are in
close proximity to each other. NFC is commonly used for con‐
tactless payment systems. Another potential option would be
Apple’s AirDrop, but this relies on Bluetooth (for finding de‐
vices) and Wi-Fi (for transmitting data), and of course relies
on using Apple-based devices. A mobile hotspot enables a
smartphone or tablet to act as an Internet gateway for other
mobile devices and computers. IoT stands for the Internet of
Things. In the question, it said employees can use their mobile
devices in a BYOD manner, but CYOD is a bit different. This
means that employees can choose a device to use for work pur‐
poses (most likely whichever type they are more familiar
with). Whether or not the employees can use those for personal
purposes is usually defined by company policy. IR stands for
infrared, which is less commonly found on smartphones as of
2017.
    \end{itemize}
    \item Which of the following can be useful in areas where a smart‐
phone has cellular access, but the PC (or laptop) cannot con‐
nect to the Internet?
    \begin{itemize}
        \item D. Tethering can allow a desktop computer or laptop to share
the mobile device’s Internet connection. Tethering functional‐
ity can be very useful in areas where a smartphone has cellular
access, but the PC/laptop cannot connect to the Internet. Mo‐
bile device accessories such as headsets, speakers, gamepads,extra battery packs and protective covers are useful, however
they are not used to connect to the Internet. IP codes are used
to classify and rate the degree of protection against dust and
water (for example, IP68). A perfect example of a proprietary
vendor-specific connector is the Apple Lightning connector
that can only be used on iOS devices.
    \end{itemize}
\end{enumerate}
\subsection{Smartphones, Tablets, and Other Mobile Devices, Part 2}
\begin{enumerate}
    \item Which of the following connections require a username, pass‐
word, and SMTP server? (Select the two best answers.)
    \begin{itemize}
        \item C and E. POP3 and IMAP e-mail connections require an in‐
coming mail server (either POP3 or IMAP) and an outgoing
mail server (SMTP). Bluetooth and Wi-Fi connections do not
require a username or SMTP server. Bluetooth might require a
PIN, and Wi-Fi will almost always require a passcode. Ex‐
change connections require a username and password, but no
SMTP server. The Exchange server acts as the incoming and
outgoing mail server.
    \end{itemize}
    \item When manually configuring a Wi-Fi connection, which step
occurs after successfully entering the SSID?
    \begin{itemize}
        \item C. After you enter the SSID (if it’s correct) you would enter
the passcode for the network. POP3 has to do with configuring
an e-mail account. If you have already entered the SSID, then
you should be within range of the wireless access point
(WAP). Scanning for networks is the first thing you do when
setting up a Wi-Fi connection.
    \end{itemize}
    \item Which of the following allows other mobile devices to wire‐
lessly share your mobile device’s Internet connection?
    \begin{itemize}
        \item . D. Mobile hotspot technology (sometimes referred to as Wi-Fi
tethering) allows a mobile device to share its Internet connec‐
tion with other Wi-Fi capable devices. Another possibility
would be USB tethering, but that is done in a wired fashion.
NFC stands for near field communication—a technology that
allows two mobile devices to send information to each other
when they are in close proximity. Airplane mode will disableall wireless connectivity including (but not limited to) cellular,
Wi-Fi, and Bluetooth. IMAP is another e-mail protocol similar
to POP3.
    \end{itemize}
     \item Which of the following identifies the user of the device?
    \begin{itemize}
        \item A. International Mobile Subscriber Identity (IMSI) or IMSI ID
is used to identify the user of the device. IMEI stands for Inter‐
national Mobile Station Equipment Identity and identifies the
phone used. In other words, the IMEI ID identifies the device
itself. S/MIME (Secure/Multipurpose Internet Mail Exten‐
sions) is used for authentication and message integrity and is
built-in to some e-mail clients. In other words, it is used to en‐
crypt email. Virtual private networking (VPN) technology is
used to make secure connections—tunneling though the
provider’s radio network.
    \end{itemize}
     \item Which of the following is the most common connection
method when synchronizing data from a mobile device to a
PC?
    \begin{itemize}
        \item C) USB is the most common connection method used when
synchronizing data from a mobile device to a PC. Though Wi-
Fi and Bluetooth are also possible, they are less common.
Lightning is the port found on some of Apple’s mobile de‐
vices, but still ends in USB when connecting to a PC or Mac.
    \end{itemize}
     \item Which of the following is used to synchronize contacts from an
iPad to a PC?
    \begin{itemize}
        \item C. PC users need iTunes to synchronize contacts and other data
from an iPad to a PC. While Gmail can work to synchronize
contacts, it is all based on web storage; nothing is actually
stored on the iPad. Google Play is a place to get applications
and other items for Android. Sync Center is a Control Panel
utility that enables synchronization across Windows 10 de‐
vices.
    \end{itemize}
     \item What is it known as when a user connects to several services
using several apps but with only one username and password?
    \begin{itemize}
        \item B. SSO (single sign-on) is a type of authentication where a user
logs in once but is granted access to multiple services. Android
Auto is a screen sharing/synchronizing app used on Android-
based mobile devices to communicate with a properly
equipped automobile. iTunes is a music/media program that
can be used to sync up a mobile device to a PC or Apple de‐
vice. BT is short for Bluetooth. Exchange ActiveSync is a
client-based protocol that allows a user to sync a mobile device
with an Exchange Server mailbox.
    \end{itemize}
\end{enumerate}
\subsection{Ports, Protocols, and Network Devices} 
\begin{enumerate}
    \item Which protocol uses port 22?
    \begin{itemize}
        \item  C. SSH (Secure Shell) uses port 22, FTP uses port 21, Telnet
uses port 23, and HTTPS uses port 443.
    \end{itemize}
    \item Which of these would be used for streaming media?
    \begin{itemize}
        \item C. User Datagram Protocol (UDP) is used for streaming media.
It is connectionless, whereas TCP is connection-oriented and
not a good choice for streaming media. RDP is the Remote
Desktop Protocol used to make connections to other comput‐
ers. DHCP is the Dynamic Host Configuration Protocol used
to assign IP addresses to clients automatically.
    \end{itemize}
    \item Which ports are used by the IMAP protocol?
    \begin{itemize}
        \item D. The Internet Message Access Protocol (IMAP) uses port
143 by default and port 993 as a secure default. DNS uses port
53. DHCP uses port 68. HTTP uses port 80. HTTPS uses port
443. POP3 uses port 110 and 995 as a secure default. Know
those ports!
    \end{itemize}
    \item A user can receive e-mail but cannot send any. Which protocol
is not configured properly?
    \begin{itemize}
        \item C. The Simple Mail Transfer Protocol (SMTP) is probably not
configured properly. It deals with sending mail. POP3 receives
mail. FTP sends files to remote computers. SNMP is used to
manage networks.
    \end{itemize}
    \item Which of the following is most often used to connect a group
of computers in a LAN? (Select all that apply.)
    \begin{itemize}
        \item B and D. Computers in a LAN are connected by a central con‐
necting device; the most common of which are the switch and
the wireless access point (WAP). Hubs can also be used, but
those are deprecated devices; they are the predecessor of the
switch. A router is designed to connect two networks together.
Now, you might say, “Wait! My router at home has four ports
on the back for computers to talk to each other.” Well, that is
actually the switch portion of a SOHO “router”. The actual
router functionality is in the connection between the two net‐
works—the switched LAN and the Internet. A bridge is used
to connect two LANs or separate a single LAN into two sec‐
tions.
    \end{itemize}
    \item What device protects a network from unwanted intrusion?
    \begin{itemize}
        \item D. A firewall is a hardware appliance or software application
that protects one or more computers from unwanted intrusion.
A switch is a device that connects multiple computers together
on a LAN. A router is used to connect two or more networks.
An access point (or wireless access point) allows Wi-Fi-en‐
abled computers and devices to communicate on the LAN
wirelessly.
    \end{itemize}
    \item Which of the following network devices moves frames of data
between a source and destination based on their MAC address‐
es?
    \begin{itemize}
        \item B) A switch sends frames of data between computers by identi‐
fying the systems by their MAC addresses. A hub broadcasts
data out to all computers. The computer that it is meant for ac‐cepts the data; the rest drop the information. Routers enable
connections with individual high-speed interconnection points
and route signals for all the computers on the LAN out to the
Internet. A modem is a device that allows a computer to access
the Internet by changing the digital signals of the computer to
analog signals used by a typical land-based phone line.
    \end{itemize}
    \item Which of the following network devices allows a remote de‐
vice to obtain Ethernet data as well as electrical power?
    \begin{itemize}
        \item B) A PoE injector sends Ethernet data and power over a single
twisted-pair cable to a remote device. PD stands for “powered
device,” the PoE-compliant remote device that is receiving the
power. A repeater extends the distance of a network connec‐
tion. While a PoE injector can act as a repeater, not all re‐
peaters are PoE injectors. A router makes connections from
one network to another or from the LAN to the Internet.
    \end{itemize}
    \item Which of the following devices can be configured when ac‐
cessed from a browser or SSH or similar configuration tool?
    \begin{itemize}
        \item A)Managed switches can be configured when accessed from a
browser or SSH or similar configuration tool. For example,
you can: change the device’s IP address; configure ports; and
monitor the switch and other devices with SNMP. On the other
hand, unmanaged switches don’t have these capabilities, they
simply connect devices and computers together for transmis‐
sion of data over the Ethernet network. A patch panel is a
physical hardware device that acts as a termination point for all
of the network cables in a building. A network interface card
(NIC) allows for connectivity to a computer network. It is a
physical device that can be added to a computer or networking
device that has an open and compatible slot.
    \end{itemize}

\end{enumerate}
\subsection{SOHO Networks and Wireless Protocols} 
\begin{enumerate}
     \item Which of the following allows for network throttling of indi‐
vidual computers or applications?
    \begin{itemize}
        \item A. Quality of Service (QoS) is a technology used in SOHO
routers that can throttle bandwidth, and give higher priority to
individual computers or applications. Port forwarding is used
to forward outside network ports to internal IP addresses. A
DMZ is a protected area between the LAN and the Internet—
often inhabited by company servers. The Dynamic Host Con‐
figuration Protocol (DHCP) is the protocol in charge of auto‐
matically handing out IP address information to clients.
    \end{itemize}
    \item Which of the following forwards an external network port to an
internal IP address/port on a computer on the LAN?
    \begin{itemize}
        \item B. Port forwarding is used to forward external network ports to
an internal IP and port. This is done so a person can host ser‐
vices such as FTP internally. Network address translation
(NAT) is used by most routers to convert the internal network
of IPs to the single public IP address used by the router. The
demilitarized zone (DMZ) is an area that is protected by the
firewall but separate from the LAN. Servers are often placed
here. DHCP is the protocol that governs the automatic assign‐
ment of IP addresses to clients by a server.
    \end{itemize}
    \item Which of the following is described as the simultaneous send‐
ing and receiving of network data?
    \begin{itemize}
        \item D. Full-duplex is when a network adapter (or other device) can
send and receive information at the same time. Half-duplex is
when only sending or receiving can be done at one time. La‐
tency is the delay it takes for data to reach a computer from a
remote location. PoE is Power over Ethernet, a technology that
allows devices to receive data and power over an Ethernet net‐
work cable.
    \end{itemize}
    \item Which of the following would a company most likely use for
authentication to a server room?
    \begin{itemize}
        \item C. RFID (radio-frequency identification) is commonly used for
access to areas of a building such as a server room. It is often
implemented as a proximity-based ID card or badge. The oth‐
ers are not usually associated with authentication. 802.11ac is
a WLAN (Wi-Fi) standard that runs on 5 GHz and can provide
1 Gbps of data transfer. 802.15-4 is the IEEE standard for Zig‐
bee. Z-Wave, like Zigbee, is a home automation and wirelesssensor control technology. MIMO (multiple-input and multi‐
ple-output) is a multiple propagation technology used to in‐
crease data transfer in 802.11n and 802.11ac wireless net‐
works.
    \end{itemize}
    \item Which standard can attain a data transfer rate of 1 Gbps over a
wireless connection?
    \begin{itemize}
        \item D. 802.11ac can attain speeds in excess of 1 Gbps over wire‐
less. 802.11a and g have a typical maximum of 54 Mbps.
802.11b (rarely used today) has a maximum of 11 Mbps.
802.3ab is the IEEE specification for 1 Gbps transfer over
twisted pair cables—it is wired, not wireless. By the way, this
is also known as 1000BASE-T
    \end{itemize}
    \item Which of the following is often broken down into groups of
channels including 1-5, 6-10, and 11?
    \begin{itemize}
        \item . B. In the United States, the 2.4 GHz frequency range is broken
down into three categories: Channel 1-5, 6-10, and 11. By
placing separate wireless networks on separate distant chan‐
nels (such as 1 and 11), you can avoid overlapping and inter‐
ference. 802.11ac and 802.11a are standards, not frequencies.
5 GHz uses channels such as 36, 40, 149, 153, and so on.
    \end{itemize}
   
\end{enumerate}
\subsection{Networked Hosts and
Network Configuration}
\begin{enumerate}
    \item While looking at the details of a server in your provider’s con‐
trol panel you notice that it says “Apache” in the HTTP sum‐
mary. What kind of server is this?
    \begin{itemize}
        \item B. Apache is a type of web server that runs on Linux. It is also
known as Apache HTTP Server. File servers are used to store
and transfer files but not websites. E-mail servers deal with the
sending and receiving of electronic mail via POP3, IMAP, and
HTTP. Authentication servers verify the identity of users log‐
ging in and computers on the network.
    \end{itemize}
    \item Which type of server acts as a go-between for clients and web‐
sites?
    \begin{itemize}
        \item A. A proxy server is a caching server used to store commonly
accessed websites by clients. It can be incorporated into a web
server but often it runs as a stand-alone server. A print server
manages network printers and their spooling of print jobs, pri‐
orities, and so on. A Syslog server gathers logging data from
network devices and allows for the easy analysis of those logs
from a client workstation. A DHCP server hands out IP ad‐dresses (and other TCP/IP information) to client computers.
    \end{itemize}
    \item Which type of server runs Microsoft Exchange?
    \begin{itemize}
        \item C. Microsoft Exchange is a type of e-mail server software.
While you could run multiple services on a single server—for
example, you could run the web server and e-mail server on
the same machine—it isn’t recommended. Unless you have a
small office, all servers (such as file servers, authentication
servers, e-mail servers, DHCP servers, and so on) should be
separate entities. SCADA is not a server—it stands for Super‐
visory Control and Data Acquisition System, a type of system
used to control larger organizations’ infrastructures such as
heating/cooling, electricity, and so on.
    \end{itemize}
    \item Which of these addresses needs to be configured to enable a
computer access to the Internet or to other networks?
    \begin{itemize}
        \item B. The gateway address must be configured to enable a com‐
puter access to the Internet through the gateway device. By de‐
fault, the subnet mask defines the IP address’s network and
host portions. The DNS server takes care of name resolution.
The MAC address is the address that is burned into the net‐
work adapter; it is configured at the manufacturer.
    \end{itemize}
    \item Which technology assigns addresses on the 169.254.0.0 net‐
work number?
    \begin{itemize}
        \item C. If you see an address with 169.254 as the first two octets,
then it is Automatic Private IP Addressing (APIPA). This is
also the link-local range for IPv4. The Dynamic Host Configu‐
ration Protocol (DHCP) assigns IP addresses automatically to
clients but by default does not use the 169.254 network num‐
ber. Static IP addresses are configured manually by the user in
the IP Properties window. Class B is a range of IP networks
from 128 through 191.
    \end{itemize}
    \item You want to test the local loopback IPv6 address. Which ad‐
dress would you use?
    \begin{itemize}
        \item B. You would use the ::1 address. That is the local loopback
address for IPv6. 127.0.0.1 is the local loopback for IPv4.
FE80::/10 is the range of unicast auto-configured addresses. ::0
is not valid but looks similar to how multiple zeros can be
truncated with a double colon.
    \end{itemize}
    \item You have been tasked with compartmentalizing the network.Which of the following technologies should you use?
    \begin{itemize}
        \item C) VLAN 
    \end{itemize}
\end{enumerate}
\subsection{Network Types and Networking Tools} 
\begin{enumerate}
    \item Which of the following is a group of Windows desktop com‐
puters located in a small area?
    \begin{itemize}
        \item A local area network (LAN) is a group of computers, such
as a SOHO network located in a small area. A wide area net‐
work (WAN) is a group of one or more LANs spread over a
larger geographic area. A personal area network (PAN) is a
smaller computer network used by smartphones and other
small computing devices. A metropolitan area network (MAN)
is a group of LANs in a smaller geographic area of a city.
    \end{itemize}
    \item Which Internet service makes use of PSTN?
    \begin{itemize}
        \item A) Dial-up Internet connections make use of the public
switched telephone network (PSTN) and POTS phone lines.ISDN) was developed to meet the limitations of PSTN. DSL
provides faster data transmissions over phone lines (or separate
data lines). Cable Internet is a broadband service that offers
higher speeds than DSL; it is provided by cable TV compa‐
nies.
    \end{itemize}
    \item You have been tasked with setting up a small office with the
fastest Internet service possible. There is no fiber optic avail‐
ability in the area because of the rocky, hilly terrain. Which In‐
ternet service will typically offer the best data transfer rates?
    \begin{itemize}
        \item D. The Internet service with the best data transfer rates will
typically be cable Internet. Cable Internet service is generally
“faster” than DSL. WMN stands for wireless mesh network, a
type of network that uses multiple access points—but it is not
an Internet service. FTTP stands for fiber to the premises,
which is not available in the scenario. Fixed wireless (or line-
of-sight wireless Internet service) can be offer very high data
transfer rates, but it requires an unobstructed view of the
provider’s tower. This is probably not an option due to the
rocky, hilly area where the customer’s office resides.
    \end{itemize}
    \item Which tool is used to test a network adapter not connected to
the network?
    \begin{itemize}
        \item C. To test a network adapter without a network connection,
you would use a loopback plug. This simulates a network con‐
nection. It can also be used to test a switch port. Punchdown
tools are used to punch individual wires to a patch panel. Cable
testers such as continuity testers test the entire length of a ter‐
minated cable. A tone generator and probe kit can also test a
cable’s length, but only tests one pair of wires at a time.
    \end{itemize}
    \item Your boss is concerned with overlapping wireless networks
from neighboring companies using 802.11ac. Which tool
should you use to analyze the problem, and which frequency
should you display for analysis?
    \begin{itemize}
        \item A. To analyze the problem, use a Wi-Fi analyzer! Because
your boss is concerned about wireless networks using
802.11ac, you would display the results for 5 GHz networks,
not 2.4 GHz. Cable certifiers are used to check long distance
wired connections, for example from a patch panel to an RJ45
jack. The loopback plug is used to simulate a network connec‐
tion which can help with identifying switch ports, and testing a
PC’s network connection.
    \end{itemize}
    \item What would be required to attach RJ45 plugs to the ends of a
single patch cable?
    \begin{itemize}
        \item D. RJ45 plugs are attached to the cable ends with a tool called
a crimper. A tone generator and probe kit is used to trace hard
to find telecommunication and data communication ca‐
bles/wires. A multimeter can be used to test continuity of a
patch cable. A cable stripper is used to strip a portion of the
plastic jacket off the cable, exposing the individual wires.
    \end{itemize}
    
\end{enumerate}
\subsection{Cables and Connectors}
\begin{enumerate}
    \item Which of the following would be suitable for 1000 Mbps net‐
works? (Select all that apply.)
    \begin{itemize}
        \item C and D. Category 5e and Category 6 are suitable for 1000
Mbps networks (and Cat 6 is also suitable for 10 Gbps net‐
works). Category 3 is suitable for 10 Mbps networks only. It is
outdated and you most likely won’t see it. Category 5 is suit‐
able for 100 Mbps networks. In general, Cat 3 and Cat 5 net‐
works should be upgraded.
    \end{itemize}
    \item Which type of cable would you use if you were concerned
about EMI?
    \begin{itemize}
        \item C. STP (shielded twisted pair) is the only cable listed here that
can reduce electromagnetic interference. However, fiber optic
cable is another good solution, though it will be more expen‐
sive, and more difficult to install. Plenum-rated cable is used
where fire code requires it; it doesn’t burn as fast, releasing
fewer PVC chemicals into the air.
    \end{itemize}
    \item You have been tasked with connecting a newer Android-based
smartphone to an external TV so that you can display the
CEO’s smartphone screen during a meeting. Which of the fol‐
lowing adapters would be the best solution typically.?
    \begin{itemize}
        \item E. Typically, you would use USB-C to HDMI. If it is a newer
Android-based smartphone, then chances are that it will have aUSB-C port. If you are attempting to connect it to a TV, then
HDMI is the most likely port to use. Micro-USB is used with
many mobile devices, but newer devices especially Android-
based devices) have switched to, or are moving toward USB-
C. We wouldn’t want USB-C to DVI because TVs normally
don’t have DVI inputs. USB to Ethernet helps to convert from
a computer or mobile device to the Ethernet network. These
devices can ultimately allow a device or computer with a USB
port to access the Internet. This wired connection might be fa‐
vored over wireless for its speed, quality connection, and low
latency.
    \end{itemize}
    \item Which type of cable can connect a computer to another com‐
puter directly?
    \begin{itemize}
        \item B. A crossover cable is used to connect like devices: computer
to computer or switch to switch. Straight-through cables (the
more common patch cable) do not connect like devices (for
example, they connect from a computer to a switch). 568B is
the typical wiring standard you will see in twisted-pair cables;
568A is the less common standard. A crossover cable uses the
568B wiring standard on one end and 568A on the other end.
(By the way, sometimes you will see these written as T568A
and T568B.) SATA is used to connect hard drives internally to
a desktop or laptop computer.
    \end{itemize}
    \item Which connector is used for cable Internet?
    \begin{itemize}
        \item B. Cable Internet connections use RG-6 coaxial cable (usually)
with an F-connector on the end. LC is a type of fiber optic con‐
nector. BNC is an older connector type used by coaxial net‐
works. RJ45 is the connector used on twisted-pair patch ca‐
bles. DE-9 (or DB-9) is a serial connector used with RS-232
connections.
    \end{itemize}
    \item Which cable type would be suitable for longer distances such
as connecting two cities?
    \begin{itemize}
        \item D. Single-mode fiber is used for longer distance runs, perhapsfrom one city to the next (as far as thousands of kilometers).
Coaxial is common for connections between utility poles and
houses/buildings. Twisted pair is common in LANs. Multi‐
mode cables have a larger core diameter than single-mode ca‐
bles. It is the more commonly used fiber optic cable in server
rooms and when making network backbone connections be‐
tween buildings in a campus.
    \end{itemize}

\end{enumerate}
\subsection{RAM and Storage} 
\begin{enumerate}
     \item Which of the following types of cloud services offers e-mail
through a web browser?
    \begin{itemize}
        \item A. Software as a service (SaaS) is the most commonly recog‐
nized cloud service; it allows users to use applications to ac‐
cess data that is stored on the Internet by a third party. Infras‐
tructure as a service (IaaS) is a service that offers computer
networking, storage, load balancing, routing, and VM hosting.
Platform as a service (PaaS) is used for easy-to-configure op‐
erating systems and on-demand computing. A community
cloud is mix of public and private clouds, but one where multi‐
ple organizations can share the public portion.
    \end{itemize}
    \item Your organization requires more control over its data and in‐
frastructure. Money is apparently not an issue. There are only
two admins and about 30 users that will have access to the data
on the cloud. Which of the following types of clouds is the best
option?
    \begin{itemize}
        \item B. The best option listed is a private cloud. This gives the most
control over data and resources in an environment where there
are limited users (and a healthy budget). These resources could
be entirely internal, or a portion of them could also be pro‐
vided by a third-party. Public cloud technology is used for the
general public to access applications over the Internet. Hybrid
is a mixture of the two, but not necessary in this situation be‐
cause of the budget and the limited number of users. Commu‐
nity cloud is similar to hybrid but is meant for multiple organi‐
zations that share data, which is not necessary in this scenario.
    \end{itemize}
    \item You require the ability to add on to your cloud-based network
whenever necessary, rapidly and efficiently. What is this refer‐
ring to?
    \begin{itemize}
        \item C. Rapid elasticity is the ability to build your cloud-based net‐
work, or extend upon an existing one, quickly and efficiently.
Measured services is when a provider monitors a customer’s
services used so that the customer can be properly billed.
Metered services is when the customer can access as many re‐
sources as needed but only be billed for what was accessed.
On-demand service means that the cloud service is available at
all times. The leaders of a successful organization don’t care
what it takes; they simply want high speed, secure access to
services 24/7.
    \end{itemize}
    
\end{enumerate}
\subsection{RAM and Storage} 
\begin{enumerate}
    \item What is the transfer rate of DDR4-2133?
    \begin{itemize}
        \item A. The transfer rate of DDR4-2133 is 17,066
MB/s. It is also known as PC4-17000. 19,200
MB/s is the speed of DDR4-2400 (PC4-19200).
21,333 MB/s is the speed of DDR4-2666 (PC4-
21333). 25,600 MB/s is the speed of DDR4-3200
(PC4-25600).
    \end{itemize}
    \item How many pins are on a DDR3 memory module?
    \begin{itemize}
        \item C. DDR3 is a 240-pin architecture. 288-pin is
DDR4, 184-pin is the first version of DDR
(DDR1), and you can find 200-pin architectures in
laptops; they are known as SODIMMs. To review,
Table 10.2 shows the pin configurations for PCbased
DDR 1 through 4.
    \end{itemize}
    \item Which of the following allows for a 256-bit wide
bus?
    \begin{itemize}
        \item B. The quad-channel memory architecture can
allow for a 256-bit wide bus (64-bit per channel).
However, this will only be the case if all four
channels have memory installed to them. ECC
stands for error correction code which can detect
and correct errors in RAM. Parity is when the
RAM stores an extra bit used for error detection.
DDR2 is a type of DRAM that for the most part
was used in either single-channel or dual-channel
environments.
    \end{itemize}
    \item How much data can a SATA revision 3.0 drive transfer per
second?
    \begin{itemize}
        \item D. SATA Revision 3.0 drives can transfer 6.0 Gb/s, which after
encoding amounts to 600 MB/s. SATA Revision 3.2 is 16 Gb/s
(1969 MB/s) but requires SATA Express or M.2. 50 MB/s is a
typical write speed for Blu-ray discs and some flash media. 90
MB/s is a typical write speed for an SD card.
    \end{itemize}
    \item Which level of RAID stripes data and parity across three or
more disks?
    \begin{itemize}
        \item C. RAID 5 stripes data and parity across three or more disks.
RAID 0 does not stripe parity; it stripes data only and can use
two disks or more. RAID 1 uses two disks only. Striping is an‐
other name for RAID 0. RAID 10 contains two sets of mir‐
rored disks that are then striped.
    \end{itemize}
    \item Which of the following has the largest potential for storage ca‐
pacity?
    \begin{itemize}
        \item D. Blu-ray, at a typical maximum of 50 GB, has the largest
storage capacity. CDs top out just under 1 GB. DVDs have a
maximum of 17 GB.
    \end{itemize}
    \item A customer complains that an important disc is stuck in the
computer’s DVD-ROM drive. What should you recommend to
the customer?
    \begin{itemize}
        \item C. Tell the customer to use a paper clip to eject the DVD-ROM
tray. Disassembling the drive is not necessary; the customer
shouldn’t be told to do this. If the disc is rewritable, formatting
it would erase the contents, even if you could format in thisscenario. Never tell a customer to dispose of a DVD-ROM
drive; they rarely fail.
    \end{itemize}
    \item Which of the following best describes a specification for ac‐
cessing storage while using PCI Express?
    \begin{itemize}
        \item A. Non-Volatile Memory Express (NVMe) is a specification
for accessing storage while using PCI Express. Essentially, the
M.2 slot on a motherboard taps into the PCI Express bus (x4)
and uses a portion of the total bandwidth associated with that
bus. The platters in a hard disk drive (HDD) rotate at a certain
speed, for example 7,200 RPM, which is common; other typi‐
cal speeds include 5,400 RPM and 10,000 RPM. Hot-swap‐
pable capability is when drives can be removed and inserted
while the system is on. SATA-based hard drives come in two
main widths: 3.5” and 2.5”. The 3.5” drive is used in desktop
computers, network-attached storage and other larger devices.
The 2.5” drive is used in laptops and other smaller devices.
    \end{itemize}

\end{enumerate}
\subsection{Motherboards and Cards} 
\begin{enumerate}
    \item Which motherboard form factor measures 12
inches × 9.6 inches (305 mm × 244mm)?
    \begin{itemize}
        \item C. ATX boards measure 12 inches × 9.6 inches
(305 mm × 244mm). microATX boards are square
and measure 9.6 × 9.6 inches (244 mm × 244
mm). SATA is a type of hard drive technology and
the port used to connect hard drives to the
motherboard. mITX (or Mini-ITX), also square,
measures 6.7 × 6.7 inches (17 cm × 17 cm) . 
    \end{itemize}
    \item Which component supplies power to the CMOS
when the computer is off?
    \begin{itemize}
        \item A. The lithium battery supplies power to the
CMOS when the computer is off. This is because
the CMOS is volatile and would otherwise lose the
stored settings when the computer is turned off
    \end{itemize}
    \item To implement a secure boot process, which device
should be listed first in the Boot Device Priority
screen?
    \begin{itemize}
        \item D. To ensure that other users cannot boot the
computer from removable media, set the first
device in the Boot Device Priority screen to hard
drive.
    \end{itemize}
    \item Which of the following connectors would you use
to power a video card?
    \begin{itemize}
        \item B. A video card is normally powered by a 6-pin or
8-pin PCIe connector. Lesser cards are simply
powered by the PCIe bus. The 24-pin power
connector is the main connector that leads from
the power supply to the motherboard. Molex is
used for fans, older IDE drives, and other
secondary devices. 3.5 mm TRS (or 1/8 inch) is an
audio connection.
    \end{itemize}
    \item Which of the following is a chip that stores
encryption keys?
    \begin{itemize}
        \item D. To perform hard drive encryption, some
motherboards come with a trusted platform
module (TPM), a chip that stores encryption keys
—it can be enabled in the BIOS. Intel
Virtualization Technology (VT) is part of the
firmware that supports the use of virtualization
software such as Hyper-V and VMware. Secure
boot can block rootkits and other malware from
launching boot loaders that have been tampered
with. Firmware (such as a motherboard’s BIOS) is
should be updated or “flashed” periodically to take
advantage of the latest functionality and security
updates.
    \end{itemize}
\end{enumerate}
\subsection{CPUs}
\begin{enumerate}
    \item What does Hyper-Threading do?
    \begin{itemize}
        \item C. Hyper-Threading allows for an operating
system to send two simultaneous threads to be
processed by a single CPU core. The OS views the
CPU core as two virtual processors. Multiple cores
would imply multicore technology, which means
there are two physical processing cores within the
CPU package. HyperTransport is a high-speed
connection used by AMD from the CPU to RAM.
    \end{itemize}
    \item What seals the tiny gaps between the CPU cap and
the heat sink
    \begin{itemize}
        \item D. Thermal compound/thermal paste is used to
seal the small gaps between the CPU and heat
sink. It is sometimes referred to as thermal gel or
jelly (among a variety of other names), but not
grape jelly. (Did I ever tell you about the time I
found grape jelly inside a customer’s computer?
Fun times.) Note: Never use petroleum-based
products (such as 3-in-1 oil or WD-40) inside a
computer; the oils can damage the components
over time. TDP stands for thermal design power.
    \end{itemize}
    \item Which of the following can be defined as the
amount of heat generated by the CPU, which the
cooling system is required to dissipate?
    \begin{itemize}
        \item B. TDP (thermal design power) is the amount of
power required to cool a computer and is linked
directly to the amount of heat a CPU creates.
Some CPUs come with a built-in graphics
processing unit (GPU). This means that with a
compatible motherboard, no separate video card
is necessary. PSU stands for power supply unit.
140 watts is a potential TDP rating but does not
define what TDP is.
    \end{itemize}
    \item When deciding on a CPU for use with a specific
motherboard, what does it need to be compatible
with?
    \begin{itemize}
        \item B. The CPU needs to be compatible with the
socket of the motherboard. The case doesn’t
actually make much of a difference when it comes
to the CPU. (Just make sure it’s large enough!)
There is no wattage range, but you should be
concerned with the voltage range of the CPU. PCI
Express (PCIe) slots don’t actually play into this at
all because there is no direct connectivity between
the two
    \end{itemize}
    \item Which kind of socket incorporates “lands” to
ensure connectivity to a CPU?
    \begin{itemize}
        \item C. LGA (Land Grid Array) is the type of socket
that uses “lands” to connect the socket to the CPU.
PGA sockets have pinholes that make for
connectivity to the CPU’s copper pins. AM4 is a
PGA socket that accepts AMD CPUs such as the
Ryzen 7.
    \end{itemize}
    \item Which of the following enables the user to
increase the base clock within the BIOS, thereby
increasing the clock speed of the CPU?
    \begin{itemize}
        \item A. Overclocking enables the user to increase the
clock speed of the CPU within the BIOS. Level 3
(L3) cache comes in the largest capacities of the
three types of cache and has the most latency;
therefore, it is the slowest. If the CPU can’t find
what it needs in L1, it moves to L2 and finally to
L3. An integrated GPU is a video adapter that is
built into the motherboard. The heat sink helps to
dissipate heat from the CPU and is usually aided
by a fan or liquid cooling system.
    \end{itemize}
   
\end{enumerate}
\subsection{Peripherals and Power}
\begin{enumerate}
    \item What does a KVM do?
    \begin{itemize}
        \item D. A KVM connects multiple computers to a single keyboard,
mouse, and monitor. This way, fewer resources in the way of
peripherals (input/output devices) are necessary to use the
computers.
    \end{itemize}
    \item Which of the following are considered both input and output
devices?
    \begin{itemize}
        \item D) Smart TVs, touchscreens, KVMs, and headsets are consid‐
ered both input and output devices. Keyboards, mice, touch‐pads, smart card readers, motion sensors, and biometric de‐
vices are considered input devices. Printers and speakers are
considered output devices.
    \end{itemize}
    \item Which of the following devices can be used to perform combi‐
nation shortcuts?
    \begin{itemize}
        \item A.A keyboard can be used to perform combination shortcuts.
An example of a shortcut key is Ctrl + P which initiates a print
job within an application. While a KVM will have a keyboard
connected to it, it’s at the keyboard that you perform the short‐
cut operation.
    \end{itemize}
    \item Which of the following incorporate the concept of resolution.
(Select all that apply).
    \begin{itemize}
        \item B, C, and E. The mouse, printer, and LED display all deal
with resolution. A mouse’s sensitivity is rated in DPI; for ex‐
ample, 800 DPI is a low resolution for mice. A printer will
commonly print out documents at the resolution 600 DPI
(more on that in Chapter 15, “Printers and Multifunction De‐
vices”. A monitor will commonly have a resolution of 1920 x
1080 (or greater!)
    \end{itemize}
    \item Which of the following terms describes how the light output
from a video projector is measured?
    \begin{itemize}
        \item C. A video projector’s light output is measured in lumens. In-
plane switching (IPS) technology allows for a wider viewing
angle. Some LCDs use a cold cathode fluorescent lamp
(CCFL) as the lighting source instead of LEDs. OLED stands
for organic light-emitting diode—that’s the lighting material
used in the display.
    \end{itemize}
    \item Which power connector should be used to power
an SATA hard drive?
    \begin{itemize}
        \item D. 15-pin connectors power SATA hard drives and
other SATA devices (such as optical drives). Molex
connectors power fans, older IDE devices, and
other secondary devices. 6-pin power connectors
are used for video cards (as are 8-pin connectors).
24-pin refers to the main power connection for
the motherboard.
    \end{itemize}
    \item Which voltages are supplied by a Molex power
connector?
    \begin{itemize}
        \item A. Molex connectors provide 12 volts and 5 volts.
There are four wires: if color-coded, yellow is 12 V,
red is 5 V, and the two blacks are grounds.
    \end{itemize}
    \item A company salesperson just returned to the
United States after three months in Europe. Now
the salesperson tells you that her PC, which
worked fine in Europe, won’t turn on. What is the
best solution?
    \begin{itemize}
        \item D. Most likely, the voltage selector was set to 230
V so that it could function properly in Europe (for
example, in the UK). It needs to be changed to 115
V so that the power supply can work properly in
the United States. Make sure to do this while the
computer is off and unplugged.
    \end{itemize}
    \item Which of the following can have 8 pins? (Select all
that apply).
    \begin{itemize}
        \item A and D. PCIe power can be 8-pin or 6-pin. CPU
power (EATX12V) can be 8-pin or 4-pin. SATA
power is 15-pin (and data is 7-pin). ATX main
power is typically 24-pin. Molex is a 4-wire
connector; it is sometimes also referred to as
“peripheral”.
    \end{itemize}
\end{enumerate}
\subsection{Custom PCs and
Common Devices}
\begin{enumerate}
    \item Which of the following is the best type of custom
computer for use with Pro Tools?
    \begin{itemize}
        \item B. Audio/video editing workstation
    \end{itemize}
    \item Which of the following would include a gigabit
NIC and a RAID array?
    \begin{itemize}
        \item D. NAS
    \end{itemize}
    \item Your organization needs to run Windows in a
virtual environment. The OS is expected to require
a huge amount of resources for a powerful
application it will run. What should you install
Windows to?
    \begin{itemize}
        \item C. Type 1 hypervisor
    \end{itemize}
    \item You have set up a user to work at a thin client. They will be
accessing the OS image and data from a Windows Server 2016
as well as data from the Google Cloud. Which of the following
does this configuration not require? (Select the two best an‐
swers.)
    \begin{itemize}
        \item . C and E. This configuration will not need a hard drive, be it
M.2 or other SSD. Thin clients are meant to use an OS that is
embedded in RAM (or other similar memory) or more often,
grab an image from a server, often as a virtual machine. To do
so, the thin client will need a network connection (wired or
wireless), and every computer needs a CPU.
    \end{itemize}
    \item A customer has a brand new Chromebook and needs help con‐
figuring it. Which of the following should you help the user
with? (Select the three best answers.)
    \begin{itemize}
        \item A, B and D. You should show the user how to configure the
touchpad and touchscreen, and guide the user through the ini‐
tial account setup. Chrome OS is a fairly simple system com‐
pared to Windows and other operating systems. To configure
devices, simply go to the “Home” or app launcher button, then
Settings, then Devices. The registry is a Windows configura‐
tion tool—even if this was a Windows computer, the typical
user has no place in the registry. The App Store is Apple’s ap‐
plication download site. Google uses the Play Store. The Task
Manager is another Windows utility. Consider writing a short
user guide in Word document format if you have multiple
users accessing the same type of system for the first time.
Write it once, and train many!
    \end{itemize}
   \item Where would you go in Windows to enable printer sharing?
    \begin{itemize}
        \item B. The Network and Sharing Center in Windows is where
printer sharing is enabled. Network Connections is the window
that shows the Ethernet and Wi-Fi connections a PC has to the
network. Windows sharing has to be done in Windows, it can’tbe done from the printer’s on-screen display (OSD). Bonjour is
a macOS service, that can also be run on Windows which en‐
ables automatic discovery of devices on the LAN.
    \end{itemize}
    \item Your printer supports printing to both sides of paper. What
should you enable in the Printing preferences?
    \begin{itemize}
        \item C. Duplexing (as it relates to printers) means to print to both
sides. Collating means printing multiple copies of a docu‐
ment’s pages in sequence, instead of printing all of the copies
of one page at a time. Orientation is how the print job is dis‐
played on the paper; it could be portrait (vertically—the de‐
fault), or landscape (horizontal). Quality refers to the clarity of
the print job, usually measured in dots per inch (DPI)—the
higher the DPI the better.
    \end{itemize}
    \item Which of the following address printer data privacy concerns?
(Select the two best answers.)
    \begin{itemize}
        \item A and D. Implement user authentication for the printer or print
server (PIN or password), and clear the cache on the printer.
Bluetooth ad hoc mode network printing can be used by mo‐
bile devices where no wireless access point exists. AirPrint is
an Apple technology for macOS and iOS used to automatically
locate and download drivers for printers.
    \end{itemize}
    \item During which step of the laser printing/imaging process is the
transfer corona wire involved?
    \begin{itemize}
        \item B. The transfer corona wire gets involved in the laser print‐
ing imaging process during the transferring step.
    \end{itemize}
    \item Which stage of the laser printing/imaging process involves ex‐
treme heat?
    \begin{itemize}
        \item A. The fusing step uses heat (up to 400 degrees Fahrenheit/200
degrees Celsius) and pressure to fuse the toner permanently to
the paper.
    \end{itemize}
    \item Which represents the proper order of the laser printing/imaging
process?
    \begin{itemize}
        \item D. The proper order of the laser printing/imaging process is
processing, charging, exposing, developing, transferring, fus‐
ing, cleaning.
    \end{itemize}
    \item Which of the following are associated with inkjet printers?
    \begin{itemize}
        \item .B. Inkjet printer components include ink cartridge, print head,
roller, feeder, duplexing assembly, carriage, and belt. Imaging
drum, fuser assembly, transfer belt, transfer roller, pickup
rollers, separate pads, and duplexing assembly are associated
with laser printers. Feed assembly, thermal heating unit, and
thermal paper are associated with thermal printers. Print head,
ribbon, tractor feed, and impact paper are associated with im‐
pact printers.
    \end{itemize}
    \item When finished installing a new printer and print drivers, what
should you do? (Select all that apply.)
    \begin{itemize}
        \item A and D. After the printer is installed (meaning it has been
connected and the drivers have been installed), you should cal‐
ibrate the printer (if necessary) and print a test page. You
should also consider updating the firmware for the printer. Be‐
fore starting the installation, you should check for compatibil‐
ity with operating systems, applications, and so on.
    \end{itemize}
    
\end{enumerate}

\subsection{Cloud Computing and
Client-side Virtualization}
\begin{enumerate}
    \item Which of the following types of cloud services offers e-mail
through a web browser?
    \begin{itemize}
        \item A. Software as a service (SaaS) is the most commonly recog‐
nized cloud service; it allows users to use applications to ac‐
cess data that is stored on the Internet by a third party. Infras‐
tructure as a service (IaaS) is a service that offers computer
networking, storage, load balancing, routing, and VM hosting.
Platform as a service (PaaS) is used for easy-to-configure op‐
erating systems and on-demand computing. A community
cloud is mix of public and private clouds, but one where multi‐
ple organizations can share the public portion.
    \end{itemize}
    \item Your organization requires more control over its data and in‐
frastructure. Money is apparently not an issue. There are only
two admins and about 30 users that will have access to the data
on the cloud. Which of the following types of clouds is the best
option?
    \begin{itemize}
        \item B. The best option listed is a private cloud. This gives the most
control over data and resources in an environment where there
are limited users (and a healthy budget). These resources could
be entirely internal, or a portion of them could also be pro‐
vided by a third-party. Public cloud technology is used for the
general public to access applications over the Internet. Hybrid
is a mixture of the two, but not necessary in this situation be‐
cause of the budget and the limited number of users. Commu‐
nity cloud is similar to hybrid but is meant for multiple organi‐
zations that share data, which is not necessary in this scenario.
    \end{itemize}
    \item You require the ability to add on to your cloud-based network
whenever necessary, rapidly and efficiently. What is this refer‐
ring to?
    \begin{itemize}
        \item C. Rapid elasticity is the ability to build your cloud-based net‐
work, or extend upon an existing one, quickly and efficiently.
Measured services is when a provider monitors a customer’s
services used so that the customer can be properly billed.
Metered services is when the customer can access as many re‐
sources as needed but only be billed for what was accessed.
On-demand service means that the cloud service is available at
all times. The leaders of a successful organization don’t care
what it takes; they simply want high speed, secure access to
services 24/7.
    \end{itemize}
\end{enumerate}
\end{document}



