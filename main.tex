\documentclass{article}
\usepackage[utf8]{inputenc}
\usepackage{xcolor}
\usepackage{hyperref}
\usepackage{comment}
\title{CompTIA A+ Exam Questions}
\author{Ramiro Gonzalez}
\date{December 2020}

\begin{document}

\maketitle
\section{Questions} 
\begin{enumerate}
    \item What are the six step troubleshooting process? 
    \begin{enumerate}
        \item Identify the problem
        \item Establish a theory of probable cause. (Question the obvious) 
        \item Test the theory to determine cause
        \item Establish a plan of action to resolve the problem and implement a solution
        \item Verify full system functionality and, if applicable, implement preventative measures. 
        \item Document findings, actions, and outcomes 
    \end{enumerate}
    \item To which type of technology would you install a x16 card?
    \begin{enumerate}
        \item PCIe (Peripheral Component Interconnect express) is the most common expansion slot for video cards (by sixteen) 
    \end{enumerate}
    \item Which process of the computer checks all your components dur‐
ing boot?
    \begin{enumerate}
        \item POST (Power-On Self Test) is a process that is part of the BIOS (Basic Input Output System) or Unified Extensible Firmware Interface (UEFI). It runs a self-check of the computer system during boot and stores many of the parameters of the components within the CMOS. 
    \end{enumerate}
    \item Which of the following could cause the POST to fail? (Select
the two best answers.) 
    \begin{enumerate}
        \item The CPU and Memory need to be installed properly for the POST to run (and to pass). 
    \end{enumerate}
    
    \item Which of the following might you find as part of a tablet com‐
puter? (Select the two best answers.)   
    \begin{enumerate}
        \item A tablet computer will almost always  contain flash memory as main storage and multi-touch screen. 
    \end{enumerate}
    \item Which kind of socket incorporates “lands” to ensure connectiv‐
ity to a CPU? 
    \begin{enumerate}
        \item LGA (land grid array) is the type of socket that
uses “lands” to connect the socket to the CPU.
    \end{enumerate}
    \item How many pins are inside an SATA 3.0 data connector? 
    \begin{enumerate}
        \item The SATA version 3.0 data connector has seven pins. Note: SATA express uses a tiple connector with 18 pins ( 7 + 7 + 4) 
    \end{enumerate}
    \item What is the minimum number of hard drives necessary to imple‐
ment RAID 5?
    \begin{enumerate}
        \item Because RAID 5 uses striping with parity, a third disk is needed. You can have more than three disks as well. 
    \end{enumerate}
    \item A user’s time and date keep resetting to January 1, 2012. Which
of the following is the most likely cause?
    \begin{enumerate}
        \item If the time and date keep resetting - for example, to a date such as January 1,2012 - chances are that the lithium battery needs to be replaced. These are usually nickle-sized batteries; most PCs use a CR2032 lithium battery. 
    \end{enumerate}
    \item Which type of adapter card is normally plugged into a PCIe x16
adapter card slot?
    \begin{enumerate}
        \item The PCI Express (PCIe) x16 expansion slot is used
primarily for video.
    \end{enumerate}
    \item Which of the following CPU cooling methods is the most com‐
mon?
    \begin{enumerate}
        \item The most common CPU cooling method is the heat
sink and fan combination. The heat sink helps the heat to disperse
away from the CPU, whereas the fan blows the heat down and
through the fins; the power supply exhaust fan and possibly addi‐
tional case fans help the heat escape the case. Heat sink and fan
combinations are known as active cooling methods.
    \end{enumerate}
    \item What type of power connector is used for a x16 video card?
    \begin{enumerate}
        \item PCIe 6-pin
    \end{enumerate}
    \item What does the b in 1000 Mbps stand for?
    \begin{enumerate}
        \item The b in 1000 Mbps stands for bits: 1000 Mbps is
1000 megabits per second or 1 gigabit per second. Remember
that the lowercase b is used to indicate bits when measuring net‐
work data transfer rates, USB data transfer rates, and other simi‐
lar serial data transfers.
    \end{enumerate}
    \item When running cable through drop ceilings, which type of cable do
you need?
    \begin{enumerate}
        \item Plenum-rated cable needs to be installed wherever
a sprinkler system is not able to spray water. This includes ceil‐
ings, walls, and plenums (airways). Plenum-rated cable has a pro‐
tective covering that burns slower and gives off fewer toxic
fumes than regular PVC-based cable.
    \end{enumerate}
    \item Which of the following is the default subnet mask for IP address
192.168.1.1?
    \begin{enumerate}
        \item 192.168.1.1, by default, has the subnet mask
255.255.255.0, which is the standard subnet mask for class C IP
addresses. However, remember that some networks are classless,
which means that a network can use a different subnet mask.
    \end{enumerate}
    \item Which of the following is the minimum category cable needed for
a 1000BASE-T network?
    \begin{enumerate}
        \item The minimum cable needed for 1000BASE-T net‐
works is Category 5e. Of course, Cat 6 would also work, but it is
not the minimum of the listed answers. 1000BASE-T specifies
the speed of the network (1000 Mbps), the type (baseband, single
shared channel), and the cable to be used (T = twisted pair).
    \end{enumerate}
    \item Which of the following IP addresses can be routed across the In‐
ternet?
    \begin{enumerate}
        \item The only listed answer that is a public address
(needed to get onto the Internet) is 129.52.50.13
        \item All the other answers are private IPs, meant
to be behind a firewall. 127.0.0.1 is the IPv4 local loopback IP
address. 192.168.1.1 is a common private IP address used by
SOHO networking devices. 10.52.50.13 is a private address.
Note that the 10 network is common in larger networks.
    \end{enumerate}
    \item Which port number is used by HTTPS by default?
    \begin{enumerate}
        \item The Hypertext Transfer Protocol Secure (HTTPS)
uses port 443 (by default).
        \item Port 21 is used by the File Transfer Protocol
(FTP). Port 25 is used by the Simple Mail Transfer Protocol
(SMTP). Port 80 is used by regular HTTP, which is considered to
be insecure.
    \end{enumerate}
    \item Which of the following cable types have a copper medium? (S‐
elect the three best answers.)
    \begin{enumerate}
        \item Twisted-pair, coaxial, and Category 7 cable are all
examples of network cables with a copper medium. They all send
electricity over copper wire. 
        \item ultimode is a type of fiber-optic cable; it
uses light to send data over a glass or plastic medium. Twisted
pair is the most common type of cabling used in today’s net‐
works.
    \end{enumerate}
    \item Which of the following cable types can protect from electromag‐
netic interference (EMI)? (Select the two best answers.)
    \begin{enumerate}
        \item Shielded twisted pair (STP) and fiber optic can
protect from EMI. 
        \item Unshielded twisted pair (UTP) cannot pro‐
tect from EMI. Unless otherwise mentioned, Category 6 cable is
UTP. STP is shielded twisted pair. Unlike UTP (unshielded
twisted pair), STP provides an aluminum shield that protects
from EMI. UTP and coaxial have no such protection. Fiber optic
uses a different medium altogether, transmitting light rather than
electricity; therefore, EMI cannot affect fiber-optic cables.
    \end{enumerate}
    \item You are configuring Bob’s computer to access the Internet. Which
of the following are required? (Select all that apply.)
    \begin{enumerate}
        \item To get on the Internet, the DNS server address is
required so that the computer can get the resolved IP addresses
from the domain names that are typed in. The gateway address is
necessary to get outside the network.
    \end{enumerate}
    \item A customer wants to access the Internet from many different loca‐
tions in the United States. Which of the following is the best tech‐
nology to enable the customer to do so?
    \begin{enumerate}
        \item Cellular WAN uses a phone or other mobile device
to send data over standard cellular connections.
    \end{enumerate}
    \item You just configured the IP address 192.168.0.105 in Windows.
When you press the Tab key, Windows automatically configures
the default subnet mask of 255.255.255.0. Which of the following
IP addresses is a suitable gateway address?
    \begin{enumerate}
        \item 192.168.0.1 is the only suitable gateway address.
Remember that the gateway address must be on the same net‐
work as the computer. In this case, the network is 192.168.0, as
defined by the 255.255.255.0 subnet mask.
    \end{enumerate}
    \item You have been tasked with blocking remote logins to a server.
Which of the following ports should you block?
    \begin{enumerate}
        \item ort 23 should be blocked. It is associated with the
Telnet service, which is used to remotely log in to a server at the
command line. You can block this service at the company fire‐
wall and individually at the server and other hosts. It uses port 23
by default, but it can be used with other ports as well. Telnet is
considered to be insecure, so it should be blocked and disabled.
    \end{enumerate}
    \item Which	of	the	following	connector	types	is	used	by	fiber-optic	ca‐
bling?

    \begin{enumerate}
        \item The	LC	connector	is	used	by	fiber-optic	cabling.
Other	fiber	connectors	include	SC	and	ST.
        \item RJ45	is	the	connector	used	by	twisted-pair
networks.	RG-6	is	the	cable	used	by	cable	Internet	and	TV;	an	F-
connector	is	attached	to	the	ends	of	an	RG-6	cable.	RJ11	is	the
standard	phone	line	connector.

    \end{enumerate}
    \item Which	protocol	uses	port	53?

    \begin{enumerate}
        \item The	Domain	Name	System	(DNS)	protocol	uses
port	53	by	default. 
        \item TP	uses	port	21.	SMTP	uses	port	25	(or	587 or	465).	HTTP uses port	80.

    \end{enumerate}
    \item Which	of	the	following	terms	best	describes	two	or	more	LANs
connected	over	a	large	geographic	distance?

    \begin{enumerate}
        \item A	wide-area	network	(WAN)	is	a	network	in	which
two	or	more	LANs	are	connected	over	a	large	geographic	dis‐
tance—for	example,	between	two	cities.	The	WAN	requires	con‐
nections	to	be	provided	by	a	telecommunications	or	data	commu‐
nications	company.
    \end{enumerate} 
        \item Which	device	connects	to	the	network	and	has	the	sole	purpose	of
providing	data	to	clients?
    \begin{enumerate}
        \item Network-attached	storage	(NAS)	devices	store	data
for	network	use.	They	connect	directly	to	the	network.
    \end{enumerate}
    \item You	are	making	your	own	networking	patch	cable.	You	need	to
attach	an	RJ45	plug	to	the	end	of	a	twisted-pair	cable.	Which	tool
should	you	use?

    \begin{enumerate}
        \item 	Use	an	RJ45	crimper	tool	to	permanently	attach
RJ45	plugs	to	the	end	of	a	cable

    \end{enumerate}
    \item Which	port	is	used	by	RDP?
    \begin{enumerate}
        \item The	Remote	Desktop	Protocol	(RDP)	uses	port
3389	by	default.	This	protocol	allows	one	computer	to	take control	of	another	remote	system.

    \end{enumerate}
    \item Which	of	the	following	printer	failures	can	be	described	as	a	con‐
dition	in	which	the	internal	feed	mechanism	stopped	working
temporarily?

    \begin{enumerate}
        \item A	failure	that	occurs	due	to	the	internal	feed	mech‐
anism	stopping	is	known	as	a	paper	jam.	For	example,	an	HP
LaserJet	might	show	error	code	13.1	on	the	display,	which	means
a	paper	jam	at	the	paper	feed	area.	You	should	verify	that	the	pa‐
per	trays	are	loaded	and	adjusted	properly.

    \end{enumerate}
    \item Which	type	of	printer	uses	a	toner	cartridge?

    \begin{enumerate}
        \item Laser	printers	use	toner	cartridges.

    \end{enumerate}
    \item Which	of	the	following	should	not	be	connected	to	a	UPS?

    \begin{enumerate}
        \item Laser	printers	use	large	amounts	of	electricity,
which	in	turn	could	quickly	drain	the	battery	of	the	UPS.	They
should	be	plugged	in	to	their	own	individual	power	strips.

    \end{enumerate}
    \item Special	paper	is	needed	to	print	on	which	type	of	printer?

    \begin{enumerate}
        \item 	Regular	paper	can	be	used	on	all	the	listed	printers
except	for	thermal	printers,	which	use	specially	coated	paper	that
is	heated	to	create	the	image.

    \end{enumerate}
    \item Which	of	the	following	channels	should	you	select	for	an	802.11
wireless	network?

    \begin{enumerate}
        \item Of	the	listed	answers,	use	channel	6	for	802.11
wireless	networks.	That	would	imply	a	2.4	GHz	connection	using
either	802.11n,	g,	or	b.	The	2.4	GHz	frequency	range	in	the
United	States	allows	for	channels	1	through	11.

    \end{enumerate}
    \item Which	environmental	issue	affects	a	thermal	printer	the	most?

    \begin{enumerate}
        \item Which	type	of	printer	uses	impact	to	transfer	ink	from	a	ribbon	to
the	paper?

    \end{enumerate}
    \item Which	of	the	following	steps	enables	you	to	take	control	of	a	net‐
work	printer	from	a	remote	computer?

    \begin{enumerate}
        \item 	After	you	install	the	driver	for	the	printer	locally,
you	can	then	take	control	of	it	by	going	to	the	properties	of	the
printer	and	accessing	the	Ports	tab.	Then	click	the	Add	Port	but‐
ton	and	select	the	Standard	TCP/IP	Port	option.	You	have	to
know	the	IP	address	of	the	printer	or	the	computer	that	the	printer
is	connected	to.

    \end{enumerate}
    \item 	A	color	laser	printer	produces	images	that	are	tinted	blue.	Which
of	the	following	steps	should	be	performed	to	address	this	prob‐
lem?

    \begin{enumerate}
        \item After	you	install	a	printer,	it	is	important	to	cali‐
brate	it	for	color	and	orientation,	especially	if	you	are	installing	a
color	laser	printer	or	an	inkjet	printer.	These	calibration	tools	are
usually	built	in	to	the	printer’s	software	and	can	be	accessed	from
Windows,	or	you	can	access	them	from	the	printer’s	display.

    \end{enumerate}
    \item A	desktop	computer	does	not	have	a	lit	link	light	on	the	back	of
the	computer.	Which	of	the	following	is	the	most	likely	reason	for
this?

    \begin{enumerate}
        \item The	most	likely	answer	in	this	scenario	is	that	the
network	cable	is	disconnected.	If	the	desktop	computer	is	using	a
wired	connection,	it	is	most	likely	a	twisted-pair	Ethernet	con‐
nection.	When	this	cable	is	connected	to	the	computer	on	one	end
and	to	a	switch	or	other	central	connecting	device	on	the	other
end,	it	initiates	a	network	connection	over	the	physical	link.	This
link	then	causes	the	network	adapter’s	link	light	to	light	up.	The
link	light	is	directly	next	to	the	RJ45	port	of	the	network	adapter.
The	corresponding	port	on	the	switch	(or	other	similar	device)	is
also	lit.	If	the	cable	is	disconnected,	the	link	light	becomes	unlit,
though	there	are	other	possibilities	for	this	link	light	to	be	dark—
for	example,	if	the	computer	is	off	or	if	the	switch	port	is	dis‐
abled.

    \end{enumerate}
    \item Which	of	the	following	IP	addresses	would	a	technician	see	if	a
computer	running	Windows	is	connected	to	a	multifunction	net‐
work	device	and	is	attempting	to	obtain	an	IP	address	automati‐
cally	but	is	not	receiving	an	IP	address	from	the	DHCP	server?

    \begin{enumerate}
        \item If	the	computer	fails	to	obtain	an	IP	address	from	a
DHCP	server,	Windows	will	take	over	and	apply	an	Automatic
Private	IP	Address	(APIPA).	This	address	will	be	on	the
169.254.0.0	network.

    \end{enumerate}
    \item For	which	type	of	PC	component	are	80	mm	and	120	mm	com‐
mon	sizes?

    \begin{enumerate}
        \item Case	fans	are	measured	in	mm	(millimeters);	80
mm	and	especially	120	mm	are	very	common.	They	are	used	to
exhaust	heat	out	of	the	case.	These	fans	aid	in	keeping	the	CPU
and	other	devices	cool.	The	120	mm	is	quite	common	in	desktop
and	tower	PCs,	and	the	80	mm	is	more	common	in	smaller	sys‐
tems	and	1U	and	2U	rackmount	servers.

    \end{enumerate}
    \item An	exclamation	point	next	to	a	device	in	the	Device	Manager	in‐
dicates	which	of	the	following?

    \begin{enumerate}
        \item If	you	see	an	exclamation	point	in	the	Device	Man‐
ager,	this	indicates	that	the	device	does	not	have	a	proper	driver.

    \end{enumerate}
    \item Beep	codes	are	generated	by	which	of	the	following?

    \begin{enumerate}
        \item As	the	power-on	self-test	(POST)	checks	all	the
components	of	the	computer,	it	may	present	its	findings	on	the
screen	or	in	the	form	of	beep	codes.

    \end{enumerate}
    \item Which	of	the	following	indicates	that	a	printer	is	network-ready?

    \begin{enumerate}
        \item The	RJ45	jack	enables	a	connection	to	a	twisted-
pair	(most	likely	Ethernet)	network.	Printers	with	a	built-in	RJ45
connector	are	network-ready;	so	are	printers	that	are	Wi-Fi	en‐
abled.

    \end{enumerate}
    \item You	just	turned	off	a	printer	to	maintain	it.	Which	of	the	follow‐
ing	should	you	be	careful	of	when	removing	the	fuser?

    \begin{enumerate}
        \item 	The	fuser	heats	paper	to	around	400°	Fahrenheit
(204°	Celsius).	That’s	like	an	oven.	If	you	need	to	replace	the
fuser,	let	the	printer	sit	for	10	or	15	minutes	after	shutting	it	down
and	before	maintenance.

    \end{enumerate}
    \item Which	of	the	following	connectors	is	used	for	musical	equipment?

    \begin{enumerate}
        \item The	Musical	Instrument	Digital	Interface	(MIDI)
connector	is	used	for	musical	equipment	such	as	keyboards,	syn‐
thesizers,	and	sequencers.	MIDI	is	used	to	create	a	clocking	sig‐
nal	that	all	devices	can	synchronize	to.
        
    \end{enumerate}
    \item Which	of	the	following	storage	technologies	is	used	by	hard
disk	drives?

    \begin{enumerate}
        \item Hard	disk	drives	(HDDs)	are	magnetic	disks.	Th‐
ese	are	the	type	with	moving	parts,	as	opposed	to	solid-state
drives	(SSDs)	that	have	no	moving	parts.

    \end{enumerate}
    \item Which	protocol	uses	port	389?

    \begin{enumerate}
        \item LDAP 
    \end{enumerate}
    \item Which	of	the	following	wireless	networking	standards	operates
at	5	GHz	only?	(Select	the	two	best	answers.)

    \begin{enumerate}
        \item :	802.11a	operates	at	5	GHz	only;	so	does	802.11ac.

    \end{enumerate}
    \item Which	of	the	following	types	of	RAM	has	a	peak	transfer	rate	of
21,333	MB/s?

    \begin{enumerate}
        \item DDR4-2666	has	a	peak	transfer	rate	of	21,333
MB/s.	It	runs	at	an	I/O	bus	clock	speed	of	1333	MHz	and	can
send	2666	megatransfers	per	second	(MT/s).	It	is	also	known	as
PC4-21333.

    \end{enumerate}
    \item Which	of	the	following	types	of	printers	uses	a	print	head,	ribbon,
and	tractor	feed?

    \begin{enumerate}
        \item The	impact	printer	uses	a	print	head,	ribbon,	and
tractor	feed.	An	example	of	an	impact	printer	is	the	dot	matrix.

    \end{enumerate}
    \item Which	of	the	following	is	a	possible	symptom	of	a	failing	CPU?

    \begin{enumerate}
        \item If	the	CPU	is	running	beyond	the	recommended
voltage	range	for	extended	periods	of	time,	it	can	be	a	sign	of	a
failing	CPU.	The	problem	could	also	be	caused	by	overclocking.
Check	in	the	UEFI/BIOS	to	see	whether	or	not	the	CPU	is	over‐
clocked.

    \end{enumerate}
    \item Which	of	the	following	cable	types	is	not	affected	by	EMI	but	re‐
quires	specialized	tools	to	install?

    \begin{enumerate}
        \item Fiber-optic	cable	is	the	only	answer	listed	that	is
not	affected	by	electromagnetic	interference	(EMI).	The	reason	is
that	it	does	not	use	copper	wire	or	electricity,	but	instead	uses
glass	or	plastic	fibers	and	light.
    \end{enumerate}
    \item Setting	an	administrator	password	in	the	BIOS	accomplishes
which	of	the	following?

    \begin{enumerate}
        \item Changing boot order
    \end{enumerate}
    \item What	is	an	LCD	display’s	contrast	ratio	defined	as?
    \begin{enumerate}
        \item 	Contrast	ratio	is	the	brightness	of	the	brightest
color	(measured	as	white)	compared	to	the	darkest	color	(mea‐
sured	as	black).	Static	contrast	ratio	measurements	are	static;	they
are	performed	as	tests	using	a	checkerboard	pattern.	But	there	is
also	the	dynamic	contrast	ratio,	a	technology	in	LCD	displays
that	adjusts	dynamically	during	darker	scenes	in	an	attempt	to
give	better	black	levels.	It	usually	has	a	higher	ratio,	but	it	should
be	noted	that	there	is	no	real	uniform	standard	for	measuring	con‐
trast	ratio.

    \end{enumerate}
    \item Which	of	the	following	devices	limits	network	broadcasts,	seg‐
ments	IP	address	ranges,	and	interconnects	different	physical	me‐
dia?

    \begin{enumerate}
        \item A	router	can	limit	network	broadcasts	through	seg‐
menting	and	programmed	routing	of	data.	This	is	part	of	a
router’s	job	when	connecting	two	or	more	networks.	It	is	also
used	with	different	media.	For	example,	you	might	have	a	LAN
that	uses	twisted-pair	cable,	but	the	router	connects	to	the	Inter‐
net	via	a	fiber-optic	connection.	That	one	router	will	have	ports
for	both	types	of	connections.

    \end{enumerate}
    \item Which	of	the	following	traits	and	port	numbers	are	associated
with	POP3?	(Select	the	two	best	answers.)

    \begin{enumerate}
        \item OP3	is	a	protocol	used	by	email	clients	to	receive
email.	It	makes	use	of	either	port	110	(considered	insecure)	or
port	995	(a	default	secure	port).

    \end{enumerate}
    \item Which	of	the	following	uses	port	427?

    \begin{enumerate}
        \item The	Service	Location	Protocol	(SLP)	uses	port
427.	It	enables	access	to	network	services	without	previous	con‐
figuration	of	the	client	computer.

    \end{enumerate}
    \item Which	of	the	following	ports	is	used	by	AFP?

    \begin{enumerate}
        \item The	Apple	Filing	Protocol	(AFP)	uses	port	548.
AFP	offers	file	services	for	Mac	computers	running	macOS	and
can	transfer	files	across	the	network.

    \end{enumerate}
    \item Which of the following is the minimum amount of RAM needed
to install a 64-bit version of Windows 10?
    \begin{enumerate}
        \item 
    \end{enumerate}
    \item 
    \begin{enumerate}
        \item 
    \end{enumerate}
    \item 
    \begin{enumerate}
        \item 
    \end{enumerate}
    \item 
    \begin{enumerate}
        \item 
    \end{enumerate}
    \item 
    \begin{enumerate}
        \item 
    \end{enumerate}
    \item 
    \begin{enumerate}
        \item 
    \end{enumerate}
    \item 
    \begin{enumerate}
        \item 
    \end{enumerate}
    \item 
    \begin{enumerate}
        \item 
    \end{enumerate}
    
\end{enumerate}
\section{Chapter 9}
\section{Tables}
\begin{tabular}{|c|c|c|}
\hline
    Technology & Full Name & Port \\
    \hline
   SLP  & Service Location Protocol & 427\\
   FTP & File Transfer Protocol & 21 \\
   DNS & Domain Name System & 53\\
   AFP & Apple Filing Protocol & 548 \\
   HTTP & HyperText Transfer Protocol & 80 \\
   SSH & Secure Shell & 22 \\
   Telnet & Teletype Network & 23\\
   IMAP & Internet Message Access Protocol & 143\\
   \hline
\end{tabular}



\section{Troy McMillan compTIA A+} 
\subsection{Mobile Devices}
\begin{enumerate}
    \item What is the maximum transmission speed of an ExpressCard in PCIe2 mode?
    \begin{itemize}
        \item D) The maximum transmission speeds are as follows: 280 Mbps effective (USB 2 mode) \\
        1.6 Gbps effective (PCIe 1 mode) \\
        3.2 Gbps effective (PCIe 2 or USB 3 mode) 
    \end{itemize}
    \item Which interface is natively found only in Apple devices?
    \begin{itemize}
        \item C) Thunderbolt ports are most likely to be found on Apple laptops, although they are now
showing up on others as well. USB ports are typically found on all mobile devices, while
Serial and PS/2 connecters are rarely found on mobile devices.
    \end{itemize}
     \item What special screwdriver is typically required to work on a notebook?
    \begin{itemize}
        \item B) Some models of notebook PCs require a special T-8 Torx screwdriver. Most PC toolkits
come with a T-8 bit for a screwdriver with interchangeable bits, but you may find that the
T-8 screws are countersunk in deep holes so that you can’t fit the screwdriver into them. In
such cases, you need to buy a separate T-8 screwdriver, available at most hardware stores
or auto parts stores. Phillips-head screwdrivers have a cross pattern on the tip and may be
required. Hex heads are another type you may encounter, and metric drivers are those that
are sized with the metric system.
    \end{itemize}
     \item What is the easiest thing to damage when removing a laptop keyboard?
    \begin{itemize}
        \item B) When replacing the keyboard, one of the main things you want to keep in mind is to not
damage the data cable connector to the system board.
    \end{itemize}
     \item Which component if damaged can render the hard drive useless?
    \begin{itemize}
        \item C) If required, remove the connector attached to the old drive’s signal pins and attach it
to the new drive. Make sure it’s right side up and do not force it. Damaging the signal pins
may render the drive useless. The caddy, rails, and chassis are not easily damaged.
    \end{itemize}
     \item What size hard drive goes in a laptop?
    \begin{itemize}
        \item C) The 2.5-inch hard drives are smaller (which makes them attractive for a laptop where
space is at a minimum); however, in comparison to 3.5-inch hard drives, they have less
capacity and cache, and they operate at a lower speed.
    \end{itemize}
     \item Which is not an advantage of solid-state drives?
    \begin{itemize}
        \item A) The advantage of solid-state drives is that they are not as susceptible to damage if the
device is dropped, and they are, generally speaking, faster as no moving parts are involved.
They are, however, more expensive, and when they fail they don’t generally give some
advance symptoms like a magnetic drive will do.
    \end{itemize}
     \item Which display uses a row of transistors across the top of the screen and a column of them
down the side?
    \begin{itemize}
        \item A passive matrix screen uses a row of transistors across the top of the screen and a
column of them down the side. It sends pulses to each pixel at the intersection of each row
and column combination, telling it what to display. An active matrix screen uses a separate
transistor for each individual pixel in the display, resulting in higher refresh rates and
brighter display quality. Twisted nematic (TN) is the older of the two major technologies
for flat-panel displays. While it provides the shortest response time, has high brightness,
and draws less power than competing technologies, it suffers from poor quality when
viewed from wide angles. In-Plane Switching (IPS) is a newer technology that solves the
issue of poor quality at angles other than straight on.
    \end{itemize}
    \item Which display is a newer technology that solves the issue of poor quality at angles other
than straight on?
    \begin{itemize}
        \item D) In-Plane Switching (IPS) is a newer technology that solves the issue of poor quality
at angles other than straight on. A passive matrix screen uses a row of transistors across
the top of the screen and a column of them down the side. It sends pulses to each pixel at
the intersection of each row and column combination, telling it what to display. An active
matrix screen uses a separate transistor for each individual pixel in the display, resulting in
higher refresh rates and brighter display quality. Twisted nematic (TN) is the older of the
two major technologies for flat-panel displays. While it provides the shortest response time,
has high brightness, and draws less power than competing technologies, it suffers from
poor quality when viewed from wide angles.
    \end{itemize}
    \item Which of the following lets you quickly connect/disconnect with external peripherals and
may also provide extra ports that the notebook PC doesn’t normally have?
    \begin{itemize}
        \item D) With a hot dock, a laptop once put into suspended mode will recognize plug-and-play
devices. A docking station essentially allows a laptop computer to be converted to a desktop
computer. Laptop and table locks are used to secure mobile devices.
    \end{itemize}
    \item In what mode of plug and play must the laptop be turned off and back on for the change to
be recognized?
    \begin{itemize}
        \item C) In cold docking, the laptop must be turned off and back on for the change to be
recognized. In warm docking, the laptop must be put in and out of suspended mode for the
change to be recognized. In hot docking, the change can be made and is recognized while
running normal operations.
    \end{itemize}
    \item Which of the following is a class of devices that specializes in tracking your movement?
    \begin{itemize}
        \item A)While many smart watches can also act as fitness monitors, there is a class of devices
that specializes in tracking your movement. Fitness monitors read your body temperature,
heart rate, and blood pressure. Extended reality is an exciting new field that includes
both augmented reality and virtual reality. Today’s smartphones are really computers
that can make calls, and tablets have been in existence in some form or fashion since the
early 1990s. Early on they were proprietary devices that didn’t have a lot in common with
desktop computers, but increasingly the two form factors have gravitated toward one
another.
    \end{itemize}
    \item Which of the following uses satellite information to plot the global location of an object
and uses that information to plot the route to a second location?
    \begin{itemize}
        \item A) A global positioning system (GPS) uses satellite information to plot the global location
of an object and uses that information to plot the route to a second location. Geofencing is
the use of GPS to restrict communication to an area. Remote wipe is the cleaning of data
from a lost or stolen device. There is no such thing as local wipe.
    \end{itemize}
    \item Which interface is the most common port found on mobile devices?
    \begin{itemize}
        \item A) he two most common ports found on mobile devices are micro-USB and mini-USB.
Both are small–form-factor implementations of the USB standard, the latest of which is
USB 3.1. Thunderbolt ports are most likely to be found on Apple laptops, but they are
now showing up on others as well. Serial and PS/2 connecters are rarely found on mobile
devices. 
    \end{itemize}
    \item Which is the most common pin code when selecting discovered Bluetooth devices?
    \begin{itemize}
        \item A) Many external devices will ask for a PIN when you select the external device from the
list of discovered devices. In many cases, the PIN is 0000, but you should check the manual
of the external device
    \end{itemize}
    \item Which of the following is the connection between the mobile device and the radio?
    \begin{itemize}
        \item A) The product release information (PRI) is the connection between the mobile device and
the radio. From time to time this may need updating, which, when done, may add features or
increase data speed. The preferred roaming list (PRL) is a list of radio frequencies residing in
the memory of some kinds of digital phones. International Mobile Equipment Identification
(IMEI) is used to identify a physical phone device, while International Mobile Subscriber
Identification (IMSI) is used to identify a Subscriber Identification Module (SIM) card.
    \end{itemize}
    \item Which of the following is a process whereby not only does the server verify the credential of
the client but the client also verifies the credential of the server?
    \begin{itemize}
        \item A) Mutual authentication is a process whereby not only does the server verify the
credential of the client but the client also verifies the credential of the server. It adds
additional security to the process. Single sign-on is a service that allows users to sign in
once and have access to all resources. Multifactor authentication makes use of multiple
factors of authentication to increase security. Biometrics is the use of physical factors of
authentication.
    \end{itemize}
    \item Which of the following is the use of physical factors of authentication?
    \begin{itemize}
        \item D) Mutual authentication is a process whereby not only does the server verify the
credential of the client but the client also verifies the credential of the server. It adds
additional security to the process. Single sign-on is a service that allows users to sign in
once and have access to all resources. Multifactor authentication makes use of multiple
factors of authentication to increase security. Biometrics is the use of physical factors of
authentication.
    \end{itemize}
\end{enumerate}
\subsection{Networking} 
\begin{enumerate}
    \item Which the following uses port 110?
    \begin{itemize}
        \item D) POP3 uses port 110. SSH uses port 22, FTP uses ports 20 and 21, and Telnet uses port 23.
    \end{itemize}
    \item Which of the following uses two ports?
     \begin{itemize}
        \item A) FTP uses ports 20 and 21. POP3 uses port 110, SSH uses port 22, and Telnet uses port 23.
    \end{itemize}
    \item Which the following uses port 22?
     \begin{itemize}
        \item B)SSH uses port 22, POP3 uses port 110, FTP uses ports 20 and 21, and Telnet uses port 23.
    \end{itemize}
    \item Which device operates at layer 2?
     \begin{itemize}
        \item A) Switches operate at layer 2. Routers operate at layer 3. Repeaters and hubs operate at layer 1.
    \end{itemize}
    \item Which device operates at layer 1?
     \begin{itemize}
        \item D. Hubs operate at layer 1. Switches and bridges operate at layer 2. Routers operate at layer 3.
    \end{itemize}
    \item Which device operates at layer 2?
     \begin{itemize}
        \item B. Switches operate at layer 2. Routers operate at layer 3. Hubs and repeaters operate at layer 1.
    \end{itemize}
    \item Which of the following is not a private IP address range?
     \begin{itemize}
        \item B. The class B range is 172.16.0.0–172.31.255.255. The other ranges are correct.
    \end{itemize}
    \item Which of the following delivers an upload speed equal to the download speed?
     \begin{itemize}
        \item A) Symmetric DSL (SDSL) offers an upload equal to the download speed. The other
versions all have slower upload speed than download speed. 
    \end{itemize}
    \item Which of the following is an area where you can place a public server for access by people
you might not trust otherwise?
     \begin{itemize}
        \item B). A demilitarized zone (DMZ) is an area where you can place a public server for access
by people you might not trust otherwise. NAT is a service that maps private IP addresses
to public IP addresses. The intranet is the internal network that should be protected. The
Internet is the untrusted public network.
    \end{itemize}
    \item Which of the following operates in the 5.0 GHz range?
     \begin{itemize}
        \item A)802.11a operates in the 5.0 GHz range. The other standards all operate in the 2.4 GHz
range.
    \end{itemize}
    \item Which of the following operates at a maximum of 2 MB?
     \begin{itemize}
        \item D) 802.11a and 802.11g have a maximum rate of 54 MB, 802.11b has a maximum of 11 MB,
and 802.11 has a maximum of 2 MB.
    \end{itemize}
    \item Which of the following has the largest cell size?
     \begin{itemize}
        \item C) 802.11g has a distance that is the cell size of 125 ft. The others have a distance of 115 ft.
    \end{itemize}
    \item Which type of server resolves IP addresses to hostnames?
     \begin{itemize}
        \item B) DNS servers resolve IP addresses to hostnames. HTTP servers are web servers. DHCP
servers provide automatic IP configurations. SQL is a database server.
    \end{itemize}
    \item Which type of server provides automatic IP configurations?
     \begin{itemize}
        \item C) DHCP servers provide automatic IP configurations. DNS servers resolve IP addresses to
hostnames. HTTP servers are web servers. SQL is a database server.
    \end{itemize}
    
    \item Which type of server is a database server?
     \begin{itemize}
        \item D) A SQL server is a database server. DNS servers resolve IP addresses to hostnames.
HTTP servers are web servers. DHCP servers provide automatic IP configurations.
    \end{itemize}
    \item Which of the following is a Class B address?
     \begin{itemize}
        \item The Class B range is 128–191. The class A range is 1–126. The Class C range is 192–223.
    \end{itemize}
    \item Which of the following is a Class A address?
     \begin{itemize}
        \item The Class A range is 1–126. The class B range is 128–191. The Class C range is 192–223.
    \end{itemize}
    \item Which of the following is a Class C address?
     \begin{itemize}
        \item The Class C range is 192–223. The class A range is 1–126. The class B range is 128–191.
The 224 range is for multicasting.
    \end{itemize}
    \item When personal devices include networking capabilities and can communicate directly with
one another, they create which type of network?
     \begin{itemize}
        \item A personal area network (PAN) is a LAN created by personal devices. A wide area
network (WAN) is a collection of two or more LANs, typically connected by routers and
dedicated leased lines. Occasionally, a WAN will be referenced as a metropolitan area
network (MAN) when it is confined to a certain geographic area, such as a university
campus or city. Wireless mesh networks (WMN) are a form of an ad hoc WLAN that often
consist of mesh clients, mesh routers, and gateways.
    \end{itemize}
    \item Which of the following is a collection of two or more LANs, typically connected by routers
and dedicated leased lines?
     \begin{itemize}
        \item Metropolitan area network (MAN) is the term occasionally used for a WAN that is
confined to a certain geographic area, such as a university campus or city. A personal area
network (PAN) is a LAN created by personal devices. A wide area network (WAN) is a
collection of two or more LANs, typically connected by routers and dedicated leased lines.
Wireless mesh networks (WMN) are a form of an ad hoc WLAN that often consist of mesh
clients, mesh routers, and gateways.
    \end{itemize}
    \item Which of the following is a form of ad hoc WLAN?
     \begin{itemize}
        \item Wireless mesh networks (WMN) are a form of an ad hoc WLAN that often consist
of mesh clients, mesh routers, and gateways. A personal area network (PAN) is a LAN
created by personal devices. A wide area network (WAN) is a collection of two or more
LANs, typically connected by routers and dedicated leased lines. Occasionally, a WAN
will be referenced as a metropolitan area network (MAN) when it is confined to a certain
geographic area, such as a university campus or city.
    \end{itemize}
    \item Which of the following is used to attach media connectors to the ends of cables?
     \begin{itemize}
        \item Wire crimpers look like pliers but are used to attach media connectors to the ends
of cables. A cable stripper is used to remove the outer covering of the cable to get to the
wire pairs within. A multimeter combines a number of tools into one. There can be slight
variations, but a multimeter always includes a voltmeter, an ohmmeter, and an ammeter
(and is sometimes called VOM as an acronym). A toner probe has two parts: the tone
generator (called the toner) and the tone locator (called the probe). The toner sends the
tone, and at the other end of the cable, the probe receives the toner’s signal. This tool makes
it easier to find the beginning and end of a cable.
    \end{itemize}
    \item Which of the following includes a voltmeter, an ohmmeter, and an ammeter?
    \begin{itemize}
        \item A multimeter combines a number of tools into one. There can be slight variations, but a
multimeter always includes a voltmeter, an ohmmeter, and an ammeter (and is sometimes
called VOM as an acronym). Wire crimpers look like pliers but are used to attach media
connectors to the ends of cables. A cable stripper is used to remove the outer covering of the
cable to get to the wire pairs within. A toner probe has two parts: the tone generator (called
the toner) and the tone locator (called the probe). The toner sends the tone, and at the other
end of the cable, the probe receives the toner’s signal. This tool makes it easier to find the
beginning and end of a cable.
    \end{itemize}
    \item Which of the following makes it easier to find the beginning and end of a cable?
    \begin{itemize}
        \item A toner probe has two parts: the tone generator (called the toner) and the tone locator
(called the probe). The toner sends the tone, and at the other end of the cable, the probe
receives the toner’s signal. This tool makes it easier to find the beginning and end of a cable.
Wire crimpers look like pliers but are used to attach media connectors to the ends of cables.
A cable stripper is used to remove the outer covering of the cable to get to the wire pairs
within. A multimeter combines a number of tools into one. There can be slight variations,
but a multimeter always includes a voltmeter, an ohmmeter, and an ammeter (and is
sometimes called VOM as an acronym).
    \end{itemize}
\end{enumerate}
\subsection{Hardware} 
\begin{enumerate}
    \item Which cable type comes in two varieties: unshielded and shielded?
    \begin{itemize}
        \item C) Twisted pair is commonly used in office settings to connect workstations to hubs or
switches. It comes in two varieties: unshielded (UTP) and shielded (STP). Fiber optic, serial,
and coaxial do not come in shielded and unshielded versions.
    \end{itemize}
    \item Which cable type transmits data at speeds up to 100 Mbps and was used with Fast Ethernet
(operating at 100 Mbps) with a transmission range of 100 meters?
    \begin{itemize}
        \item B) Cat 5 transmits data at speeds up to 100 Mbps and was used with Fast Ethernet (operating
at 100 Mbps) with a transmission range of 100 meters. It contains four twisted pairs of copper
wire to give the most protection. Although it had its share of popularity (it’s used primarily for
10/100 Ethernet networking), it is now an outdated standard. Newer implementations use the
5e standard. Cat 4 transmits at 16 Mbps, and Cat 6 transmits at 1 Gbps.
    \end{itemize}
    \item Which cable type has a glass core within a rubber outer coating?
    \begin{itemize}
        \item A) Fiber-optic cabling is the most expensive type of those discussed for this exam.
Although it’s an excellent medium, it’s often not used because of the cost of implementing
it. It has a glass core within a rubber outer coating and uses beams of light rather than
electrical signals to relay data. None of the other options uses glass in its construction.
    \end{itemize}
    \item Which connector is used for telephone cord?
    \begin{itemize}
        \item A) An RJ-11 is a standard connector for a telephone line and is used to connect a computer
modem to a phone line. It looks much like an RJ-45 but is noticeably smaller. The RJ-45 is
used for networking. RS 232 is a serial connector. BNC is a coaxial connector.
    \end{itemize}
    \item Which standard has been commonly used in computer serial ports?
    \begin{itemize}
        \item C) The RS-232 standard had been commonly used in computer serial ports. A serial
cable (and port) uses only one wire to carry data in each direction; all the rest are wires
for signaling and traffic control. An RJ-11 is a standard connector for a telephone line and
is used to connect a computer modem to a phone line. It looks much like an RJ-45 but is
noticeably smaller. The RJ-45 is used for networking.
    \end{itemize}
    \item Which connectors are sometimes used in the place of RCA connectors for video electronics?
    \begin{itemize}
        \item D) Bayonet Neill–Concelman (BNC) connectors are sometimes used in the place of RCA
connectors for video electronics, so you may encounter these connectors, especially when
video equipment connects to a PC. In many cases, you may be required to purchase an
adapter to convert this to another form of connection because it is rare to find one on the
PC. An RJ-11 is a standard connector for a telephone line and is used to connect a computer
modem to a phone line. It looks much like an RJ-45 but is noticeably smaller. The RJ45 is
used for networking. RS-232 is a serial connector.
    \end{itemize}
    \item Which RAM type is used in laptops?
    \begin{itemize}
        \item B)Portable computers (notebooks and subnotebooks) require smaller sticks of RAM
because of their smaller size. One of the two types is small outline DIMM (SODIMM),
which can have 72, 144, or 200 pins. DIMM is a full-size RAM type. Rambus is a type of
RAM but not used in laptops, and BNC is a connector for coaxial cabling.
    \end{itemize}
    \item Which RAM type allows for two memory accesses for each rising and falling clock?
    \begin{itemize}
        \item D) DDR SDRAM is Double Data Rate 2 (DDR2). This allows for two memory accesses
for each rising and falling clock and effectively doubles the speed of DDR. DDR2-667
chips work with speeds at 667 MHz and are also referred to as PC2-5300 modules. DDR3
is the higher-speed successor to DDR and DDR2. Portable computers (notebooks and
subnotebooks) require smaller sticks of RAM because of their smaller size. One of the two
types is small outline DIMM (SODIMM), which can have 72, 144, or 200 pins.
    \end{itemize}
    \item Which RAM type is not compatible with any earlier type of random-access memory?
    \begin{itemize}
        \item B) DDR4 SDRAM is an abbreviation for double data rate fourth-generation synchronous dynamic random-access memory. DDR4 is not compatible with any earlier type of random-access memory (RAM). The DDR4 standard allows for DIMMs of up to 64 GB in capacity, compared to DDR3’s maximum of 16 GB per DIMM. DDR3 and DDR2 are backward compatible, and there is no DDR5
    \end{itemize}
    \item Which of the following is a rewritable optical disc?
    \begin{itemize}
        \item B) Compact Disc-ReWritable (CD-RW) media is a rewritable optical disc. A CD-RW drive requires more sensitive laser optics. It can write data to the disc but also has the ability to erase that data and write more data to the disc. CD, DVD, and CD-ROM are all read-only.
    \end{itemize}
    \item Which of the following is a specification for internally mounted computer expansion cards
and associated connectors that replaces the mSATA?
    \begin{itemize}
        \item A) M.2, formerly known as the Next Generation Form Factor (NGFF), is a specification for internally mounted computer expansion cards and associated connectors. It replaces the mSATA standard. M.2 modules are rectangular, with an edge connector on one side, and a semicircular mounting hole at the center of the opposite edge. Non-Volatile Memory Host Controller Interface Specification (NVME) is an open logical device interface specification for accessing nonvolatile storage media attached via a PCI Express (PCIe) bus. Serial ATA and serial ATA 2.5 are computer bus interfaces that connects host bus adapters to mass storage devices such as hard disk drives, optical drives, and solid-state drives
    \end{itemize}
    \item At what speed will latency on a magnetic drive decrease to about 3 ms?
    \begin{itemize}
        \item C) At 10,000 rpm, the latency will decrease to about 3 ms. Data transfer rates also generally go up with a higher rotational speed but are influenced by the density of the disk (the number of tracks and sectors present in a given area). Latency at 5400 rpm will be  5.56 ms. At 7200 it will be 4.17, and at 15000 it will drop to 2.
    \end{itemize}
    \item Laptops and other portable devices utilize which expansion card?
    \begin{itemize}
        \item A) Laptops and other portable devices utilize an expansion card called the miniPCI. It has the same functionality as the PCI but has a much smaller form factor. PCI and PCIe are used in desktops. SATA is a drive connector.
    \end{itemize}
    \item Which of the following is a standard firmware interface for PCs, designed to replace BIOS?
    \begin{itemize}
        \item A) Unified Extensible Firmware Interface (UEFI) is a standard firmware interface for PCs, designed to replace BIOS. NVRAM is RAM that retains its data during a reboot. CMOS is a battery type found on motherboards, and CHS is a drive geometry concept
    \end{itemize}
    \item Which of the following is memory that does not lose its content when power is lost to the
machine?
    \begin{itemize}
        \item  B) NVRAM is memory that does not lose its content when power is lost to the machine. Unified Extensible Firmware Interface (UEFI) is a standard firmware interface for PCs, designed to replace BIOS. CMOS is a battery type found on motherboards, and CHS is a drive geometry concept.
    \end{itemize}
    \item Which of the following devices allows you to plug multiple PCs (usually servers) into the
device and to switch easily back and forth from system to system using the same mouse,
monitor, and keyboard?
    \begin{itemize}
        \item A) A keyboard, video, and mouse (KVM) device allows you to plug multiple PCs (usually servers) into the device and to switch easily back and forth from system to system using the same mouse, monitor, and keyboard. The KVM is actually a switch that all the systems plug into. There is usually no software to install. Just turn off all the systems, plug them all into the switch and turn them back on; then you can switch from one to another using the same keyboard, monitor, and mouse device connected to the KVM switch. CMOS is a battery type found on motherboards, and CHS is a drive geometry concept. NVRAM is memory that does not lose its content when power is lost to the machine.
    \end{itemize}
    \item Which of the following is a description of light output?
    \begin{itemize}
        \item B) When discussing bulbs for projectors, brightness is a description of light output, which is measured in lumens (not watts). Ensure that you are purchasing the correct bulb for the projector and maximize the life of the bulb by turning the projector off when not in use. A keyboard, video, and mouse (KVM) device allows you to plug multiple PCs (usually servers) into the device and to switch easily back and forth from system to system using the same mouse, monitor, and keyboard. Contrast is the relationship between dark and light. CHS is a drive geometry concept.
    \end{itemize}
    \item Which of the following is a standard managed by the ISO and uses tags that are embedded
in the devices?
    \begin{itemize}
        \item C) NFC components include an initiator and a target; the initiator actively generates an RF field that can power a passive target. This enables NFC targets to take simple form factors such as tags, stickers, key fobs, or cards that do not require batteries. When discussing bulbs for projectors, brightness is a description of light output, which is measured in lumens (not watts). Ensure that you are purchasing the correct bulb for the projector and maximize the life of the bulb by turning the projector off when not in use. A keyboard, video, and mouse (KVM) device allows you to plug multiple PCs (usually servers) into the device and to switch easily back and forth from system to system using the same mouse, monitor, and keyboard. CHS is a drive geometry concept
    \end{itemize}
    \item In 2004, the ATX 12V 2.0 (now 2.03) standard was passed, changing the main connector
from 20 pins to how many?
    \begin{itemize}
        \item C) In 2004, the ATX 12V 2.0 (now 2.03) standard was passed, changing the main connector from 20 pins to 24. The additional pins provide +3.3V, +5V, and +12V (the fourth pin is a ground) for use by PCIe cards. When a 24-pin connector is used, there is no need for the optional four- or six-pin auxiliary power connectors.
    \end{itemize}
    \item When using the AT power connector, the power cable coming from the power supply will
have two separate connectors, labeled what?
    \begin{itemize}
        \item When using the AT power connector, the power cable coming from the power supply will have two separate connectors, labeled P8 and P9. When you are attaching the two parts to the motherboard, the black wires on one should be next to the black wires on the other for proper function
    \end{itemize}
      \item The SATA power connector has how many pins?
    \begin{itemize}
        \item C)The SATA power connector consists of 15 pins, with 3 pins designated for 3.3V, 5V, and 12V and with each pin carrying 1.5 amps. This results in a total draw of 4.95 watts + 7.5 watts + 18 watts, or about 30 watts
    \end{itemize}
      \item Which of the following is a desktop computer system?
    \begin{itemize}
        \item C) A thick client has the applications installed locally and will need to have sufficient resources to support the applications. A thin client only sends commands and displays output with the application on the server. Network attached storage is a storage network that is IP based, while Storage Area Networks use a storage area protocol
    \end{itemize}
     \item Which of the following is a PC that has all the capabilities of a standard PC?
    \begin{itemize}
        \item C) When discussing thin and thick clients, you should understand that a thick client is a PC
that has all the capabilities of a standard PC. It runs all applications locally from its own hard
drive. A thin client is one that has minimal capabilities and runs the applications (and perhaps
even the operating system itself) from a remote server. There is no standard client or thin host.
    \end{itemize}
     \item The amount of RAM that is required in a virtualization workstation depends on which of
the following?
    \begin{itemize}
        \item C) Number of VMs
    \end{itemize}
     \item Which IP setting is optional for network connectivity on a thin client?
    \begin{itemize}
        \item D) proxy server address is optional.
    \end{itemize}
     \item Which of the following needs the most resources?
    \begin{itemize}
        \item B) A thick client is a standard PC. When discussing thin and thick clients, you should understand that a thick client is a PC that has all the capabilities of a standard PC. It runs all applications locally from its own hard drive. A thin client is one that has minimal capabilities and runs the applications (and perhaps even the operating system itself) from a remote server. There is no medium client or stationary client
    \end{itemize}
     \item How is accountability ensured?
    \begin{itemize}
        \item C) Ensure accountability by using no shared accounts. Each user should have a unique username/password combination. Audit trails should always be created.
    \end{itemize}
     \item What software controls how the printer processes the print job?
    \begin{itemize}
        \item A) When you install a printer driver for the printer you are using, it allows the computer to print to that printer correctly (assuming you have the correct interface configured between the computer and printer). Also, keep in mind that drivers are specific to the operating system, so you need to select the one that is both for the correct printer and for the correct operating system.
    \end{itemize}
     \item What printer component turns the printed sheet over so it can be run back through the
printer and allow printing on both sides?
    \begin{itemize}
        \item B) An optional component that can be added to printers (usually laser but also inkjet) is a duplexer. This can be an optional assembly added to the printer, or built into it, but the sole purpose of duplexing is to turn the printed sheet over so it can be run back through the printer and allow printing on both sides
    \end{itemize}
     \item Which of the following refers to how the printed matter is laid out on the page?
    \begin{itemize}
        \item C) .The orientation of a document refers to how the printed matter is laid out on the page. In the landscape orientation, the printing is written across the paper turned on its long side, while in portrait the paper is turned up vertically and printed top to bottom. The driver is the software that talks between the printer and the operating system. Duplexing makes it possible to print on both sides. To collate is to create multiple copies with all sets in correct page order
    \end{itemize}
     \item Which of the following feeds through the printer using a system of sprockets and tractors?
    \begin{itemize}
        \item A)Continuous-feed paper feeds through the printer using a system of sprockets and tractors. Sheet-fed printers accept plain paper in a paper tray. Dot matrix is continuous feed; everything else is sheet fed
    \end{itemize}
     \item Which of the following should not be used more than once?
    \begin{itemize}
        \item B) Never reuse paper in a laser printer that has been through the printer once. Although it may look blank, you’re repeating the charging and fusing process on a piece of paper that most likely has something already on it.
    \end{itemize}
     \item Which of the following is a large circuit board that acts as the motherboard for the printer?
    \begin{itemize}
        \item A) This is a large circuit board that acts as the motherboard for the printer. It contains the processor and RAM to convert data coming in from the computer into a picture of a page to be printed. The imaging drum is the drum where the toner is placed on the correctly charged area. The toner cartridge is the container holding the toner. The maintenance kit contains items that should be changed periodically like rollers
    \end{itemize}
\end{enumerate}
\subsection{Virtualization and Cloud Computing}
\begin{enumerate}
    \item Which of the following involves the vendor providing the entire solution?
    \begin{itemize}
        \item Software as a service (SaaS) involves the vendor providing the entire solution. This
includes the operating system, the infrastructure software, and the application. Infrastructure
as a service ( IaaS) provides only the hardware platform to the customer. Platform as a service
( PaaS) provides a development environment. Security Information and Event Management
(SIEM) is a system that aggregates all log files and analyzes them in real time for attacks.
    \end{itemize}
    \item When a company pays another company to host and manage a cloud environment, it is
called what?
    \begin{itemize}
        \item B) When a company pays another company to host and manage a cloud environment, it is
called a public cloud solution. If the company hosts this environment itself, it is a private
cloud solution. A hybrid cloud solution is one in which both public and private clouds are
part of the solution. A community cloud is one in which multiple entities use the cloud.
    \end{itemize}
    \item Which of the following is the ability to add resources as needed on the fly and release those resources when they are no longer required?
    \begin{itemize}
        \item B) One of the advantages of a cloud environment is the ability to add resources as needed
on the fly and release those resources when they are no longer required. This makes for
more efficient use of resources, placing them where needed at any particular point in
time. These include CPU and memory resources. This is called rapid elasticity because it
occurs automatically according to the rules for resource sharing that have been deployed.
On-demand refers to the ability of the customer to add resources as needed. Virtual sharing
and stretched resources are not terms used when discussing the cloud.
    \end{itemize}
    \item In which VDI model are all desktop instances stored in a single server, requiring significant
processing power on the server?
    \begin{itemize}
        \item A) There are three models for implementing VDI:
Centralized model: All desktop instances are stored in a single server, requiring significant
processing power on the server.
Hosted model: Desktops are maintained by a service provider. This model eliminates capital
cost and is instead subject to operation cost.
Remote virtual desktops model: An image is copied to the local machine, making a constant
network connection unnecessary.
There is no local model.
    \end{itemize}
    \item Which of the following involves the vendor providing the hardware platform or data center
and the software running on the platform?
    \begin{itemize}
        \item A) PaaS (Platform as a Service) 
    \end{itemize}
    \item What is the benefit derived from using hardware-assisted virtualization?
    \begin{itemize}
        \item A) Better performance 
    \end{itemize}
    \item Which of the following is the software that allows the VMs to exist?
    \begin{itemize}
        \item B) Hypervisor
    \end{itemize}
    \item Which hypervisor type runs directly on the host’s hardware?
    \begin{itemize}
        \item B) Type I 
    \end{itemize}
    \item Which of the following is an example of a Type II hypervisor?
    \begin{itemize}
        \item A) Oracle VirtualBox
    \end{itemize}
    \item Which of the following hypervisors runs within a conventional operating system?
    \begin{itemize}
        \item Type I 
    \end{itemize}
\end{enumerate}
\subsection{Hardware and Network Troubleshooting}
\begin{enumerate}
    \item Which of the following is the final step in the CompTIA troubleshooting method?
    \begin{itemize}
        \item B) Document findings, actions, and outcomes.
    \end{itemize}
    \item Which of the following is the first step in the CompTIA troubleshooting method?
    \begin{itemize}
        \item (D) Identify the problem 
    \end{itemize}
    \item What is the most common reason for an unexpected reboot?
    \begin{itemize}
        \item A. One common reason for shutdowns is overheating. Often when that is the case,
however, the system reboots itself rather than just shutting down.
    \end{itemize}
    \item Which of the following is typically not a cause of system lockups?
    \begin{itemize}
        \item D. A bad NIC driver would cause the NIC not to work but would not cause a system
lockup.
    \end{itemize}
    \item What are proprietary screen crashes called in Windows?
    \begin{itemize}
        \item B. Once a regular occurrence when working with Windows, blue screens (also known as
the blue screen of death) have become much less frequent.
    \end{itemize}
    \item Which operating system uses the Pinwheel of Death as a proprietary screen crash?
    \begin{itemize}
        \item A. While Microsoft users have the BSOD to deal with, Apple users have also come to have
the same negative feelings about the Pinwheel of Death. This is a multicolored pinwheel
mouse pointer.
    \end{itemize}
    \item What are the small dots on the screen that are filled with a color?
    \begin{itemize}
        \item A. Pixels are the small dots on the screen that are filled with a color; as a group they
present the image you see on the screen.
    \end{itemize}
    \item What are visual anomalies that appear on the screen called? 
    \begin{itemize}
        \item B. Artifacts are visual anomalies that appear on the screen. They might be pieces of images
left over from a previous image or a “tear in the image” (it looks like the image is divided
into two parts and the parts don’t line up).
    \end{itemize}
    \item What is the light in the device that powers the LCD screen?
    \begin{itemize}
        \item A. The backlight is the light in the device that powers the LCD screen. It can go bad over
time and need to be replaced, and it can also be held captive by the inverter. The inverter
takes the DC power the laptop is providing and boosts it up to AC to run the backlight. If
the inverter goes bad, you can replace it on most models (it’s cheaper than the backlight).
    \end{itemize}
    \item Which of the following is a user interface feature designed by HTC?
    \begin{itemize}
        \item B. Touch flow, or TouchFLO, is a user interface feature designed by HTC. It is used by
dragging your finger up and down or left and right to access common tasks on the screen. This
movement is akin to scrolling the screen up and down or scrolling the screen left and right.
    \end{itemize}
    \item Which of the following indicates that the fuser is not fusing the toner properly on the
paper?
    \begin{itemize}
        \item B. With laser printers, streaks usually indicate that the fuser is not fusing the toner
properly on the paper. It could also be that the incorrect paper is being used. In laser
printers, you can sometimes tell the printer that you are using a heavier paper. For dot-
matrix, you can adjust the platen for thicker paper.
    \end{itemize}
    \item Which of the following indicates that the toner cartridge is just about empty?
    \begin{itemize}
        \item C. In laser printers, faded output usually indicates that the toner cartridge is just about
empty. You can usually remove it, shake it, and replace it and then get a bit more life out of
it before it is completely empty, but it is a signal that you are near the end.
    \end{itemize}
    \item If you can ping resources by IP address but not by name,
functional.
   \begin{itemize}
       \item B) DNS . You may be able to ping the entire network using IP addresses, but most access is done
by name, not IP address. If you can’t ping resources by name, DNS is not functional,
meaning either the DNS server is down or the local machine is not configured with the
correct IP address of the DNS server.
   \end{itemize}
   \item Which of the following should be set to the IP address of the router interface connecting to
the local network?
    \begin{itemize}
        \item C) If the computer cannot connect to the default gateway, it will be confined to
communicating with devices on the local network. This IP address should be that of the
router interface connecting to the local network.
    \end{itemize}
\end{enumerate}
\section{Mike Meyers} 
\subsection{The Visible Computer}
\begin{enumerate}
    \item Which version of Windows introduced the Metro UI?
    \begin{itemize}
        \item B)Microsoft introduced Metro UI with Windows 8.
    \end{itemize}
    \item Which Windows 8 feature did Microsoft not include in Windows 10?
    \begin{itemize}
        \item D) Microsoft did not include the Charms bar in Windows 10. Bye!
    \end{itemize}
    \item What macOS feature is essentially multiple Desktops?
    \begin{itemize}
        \item D) Spaces is the term Apple uses for multiple Desktops in macOS.
    \end{itemize}
    \item What KDE feature is essentially the Start button?
    \begin{itemize}
        \item Kickoff functions like a Start button for KDE desktops.
    \end{itemize}
    \item The user Mike has downloaded files with his Web browser. Where
will they be stored by default?
    \begin{itemize}
        \item C:$\backslash Users\backslash Mike\backslash$ Downloads
    \end{itemize}
    \item 32-bit programs are installed into which folder by default in a 64-bit
edition of Windows?
    \begin{itemize}
        \item By default, 32-bit applications install into the $C:\backslash Program Files
(x86)$ folder.
    \end{itemize}
    \item Which macOS feature is functionally equivalent to Windows File
Explorer?
    \begin{itemize}
        \item A) Finder is the equivalent of File Explorer.
    \end{itemize}
    \item Which of the following paths would open Administrative Tools in
Windows 8.1?
    \begin{itemize}
        \item B) To open Administrative Tools, right-click the Start button and
select Administrative Tools. Easy!
    \end{itemize}
    \item What feature of macOS is the equivalent of the command-line
interface in Windows
    \begin{itemize}
        \item C) Terminal is the equivalent of the Windows command-line
interface.
    \end{itemize}
    \item What Windows app in Windows 10 combines many utilities into a
unified tool?
    \begin{itemize}
        \item C)The Settings app in Windows 10 offers many utilities in a unified
interface.
    \end{itemize}

\end{enumerate}
\subsection{CPUs}
\begin{enumerate}
     \item What do registers provide for the CPU?
    \begin{itemize}
        \item B) The CPU uses registers for temporary storage of internal
commands and data.
    \end{itemize}
    \item C) What function does the external data bus have in the PC?
    \begin{itemize}
        \item The external data bus provides a channel for the flow of data and
commands between the CPU and RAM.
    \end{itemize}
    \item What is the function of the address bus in the PC?
    \begin{itemize}
        \item A) The address bus enables the CPU to communicate with the
memory controller chip.
    \end{itemize}
    \item Which of the following terms are measures of CPU speed?
    \begin{itemize}
        \item A) Megahertz and gigahertz
    \end{itemize}
    \item Which CPU feature enables the microprocessor to support running
multiple operating systems at the same time?
    \begin{itemize}
        \item D) Virtualization support
    \end{itemize}
    \item Into which socket could you place an Intel Core i5?
    \begin{itemize}
        \item B) Socket LGA 1151
    \end{itemize}
    \item Which feature enables a single-core CPU to function like two CPUs?
    \begin{itemize}
        \item A) Hyper-Threading
    \end{itemize}
    \item What steps do you need to take to install a Core i3 CPU into an FM2+
motherboard?
    \begin{itemize}
        \item D) Take all of the steps you want to take because it’s not going to
work
    \end{itemize}
    \item A client calls to complain that his computer starts up, but crasheswhen Windows starts to load. After a brief set of questions, you findout that his nephew upgraded his RAM for him over the weekend andcouldn’t get the computer to work right afterward. What could be theproblem?
    \begin{itemize}
        \item A)   Most likely, the nephew disconnected the CPU fan to get at the RAM slots and simply forgot to plug it back in. 
    \end{itemize}
    \item Darren has installed a new CPU in a client’s computer, but nothinghappens when he pushes the power button on the case. The LED onthe motherboard is lit up, so he knows the system has power. Whatcould the problem be?
    \begin{itemize}
        \item The best answer here is that he forgot the thermal paste, thoughyou can also make an argument for a disconnected fan.
    \end{itemize}
\end{enumerate}
\subsection{RAM}
\begin{enumerate}
    \item Steve adds a second 8-GB 288-pin DIMM to his PC, which should
bring the total RAM in the system up to 16 GB. The PC has an Intel
Core i7 4-GHz processor and four 288-pin DIMM slots on the
motherboard. When he turns on the PC, however, only 8 GB of RAM
shows up in Windows Settings app. Which of the following is most
likely to be the problem?
      \begin{itemize}
          \item 
      \end{itemize}
    \item 
    \begin{itemize}
        \item 
    \end{itemize}
    \item 
    \begin{itemize}
        \item 
    \end{itemize}
    \item 
    \begin{itemize}
        \item 
    \end{itemize}
    \item 
    \begin{itemize}
        \item 
    \end{itemize}
    \item 
    \begin{itemize}
        \item 
    \end{itemize}
    \item 
    \begin{itemize}
        \item 
    \end{itemize}
    \item 
    \begin{itemize}
        \item 
    \end{itemize}
    \item 
    \begin{itemize}
        \item 
    \end{itemize}
    \item 
    \begin{itemize}
        \item 
    \end{itemize}
\end{enumerate}
\end{document}



